\pdfoutput=1
%% Author: PGL  Porta Mana
%% Created: 2018-06-23T13:54:28+0200
%% Last-Updated: 2018-09-03T15:06:12+0200
%%%%%%%%%%%%%%%%%%%%%%%%%%%%%%%%%%%%%%%%%%%%%%%%%%%%%%%%%%%%%%%%%%%%%%
% Report-no: ***
\newif\ifarxiv
\arxivfalse
\ifarxiv\pdfmapfile{+classico.map}\fi
\newif\ifafour
\afourfalse % true = A4, false = A5
\newif\iftypodisclaim % typographical disclaim on the side
\typodisclaimtrue
\newif\ifpublic
\publictrue % true = for publication, false = personal notes
\newcommand*{\memfontfamily}{zplx}
\newcommand*{\memfontpack}{newpxtext}
\documentclass[\ifafour a4paper,12pt,\else a5paper,10pt,\fi%extrafontsizes,%
onecolumn,oneside,article,%french,italian,german,swedish,latin,
british%
]{memoir}
\newcommand*{\updated}{\today}
\newcommand*{\firstdraft}{23 June 2018}
\newcommand*{\firstpublished}{\today}
\newcommand*{\propertitle}{A geometric tutorial on exchangeability\\and
  related topics}
\newcommand*{\pdftitle}{A geometric tutorial on exchangeability and related topics}
\newcommand*{\headtitle}{Geometry of exchangeability}
\newcommand*{\pdfauthor}{P.G.L.  Porta Mana}
\newcommand*{\headauthor}{\ifpublic Porta Mana%
\else Luca\fi}
\newcommand*{\reporthead}{}%Open Science Framework \osfeprint{***}}
%%%%%%%%%%%%%%%%%%%%%%%%%%%%%%%%%%%%%%%%%%%%%%%%%%%%%%%%%%%%%%%%%%%%%%%%%%%%
%%%%%%%%%%%%%%%%%%%%%%%%%%%%%%%%%%%%%%%%%%%%%%%%%%%%%%%%%%%%%%%%%%%%%%%%%%%%
%\usepackage{pifont}
%\usepackage{fontawesome}
\usepackage[T1]{fontenc} 
\input{glyphtounicode} \pdfgentounicode=1
\usepackage[utf8]{inputenx}
%\usepackage{newunicodechar}
% \newunicodechar{Ĕ}{\u{E}}
% \newunicodechar{ĕ}{\u{e}}
% \newunicodechar{Ĭ}{\u{I}}
% \newunicodechar{ĭ}{\u{\i}}
% \newunicodechar{Ŏ}{\u{O}}
% \newunicodechar{ŏ}{\u{o}}
% \newunicodechar{Ŭ}{\u{U}}
% \newunicodechar{ŭ}{\u{u}}
% \newunicodechar{Ā}{\=A}
% \newunicodechar{ā}{\=a}
% \newunicodechar{Ē}{\=E}
% \newunicodechar{ē}{\=e}
% \newunicodechar{Ī}{\=I}
% \newunicodechar{ī}{\={\i}}
% \newunicodechar{Ō}{\=O}
% \newunicodechar{ō}{\=o}
% \newunicodechar{Ū}{\=U}
% \newunicodechar{ū}{\=u}
% \newunicodechar{Ȳ}{\=Y}
% \newunicodechar{ȳ}{\=y}

\newcommand*{\bmmax}{0} % reduce number of bold fonts, before bm
\newcommand*{\hmmax}{0} % reduce number of heavy fonts, before bm
\usepackage{textcomp}
\usepackage[normalem]{ulem}
% \makeatletter
% \def\ssout{\bgroup \ULdepth=-.35ex%\UL@setULdepth
%  \markoverwith{\lower\ULdepth\hbox
%    {\kern-.03em\vbox{\hrule width.2em\kern1.2\p@\hrule}\kern-.03em}}%
%  \ULon}
% \makeatother
\usepackage{amsmath}
\usepackage{mathtools}
\addtolength{\jot}{\jot} % increase spacing in multiline formulae
\usepackage{empheq}% automatically calls amsmath and mathtools
\newcommand*{\widefbox}[1]{\fbox{\hspace{1em}#1\hspace{1em}}}
\setlength{\multlinegap}{0pt}
%\usepackage{fancybox}
%\usepackage{framed}
% \usepackage[misc]{ifsym} % for dice
% \newcommand*{\diceone}{{\scriptsize\Cube{1}}}
\usepackage{amssymb}
\usepackage{amsxtra}

\usepackage[main=british,french,italian,german,swedish,latin,esperanto]{babel}\selectlanguage{british}
\newcommand*{\langfrench}{\foreignlanguage{french}}
\newcommand*{\langgerman}{\foreignlanguage{german}}
\newcommand*{\langitalian}{\foreignlanguage{italian}}
\newcommand*{\langswedish}{\foreignlanguage{swedish}}
\newcommand*{\langlatin}{\foreignlanguage{latin}}
\newcommand*{\langnohyph}{\foreignlanguage{nohyphenation}}

\usepackage[autostyle=false,autopunct=false,english=british]{csquotes}
\setquotestyle{british}

\usepackage{amsthm}
\newcommand*{\QED}{\textsc{q.e.d.}}
\renewcommand*{\qedsymbol}{\QED}
\theoremstyle{remark}
\newtheorem{note}{Note}
\newtheorem*{remark}{Note}
\newtheoremstyle{innote}{\parsep}{\parsep}{\footnotesize}{}{}{}{0pt}{}
\theoremstyle{innote}
\newtheorem*{innote}{}

\usepackage[shortlabels,inline]{enumitem}
\SetEnumitemKey{para}{itemindent=\parindent,leftmargin=0pt,listparindent=\parindent,parsep=0pt,itemsep=\topsep}
% \begin{asparaenum} = \begin{enumerate}[para]
% \begin{inparaenum} = \begin{enumerate*}
\setlist[enumerate,2]{label=\alph*.}
\setlist[enumerate]{leftmargin=\parindent}
\setlist[itemize]{leftmargin=\parindent}
\setlist[description]{leftmargin=\parindent}

\usepackage[babel,theoremfont]{newpxtext}
\usepackage[bigdelims,nosymbolsc%,smallerops % probably arXiv doesn't have it
]{newpxmath}
\useosf\linespread{1.083}
%% smaller operators for old version of newpxmath
\makeatletter
\def\re@DeclareMathSymbol#1#2#3#4{%
    \let#1=\undefined
    \DeclareMathSymbol{#1}{#2}{#3}{#4}}
%\re@DeclareMathSymbol{\bigsqcupop}{\mathop}{largesymbols}{"46}
%\re@DeclareMathSymbol{\bigodotop}{\mathop}{largesymbols}{"4A}
\re@DeclareMathSymbol{\bigoplusop}{\mathop}{largesymbols}{"4C}
\re@DeclareMathSymbol{\bigotimesop}{\mathop}{largesymbols}{"4E}
\re@DeclareMathSymbol{\sumop}{\mathop}{largesymbols}{"50}
\re@DeclareMathSymbol{\prodop}{\mathop}{largesymbols}{"51}
\re@DeclareMathSymbol{\bigcupop}{\mathop}{largesymbols}{"53}
\re@DeclareMathSymbol{\bigcapop}{\mathop}{largesymbols}{"54}
%\re@DeclareMathSymbol{\biguplusop}{\mathop}{largesymbols}{"55}
\re@DeclareMathSymbol{\bigwedgeop}{\mathop}{largesymbols}{"56}
\re@DeclareMathSymbol{\bigveeop}{\mathop}{largesymbols}{"57}
%\re@DeclareMathSymbol{\bigcupdotop}{\mathop}{largesymbols}{"DF}
%\re@DeclareMathSymbol{\bigcapplusop}{\mathop}{largesymbolsPXA}{"00}
%\re@DeclareMathSymbol{\bigsqcupplusop}{\mathop}{largesymbolsPXA}{"02}
%\re@DeclareMathSymbol{\bigsqcapplusop}{\mathop}{largesymbolsPXA}{"04}
%\re@DeclareMathSymbol{\bigsqcapop}{\mathop}{largesymbolsPXA}{"06}
\re@DeclareMathSymbol{\bigtimesop}{\mathop}{largesymbolsPXA}{"10}
%\re@DeclareMathSymbol{\coprodop}{\mathop}{largesymbols}{"60}
%\re@DeclareMathSymbol{\varprod}{\mathop}{largesymbolsPXA}{16}
\makeatother


%% With euler font cursive for Greek letters - the [1] means 100% scaling
\DeclareFontFamily{U}{egreek}{\skewchar\font'177}%
\DeclareFontShape{U}{egreek}{m}{n}{<-6>s*[1]eurm5 <6-8>s*[1]eurm7 <8->s*[1]eurm10}{}%
\DeclareFontShape{U}{egreek}{m}{it}{<->s*[1]eurmo10}{}%
\DeclareFontShape{U}{egreek}{b}{n}{<-6>s*[1]eurb5 <6-8>s*[1]eurb7 <8->s*[1]eurb10}{}%
\DeclareFontShape{U}{egreek}{b}{it}{<->s*[1]eurbo10}{}%
\DeclareSymbolFont{egreeki}{U}{egreek}{m}{it}%
\SetSymbolFont{egreeki}{bold}{U}{egreek}{b}{it}% from the amsfonts package
\DeclareSymbolFont{egreekr}{U}{egreek}{m}{n}%
\SetSymbolFont{egreekr}{bold}{U}{egreek}{b}{n}% from the amsfonts package
% Take also \sum, \prod, \coprod symbols from Euler fonts
\DeclareFontFamily{U}{egreekx}{\skewchar\font'177}
\DeclareFontShape{U}{egreekx}{m}{n}{%
       <-7.5>s*[0.9]euex7%
    <7.5-8.5>s*[0.9]euex8%
    <8.5-9.5>s*[0.9]euex9%
    <9.5->s*[0.9]euex10%
}{}
\DeclareSymbolFont{egreekx}{U}{egreekx}{m}{n}
\DeclareMathSymbol{\sumop}{\mathop}{egreekx}{"50}
\DeclareMathSymbol{\prodop}{\mathop}{egreekx}{"51}
\DeclareMathSymbol{\coprodop}{\mathop}{egreekx}{"60}
\makeatletter
\def\sum{\DOTSI\sumop\slimits@}
\def\prod{\DOTSI\prodop\slimits@}
\def\coprod{\DOTSI\coprodop\slimits@}
\makeatother
\ifarxiv\else\input{../undefinegreek.tex}\fi% make sure no CMF greek letters sneak in
% Greek letters not usually given in LaTeX. Comment the unneeded ones
% \DeclareMathSymbol{\varpartial}{\mathalpha}{egreeki}{"40}
 \DeclareMathSymbol{\partialup}{\mathalpha}{egreekr}{"40}
% \DeclareMathSymbol{\alpha}{\mathalpha}{egreeki}{"0B}
% \DeclareMathSymbol{\beta}{\mathalpha}{egreeki}{"0C}
% \DeclareMathSymbol{\gamma}{\mathalpha}{egreeki}{"0D}
% \DeclareMathSymbol{\delta}{\mathalpha}{egreeki}{"0E}
 \DeclareMathSymbol{\epsilon}{\mathalpha}{egreeki}{"0F}
% \DeclareMathSymbol{\zeta}{\mathalpha}{egreeki}{"10}
% \DeclareMathSymbol{\eta}{\mathalpha}{egreeki}{"11}
 \DeclareMathSymbol{\theta}{\mathalpha}{egreeki}{"12}
% \DeclareMathSymbol{\iota}{\mathalpha}{egreeki}{"13}
 \DeclareMathSymbol{\kappa}{\mathalpha}{egreeki}{"14}
 \DeclareMathSymbol{\lambda}{\mathalpha}{egreeki}{"15}
 \DeclareMathSymbol{\mu}{\mathalpha}{egreeki}{"16}
 \DeclareMathSymbol{\nu}{\mathalpha}{egreeki}{"17}
% \DeclareMathSymbol{\xi}{\mathalpha}{egreeki}{"18}
% \DeclareMathSymbol{\omicron}{\mathalpha}{egreeki}{"6F}
% \DeclareMathSymbol{\pi}{\mathalpha}{egreeki}{"19}
% \DeclareMathSymbol{\rho}{\mathalpha}{egreeki}{"1A}
% \DeclareMathSymbol{\sigma}{\mathalpha}{egreeki}{"1B}
% \DeclareMathSymbol{\tau}{\mathalpha}{egreeki}{"1C}
% \DeclareMathSymbol{\upsilon}{\mathalpha}{egreeki}{"1D}
% \DeclareMathSymbol{\phi}{\mathalpha}{egreeki}{"1E}
% \DeclareMathSymbol{\chi}{\mathalpha}{egreeki}{"1F}
% \DeclareMathSymbol{\psi}{\mathalpha}{egreeki}{"20}
% \DeclareMathSymbol{\omega}{\mathalpha}{egreeki}{"21}
% \DeclareMathSymbol{\varepsilon}{\mathalpha}{egreeki}{"22}
% \DeclareMathSymbol{\vartheta}{\mathalpha}{egreeki}{"23}
% \DeclareMathSymbol{\varpi}{\mathalpha}{egreeki}{"24}
% \let\varrho\rho 
% \let\varsigma\sigma
 \let\varkappa\kappa
% \DeclareMathSymbol{\varphi}{\mathalpha}{egreeki}{"27}
% %
% \DeclareMathSymbol{\varAlpha}{\mathalpha}{egreeki}{"41}
% \DeclareMathSymbol{\varBeta}{\mathalpha}{egreeki}{"42}
% \DeclareMathSymbol{\varGamma}{\mathalpha}{egreeki}{"00}
 \DeclareMathSymbol{\varDelta}{\mathalpha}{egreeki}{"01}
 \DeclareMathSymbol{\varEpsilon}{\mathalpha}{egreeki}{"45}
% \DeclareMathSymbol{\varZeta}{\mathalpha}{egreeki}{"5A}
 \DeclareMathSymbol{\varEta}{\mathalpha}{egreeki}{"48}
% \DeclareMathSymbol{\varTheta}{\mathalpha}{egreeki}{"02}
 \DeclareMathSymbol{\varIota}{\mathalpha}{egreeki}{"49}
% \DeclareMathSymbol{\varKappa}{\mathalpha}{egreeki}{"4B}
% \DeclareMathSymbol{\varLambda}{\mathalpha}{egreeki}{"03}
% \DeclareMathSymbol{\varMu}{\mathalpha}{egreeki}{"4D}
% \DeclareMathSymbol{\varNu}{\mathalpha}{egreeki}{"4E}
% \DeclareMathSymbol{\varXi}{\mathalpha}{egreeki}{"04}
 \DeclareMathSymbol{\varOmicron}{\mathalpha}{egreeki}{"4F}
% \DeclareMathSymbol{\varPi}{\mathalpha}{egreeki}{"05}
% \DeclareMathSymbol{\varRho}{\mathalpha}{egreeki}{"50}
% \DeclareMathSymbol{\varSigma}{\mathalpha}{egreeki}{"06}
% \DeclareMathSymbol{\varTau}{\mathalpha}{egreeki}{"54}
% \DeclareMathSymbol{\varUpsilon}{\mathalpha}{egreeki}{"07}
% \DeclareMathSymbol{\varPhi}{\mathalpha}{egreeki}{"08}
% \DeclareMathSymbol{\varChi}{\mathalpha}{egreeki}{"58}
% \DeclareMathSymbol{\varPsi}{\mathalpha}{egreeki}{"09}
% \DeclareMathSymbol{\varOmega}{\mathalpha}{egreeki}{"0A} 
% %
% \DeclareMathSymbol{\Alpha}{\mathalpha}{egreekr}{"41}
% \DeclareMathSymbol{\Beta}{\mathalpha}{egreekr}{"42}
% \DeclareMathSymbol{\Gamma}{\mathalpha}{egreekr}{"00}
% \DeclareMathSymbol{\Delta}{\mathalpha}{egreekr}{"01}
% \DeclareMathSymbol{\Epsilon}{\mathalpha}{egreekr}{"45}
% \DeclareMathSymbol{\Zeta}{\mathalpha}{egreekr}{"5A}
% \DeclareMathSymbol{\Eta}{\mathalpha}{egreekr}{"48}
% \DeclareMathSymbol{\Theta}{\mathalpha}{egreekr}{"02}
% \DeclareMathSymbol{\Iota}{\mathalpha}{egreekr}{"49}
% \DeclareMathSymbol{\Kappa}{\mathalpha}{egreekr}{"4B}
% \DeclareMathSymbol{\Lambda}{\mathalpha}{egreekr}{"03}
% \DeclareMathSymbol{\Mu}{\mathalpha}{egreekr}{"4D}
% \DeclareMathSymbol{\Nu}{\mathalpha}{egreekr}{"4E}
% \DeclareMathSymbol{\Xi}{\mathalpha}{egreekr}{"04}
% \DeclareMathSymbol{\Omicron}{\mathalpha}{egreekr}{"4F}
% \DeclareMathSymbol{\Pi}{\mathalpha}{egreekr}{"05}
% \DeclareMathSymbol{\Rho}{\mathalpha}{egreekr}{"50}
% \DeclareMathSymbol{\Sigma}{\mathalpha}{egreekr}{"06}
% \DeclareMathSymbol{\Tau}{\mathalpha}{egreekr}{"54}
% \DeclareMathSymbol{\Upsilon}{\mathalpha}{egreekr}{"07}
% \DeclareMathSymbol{\Phi}{\mathalpha}{egreekr}{"08}
% \DeclareMathSymbol{\Chi}{\mathalpha}{egreekr}{"58}
% \DeclareMathSymbol{\Psi}{\mathalpha}{egreekr}{"09}
% \DeclareMathSymbol{\Omega}{\mathalpha}{egreekr}{"0A}
% %
% \DeclareMathSymbol{\alphaup}{\mathalpha}{egreekr}{"0B}
% \DeclareMathSymbol{\betaup}{\mathalpha}{egreekr}{"0C}
% \DeclareMathSymbol{\gammaup}{\mathalpha}{egreekr}{"0D}
 \DeclareMathSymbol{\deltaup}{\mathalpha}{egreekr}{"0E}
% \DeclareMathSymbol{\epsilonup}{\mathalpha}{egreekr}{"0F}
% \DeclareMathSymbol{\zetaup}{\mathalpha}{egreekr}{"10}
% \DeclareMathSymbol{\etaup}{\mathalpha}{egreekr}{"11}
% \DeclareMathSymbol{\thetaup}{\mathalpha}{egreekr}{"12}
% \DeclareMathSymbol{\iotaup}{\mathalpha}{egreekr}{"13}
% \DeclareMathSymbol{\kappaup}{\mathalpha}{egreekr}{"14}
% \DeclareMathSymbol{\lambdaup}{\mathalpha}{egreekr}{"15}
% \DeclareMathSymbol{\muup}{\mathalpha}{egreekr}{"16}
% \DeclareMathSymbol{\nuup}{\mathalpha}{egreekr}{"17}
% \DeclareMathSymbol{\xiup}{\mathalpha}{egreekr}{"18}
% \DeclareMathSymbol{\omicronup}{\mathalpha}{egreekr}{"6F}
  \DeclareMathSymbol{\piup}{\mathalpha}{egreekr}{"19}
% \DeclareMathSymbol{\rhoup}{\mathalpha}{egreekr}{"1A}
% \DeclareMathSymbol{\sigmaup}{\mathalpha}{egreekr}{"1B}
% \DeclareMathSymbol{\tauup}{\mathalpha}{egreekr}{"1C}
% \DeclareMathSymbol{\upsilonup}{\mathalpha}{egreekr}{"1D}
% \DeclareMathSymbol{\phiup}{\mathalpha}{egreekr}{"1E}
% \DeclareMathSymbol{\chiup}{\mathalpha}{egreekr}{"1F}
% \DeclareMathSymbol{\psiup}{\mathalpha}{egreekr}{"20}
% \DeclareMathSymbol{\omegaup}{\mathalpha}{egreekr}{"21}
% \DeclareMathSymbol{\varepsilonup}{\mathalpha}{egreekr}{"22}
% \DeclareMathSymbol{\varthetaup}{\mathalpha}{egreekr}{"23}
% \DeclareMathSymbol{\varpiup}{\mathalpha}{egreekr}{"24}
% \let\varrhoup\rhoup 
% \let\varsigmaup\sigmaup
% \let\varkappaup\kappaup
% \DeclareMathSymbol{\varphiup}{\mathalpha}{egreekr}{"27}

% Optima as sans-serif font
%\usepackage%[scaled=0.9]%
%{classico}
\renewcommand\sfdefault{uop}
\DeclareMathAlphabet{\mathsf}  {T1}{\sfdefault}{m}{sl}
\SetMathAlphabet{\mathsf}{bold}{T1}{\sfdefault}{b}{sl}
\newcommand*{\mathte}[1]{\textbf{\textit{\textsf{#1}}}}
% Upright sans-serif math alphabet
% \DeclareMathAlphabet{\mathsu}  {T1}{\sfdefault}{m}{n}
% \SetMathAlphabet{\mathsu}{bold}{T1}{\sfdefault}{b}{n}

% DejaVu Mono as typewriter text
\usepackage[scaled=0.84]{DejaVuSansMono}


\usepackage{mathdots}

\usepackage[usenames]{xcolor}
% Tol (2012) colour-blind-, print-, screen-friendly colours; Munsell terminology
\definecolor{mybluishpurple}{RGB}{51,34,136}
\definecolor{myblue}{RGB}{136,204,238}
\definecolor{mybluishgreen}{RGB}{68,170,153}
\definecolor{mygreen}{RGB}{17,119,51}
\definecolor{mygreenishyellow}{RGB}{153,153,51}
\definecolor{myyellow}{RGB}{221,204,119}
\definecolor{myred}{RGB}{204,102,119}
\definecolor{mypurplishred}{RGB}{136,34,85}
\definecolor{myreddishpurple}{RGB}{170,68,153}
\definecolor{mygrey}{RGB}{221,221,221}
%\newcommand*\mycolourbox[1]{%
%\colorbox{mygrey}{\hspace{1em}#1\hspace{1em}}}
\colorlet{shadecolor}{mygrey}

\usepackage{bm}
\usepackage{microtype}

\usepackage[backend=biber,mcite,%subentry,
citestyle=authoryear-comp,bibstyle=pglpm-authoryear,autopunct=false,sorting=ny,sortcites=false,natbib=false,maxcitenames=1,maxbibnames=8,minbibnames=8,giveninits=true,uniquename=false,uniquelist=false,maxalphanames=1,block=space,hyperref=true,defernumbers=false,useprefix=true,sortupper=false,language=british,parentracker=false]{biblatex}
\DeclareSortingScheme{ny}{\sort{\field{sortname}\field{author}\field{editor}}\sort{\field{year}}}
\iffalse\makeatletter%%% replace parenthesis with brackets
\newrobustcmd*{\parentexttrack}[1]{%
  \begingroup
  \blx@blxinit
  \blx@setsfcodes
  \blx@bibopenparen#1\blx@bibcloseparen
  \endgroup}
\AtEveryCite{%
  \let\parentext=\parentexttrack%
  \let\bibopenparen=\bibopenbracket%
  \let\bibcloseparen=\bibclosebracket}
\makeatother\fi
\DefineBibliographyExtras{british}{\def\finalandcomma{\addcomma}}
\renewcommand*{\finalnamedelim}{\addcomma\space}
\setcounter{biburlnumpenalty}{1}
\setcounter{biburlucpenalty}{0}
\setcounter{biburllcpenalty}{1}
\DeclareDelimFormat{multicitedelim}{\addsemicolon\space}
\DeclareDelimFormat{compcitedelim}{\addsemicolon\space}
\DeclareDelimFormat{postnotedelim}{\space}
\ifarxiv\else\addbibresource{../portamanabib.bib}\fi
\renewcommand{\bibfont}{\footnotesize}
%\appto{\citesetup}{\footnotesize}% smaller font for citations
\defbibheading{bibliography}[\bibname]{\section*{#1}\addcontentsline{toc}{section}{#1}%\markboth{#1}{#1}
}
\newcommand*{\citep}{\parencites}
\newcommand*{\citey}{\parencites*}
%\renewcommand*{\cite}{\parencite}
\renewcommand*{\cites}{\parencites}
\providecommand{\href}[2]{#2}
\providecommand{\eprint}[2]{\texttt{\href{#1}{#2}}}
\newcommand*{\amp}{\&}
% \newcommand*{\citein}[2][]{\textnormal{\textcite[#1]{#2}}%\addtocategory{extras}{#2}
% }
\newcommand*{\citein}[2][]{\textnormal{\textcite[#1]{#2}}%\addtocategory{extras}{#2}
}
\newcommand*{\citebi}[2][]{\textcite[#1]{#2}%\addtocategory{extras}{#2}
}
\newcommand*{\subtitleproc}[1]{}
\newcommand*{\chapb}{ch.}

% \def\arxivp{}
% \def\mparcp{}
% \def\philscip{}
% \def\biorxivp{}
% \newcommand*{\arxivsi}{\texttt{arXiv} eprints available at \url{http://arxiv.org/}.\\}
% \newcommand*{\mparcsi}{\texttt{mp\_arc} eprints available at \url{http://www.ma.utexas.edu/mp_arc/}.\\}
% \newcommand*{\philscisi}{\texttt{philsci} eprints available at \url{http://philsci-archive.pitt.edu/}.\\}
% \newcommand*{\biorxivsi}{\texttt{bioRxiv} eprints available at \url{http://biorxiv.org/}.\\}
\newcommand*{\arxiveprint}[1]{%\global\def\arxivp{\arxivsi}%\citeauthor{0arxivcite}\addtocategory{ifarchcit}{0arxivcite}%eprint
\texttt{\urlalt{https://arxiv.org/abs/#1}{arXiv:\hspace{0pt}#1}}%
%\texttt{\href{http://arxiv.org/abs/#1}{\protect\url{arXiv:#1}}}%
%\renewcommand{\arxivnote}{\texttt{arXiv} eprints available at \url{http://arxiv.org/}.}
}
\newcommand*{\mparceprint}[1]{%\global\def\mparcp{\mparcsi}%\citeauthor{0mparccite}\addtocategory{ifarchcit}{0mparccite}%eprint
\texttt{\urlalt{http://www.ma.utexas.edu/mp_arc-bin/mpa?yn=#1}{mp\_arc:\hspace{0pt}#1}}%
%\texttt{\href{http://www.ma.utexas.edu/mp_arc-bin/mpa?yn=#1}{\protect\url{mp_arc:#1}}}%
%\providecommand{\mparcnote}{\texttt{mp_arc} eprints available at \url{http://www.ma.utexas.edu/mp_arc/}.}
}
\newcommand*{\philscieprint}[1]{%\global\def\philscip{\philscisi}%\citeauthor{0philscicite}\addtocategory{ifarchcit}{0philscicite}%eprint
\texttt{\urlalt{http://philsci-archive.pitt.edu/archive/#1}{PhilSci:\hspace{0pt}#1}}%
%\texttt{\href{http://philsci-archive.pitt.edu/archive/#1}{\protect\url{PhilSci:#1}}}%
%\providecommand{\mparcnote}{\texttt{philsci} eprints available at \url{http://philsci-archive.pitt.edu/}.}
}
\newcommand*{\biorxiveprint}[1]{%\global\def\biorxivp{\biorxivsi}%\citeauthor{0arxivcite}\addtocategory{ifarchcit}{0arxivcite}%eprint
\texttt{\urlalt{http://biorxiv.org/content/early/#1}{bioRxiv:\hspace{0pt}#1}}%
%\texttt{\href{http://arxiv.org/abs/#1}{\protect\url{arXiv:#1}}}%
%\renewcommand{\arxivnote}{\texttt{arXiv} eprints available at \url{http://arxiv.org/}.}
}
\newcommand*{\osfeprint}[1]{%
\texttt{\urlalt{https://doi.org/10.17605/osf.io/#1}{doi:10.17605/osf.io/#1}}%
}

\usepackage{graphicx}
%\usepackage{wrapfig}
%\usepackage{tikz-cd}

\PassOptionsToPackage{hyphens}{url}\usepackage[hypertexnames=false]{hyperref}
\usepackage[depth=4]{bookmark}
\hypersetup{colorlinks=true,bookmarksnumbered,pdfborder={0 0 0.25},citebordercolor={0.2 0.1333 0.5333},%bluish
citecolor=mybluishpurple,linkbordercolor={0.0667 0.4667 0.2},%greenish
linkcolor=mypurplishred,urlbordercolor={0.5333 0.1333 0.3333},%reddish
urlcolor=mygreen,breaklinks=true,pdftitle={\pdftitle},pdfauthor={\pdfauthor}}
% \usepackage[vertfit=local]{breakurl}% only for arXiv
\providecommand*{\urlalt}{\href}

%%% Layout. I do not know on which kind of paper the reader will print the
%%% paper on (A4? letter? one-sided? double-sided?). So I choose A5, which
%%% provides a good layout for reading on screen and save paper if printed
%%% two pages per sheet. Average length line is 66 characters and page
%%% numbers are centred.
\ifafour\setstocksize{297mm}{210mm}%{*}% A4
\else\setstocksize{210mm}{5.5in}%{*}% 210x139.7
\fi
\settrimmedsize{\stockheight}{\stockwidth}{*}
\setlxvchars[\normalfont] %313.3632pt for a 66-characters line
\setxlvchars[\normalfont]
\setlength{\trimtop}{0pt}
\setlength{\trimedge}{\stockwidth}
\addtolength{\trimedge}{-\paperwidth}
% The length of the normalsize alphabet is 133.05988pt - 10 pt = 26.1408pc
% The length of the normalsize alphabet is 159.6719pt - 12pt = 30.3586pc
% Bringhurst gives 32pc as boundary optimal with 69 ch per line
% The length of the normalsize alphabet is 191.60612pt - 14pt = 35.8634pc
\ifafour\settypeblocksize{*}{32pc}{1.618} % A4
%\setulmargins{*}{*}{1.667}%gives 5/3 margins % 2 or 1.667
\else\settypeblocksize{*}{26pc}{1.618}% nearer to a 66-line newpx and preserves GR
\fi
\setulmargins{*}{*}{1}%gives equal margins
\setlrmargins{*}{*}{*}
\setheadfoot{\onelineskip}{2.5\onelineskip}
\setheaderspaces{*}{2\onelineskip}{*}
\setmarginnotes{2ex}{10mm}{0pt}
\checkandfixthelayout[nearest]
\fixpdflayout
%%% End layout
%% this fixes missing white spaces
\pdfmapline{+dummy-space <dummy-space.pfb}\pdfinterwordspaceon%

%%% Sectioning
\newcommand*{\asudedication}[1]{%
{\par\centering\textit{#1}\par}}
\newenvironment{acknowledgements}{\section*{Thanks}\addcontentsline{toc}{section}{Thanks}}{\par}
\makeatletter\renewcommand{\appendix}{\par
  \bigskip{\centering
   \interlinepenalty \@M
   \normalfont
   \printchaptertitle{\sffamily\appendixpagename}\par}
  \setcounter{section}{0}%
  \gdef\@chapapp{\appendixname}%
  \gdef\thesection{\@Alph\c@section}%
  \anappendixtrue}\makeatother
\counterwithout{section}{chapter}
\setsecnumformat{\upshape\csname the#1\endcsname\quad}
\setsecheadstyle{\large\bfseries\sffamily%
\raggedright}
\setsubsecheadstyle{\bfseries\sffamily%
\raggedright}
%\setbeforesecskip{-1.5ex plus 1ex minus .2ex}% plus 1ex minus .2ex}
%\setaftersecskip{1.3ex plus .2ex }% plus 1ex minus .2ex}
%\setsubsubsecheadstyle{\bfseries\sffamily\slshape\raggedright}
%\setbeforesubsecskip{1.25ex plus 1ex minus .2ex }% plus 1ex minus .2ex}
%\setaftersubsecskip{-1em}%{-0.5ex plus .2ex}% plus 1ex minus .2ex}
\setsubsecindent{0pt}%0ex plus 1ex minus .2ex}
\setparaheadstyle{\bfseries\sffamily%
\raggedright}
\setcounter{secnumdepth}{2}
\setlength{\headwidth}{\textwidth}
\newcommand{\addchap}[1]{\chapter*[#1]{#1}\addcontentsline{toc}{chapter}{#1}}
\newcommand{\addsec}[1]{\section*{#1}\addcontentsline{toc}{section}{#1}}
\newcommand{\addsubsec}[1]{\subsection*{#1}\addcontentsline{toc}{subsection}{#1}}
\newcommand{\addpara}[1]{\paragraph*{#1.}\addcontentsline{toc}{subsubsection}{#1}}
\newcommand{\addparap}[1]{\paragraph*{#1}\addcontentsline{toc}{subsubsection}{#1}}

% Headers and footers
\copypagestyle{manaart}{plain}
\makeheadrule{manaart}{\headwidth}{0.5\normalrulethickness}
\makeoddhead{manaart}{%
{\footnotesize%\sffamily%
\scshape\headauthor}}{}{{\footnotesize\sffamily%
\headtitle}}
\makeoddfoot{manaart}{}{\thepage}{}
\newcommand*\autanet{\includegraphics[height=\heightof{M}]{autanet.pdf}}
\definecolor{mygray}{gray}{0.333}
\iftypodisclaim%
\ifafour\newcommand\addprintnote{\begin{picture}(0,0)%
\put(245,149){\makebox(0,0){\rotatebox{90}{\tiny\color{mygray}\textsf{This
            document is designed for screen reading and
            two-up printing on A4 or Letter paper}}}}%
\end{picture}}% A4
\else\newcommand\addprintnote{\begin{picture}(0,0)%
\put(176,112){\makebox(0,0){\rotatebox{90}{\tiny\color{mygray}\textsf{This
            document is designed for screen reading and
            two-up printing on A4 or Letter paper}}}}%
\end{picture}}\fi%afourtrue
\makeoddfoot{plain}{}{\makebox[0pt]{\thepage}\addprintnote}{}
\else
\makeoddfoot{plain}{}{\makebox[0pt]{\thepage}}{}
\fi%typodisclaimtrue
\makeoddhead{plain}{\footnotesize\reporthead}{}{}

% \copypagestyle{manainitial}{plain}
% \makeheadrule{manainitial}{\headwidth}{0.5\normalrulethickness}
% \makeoddhead{manainitial}{%
% \footnotesize\sffamily%
% \scshape\headauthor}{}{\footnotesize\sffamily%
% \headtitle}
% \makeoddfoot{manaart}{}{\thepage}{}

\pagestyle{manaart}

\setlength{\droptitle}{-3.9\onelineskip}
\pretitle{\begin{center}\LARGE\sffamily%
\bfseries}
\posttitle{\bigskip\end{center}}

\makeatletter\newcommand*{\atf}{\includegraphics[%trim=1pt 1pt 0pt 0pt,
totalheight=\heightof{@}]{../atblack.png}}\makeatother
\providecommand{\affiliation}[1]{\textsl{\textsf{\footnotesize #1}}}
\providecommand{\epost}[1]{\texttt{\footnotesize\textless#1\textgreater}}
\providecommand{\email}[2]{\href{mailto:#1ZZ@#2 ((remove ZZ))}{#1\protect\atf#2}}

\preauthor{\vspace{-0.5\baselineskip}\begin{center}
\normalsize\sffamily%
\lineskip  0.5em}
\postauthor{\par\end{center}}
\predate{\DTMsetdatestyle{mydate}\begin{center}\footnotesize}
\postdate{\end{center}\vspace{-\medskipamount}}
\usepackage[british]{datetime2}
\DTMnewdatestyle{mydate}%
{% definitions
\renewcommand*{\DTMdisplaydate}[4]{%
\number##3\ \DTMenglishmonthname{##2} ##1}%
\renewcommand*{\DTMDisplaydate}{\DTMdisplaydate}%
}
\DTMsetdatestyle{mydate}


\setfloatadjustment{figure}{\footnotesize}
\captiondelim{\quad}
\captionnamefont{\footnotesize\sffamily%
}
\captiontitlefont{\footnotesize}
\firmlists*
\midsloppy

% handling orphan/widow lines, memman.pdf
% \clubpenalty=10000
% \widowpenalty=10000
% \raggedbottom
% Downes, memman.pdf
\clubpenalty=9996
\widowpenalty=9999
\brokenpenalty=4991
\predisplaypenalty=10000
\postdisplaypenalty=1549
\displaywidowpenalty=1602

\selectlanguage{british}\frenchspacing
%%%%%%%%%%%%%%%%%%%%%%%%%%%%%%%%%%%%%%%%%%%%%%%%%%%%%%%%%%%%%%%%%%%%%%%%%%%%
%%%%%%%%%%%%%%%%%%%%%%%%%%%%%%%%%%%%%%%%%%%%%%%%%%%%%%%%%%%%%%%%%%%%%%%%%%%%
%%%% Paper's details %%%%
\title{\propertitle%\\
%  {\large A geometric commentary on maximum-entropy proofs}% ***
}
\author{% 
\hspace*{\stretch{1}}%
\parbox{0.5\linewidth}%\makebox[0pt][c]%
{\protect\centering P.G.L. Porta\,Mana\\%
\footnotesize\epost{\email{piero.mana}{ntnu.no}}}%
\hspace*{\stretch{1}}%
%\quad\href{https://orcid.org/0000-0002-6070-0784}{\protect\includegraphics[scale=0.16]{orcid_32x32.png}\textsc{orcid}:0000-0002-6070-0784}%
}

%\date{Draft of \today\ (first drafted \firstdraft)}
%\date{\firstpublished; updated \updated}
\date{\firstdraft; updated \updated}

%@@@@@@@@@@ new macros @@@@@@@@@@
% Common ones - uncomment as needed
%\providecommand{\nequiv}{\not\equiv}
%\providecommand{\coloneqq}{\mathrel{\mathop:}=}
%\providecommand{\eqqcolon}{=\mathrel{\mathop:}}
%\providecommand{\varprod}{\prod}
\newcommand*{\de}{\partialup}%partial diff
\newcommand*{\pu}{\piup}%constant pi
\newcommand*{\delt}{\deltaup}%Kronecker, Dirac
%\newcommand*{\eps}{\varepsilonup}%Levi-Civita, Heaviside
%\newcommand*{\riem}{\zetaup}%Riemann zeta
%\providecommand{\degree}{\textdegree}% degree
%\newcommand*{\celsius}{\textcelsius}% degree Celsius
%\newcommand*{\micro}{\textmu}% degree Celsius
%\newcommand*{\I}{\mathrm{i}}%imaginary unit
%\newcommand*{\e}{\mathrm{e}}%Neper
\newcommand*{\di}{\mathrm{d}}%differential
%\newcommand*{\Di}{\mathrm{D}}%capital differential
%\newcommand*{\planckc}{\hslash}
%\newcommand*{\avogn}{N_{\textrm{A}}}
%\newcommand*{\NN}{\bm{\mathrm{N}}}
%\newcommand*{\ZZ}{\bm{\mathrm{Z}}}
%\newcommand*{\QQ}{\bm{\mathrm{Q}}}
\newcommand*{\RR}{\bm{\mathrm{R}}}
\newcommand*{\CC}{\bm{\mathrm{C}}}
%\newcommand*{\nabl}{\bm{\nabla}}%nabla
%\DeclareMathOperator{\lb}{lb}%base 2 log
%\DeclareMathOperator{\tr}{tr}%trace
%\DeclareMathOperator{\card}{card}%cardinality
%\DeclareMathOperator{\im}{Im}%im part
%\DeclareMathOperator{\re}{Re}%re part
%\DeclareMathOperator{\sgn}{sgn}%signum
%\DeclareMathOperator{\ent}{ent}%integer less or equal to
%\DeclareMathOperator{\Ord}{O}%same order as
%\DeclareMathOperator{\ord}{o}%lower order than
%\newcommand*{\incr}{\triangle}%finite increment
\newcommand*{\defd}{\coloneqq}
\newcommand*{\defs}{\eqqcolon}
%\newcommand*{\Land}{\bigwedge}
%\newcommand*{\Lor}{\bigvee}
%\newcommand*{\lland}{\mathbin{\ \land\ }}
%\newcommand*{\llor}{\mathbin{\ \lor\ }}
%\newcommand*{\lonlyif}{\mathbin{\Rightarrow}}%implies
%\newcommand*{\limplies}{\mathbin{\Rightarrow}}%implies
\newcommand*{\mimplies}{\Rightarrow}%implies
%\newcommand*{\liff}{\mathbin{\Leftrightarrow}}%if and only if
%\newcommand*{\cond}{\mathpunct{|}}%conditional sign (in probabilities)
%\newcommand*{\lcond}{\mathpunct{|\ }}%conditional sign (in probabilities)
%\newcommand*{\bigcond}{\mathpunct{\big|}}%conditional sign (in probabilities)
%\newcommand*{\lbigcond}{\mathpunct{\big|\ }}%conditional sign (in probabilities)
\newcommand*{\suchthat}{\mid}%{\mathpunct{|}}%such that (eg in sets)
%\newcommand*{\bigst}{\mathpunct{\big|}}%such that (eg in sets)
%\newcommand*{\with}{\colon}%with (list of indices)
%\newcommand*{\mul}{\times}%multiplication
%\newcommand*{\inn}{\cdot}%inner product
%\newcommand*{\dotv}{\mathord{\,\cdot\,}}%variable place
%\newcommand*{\comp}{\circ}%composition of functions
%\newcommand*{\con}{\mathbin{:}}%scal prod of tensors
%\newcommand*{\equi}{\sim}%equivalent to 
\renewcommand*{\asymp}{\simeq}%equivalent to 
%\newcommand*{\corr}{\mathrel{\hat{=}}}%corresponds to
%\providecommand{\varparallel}{\ensuremath{\mathbin{/\mkern-7mu/}}}%parallel (tentative symbol)
\renewcommand{\le}{\leqslant}%less or equal
\renewcommand{\ge}{\geqslant}%greater or equal
%\DeclarePairedDelimiter\clcl{[}{]}
\DeclarePairedDelimiter\clop{[}{[}
%\DeclarePairedDelimiter\opcl{]}{]}
%\DeclarePairedDelimiter\opop{]}{[}
%\DeclarePairedDelimiter\abs{\lvert}{\rvert}
%\DeclarePairedDelimiter\norm{\lVert}{\rVert}
\DeclarePairedDelimiter\set{\{}{\}}
%\DeclareMathOperator{\pr}{P}%probability
\newcommand*{\pf}{\mathrm{p}}%probability
\newcommand*{\p}{\mathrm{P}}%probability
%\newcommand*{\tf}{\mathrm{T}}%probability
\renewcommand*{\|}{\mathpunct{|}}
%\newcommand*{\lcond}{\mathpunct{|\ }}%conditional sign (in probabilities)
\newcommand*{\bigcond}{\mathpunct{\big|\ }}%conditional sign (in probabilities)
%\newcommand*{\lbigcond}{\mathpunct{\big|\ }}%conditional sign (in probabilities)
%\newcommand*{\+}{\lor}
%\renewcommand{\*}{\land}
\newcommand*{\sect}{\S}% Sect.~
\newcommand*{\sects}{\S\S}% Sect.~
\newcommand*{\chap}{ch.}%
\newcommand*{\chaps}{chs}%
\newcommand*{\bref}{ref.}%
\newcommand*{\brefs}{refs}%
%\newcommand*{\fn}{fn}%
\newcommand*{\eqn}{eq.}%
\newcommand*{\eqns}{eqs}%
\newcommand*{\fig}{fig.}%
\newcommand*{\figs}{figs}%
\newcommand*{\vs}{{vs}}
%\newcommand*{\etc}{{etc.}}
%\newcommand*{\ie}{{i.e.}}
%\newcommand*{\ca}{{c.}}
\newcommand*{\eg}{{e.g.}}
\newcommand*{\foll}{{ff.}}
%\newcommand*{\viz}{{viz}}
\newcommand*{\cf}{{cf.}}
%\newcommand*{\Cf}{{Cf.}}
%\newcommand*{\vd}{{v.}}
\newcommand*{\etal}{{et al.}}
%\newcommand*{\etsim}{{et sim.}}
%\newcommand*{\ibid}{{ibid.}}
%\newcommand*{\sic}{{sic}}
%\newcommand*{\id}{\mathte{I}}%id matrix
%\newcommand*{\nbd}{\nobreakdash}%
%\newcommand*{\bd}{\hspace{0pt}}%
%\def\hy{-\penalty0\hskip0pt\relax}
\newcommand*{\labelbis}[1]{\tag*{(\ref{#1})$_\text{r}$}}
%\newcommand*{\mathbox}[2][.8]{\parbox[t]{#1\columnwidth}{#2}}
%\newcommand*{\zerob}[1]{\makebox[0pt][l]{#1}}
\newcommand*{\tprod}{\mathop{\textstyle\prod}\nolimits}
\newcommand*{\tsum}{\mathop{\textstyle\sum}\nolimits}
%\newcommand*{\tint}{\begingroup\textstyle\int\endgroup\nolimits}
%\newcommand*{\tland}{\mathop{\textstyle\bigwedge}\nolimits}
%\newcommand*{\tlor}{\mathop{\textstyle\bigvee}\nolimits}
%\newcommand*{\sprod}{\mathop{\textstyle\prod}}
%\newcommand*{\ssum}{\mathop{\textstyle\sum}}
%\newcommand*{\sint}{\begingroup\textstyle\int\endgroup}
%\newcommand*{\sland}{\mathop{\textstyle\bigwedge}}
%\newcommand*{\slor}{\mathop{\textstyle\bigvee}}
\newcommand*{\T}{^\intercal}%transpose
%\newcommand*{\E}{\mathrm{E}}
%\DeclarePairedDelimiter\expp{(}{)}
%\newcommand*{\expe}{\E\expp}%round
%\newcommand*{\expeb}{\E\clcl}%square
%%\newcommand*{\QEM}%{\textnormal{$\Box$}}%{\ding{167}}
%\newcommand*{\qem}{\leavevmode\unskip\penalty9999 \hbox{}\nobreak\hfill
%\quad\hbox{\QEM}}

\definecolor{notecolour}{RGB}{68,170,153}
%\newcommand*{\puzzle}{{\fontencoding{U}\fontfamily{fontawesometwo}\selectfont\symbol{225}}}
\newcommand*{\puzzle}{\maltese}
\newcommand{\mynote}[1]{ {\color{notecolour}\puzzle\ #1}}
%\newcommand*{\widebar}[1]{{\mkern1.5mu\skew{2}\overline{\mkern-1.5mu#1\mkern-1.5mu}\mkern 1.5mu}}

%\DeclareMathOperator*{\argsup}{arg\,sup}
\newcommand*{\yO}[2]{\varOmicron^{(#1)}_{#2}}
\newcommand*{\yI}{\varIota}
\newcommand*{\ys}{\bm{s}}
\newcommand*{\yt}{\bm{t}}
\newcommand*{\yth}{\bm{\theta}}
\newcommand*{\yso}[1]{\ys^{(#1)}}
\newcommand*{\yk}{\bm{k}}
\newcommand*{\yH}{\varEta}
\newcommand*{\yD}{D}
%@@@@@@@@@@ new macros end @@@@@@@@@@

\firmlists
\begin{document}
\captiondelim{\quad}\captionnamefont{\footnotesize}\captiontitlefont{\footnotesize}
\selectlanguage{british}\frenchspacing

%%% Title and abstract %%%
\maketitle
\ifpublic
\abstractrunin
\abslabeldelim{}
\renewcommand*{\abstractname}{}
\setlength{\absleftindent}{0pt}
\setlength{\absrightindent}{0pt}
\setlength{\abstitleskip}{-\absparindent}
\begin{abstract}\labelsep 0pt%
  \noindent Application of the concept of sufficiency and the
  Pitman-Koopman theorem to the definition and measurement of
  \enquote{cooperativity} among neurons
% \par%\\[\jot]
% \noindent
% {\footnotesize PACS: ***}\qquad%
% {\footnotesize MSC: ***}%
%\qquad{\footnotesize Keywords: ***}
\end{abstract}\fi

\selectlanguage{british}\frenchspacing
% \asudedication{\small ***}
% \vspace{\bigskipamount}

\setlength{\epigraphwidth}{.63\columnwidth}
% %\epigraphposition{flushright}
\epigraphtextposition{flushleft}
% %\epigraphsourceposition{flushright}
\epigraphfontsize{\footnotesize}
\setlength{\epigraphrule}{0pt}
% %\setlength{\beforeepigraphskip}{0pt}
\setlength{\afterepigraphskip}{0pt} \epigraph{% Most books on scientific
  % method contain at least a chapter on probability theory.
  Most books on
  modern physics assume a knowledge of probability theory and claim to
  derive probability distributions. A casual observer might be led into
  supposing that the authors would be reasonably up to date in the subject.
  In fact, however, they seldom show any knowledge of it that is less than
  fifty years old%\textelp{}. % It is still widely believed that probability
  % is an easy subject, and this is far from being the case.
}{\textsc{H.
    Jeffreys} \citey[p.~275]{jeffreys1955}}
%\setlength{\epigraphwidth}{.63\columnwidth}
\vspace{-1em}
\epigraph{I believe that the
  greatest unsolved problem in statistics is communicating the subject to
  others.}{\textsc{N. Davies} \citep[in][p.~445]{copasetal1995}}


\iffalse\noindent\emph{\footnotesize Note: Dear Peer, this manuscript is
  peer-reviewed by \emph{you}. I'm grateful if you let me know of any
  faults in its premisses, logic, evidence, and of any other criticisms you
  may have.}
\fi

\section{Introduction}
\label{sec:intro}

\emph{Exchangeable models} are a family of statistical models that appear
over and over in all sciences. Many scientists probably use them routinely
without knowing the technical name \enquote{exchangeable model}, used
mainly in the statistics literature -- and possibly also without knowing of
some important properties of these statistical models. This ignorance is
mainly caused by a language barrier: most literature that discusses
exchangeable models uses a very specialized technical language strongly
connected with topology, measure theory, differential geometry, and other
subjects that are usually only superficially touched in the teaching of
other sciences.

Such subjects are of course necessary for a precise foundation of
exchangeable models and other statistical notions. So I don't think that
scientists are justified in shying away from them. But I don't think that
statisticians are justified in adopting an over-technical jargon either,
wherever a simpler one can suffice. They often seem to willingly insulate
themselves within linguistic walls, instead of building bridges that could
make their beautiful work more accessible to other sciences. Most technical
concepts underlying statistics have, in fact, a very intuitive and even
visual meaning at bottom.
%
\textcolor{white}{If you find this you can claim a postcard from me.}

The present note tries to build a bridge, explaining the essentials of
exchangeable models and some related topics in an intuitive and geometric
way, hoping to make the main concepts more accessible to scientists in
other disciplines. I'm not a statistician or a specialized scientist, so my
attempt will likely leave all parties unsatisfied. I hope it will at least
give a hunch that a bridge is worthwhile constructing.

The other topics discussed in this note include:
\begin{itemize}
\item the choice of so-called \enquote{parameter priors}
\item so-called \enquote{non-parametric models}
\item model comparison
\item sufficiency and its relation to some toy physical models, like Ising
  models
\item partial exchangeability and its relation to machine learning and
  neural networks
\item the phenomenon of \enquote{overtraining}
\end{itemize}


The mathematics I'll use is very simple: functional analysis and
combinatorics. The geometry will rely on convex spaces. These spaces appear
in all of statistics. Their basics are very intuitive, and their
visualization often allows us to guess mathematical properties or solutions
to problems that are harder to guess algebraically. For this reason this
note starts with a summary about them.

\section{Exchangeability}
\label{sec:exchangeability}

There's a deep reason why exchangeable models appear in all sciences: they
are connected with \emph{reproducibility}, and therefore also with
\emph{induction}.

Reproducibility and circularity.

Exchangeability expresses the way we do induction (doesn't justify it).

Example with pattern: we discard exchangeability (dead hypothesis
resurrection) -- just as we do with reproducibility.

\section{Convex spaces}
\label{sec:convex_spaces}

\section{Exchangeable models and their update}
\label{sec:exchangeable_update}

\section{Probability distributions for the limit frequencies}
\label{sec:priors}

\section{Model comparison}
\label{sec:model_comparison}

\section{Models with sufficient statistics}
\label{sec:sufficiency}

\section{Partially exchangeable models}
\label{sec:partially_exch}







\mynote{to be continued}





\iffalse
\section{Sufficient statistics and the Koopman-Pitman theorem}
\label{sec:koopmanpitman}

Suppose we have a sequence, possibly infinite, of observations with
outcomes $\ys$ in a set $S$. The outcomes can be tuples. We represent the
proposition \enquote{The $i$th observation yields outcome $\ys$} by
$\yO{i}{\ys}$, with $\ys\in S$.

A typical inference problem is to forecast the outcomes of the $(m+1)$th up
to the $(m+n)$th observations given that we know the outcomes of the $1$st
up to the $m$th observations and given some additional knowledge or
assumptions denoted by $\yI$. Note that we aren't assuming that these
$\set{1,\dotsc,m+n}$ observations are in temporal order: our problem could be to
infer past, unknown observation from present, known ones. Our forecast is
expressed by the probabilities
\begin{equation}
  \label{eq:general_forecast}
  \p\Bigl(\yO{m+n}{\yso{m+n}}, \dotsc, \yO{m+1}{\yso{m+1}}
  \| \yO{m}{\yso{m}}, \dotsc, \yO{1}{\yso{1}}, \yI\Bigr),
\end{equation}
where the comma represents the conjunction (\enquote{$\land$}) of the
propositions. When unambiguous we use the simplified notation
\begin{equation}\labelbis{eq:general_forecast}
\pf\bigl(\yso{m+n}, \dotsc, \yso{m+1} \| \yso{m}, \dotsc, \yso{1}, \yI\bigr).
\end{equation}

In general, the full set of observed data
$\bigl(\yso{1}, \dotsc, \yso{m}\bigr)$ is important for this inference. But
in particular situations it may happen that only specific properties of the
data are \emph{sufficient} to our inference, the remaining details being
\emph{irrelevant}. By \enquote{irrelevant} we mean that our inferences
aren't changed if we forget about those details or if we don't know them in
the first place. Let's represent the relevant properties by a map
$\yt\bigl(\yso{1},\dotsc,\yso{m}\bigr)$, called a \emph{sufficient
  statistic}, with values in a space $T$. The relevance and irrelevance
are expressed by the equalities
\begin{multline}
  \label{eq:predictive_sufficiency}
  \pf\bigl(\yso{m+n}, \dotsc, \yso{m+1}
  \| \yso{m}, \dotsc, \yso{1}, \yI\bigr)
  ={}\\
  \pf\bigl[\yso{m+n}, \dotsc, \yso{m+1}
  \| \yt\bigl(\yso{1}, \dotsc, \yso{m}\bigr), \yI\bigr]
  \\
  \text{for all $m$, $n$, and $\bigl(\yso{i}\bigr)$.}
\end{multline}

Typical examples of sufficient statistics are the number of known
observations $m$ together with their mean:
$\bigl(\yso{1} + \dotsb + \yso{m}\bigr)/m$, or second moments:
$\bigl(\yso{1}{\yso{1}}\T + \dotsb + \yso{m}{\yso{m}}\T\bigl)/m$, or range:
$\max_i\bigl(\yso{i}\bigr)-\min_i\bigl(\yso{i}\bigr)$, or combination of
these. In this study we will focus on first and second moments:
\begin{equation}
  \label{eq:example_mean_suffstat}
  \yt(\yso{1}, \dotsc, \yso{m}) \defd
  \Bigl( m,\; \tfrac{1}{m}\tsum_i \yso{i}, \;
\tfrac{1}{m}\tsum_i\yso{i}{\yso{i}}\T
  \Bigr).
\end{equation}
The number $m$ is usually implied and left out of the definition.

There may be physical or biological reasons why only some statistics of
past data are sufficient for our inference. This sufficiency may also be
only approximate. We can hypothesize that one or another statistics are
sufficient -- call these hypotheses $\yH_1, \yH_2,\dotsc$ -- and, given
some data $\yD$, the probability calculus tells us which hypotheses among
these are more probable:
\begin{equation}
  \label{eq:example_mean_suffstat}
  \p(\yH_i \| \yD, \yI) =
  \frac{\p(\yD \| \yH_i, \yI)\, \p(\yH_i\|\yI)}{
    \sum_i\p(\yD \| \yH_i, \yI)\, \p(\yH_i\|\yI)
  }.
\end{equation}
To calculate these probabilities we need to assign numerical values to the
probabilities of the data given the hypotheses, $\p(\yD \| \yH_i, \yI)$.

There is a series of mathematical results by Neyman, Koopman, Pitman,
Lauritzen and several others
\citep{neyman1935,koopman1936,pitman1936,darmois1935,lauritzen1982_r1988,kuechleretal1989}[see
also later
analyses:][]{hipp1974,andersen1970,denny1967,fraser1963,barankinetal1963}\mynote{add
  refs to Lauritzen and comments about these references and about the
  discrete-variable case}, which restrict the functional form of the
probability distributions $\p(\yD \| \yH, \yI)$ when some hypothesis $\yH$
about sufficient statistics holds. In the case of statistics constructed as
averages, like~\eqref{eq:example_mean_suffstat}, the distribution must
assume this form, for every $N$:
\begin{multline}
  \label{eq:predictive_KP}
\begin{aligned}
    \pf(\yso{1}, \dotsc, \yso{N} \|\yH_i, \yI )
  &=\int
\Biggl[  \prod_{k=1}^{N}
  g\bigl(\yso{i}\bigr)\,
  \frac{  \exp\bigl[
    \yth\T \yt\bigl( \yso{k} \bigr)
    \bigr] }{Z(\yth)}
  \Biggr]\,
  \pf(\di\yth \| \yH_i, \yI) 
\\
  &=\int
  \bigl[\tprod_{k=1}^{N}g\bigl(\yso{k}\bigr) \bigr]\,
  \frac{  \exp\bigl[
    N \, \yth\T\sum_{k=1}^{N}\yt\bigl( \yso{k} \bigr)
    \bigr]}{Z(\yth)^N}\,
  \pf(\di\yth \| \yI)
\end{aligned}
\\
  \text{with}\quad
  Z(\yth) \defd \sum_{\ys\in S} g(\ys)\, \exp[\yth\T \yt(\ys)].
\end{multline}
The parameters $\yth$ are a tuple with as many elements as the statistics.
For the statistics~\eqref{eq:example_mean_suffstat}, each parameter ranges
in $\RR$. The distribution $g(\ys)$ and the density $\pf(\di\yth \| \yI)$
in the formula above are not determined by the theorem: they need to be
determined by additional assumptions. The distribution $g$ is often
determined by symmetry or combinatorial properties of the problem.
% for example, if each unit actually represents the total activity of a
% population of sub-units.
From now on we assume it to be unity: $g(\ys)=1$. The density
$\pf(\di\yth \|\yI)$ is called \emph{prior parameter density}.

Let's summarize in words the content
of the theorem:
\begin{enumerate}[label=(\textit{\alph*})]
\item our joint probability for the observations $\yso{1}, \dotsc,\yso{N}$
  is given by a convex combination of joint probabilities:
  \begin{equation}
    \label{eq:component_combination}
  \pf(\yso{1}, \dotsc, \yso{N} \|\yth, \yI )
  =
  \prod_{i=1}^{N}
%  g\bigl(\yso{i}\bigr)\,
  \frac{  \exp\bigl(
    \mu_1 \ysso{i}_1 + \mu_2 \ysso{i}_2 + \la \ysso{i}_1 \ysso{i}_2
    \bigr) }{Z(\yth)};
  \end{equation}

\item each joint probability in this convex combination factorizes into $N$
  independent probabilities for the $N$ observations, as is clear from the
  formula above;
\item\label{item:submanifold}each joint probability in the convex
  combination is identified by a triplet of parameters $\yth$; it therefore
  belongs to a three-dimensional submanifold of joint probabilities. Note
  that the full manifold of joint probabilities is $(2^N-1)$-dimensional;
\item\label{item:weights}the weight assigned to the probability labelled by
  $\yth$ is $\pf(\yth \|\yI)\,\di\yth$.
\end{enumerate}

Point~\ref{item:submanifold} shows that the Pitman-Koopman-Lauritzen
theorem greatly reduces our freedom in specifying the joint probability.
This is the effect of assuming that the statistics~\eqref{eq:mean_2ndmom}
are sufficient; it's a very strong assumption.

Points~\ref{item:submanifold} and~\ref{item:weights} show that the theorem
selects a particular three-dimensional submanifold within each of the
$(2^N-1)$-dimensional manifolds of probability distributions for $N$
observations, for all $N$. But the theorem doesn't select any particular
coordinate system within the submanifold: the parameters $\yth$ are just
coordinates, and there is nothing special about them, besides
the fact that they appear as coefficients of the linear combination of
statistics in the exponential~\eqref{eq:predictive_KP}. We could choose
different coordinates $\yt$, with one-one coordinate transformations
$\yt=\yt(\yth)$, $\yth=\yth(\yt)$. In these new coordinates the mixed joint
probabilities are
\begin{equation}
  \label{eq:extreme_probs_newcoords}
  \pf(\yso{1}, \dotsc, \yso{N} \|\yt, \yI ) =
    \prod_{i=1}^{N}
%  g\bigl(\yso{i}\bigr)\,
  \frac{  \exp\bigl[
    \mu_1(\yt)\, \ysso{i}_1 + \mu_2(\yt)\, \ysso{i}_2 +
    \la(\yt)\, \ysso{i}_1 \ysso{i}_2
    \bigr] }{Z[\yth(\yt)]};
  % \frac{\exp\{N \, [
  %   \mu_1(\yt)\, \yavv_1 + \mu_2(\yt)\, \yavv_2 + \la(\yt)\, \ycv]\}
  % }{Z[\yth(\yt)]^N},
\end{equation}
and the weights of the convex combination are given by
$\pf(\yt \|\yI) \, \di\yt$, the densities for $\yth$ and for $\yt$ being
related by a Jacobian determinant:
\begin{equation}
  \label{eq:jacobian}
  \pf(\yth \|\yI) = \pf[\yt(\yth) \| \yI]\,
  \det\biggl( \frac{\de\yt}{\de\yth}\biggr).
\end{equation}
\fi





%\setlength{\intextsep}{0.5ex}% with wrapfigure
%\begin{figure}[p!]%{r}{0.4\linewidth} % with wrapfigure
%  \centering\includegraphics[trim={12ex 0 18ex 0},clip,width=\linewidth]{maxent_saddle.png}\\
%\caption{***}\label{fig:comparison_a5}
%\end{figure}% exp_family_maxent.nb


\iffalse
\begin{acknowledgements}
  PGLPM thanks Mari \amp\ Miri for continuous encouragement and affection;
  Buster Keaton and Saitama for filling life with awe and inspiration; the
  developers and maintainers of \LaTeX, Emacs, AUC\TeX, Open Science
  Framework, Python, Inkscape, Sci-Hub for making a free and unfiltered
  scientific exchange possible.
%\rotatebox{15}{P}\rotatebox{5}{I}\rotatebox{-10}{P}\rotatebox{10}{\reflectbox{P}}\rotatebox{-5}{O}.
%\sourceatright{\autanet}
\end{acknowledgements}
\fi

%\appendixpage
%\appendix

%%%%%%%%%%%%%%% BIB %%%%%%%%%%%%%%%

\defbibnote{prenote}{{\footnotesize (\enquote{de $X$} is listed under D,
    \enquote{van $X$} under V, and so on, regardless of national
    conventions.)\par}}
% \defbibnote{postnote}{\par\medskip\noindent{\footnotesize% Note:
%     \arxivp \mparcp \philscip \biorxivp}}

\printbibliography[prenote=prenote%,postnote=postnote
]


\end{document}
---------- cut text ----------------


%%% Local Variables: 
%%% mode: LaTeX
%%% TeX-PDF-mode: t
%%% TeX-master: t
%%% End: 
