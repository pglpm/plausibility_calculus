\pdfoutput=1
%% Author: PGL  Porta Mana
%% Created: 2020-03-17T00:29:23+0100
%% Last-Updated: 2020-06-03T14:44:39+0200
%%%%%%%%%%%%%%%%%%%%%%%%%%%%%%%%%%%%%%%%%%%%%%%%%%%%%%%%%%%%%%%%%%%%%%%%%%%%
\newif\ifarxiv
\arxivfalse
\ifarxiv\pdfmapfile{+classico.map}\fi
\newif\ifafour
\afourfalse% true = A4, false = A5
\newif\iftypodisclaim % typographical disclaim on the side
\typodisclaimtrue
\newcommand*{\memfontfamily}{zplx}
\newcommand*{\memfontpack}{newpxtext}
\documentclass[\ifafour a4paper,12pt,\else a5paper,10pt,\fi%extrafontsizes,%
onecolumn,oneside,article,%french,italian,german,swedish,latin,
british%
]{memoir}
\newcommand*{\firstdraft}{14 May 2020}
\newcommand*{\firstpublished}{\firstdraft}
\newcommand*{\updated}{\ifarxiv***\else\today\fi}
\newcommand*{\propertitle}{A formula for partial and conditional\\infinite exchangeability%\\{\large ***}%
}% title uses LARGE; set Large for smaller
\newcommand*{\pdftitle}{A formula for partial and conditional infinite exchangeability}
\newcommand*{\headtitle}{Partial and conditional exchangeability}
\newcommand*{\pdfauthor}{P.G.L.  Porta Mana}
\newcommand*{\headauthor}{Porta Mana}
\newcommand*{\reporthead}{\ifarxiv\else Open Science Framework \href{https://doi.org/10.31219/osf.io/***}{\textsc{doi}:10.31219/osf.io/***}\fi}% Report number

%%%%%%%%%%%%%%%%%%%%%%%%%%%%%%%%%%%%%%%%%%%%%%%%%%%%%%%%%%%%%%%%%%%%%%%%%%%%
%%% Calls to packages (uncomment as needed)
%%%%%%%%%%%%%%%%%%%%%%%%%%%%%%%%%%%%%%%%%%%%%%%%%%%%%%%%%%%%%%%%%%%%%%%%%%%%

%\usepackage{pifont}

%\usepackage{fontawesome}

\usepackage[T1]{fontenc} 
\input{glyphtounicode} \pdfgentounicode=1

\usepackage[utf8]{inputenx}

%\usepackage{newunicodechar}
% \newunicodechar{Ĕ}{\u{E}}
% \newunicodechar{ĕ}{\u{e}}
% \newunicodechar{Ĭ}{\u{I}}
% \newunicodechar{ĭ}{\u{\i}}
% \newunicodechar{Ŏ}{\u{O}}
% \newunicodechar{ŏ}{\u{o}}
% \newunicodechar{Ŭ}{\u{U}}
% \newunicodechar{ŭ}{\u{u}}
% \newunicodechar{Ā}{\=A}
% \newunicodechar{ā}{\=a}
% \newunicodechar{Ē}{\=E}
% \newunicodechar{ē}{\=e}
% \newunicodechar{Ī}{\=I}
% \newunicodechar{ī}{\={\i}}
% \newunicodechar{Ō}{\=O}
% \newunicodechar{ō}{\=o}
% \newunicodechar{Ū}{\=U}
% \newunicodechar{ū}{\=u}
% \newunicodechar{Ȳ}{\=Y}
% \newunicodechar{ȳ}{\=y}

\newcommand*{\bmmax}{0} % reduce number of bold fonts, before font packages
\newcommand*{\hmmax}{0} % reduce number of heavy fonts, before font packages

\usepackage{textcomp}

%\usepackage[normalem]{ulem}% package for underlining
% \makeatletter
% \def\ssout{\bgroup \ULdepth=-.35ex%\UL@setULdepth
%  \markoverwith{\lower\ULdepth\hbox
%    {\kern-.03em\vbox{\hrule width.2em\kern1.2\p@\hrule}\kern-.03em}}%
%  \ULon}
% \makeatother

\usepackage{amsmath}

\usepackage[mathic]{mathtools}
%\addtolength{\jot}{\jot} % increase spacing in multiline formulae
\setlength{\multlinegap}{0pt}

\usepackage{empheq}% automatically calls amsmath and mathtools
\newcommand*{\widefbox}[1]{\fbox{\hspace*{1ex}#1\hspace*{1ex}}}
\newcommand*\mycolbox[1]{%
\colorbox{lgrey}{\hspace*{1ex}#1\hspace*{1ex}}}

%%%% empheq above seems more versatile than these:
%\usepackage{fancybox}
%\usepackage{framed}

% \usepackage[misc]{ifsym} % for dice
% \newcommand*{\diceone}{{\scriptsize\Cube{1}}}

\usepackage{amssymb}

\usepackage{amsxtra}

\usepackage[main=british,french,italian,german,swedish,latin,esperanto]{babel}\selectlanguage{british}
\newcommand*{\langfrench}{\foreignlanguage{french}}
\newcommand*{\langgerman}{\foreignlanguage{german}}
\newcommand*{\langitalian}{\foreignlanguage{italian}}
\newcommand*{\langswedish}{\foreignlanguage{swedish}}
\newcommand*{\langlatin}{\foreignlanguage{latin}}
\newcommand*{\langnohyph}{\foreignlanguage{nohyphenation}}

\usepackage[autostyle=false,autopunct=false,english=british]{csquotes}
\setquotestyle{british}

\usepackage{amsthm}
\newcommand*{\QED}{\textsc{q.e.d.}}
\renewcommand*{\qedsymbol}{\QED}
\theoremstyle{remark}
\newtheorem{note}{Note}
\newtheorem*{remark}{Note}
\newtheoremstyle{innote}{\parsep}{\parsep}{\footnotesize}{}{}{}{0pt}{}
\theoremstyle{innote}
\newtheorem*{innote}{}

\usepackage[shortlabels,inline]{enumitem}
\SetEnumitemKey{para}{itemindent=\parindent,leftmargin=0pt,listparindent=\parindent,parsep=0pt,itemsep=\topsep}
% \begin{asparaenum} = \begin{enumerate}[para]
% \begin{inparaenum} = \begin{enumerate*}
\setlist{itemsep=0pt,topsep=\parsep}
\setlist[enumerate,2]{label=\alph*.}
\setlist[enumerate]{label=\arabic*.,leftmargin=1.5\parindent}
\setlist[itemize]{leftmargin=1.5\parindent}
\setlist[description]{leftmargin=1.5\parindent}
% old alternative:
% \setlist[enumerate,2]{label=\alph*.}
% \setlist[enumerate]{leftmargin=\parindent}
% \setlist[itemize]{leftmargin=\parindent}
% \setlist[description]{leftmargin=\parindent}

\usepackage[babel,theoremfont,largesc]{newpxtext}

\usepackage[bigdelims,nosymbolsc%,smallerops % probably arXiv doesn't have it
]{newpxmath}
%\linespread{1.083}%\useosf
\linespread{1.1}%\useosf
%% smaller operators for old version of newpxmath
\makeatletter
\def\re@DeclareMathSymbol#1#2#3#4{%
    \let#1=\undefined
    \DeclareMathSymbol{#1}{#2}{#3}{#4}}
%\re@DeclareMathSymbol{\bigsqcupop}{\mathop}{largesymbols}{"46}
%\re@DeclareMathSymbol{\bigodotop}{\mathop}{largesymbols}{"4A}
\re@DeclareMathSymbol{\bigoplusop}{\mathop}{largesymbols}{"4C}
\re@DeclareMathSymbol{\bigotimesop}{\mathop}{largesymbols}{"4E}
\re@DeclareMathSymbol{\sumop}{\mathop}{largesymbols}{"50}
\re@DeclareMathSymbol{\prodop}{\mathop}{largesymbols}{"51}
\re@DeclareMathSymbol{\bigcupop}{\mathop}{largesymbols}{"53}
\re@DeclareMathSymbol{\bigcapop}{\mathop}{largesymbols}{"54}
%\re@DeclareMathSymbol{\biguplusop}{\mathop}{largesymbols}{"55}
\re@DeclareMathSymbol{\bigwedgeop}{\mathop}{largesymbols}{"56}
\re@DeclareMathSymbol{\bigveeop}{\mathop}{largesymbols}{"57}
%\re@DeclareMathSymbol{\bigcupdotop}{\mathop}{largesymbols}{"DF}
%\re@DeclareMathSymbol{\bigcapplusop}{\mathop}{largesymbolsPXA}{"00}
%\re@DeclareMathSymbol{\bigsqcupplusop}{\mathop}{largesymbolsPXA}{"02}
%\re@DeclareMathSymbol{\bigsqcapplusop}{\mathop}{largesymbolsPXA}{"04}
%\re@DeclareMathSymbol{\bigsqcapop}{\mathop}{largesymbolsPXA}{"06}
\re@DeclareMathSymbol{\bigtimesop}{\mathop}{largesymbolsPXA}{"10}
%\re@DeclareMathSymbol{\coprodop}{\mathop}{largesymbols}{"60}
%\re@DeclareMathSymbol{\varprod}{\mathop}{largesymbolsPXA}{16}
\makeatother
%%
%% With euler font cursive for Greek letters - the [1] means 100% scaling
\DeclareFontFamily{U}{egreek}{\skewchar\font'177}%
\DeclareFontShape{U}{egreek}{m}{n}{<-6>s*[1]eurm5 <6-8>s*[1]eurm7 <8->s*[1]eurm10}{}%
\DeclareFontShape{U}{egreek}{m}{it}{<->s*[1]eurmo10}{}%
\DeclareFontShape{U}{egreek}{b}{n}{<-6>s*[1]eurb5 <6-8>s*[1]eurb7 <8->s*[1]eurb10}{}%
\DeclareFontShape{U}{egreek}{b}{it}{<->s*[1]eurbo10}{}%
\DeclareSymbolFont{egreeki}{U}{egreek}{m}{it}%
\SetSymbolFont{egreeki}{bold}{U}{egreek}{b}{it}% from the amsfonts package
\DeclareSymbolFont{egreekr}{U}{egreek}{m}{n}%
\SetSymbolFont{egreekr}{bold}{U}{egreek}{b}{n}% from the amsfonts package
% Take also \sum, \prod, \coprod symbols from Euler fonts
\DeclareFontFamily{U}{egreekx}{\skewchar\font'177}
\DeclareFontShape{U}{egreekx}{m}{n}{%
       <-7.5>s*[0.9]euex7%
    <7.5-8.5>s*[0.9]euex8%
    <8.5-9.5>s*[0.9]euex9%
    <9.5->s*[0.9]euex10%
}{}
\DeclareSymbolFont{egreekx}{U}{egreekx}{m}{n}
\DeclareMathSymbol{\sumop}{\mathop}{egreekx}{"50}
\DeclareMathSymbol{\prodop}{\mathop}{egreekx}{"51}
\DeclareMathSymbol{\coprodop}{\mathop}{egreekx}{"60}
\makeatletter
\def\sum{\DOTSI\sumop\slimits@}
\def\prod{\DOTSI\prodop\slimits@}
\def\coprod{\DOTSI\coprodop\slimits@}
\makeatother
\input{definegreek.tex}% Greek letters not usually given in LaTeX.

%\usepackage%[scaled=0.9]%
%{classico}%  Optima as sans-serif font
\renewcommand\sfdefault{uop}
\DeclareMathAlphabet{\mathsf}  {T1}{\sfdefault}{m}{sl}
\SetMathAlphabet{\mathsf}{bold}{T1}{\sfdefault}{b}{sl}
%\newcommand*{\mathte}[1]{\textbf{\textit{\textsf{#1}}}}
% Upright sans-serif math alphabet
% \DeclareMathAlphabet{\mathsu}  {T1}{\sfdefault}{m}{n}
% \SetMathAlphabet{\mathsu}{bold}{T1}{\sfdefault}{b}{n}

% DejaVu Mono as typewriter text
\usepackage[scaled=0.84]{DejaVuSansMono}

\usepackage{mathdots}

\usepackage[usenames]{xcolor}
% Tol (2012) colour-blind-, print-, screen-friendly colours, alternative scheme; Munsell terminology
\definecolor{mypurpleblue}{RGB}{68,119,170}
\definecolor{myblue}{RGB}{102,204,238}
\definecolor{mygreen}{RGB}{34,136,51}
\definecolor{myyellow}{RGB}{204,187,68}
\definecolor{myred}{RGB}{238,102,119}
\definecolor{myredpurple}{RGB}{170,51,119}
\definecolor{mygrey}{RGB}{187,187,187}
% Tol (2012) colour-blind-, print-, screen-friendly colours; Munsell terminology
% \definecolor{lbpurple}{RGB}{51,34,136}
% \definecolor{lblue}{RGB}{136,204,238}
% \definecolor{lbgreen}{RGB}{68,170,153}
% \definecolor{lgreen}{RGB}{17,119,51}
% \definecolor{lgyellow}{RGB}{153,153,51}
% \definecolor{lyellow}{RGB}{221,204,119}
% \definecolor{lred}{RGB}{204,102,119}
% \definecolor{lpred}{RGB}{136,34,85}
% \definecolor{lrpurple}{RGB}{170,68,153}
\definecolor{lgrey}{RGB}{221,221,221}
%\newcommand*\mycolourbox[1]{%
%\colorbox{mygrey}{\hspace{1em}#1\hspace{1em}}}
\colorlet{shadecolor}{lgrey}

\usepackage{bm}

\usepackage{microtype}

\usepackage[backend=biber,mcite,%subentry,
citestyle=authoryear-comp,bibstyle=pglpm-authoryear,autopunct=false,sorting=ny,sortcites=false,natbib=false,maxcitenames=2,maxbibnames=8,minbibnames=8,giveninits=true,uniquename=false,uniquelist=false,maxalphanames=1,block=space,hyperref=true,defernumbers=false,useprefix=true,sortupper=false,language=british,parentracker=false]{biblatex}
\DeclareSortingScheme{ny}{\sort{\field{sortname}\field{author}\field{editor}}\sort{\field{year}}}
\iffalse\makeatletter%%% replace parenthesis with brackets
\newrobustcmd*{\parentexttrack}[1]{%
  \begingroup
  \blx@blxinit
  \blx@setsfcodes
  \blx@bibopenparen#1\blx@bibcloseparen
  \endgroup}
\AtEveryCite{%
  \let\parentext=\parentexttrack%
  \let\bibopenparen=\bibopenbracket%
  \let\bibcloseparen=\bibclosebracket}
\makeatother\fi
\DefineBibliographyExtras{british}{\def\finalandcomma{\addcomma}}
\renewcommand*{\finalnamedelim}{\addspace\amp\space}
%\renewcommand*{\finalnamedelim}{\addcomma\space}
\setcounter{biburlnumpenalty}{1}
\setcounter{biburlucpenalty}{0}
\setcounter{biburllcpenalty}{1}
\DeclareDelimFormat{multicitedelim}{\addsemicolon\addspace\space}
\DeclareDelimFormat{compcitedelim}{\addsemicolon\addspace\space}
\DeclareDelimFormat{postnotedelim}{\addspace}
\ifarxiv\else\addbibresource{portamanabib.bib}\fi
\renewcommand{\bibfont}{\footnotesize}
%\appto{\citesetup}{\footnotesize}% smaller font for citations
\defbibheading{bibliography}[\bibname]{\section*{#1}\addcontentsline{toc}{section}{#1}%\markboth{#1}{#1}
}
\newcommand*{\citep}{\footcites}
\newcommand*{\citey}{\footcites}%{\parencites*}
\newcommand*{\ibid}{\unspace\addtocounter{footnote}{-1}\footnotemark{}}
%\renewcommand*{\cite}{\parencite}
%\renewcommand*{\cites}{\parencites}
\providecommand{\href}[2]{#2}
\providecommand{\eprint}[2]{\texttt{\href{#1}{#2}}}
\newcommand*{\amp}{\&}
% \newcommand*{\citein}[2][]{\textnormal{\textcite[#1]{#2}}%\addtocategory{extras}{#2}
% }
\newcommand*{\citein}[2][]{\textnormal{\textcite[#1]{#2}}%\addtocategory{extras}{#2}
}
\newcommand*{\citebi}[2][]{\textcite[#1]{#2}%\addtocategory{extras}{#2}
}
\newcommand*{\subtitleproc}[1]{}
\newcommand*{\chapb}{ch.}
%
% \def\arxivp{}
% \def\mparcp{}
% \def\philscip{}
% \def\biorxivp{}
% \newcommand*{\arxivsi}{\texttt{arXiv} eprints available at \url{http://arxiv.org/}.\\}
% \newcommand*{\mparcsi}{\texttt{mp\_arc} eprints available at \url{http://www.ma.utexas.edu/mp_arc/}.\\}
% \newcommand*{\philscisi}{\texttt{philsci} eprints available at \url{http://philsci-archive.pitt.edu/}.\\}
% \newcommand*{\biorxivsi}{\texttt{bioRxiv} eprints available at \url{http://biorxiv.org/}.\\}
\newcommand*{\arxiveprint}[1]{%\global\def\arxivp{\arxivsi}%\citeauthor{0arxivcite}\addtocategory{ifarchcit}{0arxivcite}%eprint
\texttt{arXiv:\urlalt{https://arxiv.org/abs/#1}{#1}}%
%\texttt{\href{http://arxiv.org/abs/#1}{\protect\url{arXiv:#1}}}%
%\renewcommand{\arxivnote}{\texttt{arXiv} eprints available at \url{http://arxiv.org/}.}
}
\newcommand*{\mparceprint}[1]{%\global\def\mparcp{\mparcsi}%\citeauthor{0mparccite}\addtocategory{ifarchcit}{0mparccite}%eprint
\texttt{mp\_arc:\urlalt{http://www.ma.utexas.edu/mp_arc-bin/mpa?yn=#1}{#1}}%
%\texttt{\href{http://www.ma.utexas.edu/mp_arc-bin/mpa?yn=#1}{\protect\url{mp_arc:#1}}}%
%\providecommand{\mparcnote}{\texttt{mp_arc} eprints available at \url{http://www.ma.utexas.edu/mp_arc/}.}
}
\newcommand*{\haleprint}[1]{%\global\def\arxivp{\arxivsi}%\citeauthor{0arxivcite}\addtocategory{ifarchcit}{0arxivcite}%eprint
\texttt{HAL:\urlalt{https://hal.archives-ouvertes.fr/#1}{#1}}%
%\texttt{\href{http://arxiv.org/abs/#1}{\protect\url{arXiv:#1}}}%
%\renewcommand{\arxivnote}{\texttt{arXiv} eprints available at \url{http://arxiv.org/}.}
}
\newcommand*{\philscieprint}[1]{%\global\def\philscip{\philscisi}%\citeauthor{0philscicite}\addtocategory{ifarchcit}{0philscicite}%eprint
\texttt{PhilSci:\urlalt{http://philsci-archive.pitt.edu/archive/#1}{#1}}%
%\texttt{\href{http://philsci-archive.pitt.edu/archive/#1}{\protect\url{PhilSci:#1}}}%
%\providecommand{\mparcnote}{\texttt{philsci} eprints available at \url{http://philsci-archive.pitt.edu/}.}
}
\newcommand*{\biorxiveprint}[1]{%\global\def\biorxivp{\biorxivsi}%\citeauthor{0arxivcite}\addtocategory{ifarchcit}{0arxivcite}%eprint
bioRxiv \texttt{doi:\urlalt{https://doi.org/10.1101/#1}{10.1101/#1}}%
%\texttt{\href{http://arxiv.org/abs/#1}{\protect\url{arXiv:#1}}}%
%\renewcommand{\arxivnote}{\texttt{arXiv} eprints available at \url{http://arxiv.org/}.}
}
\newcommand*{\osfeprint}[1]{%
Open Science Framework \texttt{doi:\urlalt{https://doi.org/10.17605/osf.io/#1}{10.17605/osf.io/#1}}%
}

\usepackage{graphicx}

%\usepackage{wrapfig}

%\usepackage{tikz-cd}

\PassOptionsToPackage{hyphens}{url}\usepackage[hypertexnames=false]{hyperref}

\usepackage[depth=4]{bookmark}
\hypersetup{colorlinks=true,bookmarksnumbered,pdfborder={0 0 0.25},citebordercolor={0.2667 0.4667 0.6667},citecolor=mypurpleblue,linkbordercolor={0.6667 0.2 0.4667},linkcolor=myredpurple,urlbordercolor={0.1333 0.5333 0.2},urlcolor=mygreen,breaklinks=true,pdftitle={\pdftitle},pdfauthor={\pdfauthor}}
% \usepackage[vertfit=local]{breakurl}% only for arXiv
\providecommand*{\urlalt}{\href}

\usepackage[british]{datetime2}
\DTMnewdatestyle{mydate}%
{% definitions
\renewcommand*{\DTMdisplaydate}[4]{%
\number##3\ \DTMenglishmonthname{##2} ##1}%
\renewcommand*{\DTMDisplaydate}{\DTMdisplaydate}%
}
\DTMsetdatestyle{mydate}

%%%%%%%%%%%%%%%%%%%%%%%%%%%%%%%%%%%%%%%%%%%%%%%%%%%%%%%%%%%%%%%%%%%%%%%%%%%%
%%% Layout. I do not know on which kind of paper the reader will print the
%%% paper on (A4? letter? one-sided? double-sided?). So I choose A5, which
%%% provides a good layout for reading on screen and save paper if printed
%%% two pages per sheet. Average length line is 66 characters and page
%%% numbers are centred.
%%%%%%%%%%%%%%%%%%%%%%%%%%%%%%%%%%%%%%%%%%%%%%%%%%%%%%%%%%%%%%%%%%%%%%%%%%%%
\ifafour\setstocksize{297mm}{210mm}%{*}% A4
\else\setstocksize{210mm}{5.5in}%{*}% 210x139.7
\fi
\settrimmedsize{\stockheight}{\stockwidth}{*}
\setlxvchars[\normalfont] %313.3632pt for a 66-characters line
\setxlvchars[\normalfont]
\setlength{\trimtop}{0pt}
\setlength{\trimedge}{\stockwidth}
\addtolength{\trimedge}{-\paperwidth}
% The length of the normalsize alphabet is 133.05988pt - 10 pt = 26.1408pc
% The length of the normalsize alphabet is 159.6719pt - 12pt = 30.3586pc
% Bringhurst gives 32pc as boundary optimal with 69 ch per line
% The length of the normalsize alphabet is 191.60612pt - 14pt = 35.8634pc
\ifafour\settypeblocksize{*}{32pc}{1.618} % A4
%\setulmargins{*}{*}{1.667}%gives 5/3 margins % 2 or 1.667
\else\settypeblocksize{*}{26pc}{1.618}% nearer to a 66-line newpx and preserves GR
\fi
\setulmargins{*}{*}{1}%gives equal margins
\setlrmargins{*}{*}{*}
\setheadfoot{\onelineskip}{2.5\onelineskip}
\setheaderspaces{*}{2\onelineskip}{*}
\setmarginnotes{2ex}{10mm}{0pt}
\checkandfixthelayout[nearest]
\fixpdflayout
%%% End layout
%% this fixes missing white spaces
\pdfmapline{+dummy-space <dummy-space.pfb}\pdfinterwordspaceon%

%%% Sectioning
\newcommand*{\asudedication}[1]{%
{\par\centering\textit{#1}\par}}
\newenvironment{acknowledgements}{\section*{Thanks}\addcontentsline{toc}{section}{Thanks}}{\par}
% \makeatletter\renewcommand{\appendix}{\par
%   \bigskip{\centering
%    \interlinepenalty \@M
%    \normalfont
%    \printchaptertitle{\sffamily\appendixpagename}\par}
%   \setcounter{section}{0}%
%   \gdef\@chapapp{\appendixname}%
%   \gdef\thesection{\@Alph\c@section}%
%   \anappendixtrue}\makeatother
\renewcommand*\cftappendixname{\appendixname}
\counterwithout{section}{chapter}
\setsecnumformat{\upshape\csname the#1\endcsname\quad}
\setsecheadstyle{\large\bfseries\sffamily%
\centering}
\setsubsecheadstyle{\bfseries\sffamily%
\raggedright}
%\setbeforesecskip{-1.5ex plus 1ex minus .2ex}% plus 1ex minus .2ex}
%\setaftersecskip{1.3ex plus .2ex }% plus 1ex minus .2ex}
%\setsubsubsecheadstyle{\bfseries\sffamily\slshape\raggedright}
%\setbeforesubsecskip{1.25ex plus 1ex minus .2ex }% plus 1ex minus .2ex}
%\setaftersubsecskip{-1em}%{-0.5ex plus .2ex}% plus 1ex minus .2ex}
\setsubsecindent{0pt}%0ex plus 1ex minus .2ex}
\setparaheadstyle{\bfseries\sffamily%
\raggedright}
\setcounter{secnumdepth}{2}
\setlength{\headwidth}{\textwidth}
\newcommand{\addchap}[1]{\chapter*[#1]{#1}\addcontentsline{toc}{chapter}{#1}}
\newcommand{\addsec}[1]{\section*{#1}\addcontentsline{toc}{section}{#1}}
\newcommand{\addsubsec}[1]{\subsection*{#1}\addcontentsline{toc}{subsection}{#1}}
\newcommand{\addpara}[1]{\paragraph*{#1.}\addcontentsline{toc}{subsubsection}{#1}}
\newcommand{\addparap}[1]{\paragraph*{#1}\addcontentsline{toc}{subsubsection}{#1}}

%%% Headers, footers, pagestyle
\copypagestyle{manaart}{plain}
\makeheadrule{manaart}{\headwidth}{0.5\normalrulethickness}
\makeoddhead{manaart}{%
{\footnotesize%\sffamily%
\scshape\headauthor}}{}{{\footnotesize\sffamily%
\headtitle}}
\makeoddfoot{manaart}{}{\thepage}{}
\newcommand*\autanet{\includegraphics[height=\heightof{M}]{autanet.pdf}}
\definecolor{mygray}{gray}{0.333}
\iftypodisclaim%
\ifafour\newcommand\addprintnote{\begin{picture}(0,0)%
\put(245,149){\makebox(0,0){\rotatebox{90}{\tiny\color{mygray}\textsf{This
            document is designed for screen reading and
            two-up printing on A4 or Letter paper}}}}%
\end{picture}}% A4
\else\newcommand\addprintnote{\begin{picture}(0,0)%
\put(176,112){\makebox(0,0){\rotatebox{90}{\tiny\color{mygray}\textsf{This
            document is designed for screen reading and
            two-up printing on A4 or Letter paper}}}}%
\end{picture}}\fi%afourtrue
\makeoddfoot{plain}{}{\makebox[0pt]{\thepage}\addprintnote}{}
\else
\makeoddfoot{plain}{}{\makebox[0pt]{\thepage}}{}
\fi%typodisclaimtrue
\makeoddhead{plain}{\scriptsize\reporthead}{}{}
% \copypagestyle{manainitial}{plain}
% \makeheadrule{manainitial}{\headwidth}{0.5\normalrulethickness}
% \makeoddhead{manainitial}{%
% \footnotesize\sffamily%
% \scshape\headauthor}{}{\footnotesize\sffamily%
% \headtitle}
% \makeoddfoot{manaart}{}{\thepage}{}

\pagestyle{manaart}

\setlength{\droptitle}{-3.9\onelineskip}
\pretitle{\begin{center}\LARGE\sffamily%
\bfseries}
\posttitle{\bigskip\end{center}}

\makeatletter\newcommand*{\atf}{\includegraphics[%trim=1pt 1pt 0pt 0pt,
totalheight=\heightof{@}]{atblack.png}}\makeatother
\providecommand{\affiliation}[1]{\textsl{\textsf{\footnotesize #1}}}
\providecommand{\epost}[1]{\texttt{\footnotesize\textless#1\textgreater}}
\providecommand{\email}[2]{\href{mailto:#1ZZ@#2 ((remove ZZ))}{#1\protect\atf#2}}

\preauthor{\vspace{-0.5\baselineskip}\begin{center}
\normalsize\sffamily%
\lineskip  0.5em}
\postauthor{\par\end{center}}
\predate{\DTMsetdatestyle{mydate}\begin{center}\footnotesize}
\postdate{\end{center}\vspace{-\medskipamount}}

\setfloatadjustment{figure}{\footnotesize}
\captiondelim{\quad}
\captionnamefont{\footnotesize\sffamily%
}
\captiontitlefont{\footnotesize}
%\firmlists*
\midsloppy
% handling orphan/widow lines, memman.pdf
% \clubpenalty=10000
% \widowpenalty=10000
% \raggedbottom
% Downes, memman.pdf
\clubpenalty=9996
\widowpenalty=9999
\brokenpenalty=4991
\predisplaypenalty=10000
\postdisplaypenalty=1549
\displaywidowpenalty=1602
\raggedbottom

\paragraphfootnotes
\setlength{\footmarkwidth}{2ex}
% \threecolumnfootnotes
%\setlength{\footmarksep}{0em}
\footmarkstyle{\textsuperscript{%\color{myred}
\scriptsize\bfseries#1}~}
%\footmarkstyle{\textsuperscript{\color{myred}\scriptsize\bfseries#1}~}
%\footmarkstyle{\textsuperscript{[#1]}~}

\selectlanguage{british}\frenchspacing

%%%%%%%%%%%%%%%%%%%%%%%%%%%%%%%%%%%%%%%%%%%%%%%%%%%%%%%%%%%%%%%%%%%%%%%%%%%%
%%% Paper's details
%%%%%%%%%%%%%%%%%%%%%%%%%%%%%%%%%%%%%%%%%%%%%%%%%%%%%%%%%%%%%%%%%%%%%%%%%%%%
\title{\propertitle}
\author{%
\hspace*{\stretch{1}}%
%% uncomment if additional authors present
% \parbox{0.5\linewidth}%\makebox[0pt][c]%
% {\protect\centering ***\\%
% \footnotesize\epost{\email{***}{***}}}%
% \hspace*{\stretch{1}}%
\parbox{0.75\linewidth}%\makebox[0pt][c]%
{\protect\centering P.G.L.\;  Porta\,Mana  \href{https://orcid.org/0000-0002-6070-0784}{\protect\includegraphics[scale=0.16]{orcid_32x32.png}}\\%
\footnotesize Kavli Institute, Trondheim\quad\epost{\email{pgl}{portamana.org}}}%
%% uncomment if additional authors present
% \hspace*{\stretch{1}}%
% \parbox{0.5\linewidth}%\makebox[0pt][c]%
% {\protect\centering ***\\%
% \footnotesize\epost{\email{***}{***}}}%
\hspace*{\stretch{1}}%
}

%\date{Draft of \today\ (first drafted \firstdraft)}
\date{\firstpublished; updated \updated}

%%%%%%%%%%%%%%%%%%%%%%%%%%%%%%%%%%%%%%%%%%%%%%%%%%%%%%%%%%%%%%%%%%%%%%%%%%%%
%%% Macros @@@
%%%%%%%%%%%%%%%%%%%%%%%%%%%%%%%%%%%%%%%%%%%%%%%%%%%%%%%%%%%%%%%%%%%%%%%%%%%%

% Common ones - uncomment as needed
%\providecommand{\nequiv}{\not\equiv}
%\providecommand{\coloneqq}{\mathrel{\mathop:}=}
%\providecommand{\eqqcolon}{=\mathrel{\mathop:}}
%\providecommand{\varprod}{\prod}
\newcommand*{\de}{\partialup}%partial diff
\newcommand*{\pu}{\piup}%constant pi
\newcommand*{\delt}{\deltaup}%Kronecker, Dirac
%\newcommand*{\eps}{\varepsilonup}%Levi-Civita, Heaviside
%\newcommand*{\riem}{\zetaup}%Riemann zeta
%\providecommand{\degree}{\textdegree}% degree
%\newcommand*{\celsius}{\textcelsius}% degree Celsius
%\newcommand*{\micro}{\textmu}% degree Celsius
\newcommand*{\I}{\mathrm{i}}%imaginary unit
\newcommand*{\e}{\mathrm{e}}%Neper
\newcommand*{\di}{\mathrm{d}}%differential
%\newcommand*{\Di}{\mathrm{D}}%capital differential
%\newcommand*{\planckc}{\hslash}
%\newcommand*{\avogn}{N_{\textrm{A}}}
\newcommand*{\NN}{\bm{\mathrm{N}}}
%\newcommand*{\ZZ}{\bm{\mathrm{Z}}}
%\newcommand*{\QQ}{\bm{\mathrm{Q}}}
\newcommand*{\RR}{\bm{\mathrm{R}}}
%\newcommand*{\CC}{\bm{\mathrm{C}}}
%\newcommand*{\nabl}{\bm{\nabla}}%nabla
%\DeclareMathOperator{\lb}{lb}%base 2 log
%\DeclareMathOperator{\tr}{tr}%trace
%\DeclareMathOperator{\card}{card}%cardinality
%\DeclareMathOperator{\im}{Im}%im part
%\DeclareMathOperator{\re}{Re}%re part
%\DeclareMathOperator{\sgn}{sgn}%signum
%\DeclareMathOperator{\ent}{ent}%integer less or equal to
%\DeclareMathOperator{\Ord}{O}%same order as
%\DeclareMathOperator{\ord}{o}%lower order than
\newcommand*{\incr}{\triangle}%finite increment
\newcommand*{\defd}{\coloneqq}
\newcommand*{\defs}{\eqqcolon}
\newcommand*{\Land}{\mathop{\textstyle\bigwedge}}
%\newcommand*{\Lor}{\bigvee}
%\newcommand*{\lland}{\DOTSB\;\land\;}
%\newcommand*{\llor}{\DOTSB\;\lor\;}
%\newcommand*{\limplies}{\mathbin{\Rightarrow}}%implies
%\newcommand*{\suchthat}{\mid}%{\mathpunct{|}}%such that (eg in sets)
%\newcommand*{\with}{\colon}%with (list of indices)
%\newcommand*{\mul}{\times}%multiplication
%\newcommand*{\inn}{\cdot}%inner product
%\newcommand*{\dotv}{\mathord{\,\cdot\,}}%variable place
%\newcommand*{\comp}{\circ}%composition of functions
%\newcommand*{\con}{\mathbin{:}}%scal prod of tensors
%\newcommand*{\equi}{\sim}%equivalent to 
\renewcommand*{\asymp}{\simeq}%equivalent to 
%\newcommand*{\corr}{\mathrel{\hat{=}}}%corresponds to
%\providecommand{\varparallel}{\ensuremath{\mathbin{/\mkern-7mu/}}}%parallel (tentative symbol)
\renewcommand*{\le}{\leqslant}%less or equal
\renewcommand*{\ge}{\geqslant}%greater or equal
%\DeclarePairedDelimiter\clcl{[}{]}
%\DeclarePairedDelimiter\clop{[}{[}
%\DeclarePairedDelimiter\opcl{]}{]}
%\DeclarePairedDelimiter\opop{]}{[}
\DeclarePairedDelimiter\abs{\lvert}{\rvert}
%\DeclarePairedDelimiter\norm{\lVert}{\rVert}
\DeclarePairedDelimiter\set{\{}{\}}
%\DeclareMathOperator{\pr}{P}%probability
\newcommand*{\pf}{\mathrm{p}}%probability
\newcommand*{\p}{\mathrm{P}}%probability
%\newcommand*{\E}{\mathrm{E}}
%\renewcommand*{\|}{\nonscript\,\vert\nonscript\;\mathopen{}}
\renewcommand*{\|}[1][]{\nonscript\,#1\vert\nonscript\,\mathopen{}}
\DeclarePairedDelimiterX{\cond}[2]{(}{)}{#1\nonscript\,\delimsize\vert\nonscript\;\mathopen{}#2}
\DeclarePairedDelimiterX{\condt}[2]{[}{]}{#1\nonscript\,\delimsize\vert\nonscript\;\mathopen{}#2}
\DeclarePairedDelimiterX{\conds}[2]{\{}{\}}{#1\nonscript\,\delimsize\vert\nonscript\;\mathopen{}#2}
%\newcommand*{\+}{\lor}
%\renewcommand{\*}{\land}
\newcommand*{\sect}{\S}% Sect.~
\newcommand*{\sects}{\S\S}% Sect.~
\newcommand*{\chap}{ch.}%
\newcommand*{\chaps}{chs}%
\newcommand*{\bref}{ref.}%
\newcommand*{\brefs}{refs}%
%\newcommand*{\fn}{fn}%
\newcommand*{\eqn}{eq.}%
\newcommand*{\eqns}{eqs}%
\newcommand*{\fig}{fig.}%
\newcommand*{\figs}{figs}%
\newcommand*{\vs}{{vs}}
\newcommand*{\eg}{{e.g.}}
\newcommand*{\etc}{{etc.}}
\newcommand*{\ie}{{i.e.}}
\newcommand*{\ca}{{c.}}
\newcommand*{\foll}{{ff.}}
%\newcommand*{\viz}{{viz}}
\newcommand*{\cf}{{cf.}}
%\newcommand*{\Cf}{{Cf.}}
%\newcommand*{\vd}{{v.}}
\newcommand*{\etal}{{et al.}}
%\newcommand*{\etsim}{{et sim.}}
%\newcommand*{\ibid}{{ibid.}}
%\newcommand*{\sic}{{sic}}
%\newcommand*{\id}{\mathte{I}}%id matrix
%\newcommand*{\nbd}{\nobreakdash}%
%\newcommand*{\bd}{\hspace{0pt}}%
%\def\hy{-\penalty0\hskip0pt\relax}
%\newcommand*{\labelbis}[1]{\tag*{(\ref{#1})$_\text{r}$}}
\newcommand*{\mathbox}[2][.8]{\parbox[b]{#1\columnwidth}{#2}}
%\newcommand*{\zerob}[1]{\makebox[0pt][l]{#1}}
\newcommand*{\tprod}{\mathop{\textstyle\prod}\nolimits}
\newcommand*{\tsum}{\mathop{\textstyle\sum}\nolimits}
%\newcommand*{\tint}{\begingroup\textstyle\int\endgroup\nolimits}
\newcommand*{\tland}{\mathop{\textstyle\bigwedge}\nolimits}
%\newcommand*{\tlor}{\mathop{\textstyle\bigvee}\nolimits}
%\newcommand*{\sprod}{\mathop{\textstyle\prod}}
%\newcommand*{\ssum}{\mathop{\textstyle\sum}}
%\newcommand*{\sint}{\begingroup\textstyle\int\endgroup}
%\newcommand*{\sland}{\mathop{\textstyle\bigwedge}}
%\newcommand*{\slor}{\mathop{\textstyle\bigvee}}
%\newcommand*{\T}{^\intercal}%transpose
%%\newcommand*{\QEM}%{\textnormal{$\Box$}}%{\ding{167}}
%\newcommand*{\qem}{\leavevmode\unskip\penalty9999 \hbox{}\nobreak\hfill
%\quad\hbox{\QEM}}

%%%%%%%%%%%%%%%%%%%%%%%%%%%%%%%%%%%%%%%%%%%%%%%%%%%%%%%%%%%%%%%%%%%%%%%%%%%%
%%% Custom macros for this file @@@
%%%%%%%%%%%%%%%%%%%%%%%%%%%%%%%%%%%%%%%%%%%%%%%%%%%%%%%%%%%%%%%%%%%%%%%%%%%%
 \definecolor{notecolour}{RGB}{68,170,153}
\newcommand*{\puzzle}{{\fontencoding{U}\fontfamily{fontawesometwo}\selectfont\symbol{225}}}
%\newcommand*{\puzzle}{\maltese}
\newcommand{\mynote}[1]{ {\color{notecolour}\puzzle\ #1}}
\newcommand*{\widebar}[1]{{\mkern1.5mu\skew{2}\overline{\mkern-1.5mu#1\mkern-1.5mu}\mkern 1.5mu}}

% \newcommand{\explanation}[4][t]{%\setlength{\tabcolsep}{-1ex}
% %\smash{
% \begin{tabular}[#1]{c}#2\\[0.5\jot]\rule{1pt}{#3}\\#4\end{tabular}}%}
% \newcommand*{\ptext}[1]{\text{\small #1}}
%\DeclareMathOperator*{\argsup}{arg\,sup}
\newcommand*{\dob}{degree of belief}
\newcommand*{\dobs}{degrees of belief}
% from https://tex.stackexchange.com/a/484142/97039
\newcommand*{\eq}{\mathrel{\!=\!}}
\let\texteq\=
\renewcommand*{\=}{\TextOrMath\texteq\eq}
\newcommand*{\X}[1]{X_{#1}}
\newcommand*{\x}[1]{x_{#1}}
\newcommand*{\Y}[1]{Y_{#1}}
\newcommand*{\y}[1]{y_{#1}}
\newcommand*{\Z}[1]{Z_{#1}}
\newcommand*{\z}[1]{z_{#1}}
\newcommand*{\C}[1]{C_{#1}}
\newcommand*{\cc}[1]{c_{#1}}
\newcommand*{\A}[1]{A_{#1}}
\newcommand*{\va}[1]{a_{#1}}
\newcommand*{\B}[1]{B_{#1}}
\newcommand*{\vb}[1]{b_{#1}}
\newcommand*{\sX}{\mathfrak{X}}
\newcommand*{\sC}{\mathfrak{C}}
\newcommand*{\sA}{\mathfrak{A}}
\newcommand*{\sB}{\mathfrak{B}}
\newcommand*{\vf}{v}
\newcommand*{\ff}[1]{f_{#1}}
\newcommand*{\ffb}[1]{\bm{f_{\!#1}}}
\newcommand*{\FF}[1]{F_{#1}}
\newcommand*{\fx}{\bm{f_{\!x}}}
\newcommand*{\Fx}{\bm{F_{\!x}}}
\newcommand*{\fxc}{\bm{f_{\!x\bcond c}}}
\newcommand*{\fc}{\bm{f_{\!c}}}
\newcommand*{\fj}{\bm{f_{\!xc}}}
\newcommand*{\bg}{\bm{g}}
\newcommand*{\bh}{\bm{h}}
\newcommand*{\xs}{\textsf{S}}
\newcommand*{\xf}{\textsf{F}}
%\newcommand*{\xj}{\textsf{J}}
%\newcommand*{\xa}{\textsf{A}}
\newcommand*{\xY}{\textsf{Y}}
\newcommand*{\xZ}{\textsf{Z}}
\newcommand*{\zI}{J}
\newcommand*{\bcond}[1][]{\nonscript\,#1\bm{\vert}\nonscript\;\mathopen{}}

%%% Custom macros end @@@

%%%%%%%%%%%%%%%%%%%%%%%%%%%%%%%%%%%%%%%%%%%%%%%%%%%%%%%%%%%%%%%%%%%%%%%%%%%%
%%% Beginning of document
%%%%%%%%%%%%%%%%%%%%%%%%%%%%%%%%%%%%%%%%%%%%%%%%%%%%%%%%%%%%%%%%%%%%%%%%%%%%
% \firmlists
\newfixedcaption{\figcaption}{figure}
\setcounter{section}{-1}
\begin{document}
\captiondelim{\quad}\captionnamefont{\footnotesize}\captiontitlefont{\footnotesize}
\selectlanguage{british}\frenchspacing%\raggedyright[1fil]
\maketitle

%%%%%%%%%%%%%%%%%%%%%%%%%%%%%%%%%%%%%%%%%%%%%%%%%%%%%%%%%%%%%%%%%%%%%%%%%%%%
%%% Abstract
%%%%%%%%%%%%%%%%%%%%%%%%%%%%%%%%%%%%%%%%%%%%%%%%%%%%%%%%%%%%%%%%%%%%%%%%%%%%
\abstractrunin
\abslabeldelim{}
\renewcommand*{\abstractname}{}
\setlength{\absleftindent}{0pt}
\setlength{\absrightindent}{0pt}
\setlength{\abstitleskip}{-\absparindent}
\begin{abstract}\labelsep 0pt%
  \noindent [draft]\;
  A formula is given for conditionally, infinitely exchangeable probability
  distributions.
% \\\noindent\emph{\footnotesize Note: Dear Reader
%     \amp\ Peer, this manuscript is being peer-reviewed by you. Thank you.}
% \par%\\[\jot]
% \noindent
% {\footnotesize PACS: ***}\qquad%
% {\footnotesize MSC: ***}%
%\qquad{\footnotesize Keywords: ***}
\end{abstract}
\selectlanguage{british}\frenchspacing

%%%%%%%%%%%%%%%%%%%%%%%%%%%%%%%%%%%%%%%%%%%%%%%%%%%%%%%%%%%%%%%%%%%%%%%%%%%%
%%% Epigraph
%%%%%%%%%%%%%%%%%%%%%%%%%%%%%%%%%%%%%%%%%%%%%%%%%%%%%%%%%%%%%%%%%%%%%%%%%%%%
% \asudedication{\small ***}
% \vspace{\bigskipamount}
% \setlength{\epigraphwidth}{.7\columnwidth}
% %\epigraphposition{flushright}
% \epigraphtextposition{flushright}
% %\epigraphsourceposition{flushright}
% \epigraphfontsize{\footnotesize}
% \setlength{\epigraphrule}{0pt}
% %\setlength{\beforeepigraphskip}{0pt}
% %\setlength{\afterepigraphskip}{0pt}
% \epigraph{\emph{text}}{source}



%%%%%%%%%%%%%%%%%%%%%%%%%%%%%%%%%%%%%%%%%%%%%%%%%%%%%%%%%%%%%%%%%%%%%%%%%%%%
%%% BEGINNING OF MAIN TEXT
%%%%%%%%%%%%%%%%%%%%%%%%%%%%%%%%%%%%%%%%%%%%%%%%%%%%%%%%%%%%%%%%%%%%%%%%%%%%
\textcolor{white}{If you find this you can claim a postcard from me.}

\section{Introduction and notation}
\label{sec:inf_exch}

De Finetti's theorem for the representation of infinitely exchangeable
probability distributions yields some of the formulae, derivable from the
probability calculus, with the richest practical and philosophical
consequences.%  It belongs to family of theorems whose members are still
% under exploration. Its closest relative is the theorem for infinitely,
% \emph{partially} exchangeable probability distributions.

In this work I present three results; each may interest a different
audience.

The first result, derived in \sect~\ref{sec:partial_exch}, is a
reformulation of partial exchangeability and of its representation theorem
in a slightly unfamiliar form. Though not remarkable, this reformulation
gives some insights into the connection between partial and conditional
exchangeability, and their connection to regression.

The second result, derived is \sect~\ref{sec:result}, is an integral
representation for joint (predictive) probability distributions with
particular symmetries: some of the conditional distributions obtained from
them satisfy partial exchangeability. This integral representation has the
usual de~Finetti form and, remarkably, its density must factorize in a
specific way. This factorization is implied by, and implies, the
conditional-exchangeability symmetries.

The third result, \sect~\ref{sec:exch_nets}, brings together
exchangeability and Bayesian networks. A Bayesian network expresses
assumptions of conditional dependence and independence. Exchangeability
expresses assumptions about the mutual informational relevance of a set of
phenomena or experiments, deemed to be \enquote{similar}. When we combine
these two kinds of assumptions we obtain (predictive) probability
distributions with a particular integral representation: a mixture of
copies of the same Bayesian network. This result has some analogies with
de~Finetti's usual representation, which is a mixture of independent
probability distributions (\enquote{i.i.d.}).

% The first is to show an alternative
% representation of the theorem for partial exchangeability. This
% representation emphasizes the role of \emph{conditional} exchangeability
% and of the conditional character of the limit distributions that appear in
% the usual representation.

% The second purpose is to give a representation formula for distributions
% that satisfy a combination of conditional and full exchangeability for the
% marginals. I find the representation interesting because it expresses the
% combination of exchangeability condition as the \emph{factorization} of the
% density that appears in de~Finetti's representation.

Each of these three results builds on the preceding one(s). The final
section discusses them further and hints at possible applications for them.

The remainder of this section introduces some notation and summarizes
de~Finetti's theorem for full exchangeability. It can be skimmed through by
readers familiar with exchangeability theorems, just to grasp the notation
I use.

\subsection{Notation and summary of representation for full exchangeability}
\label{sec:setup}


For the details about exchangeable distributions I refer to Bernardo \amp\
Smith \citep[\sects~4.3, 4.6]{bernardoetal1994_r2000}, Diaconis \amp\
Freedman \citep{diaconisetal1980,diaconisetal1980b}, and Dawid's
\citep{dawid2013} review. 

Our domain of discourse consists of a countably infinite set of atomic
statements (in the logical sense)
 \begin{equation}
  \label{eq:statements}
  \conds[\big]{\X{i}\=\x{i}}{i \in \NN,\;
 \forall i\;\x{i} \in \sX}
\end{equation}
where $\sX$ is a finite set. For each $i$ the statements
$\set{\X{i}\=x \| x \in \sX}$ are assumed mutually exclusive on information
$I$. (The theorem holds for any set of statements with these properties,
even if the statements are not of the form \enquote{$X\=x$}.)

A probability distribution over these atomic statements
%$\set{\X{i}\=\x{i}}$,
is called fully (infinitely) exchangeable if
\begin{multline}
  \label{eq:exchangeable}
  \text{\small for every $N$, every set
    $\set{i_{1},\dotsc,i_{N}}\subset \NN$, and every permutation $\pi$ thereof,}
  \\
  \shoveleft{\p( \X{i_{1}}\=\x{i_{1}}, \X{i_{2}}\=\x{i_{2}},\dotsc,\X{i_{N}}\=\x{i_{N}}
    \| I) ={}}\\
  \p( \X{i_{1}}\=\x{\pi(i_{1})},
  \X{i_{2}}\=\x{\pi(i_{2})},\dotsc,\X{i_{N}}\=\x{\pi(i_{N})}   \| I),
\end{multline}
and if all such probabilities are consistently related by marginalization.
This property is equivalent to declaring the empirical frequencies of the
values $x$ to be sufficient statistics.

In the following, both a comma and \enquote{$\land$} will denote logical
conjunction. The notation $\Land_{i}$ will express conjunction with the
index $i$ running over a subset of $N$ elements from $\NN$.

Denote by $\fx \defd (f_{x})$ a normalized distribution over the values
$x \in \sX$. The set of all such distributions is a simplex of dimension
$\abs{\sX}-1$.

For each $x \in \sX$, denote by $F_{x}$ the empirical relative frequency of
$x$ in the set $\set{x_{1}, \dotsc, x_{N}}$:
\begin{equation}
  \label{eq:rel_frequencies}
  N F_{x} \defd \tsum_{i}\delt(x, \x{i}), \quad x \in \sX \;.
\end{equation}

De~Finetti's theorem states that a fully exchangeable distribution can be
written as follows:
\begin{equation}
  \label{eq:definetti_freq}
  \begin{split}
  \p\bigl( \Land_{i} \X{i}\=\x{i} \|[\big] I\bigr) &=
%  \p( \X{1}\=\x{1},\; \dotsc, \; \X{N}\=\x{N} \| I) ={}\\
  \int \prod_{i} f_{\x{i}}\;  \pf(\fx \| I)\,\di\fx
\\&\equiv  \int\prod_{x} {f_{x}}^{N F_{x}}\;   \pf(\fx \| I)\,\di\fx \;,
\end{split}
\end{equation}
where the integral is over the simplex of distributions $\set{\fx}$.

In the first integral form, the product is over the set of instances
$1,\dotsc,N$. In the second, equivalent integral form, the product is over
the set of values $x$. This form shows that the empirical frequency
distribution $(F_{x})$ is a sufficient statistic. It also hint at the
important role played in the theorem by the relative entropy of $(F_{x})$
with respect to $(f_{x})$.

The theorem establishes a one-one correspondence between the set of
(nested) joint predictive distributions and the set of densities over the
$(\abs{\sX}-1)$-dimensional simplex. Thus any special properties of a
predictive distribution must be reflected in special properties of its
corresponding density, and vice versa.

For enough large $N$, the probability of observing an empirical frequency
distribution $\Fx$ within a small volume $\vf$ centred around the
distribution $\fx$ is approximately given by the density
$\pf(\fx \| I)\,\di\fx$:
\begin{equation}
  \label{eq:longrun_freq}
  \p( \Fx \in v \| \text{$N$ large}, I)
  \approx \pf(\fx \| I)\,v \;.
\end{equation}
For this reason the parameter $\fx$ can be interpreted as a long-run
frequency distribution\footnote{\enquote{But this \emph{long run} is a
    misleading guide to current affairs. \emph{In the long run} we are all
    dead} \parencite[\sect~3.I p.~65]{keynes1923_r2013}.}. I will therefore call it so sometimes, but without
the intention to force such interpretation on you.


\section{Partial exchangeability: alternative form}
\label{sec:partial_exch}

In de~Finetti's theorem for partially exchangeable distributions, the set
$\set{\X{i}\=\x{i}}$ of \sect~\ref{sec:setup} is divided into two or more
categories represented by subsets
$\set{\Y{j}\=\y{j}}, \set{\Z{k}\=\z{k}}, \dotsc$. Partial exchangeability
of the distribution for such statements means that permutations are allowed
within each subset but not necessarily across subsets. The usual
representation in this case % , after a suitable re-indexing
% $\set{1,2,\dotsc} \mapsto \set{1', 2', \dotsc, 1'', 2'', \dotsc}$,
has the form
\begin{equation}
  \label{eq:partialdefinetti}
  \p\Bigl( \Land_{i'}\Y{i'}\=\y{i'},\;
  \Land_{i''}\Z{i''}\=\z{i''} \|[\Big] \zI\Bigr) =
  % \p( \Y{1'}\=\y{1'},\; \Y{2'}\=\y{2'},\; \dotsc,\;
  % \Z{1''}\=\z{1''},\; \Z{2''}\=\z{2''},\; \dotsc \| \zI) ={}\\
  \iint
  \prod_{j} g_{\y{j}}\;
  \prod_{k} h_{\z{k}}\;
  \pf(\bg,\bh \| \zI)\,\di\bg\,\di\bh \;,
\end{equation}
with distinct normalized distributions $\bg$, $\bh$ for each category. If
the density $\pf(\bg,\bh \| \zI)\,\di\bg\,\di\bh$ is diagonal, that is, if
it contains a term $\delt(\bg-\bh)$, the fully exchangeable
form~\eqref{eq:definetti_freq} is recovered.



A little reflection shows that if we know the quantities $\X{i}$ to belong
to category $Y$ in instances $i'$, and to category $Z$ in instances $i''$,
then
\begin{enumerate*}[label=(\alph*)]
\item there are some other quantities $\C{i}$ that allows us to distinguish
  the two categories, and \item the values %$\C{i}\=\cc{i}$
  of these quantities \emph{are known} for all instances.
\end{enumerate*}

Let us say, for example, that the quantities $\X{i}$ are the results of
animal treatments, with values \enquote{$\xs$}uccess and
\enquote{$\xf$}ailure. $Y$ refers to the results for treatments on
$\xY$aks, and $Z$ on $\xZ$ebras. If we write
$$\p(\Y{3}\=\xs,\;\Z{5}\=\xf \| \zI)=0.2\;,$$
then we must already know that animal number~$3$ is a yak: $\C{3}\=\xY$,
and animal number~$5$ is a zebra: $\C{5}\=\xZ$. This is clear from our very
notation, otherwise we would not have known whether to use the symbol $Y$
or $Z$ for those instances. This information is evidently implicit in our
background information $\zI$.

We now make the category information more explicit. A slightly different
definition of partial exchangeability is thus obtained, with a slightly
different form of its representation theorem.



Besides the statements $\set{\X{i}\=\x{i}}$, we introduce an additional set
of atomic statements
\begin{equation}
  \label{eq:category_statements}
  \conds[\big]{\C{i}\=\cc{i}}{i \in \NN,\;
 \forall i\;\cc{i} \in \sC} \;.
\end{equation}
For each $i$ the statements $\set{\C{i}\=c \| c \in \sC}$ are mutually
exclusive on information $I$.

These statements allow us to identify each instance $i$ as belonging to one
or another category out of the finite set $\sC$.

A probability distribution over the $\set{\X{i}\=\x{i}}$ atomic statements
%$\set{\X{i}\=\x{i}}$,
is called partially exchangeable if
\begin{multline}
  \label{eq:part_exchangeable}
  \mathbox[0.99]{\small for every $N$, every set of $N$ indices $\set{i}$, and every permutation $\pi$ thereof such that\quad
    $\pi(j)=k \mathrel{\;\Rightarrow\;} \cc{j}=\cc{k}$,}
  \\
  \p\Bigl( \Land_{i}\X{i}\=\x{i}  \|[\Big]
    \Land_{i}\C{i}\=\cc{i},\;    I\Bigr) =
  \p\Bigl( \Land_{i}\X{i}\=\x{\pi(i)}  \|[\Big]
  \Land_{i}\C{i}\=\cc{i},\;    I\Bigr)
  \;.
\end{multline}
that is, the only allowed permutations are those \emph{which exchange
  indices having the same $c$ value}, that is, belonging to the same
category.

% As remarked in \sect~\ref{sec:inf_exch}, partial exchangeability is usually
% written by using different symbols for indices belonging to different
% categories, and leaving the information implied by such notation implicit
% in the context.

Let us rewrite the representation formula~\eqref{eq:partialdefinetti}
accordingly.


For each category $c \in \sC$, introduce a normalized distribution
$\set{f_{x \bcond c} \| x \in \sX}$ over the values $x$. As the notation
suggests, it can be considered as a \emph{conditional} distribution over
$x$ given $c$. Denote (with some abuse of the symbols) by
$\fxc \defd (f_{x \bcond c})$ the set of all such conditional
distributions. This set is the Cartesian product of $\abs{\sC}$ simplices,
each of dimension $\abs{\sX}-1$.

Denote by $F_{x,c}$ the empirical, joint relative frequency of the pair of
values $(x,c)$ occurring in the set of pairs
$\set{(\x{1}, \cc{1}), \dotsc, (\x{N}, \cc{N})}$:
\begin{equation}
  \label{eq:rel_frequencies}
  N F_{x,c} \defd \tsum_{i}\delt(x, \x{i})\,\delt(c, \cc{i}),
  \quad x \in \sX,\; c \in \sC \;.
\end{equation}
Thus $N F_{x,c}$ is the total number of times value $x$ appears among the
pairs with $\cc{i}=c$.

De Finetti’s theorem states that the partially exchangeable
distribution~\eqref{eq:part_exchangeable} can be written as follows:
\begin{equation}
  \label{eq:partial_definetti_freq}
    \p\Bigl(\Land_{i} \X{i}\=\x{i} \|[\Big]
    \Land_{i}\C{i}\=\cc{i},\;  I\Bigr) =
    % \p( \X{1}\=\x{1}, \dotsc,  \X{N}\=\x{N} \|
    % \C{1}\=\cc{1}, \dotsc,\C{N}\=\cc{N},\;  I) ={}\\
\int\prod_{c,x} {f_{x \bcond c}}^{N F_{x, c}}\;
%\pf\condt[\big]{(f_{x \bcond c})}{I} \,\di(f_{x \bcond c}) \;.
\pf\cond[\big]{\fxc}{I} \,\di\fxc \;.
\end{equation}
Scrutiny of this formula shows that this form is equivalent to the more
familiar representation~\eqref{eq:partialdefinetti}. The integral contains
one product of $f_{\dotso \bcond c}$ terms for every category $c$. In each
such product, $f_{\x{i} \bcond c}$ terms are multiplied together for those
$i$ such that $\cc{i}=c$. There are exactly $N F_{x,c}$ such terms.

This alternative formulation of partial exchangeability shows that this
symmetry could also be called \emph{conditional} exchangeability instead.
The role of conditional distributions is clear in the
representation~\eqref{eq:partial_definetti_freq}. In the following I will
use the term \enquote{conditional} instead of \enquote{partial} to
emphasize this.


\section{Representation for joint distributions with conditional
exchangeability  symmetries}
\label{sec:result}

Suppose that we would assign a conditionally (\ie, \enquote{partially}: see
previous section) exchangeable distribution of probability to the
statements $\set{\X{i}\=\x{i}}$, if we knew the true $\set{\C{i}\=\cc{i}}$.
But we do not know the latter. What kind of properties does the joint
probability distribution of these statements have? And the marginal
distribution for $\set{\X{i}\=\x{i}}$?

The joint probability distribution can be rewritten
\begin{multline}
  \label{eq:joint_p_xc_decomposed}
  \p\Bigl(\Land_{i} (\X{i}\=\x{i}, \C{i}\=\cc{i}) \|[\Big] I\Bigr) ={}\\
  \p\Bigl(\Land_{i} \X{i}\=\x{i} \|[\Big] \Land_{i} \C{i}\=\cc{i},\; I\Bigr) \times
  \p\Bigl(\Land_{i} \C{i}\=\cc{i} \|[\Big] I\Bigr)
%   \p( \X{1}\=\x{1}, \C{1}\=\cc{1},\; \dotsc,\;  \X{N}\=\x{N},\C{N}\=\cc{N}
%   \| I) ={}\\
% \p( \X{1}\=\x{1}, \dotsc, \X{N}\=\x{N}  \|
% \C{1}\=\cc{1}, \dotsc,\C{N}\=\cc{N},\;    I)
% \times{}\\
% \p(\C{1}\=\cc{1}, \dotsc,\C{N}\=\cc{N}  \|    I)
  \;,
\end{multline}
where the first factor, conditionally exchangeable, can be represented by
the integral of \eqn~\eqref{eq:partial_definetti_freq}.

Let us suppose that our uncertainty about the statements
$\set{\C{i}\=\cc{i}}$ is expressed by a fully exchangeable marginal
probability distribution. An integral representation analogous
to~\eqref{eq:definetti_freq} then holds:
\begin{equation}
  \label{eq:definetti_freq_C}
  \p\Bigl( \Land_{i}\C{i}\=\cc{i} \|[\Big] I\Bigr) =
%  \p( \C{1}\=\cc{1},\; \dotsc, \; \C{N}\=\cc{N} \| I) =
\int\prod_{c} {f_{,c}}^{N F_{,c}}\;   \pf(\fc \| I)\,\di\fc \;,
\end{equation}
where $\fc \defd \set{f_{,c}}$ (the reason of the subscript comma will be
soon clear), and $F_{,c} \defd \sum_{x} F_{x,c}$ is the marginal empirical
distribution for the $c$ values.

We can now replace the integral
representations~\eqref{eq:partial_definetti_freq} and
\eqref{eq:definetti_freq_C} into the
product~\eqref{eq:joint_p_xc_decomposed}. The products within their
integrals can be combined considering that
\begin{equation}
  \label{eq:develop_prodfc}
  {f_{,c}}^{N F_{,c}} = {f_{,c}}^{N \sum_{x}F_{x,c}} =
  \prod_{x} {f_{,c}}^{N F_{x,c}} \;.
\end{equation}
We obtain
\begin{subequations}
  \label{eq:result}
  \begin{gather}
    \label{eq:integral_predictive_joint}
     \p\Bigl(\Land_{i} (\X{i}\=\x{i}, \C{i}\=\cc{i}) \|[\Big] I\Bigr) =
% \p( \X{1}\=\x{1}, \C{1}\=\cc{1},\; \dotsc,\;  \X{N}\=\x{N},\C{N}\=\cc{N}
% \| I) ={}\\
\int\prod_{c,x} {f_{x,c}}^{N F_{x, c}}\;  \pf(\fj\|I)\,\di\fj
\\
  \label{eq:factorizable}
\text{with}\qquad \boxed{\;\vphantom{\int}\pf(\fj\|I)\;\di\fj =
\pf(\fxc \| I)  \;
\pf(\fc \| I) \;\di\fxc\,\di\fc }
\end{gather}
\end{subequations}
where we have defined $f_{x,c} = f_{x \bcond c}\, f_{,c}$ and
$\fj \defd (f_{x,c})$. Note that $(f_{x,c})$ does indeed behave like a join
distribution.

The boxed equality above comes from the one-one correspondence between the
variables $(\fxc \,,\fc)$ and $\fj$, so that the product of density functions
for $\fxc$ and $\fc$ is just a specific case of a density function for
$\fj$, apart from a Jacobian factor.



The integral expression~\eqref{eq:result} is the representation of a fully
exchangeable predictive distribution. Thus the joint distribution for the
set of \emph{pairs} of statements $\set{(\X{i}\=\x{i}, \C{i}\=\cc{i})}$ is
fully exchangeable.

The noteworthy feature of the integral expression~\eqref{eq:result} for the
fully exchangeable predictive distribution is that \emph{the density for
  the joint distribution $\fj$ is factorizable into the product of a
  density for the conditional distribution $\fxc$ and a density for the
  marginal distribution $\fc$}.

This factorization is, on the one hand, trivial: it must be so because the
integral itself must factorize, to yield the
factorization~\eqref{eq:joint_p_xc_decomposed} of the predictive
distributions.

But from the point of view of the joint predictive
distribution~\eqref{eq:integral_predictive_joint}, on the other hand, this
factorization is generally not valid and a priori not necessary. The
following general identity holds instead:
\begin{multline}
  \label{eq:identities_factor}
  \pf(\fj\|I)\;\di\fj \equiv
  \pf[(\fxc \,,\fc) \| I] \;\di\fxc\,\di\fc
  \equiv{}\\
  \pf(\fxc \| \fc\,,\, I)  \;
  \pf(\fc \| I) \;\di\fxc\,\di\fc \;.
\end{multline}
Which does not lead to any factorization of the integral.
% The factorization condition is thus equivalent to conditional independence:
% \begin{equation}
%   \label{eq:cond_indep_ff}
%   \pf(\fxc \| \fc,\, I) \;\di\fxc =
% \pf(\fxc \| I) \;\di\fxc \;.
% \end{equation}
The factorization is thus a special property of the density, which reflects
a special property of the joint predictive distribution: namely the
conditional exchangeability for the statements $\set{\X{i}\=\x{i}}$ given
the $\set{\C{i}\=\cc{i}}$.

% It is easy to show that the reverse also holds: if the density of an
% integral representation is factorizable as in~\eqref{eq:factorizable}, then
% the corresponding probability distributions enjoy a symmetry of conditional
% exchangeability.

\section{Exchangeable Bayesian networks}
\label{sec:exch_nets}

\subsection{Graphical representation of conditional exchangeability}
\label{sec:graph_repr}

The conditional (\ie, \enquote{partial})
exchangeability~\eqref{eq:partial_definetti_freq} of the probabilities of
the $X$-statements given the $C$-statements, the full
exchangeability~\eqref{eq:definetti_freq_C} of the probabilities of the
latter, and the final integral representation~\eqref{eq:result} for their
join probability distribution, can be all together expressed in the guise
of a Bayesian network:
\begin{center}%[!h]
\includegraphics[scale=0.5]{bayesnet1.png}
\figcaption{\label{fig:simplenet}}  
\end{center}% bayesnet.svg
The two nodes represent the two sets of statements. The arrow from the
$C$-node to the $X$-node represents the conditional exchangeability of the
probabilities for the latter set of statements conditional on the former.
The absence of incoming arrows to the $C$-node represents the full
exchangeability of its probability distribution. The final integral
representation for the joint probability of the full network has a
factorizable density, with one factor per node.

Using the reasoning and integral representations of
\sects~\ref{sec:partial_exch}--\ref{sec:result} it is possible to
generalize these rules to more complex networks of statements. It can be
calculated that the following network, for example:
\begin{center}%[!h]
\includegraphics[scale=0.5]{bayesnet3.png}
\figcaption{\label{fig:dep3net}}  
\end{center}% bayesnet.svg
has the representation
\begin{multline}
  \label{eq:example1}
  \p( \X{1}\=\x{1}, \A{1}\=\va{1}, \B{1}\=\vb{1},\, \dotsc,\,
   \X{N}\=\x{N}, \A{N}\=\va{N}, \B{N}\=\vb{N} \| I) ={}\\
\int\!\prod_{x,a,b} {\underbracket[1pt][0pt]{\ff{x,a,b}}_{\mathclap{%
\begin{gathered}
\scriptstyle\big\vert\\[-2\jot]     
\scriptstyle{}=\ff{x \bcond a,b}\, \ff{,a \bcond b}\, \ff{,,b}
\end{gathered}%
}}}^{N \FF{x,a,b}}\;
\underbracket[1pt]{\pf(\ffb{x \bcond ab}\|I)\,\pf(\ffb{a \bcond b}\|I)\,\pf(\ffb{b}\|I)
  \;\di\ffb{x \bcond ab} \,\di\ffb{a \bcond b} \,\di\ffb{b}}_{{}=
\pf(\ffb{xab}\|I)\,\di\ffb{xab}} \;
\end{multline}
with $\ff{,,b} \defd \sum_{x,a}\ff{x,a,b}$ and so on.

\subsection{Representing additional independence assumptions}
\label{sec:graph_repr_indep}

In the last example the joint predictive distribution for all the
statements is decomposed in full accord with the product rule:
\begin{multline}
  \label{eq:example_usual_factorization}
  \p\Bigl( \Land_{i}\X{i}\=\x{i} \;  \Land_{i}\A{i}\=\va{i} \;
  \Land_{i}\B{i}\=\vb{i} \|[\Big] I\Bigr)  \equiv{}\\
  \p\Bigl( \Land_{i}\X{i}\=\x{i} \|[\Big]  \Land_{i}\A{i}\=\va{i} \;
  \Land_{i}\B{i}\=\vb{i}\;,\,  I\Bigr) \times{}\\
  \p\Bigl( \Land_{i}\A{i}\=\va{i} \|[\Big]
  \Land_{i}\B{i}\=\vb{i}\;,\, I\Bigr) \times
  \p\Bigl( \Land_{i}\B{i}\=\vb{i} \|[\Big]  I\Bigr)
  \;,
%   \p( \set{\X{i}\=\x{i}}, \set{\A{i}\=\va{i}}, \set{\B{i}\=\vb{i}} \| I)
%   \equiv{}\\
%   \p( \set{\X{i}\=\x{i}} \| \set{\A{i}\=\va{i}}, \set{\B{i}\=\vb{i}},\, I)
%   \times{}\\
%   \p(\set{\A{i}\=\va{i}} \| \set{\B{i}\=\vb{i}},\, I)
% \times \p(  \set{\B{i}\=\vb{i}}\| I) \;,
\end{multline}
which is an identity in the probability-calculus. In other words, no other
special properties hold, besides the conditional exchangeability of these
predictive distributions.

In this case the integral representation~\eqref{eq:example1} involves an
integration over a set of conditional or marginal long-run frequencies
which is equivalent to the set of joint frequencies. That is, the two sets
\begin{equation}
  \label{eq:sets_corr}
  \set{\ff{x \bcond a,b} \,,\, \ff{,a \bcond b} \,,\, \ff{,,b} \|
 x \in \sX, a \in \sA, b \in \sB}
\leftrightarrow
\set{\ff{x,a,b} \| x \in \sX, a \in \sA, b \in \sB}
\end{equation}
are in one-one correspondence.

The factorizability of the density $\pf(\ffb{xab}\|I)\,\di\ffb{xab}$
therefore implies, and is implied by, the exchangeability symmetries of
conditional probability distributions. But it does not imply additional
independences.

Additional independence properties, such as
\begin{multline}
  \label{eq:example_independence}
  \p\Bigl( \Land_{i}\X{i}\=\x{i} \;  \Land_{i}\A{i}\=\va{i} \;
  \Land_{i}\B{i}\=\vb{i} \|[\Big] I\Bigr)  \equiv{}\\
  \p\Bigl( \Land_{i}\X{i}\=\x{i} \|[\Big]  \Land_{i}\A{i}\=\va{i} \;
  \Land_{i}\B{i}\=\vb{i}\;,\,  I\Bigr) \times{}\\
  \p\Bigl( \Land_{i}\A{i}\=\va{i} \|[\Big] I\Bigr) \times
  \p\Bigl( \Land_{i}\B{i}\=\vb{i} \|[\Big]  I\Bigr)
  \;,
\end{multline}
which is not an identity of the probability-calculus, are those typically
expressed by Bayesian networks.

In the integral representation under discussion, these conditional
independences are expressed by \emph{a reduction in the number of
  conditional or marginal long-run frequencies that are integrated over}
(similarly to what happens when partial exchangeability reduces to full
exchangeability; see \sect~\ref{sec:partial_exch}).

For example, if the independence~\eqref{eq:example_independence} hold, a
little calculation shows that the representation~\eqref{eq:example1} becomes
\begin{multline}
  \label{eq:example1sep}
\p\Bigl( \Land_{i}\X{i}\=\x{i} \;  \Land_{i}\A{i}\=\va{i} \;
\Land_{i}\B{i}\=\vb{i} \|[\Big] I\Bigr)
% \p( \X{1}\=\x{1}, \A{1}\=\va{1}, \B{1}\=\vb{1},\, \dotsc,\,
%    \X{N}\=\x{N}, \A{N}\=\va{N}, \B{N}\=\vb{N} \| I) 
   ={}\\
   %\shoveleft{
     \int\!\prod_{x,a,b}
     (\ff{x \bcond a,b}\, \ff{,a}\, \ff{,,b})^{N \FF{x,a,b}}\;%\times{}%}\\[-\jot]
\pf(\ffb{x \bcond ab}\|I)\,\pf(\ffb{a}\|I)\,\pf(\ffb{b}\|I)
\;\di\ffb{x \bcond ab} \,\di\ffb{a} \,\di\ffb{b} \;.
\end{multline}
We see that the integration is now over the \emph{reduced} set of
distributions
\begin{equation}
  \label{eq:subset_freqs}
  \set{\ff{x \bcond a,b} \,,\, \ff{,a} \,,\, \ff{,,b} \|
    x \in \sX, a \in \sA, b \in \sB} \;.
\end{equation}
This is a reduced set in the sense that $\ff{,a \bcond b} = \ff{,a}$ for
all $b$, implying the presence of a delta term in the corresponding
density.

The independence condition~\eqref{eq:example_independence}, the various
conditional-exchangeability conditions, and the integral
representation~\eqref{eq:example1sep} can be expressed by the network\\
\begin{center}%[h!]
\includegraphics[scale=0.5]{bayesnet3s.png}
\figcaption{\label{fig:indep_net}}
\end{center}% bayesnet.svg

\subsection{Generalization and connection with Bayesian networks}
\label{sec:graph_repr_gen}

Let us summarize what we have found thus far. We have
\begin{enumerate}[label=\roman*.]
\item[0.] several sets of statements, $\set{\X{i}\=\x{i}}$,
  $\set{\A{i}\=\va{i}},\dotsc$. Each set is countably infinite. For each
  set we also have an associated set of values $x\in \sX,$
  $a \in \sA, \dotsc$;
\item\label{item:assu_bayesnet} several assumptions of conditional independence representable in the
  guise of a Bayesian network, such as in
  \eqn~\eqref{eq:example_independence} and \fig~\ref{fig:indep_net};
\item\label{item:assu_condexch} several assumptions of conditional exchangeability for the predictive
  probabilities of some groups of such statements conditional on some other
  groups, such as in \eqn~\eqref{eq:partial_definetti_freq}.
\end{enumerate}
These assumptions, taken together, must be sufficient to obtain the
predictive, joint probability distribution for all statements through the
product rule of the probability-calculus.

Then the predictive, joint probability distribution for $N$ statements from each
group has an integral representation of de~Finetti form:
\begin{framed}
  \begin{multline}
    \label{eq:general_form}
    \p( \X{1}\eq \x{1}, \A{1}\eq \va{1}, \dotsc,\,
    \X{N}\eq \x{N}, \A{N}\eq \va{N}, \dotsc \| I) ={}\\
    % \shoveleft{
    \int\!\prod_{x,a,\dotsc}
    (\xi_{x,a,\dotsc})^{N \FF{x,a,\dotsc}}\;%\times{}%}\\[-\jot]
    \pf(\bm{\xi}\|I)\,\di\bm{\xi} 
    % \pf(\ffb{x \bcond ab}\|I)\,\pf(\ffb{a}\|I)\,\pf(\ffb{b}\|I)
    % \;\di\ffb{x \bcond ab} \,\di\ffb{a} \,\di\ffb{b} \;.
  \end{multline}
%\end{framed}
where $N\,\FF{x,a,\dotsc}$ are the joint empirical frequencies. This
representation has these properties:
%\begin{framed}
  \begin{enumerate}[label=\Roman*.]
  \item the set of integration variables $\set{\xi_{x,a,\dotsc}}$ consists
    of subsets of marginal and conditional distributions: the same that
    would formally be associated with the Bayesian network of
    point~\ref{item:assu_bayesnet} For example
    \[\set{\xi_{x,a,b}} = \set[\big]{\set{\ff{x \bcond a,b}},\,
        \set{\ff{,a}},\, \set{\ff{,,b}}}\;.\]
  \item the density $\pf(\bm{\xi}\|I)\,\di\bm{\xi}$ over the integration
    variables factorizes into a product of independent densities, one for
    each subset of the point above, for example
    \[\pf(\bm{\xi}\|I)\,\di\bm{\xi} =
      \pf(\ffb{x \bcond ab}\|I)\,\di\ffb{x \bcond ab}\;\;
      \pf(\ffb{a}\|I)\,\di\ffb{a}\;\; \pf(\ffb{b}\|I)\,\di\ffb{b} \;.
    \]
  \end{enumerate}
\end{framed}

\subsection{Discussion}
\label{sec:disc_exchbayesnet}

The representation just derived combines exchangeability and Bayesian
networks. I believe such a combination to be much needed to make inferences
in concrete problems, especially when data are few with respect to the
number of factors involved.

A Bayesian network is a graphical representation of judgements (based for
example on physical theories) about the informational relevance or
irrelevance of some statements to other statements. Such relevance or
irrelevance is expressed by equalities or inequalities between conditional
probabilities involving those statements.

Unfortunately the use of Bayesian networks is a little ambiguous in the
literature. Sometimes their application seem to refer to \emph{individual}
instances, and therefore to degrees of belief; for example, our degree of
belief about the effect of a treatment upon a specific person, given the
gender and health condition of that person. Sometimes their application
seem to refer to whole \emph{populations} or even superpopulations, and
therefore to \enquote{long-run}
frequencies.\citep[\eg][]{pearl2000_r2009,wiegerincketal2013}[the
discussion by][is more precise in this respect: they assume, for
simplicity, a limit in which the two problems become numerically
similar]{lindleyetal1981}

In either case some questions arise. In the first case, the question of how
our (conditional or unconditional) degrees of belief about the specific
instance relate to, or are updated by, knowledge about similar instances.
In the second case, the question of our uncertainties about the usually
unknowable long-run frequencies, and about the empirical frequencies of
future observations of small groups of individuals. These questions are
really two sides of the same question.

And these are precisely the questions addressed by exchangeability and its
theorems. Exchangeability expresses judgements about \enquote{similarity};
more precisely, about the relevance or irrelevance of statements concerning
individual events (observations, experiments, and so on) to sets of
statements concerning other individual events. Such relevance or
irrelevance is expressed by symmetries of conditional and unconditional
probabilities involving those statements. The result is a mathematical
formula that quantitatively relates empirical frequencies from known
observations (\cf\ $F_{x,a}$), long-run frequencies of superpopulations
(\cf\ $f_{x,a}$) and degrees of beliefs about individual instances (\cf\
$\p(\X{1}\=\x{1} \| \dotso)$).

This same mathematical relation is provided by the integral representation
derived in the previous section, in the case of knowledge or assumptions
expressed by a Bayesian network.

Its most interesting feature is that it \emph{mixes} terms corresponding to
the product of conditional probabilities characteristics of the represented
Bayesian network. This is analogous to the mixture of \enquote{i.i.d.}
distributions in de~Finetti's theorem.

% can then be represented by an acyclic directed network with
% these properties:
% \begin{enumerate}
% \item nodes represent sets of statements;
% \item there is a set of integration variables for each node. These
%   variables are the long-run frequencies for the quantity of the node
%   conditional on the quantities of the parent nodes. Nodes without parents
%   have marginal, unconditional long-run frequencies;
% \item each set of integration variables has its own density. The densities
%   are multiplied in the integral representation.
% \end{enumerate}

% Moreover, s explained in \sect~\ref{sec:graph_repr_indep}, independence
% assumptions are expressed by the set of integration variables, whereas
% conditional-exchangeability assumption are expressed by the factorization
% of the density over the integration variables.


% indicates the conditional exchangeability of the latter set conditional on
% the former set. Nodes with no incoming arrows are fully exchangeable. The
% integral representation for the full network, that is, for the all the
% statements, has a factorizable density, one factor per node. The arguments
% of these factors are conditional or marginal long-run frequencies
% accordingly to the arrows incoming to their nodes.


\section{Discussion}
\label{sec:discuss}

*** first result can be useful for infinite limits, leading to regression


The result of the previous section is thus summarized: Given infinitely
countable sets of statements $\set{\X{i}\=\x{i}}$ and
$\set{\C{i}\=\cc{i}}$, and assuming that
\begin{enumerate}[wide]
\item the marginal probability distribution for the $C$ statements is fully
  exchangeable,
\item the probability distribution for the $X$ statements is partially (or
  conditionally) exchangeable given the $C$,
\end{enumerate}
Then the joint distribution for both sets is fully exchangeable, and the
density within its integral representation \emph{factorizes} into a density
for a conditional long-run frequency distribution, and a density for a
marginal long-run frequency distribution, \eqn~\eqref{eq:factorizable}.






%%%% examples use empheq
%   \begin{empheq}[left={\mathllap{\begin{aligned}    \de\yF_{\yc}/\de\yp&=0\text{:} \\
%         \de\yF_{\yc}/\de\ym&=0\text{:}\\ \de\yF_{\yc}/\de\yl&=0\text{:}\end{aligned}}\qquad}\empheqlbrace]{align}
%     \label{eq:con_p}
% %    \de\yF_{\yc}/\de\yp &\equiv
%     -\ln\yp + \ln\yq + \yl\yM + \ym\yu &=0,\\
%     \label{eq:con_u}
% %    \de\yF_{\yc}/\de\ym &\equiv
%     \yu\yp-1 &=0,\\
%     \label{eq:con_l}
%     %\de\yF_{\yc}/\de\yl &\equiv
%     \yM\yp-\yc &=0.
%   \end{empheq}
%%%%
% \begin{empheq}[box=\widefbox]{equation}
%   \label{eq:maxent_question}
%   \p\bigl[\yE{N+1}{k} \bigcond \tsum\yo\yf{N}\in\yA, \yM\bigr] = \mathord{?}
% \end{empheq}



% \[
%   \begin{tikzcd}
%       M_{n,n}(\CC) \arrow{r}{R'_{a}(\Hat{U})} & M_{n,n}(\CC)
%     \\
%     L(\mathcal{H}) \arrow{r}{\Hat{U}} \arrow[swap]{d}{R_*}\arrow[swap]{u}{R'_*} & L(\mathcal{H}) \arrow{d}{R_*}\arrow{u}{R'_*} \\
%       M_{n,n}(\CC) \arrow{r}{R_{a}(\Hat{U})} & M_{n,n}(\CC)
%   \end{tikzcd}
% \]

% \[
%   \begin{tikzcd}
%       \CC^n \arrow{r}{R'_*(A)} & \CC^n
%     \\
%     \mathcal{H} \arrow{r}{A} \arrow[swap]{d}{R}\arrow[swap]{u}{R'} & \mathcal{H} \arrow{d}{R}\arrow{u}{R'} \\
%       \CC^n \arrow{r}{R_*(A)} & \CC^n
%   \end{tikzcd}
% \]


% \[
%   \begin{tikzcd}
%     \mathcal{H} \arrow{r}{A} \arrow[swap]{d}{R} & \mathcal{H} \arrow{d}{R} \\
%       \CC^n \arrow{r}{R_*(A)} & \CC^n
%   \end{tikzcd}
% \]

%%\setlength{\intextsep}{0ex}% with wrapfigure
%%\setlength{\columnsep}{0ex}% with wrapfigure
%\begin{figure}[p!]% with figure
%\begin{wrapfigure}{r}{0.4\linewidth} % with wrapfigure
%  \centering\includegraphics[trim={12ex 0 18ex 0},clip,width=\linewidth]{maxent_saddle.png}\\
%\caption{caption}\label{fig:comparison_a5}
%\end{figure}% exp_family_maxent.nb


%%%%%%%%%%%%%%%%%%%%%%%%%%%%%%%%%%%%%%%%%%%%%%%%%%%%%%%%%%%%%%%%%%%%%%%%%%%%
%%% Acknowledgements
%%%%%%%%%%%%%%%%%%%%%%%%%%%%%%%%%%%%%%%%%%%%%%%%%%%%%%%%%%%%%%%%%%%%%%%%%%%% 
\iffalse
\begin{acknowledgements}
  \ldots to Mari \amp\ Miri for continuous encouragement and affection, and
  to Buster Keaton and Saitama for filling life with awe and inspiration.
  To the developers and maintainers of \LaTeX, Emacs, AUC\TeX, Open Science
  Framework, R, Python, Inkscape, Sci-Hub for making a free and impartial
  scientific exchange possible.
%\rotatebox{15}{P}\rotatebox{5}{I}\rotatebox{-10}{P}\rotatebox{10}{\reflectbox{P}}\rotatebox{-5}{O}.
%\sourceatright{\autanet}
\mbox{}\hfill\autanet
\end{acknowledgements}
\fi

%%%%%%%%%%%%%%%%%%%%%%%%%%%%%%%%%%%%%%%%%%%%%%%%%%%%%%%%%%%%%%%%%%%%%%%%%%%%
%%% Appendices
%%%%%%%%%%%%%%%%%%%%%%%%%%%%%%%%%%%%%%%%%%%%%%%%%%%%%%%%%%%%%%%%%%%%%%%%%%%% 
%\clearpage
% %\renewcommand*{\appendixpagename}{Appendix}
% %\renewcommand*{\appendixname}{Appendix}
% %\appendixpage
%\addtocontents{toc}{\setlength\cftsectionnumwidth{5em}}
\renewcommand\thesection{Appendix:}
\section{On the ambiguous meaning of \enquote{$X\eq x$}}
\label{sec:append}




 
%%%%%%%%%%%%%%%%%%%%%%%%%%%%%%%%%%%%%%%%%%%%%%%%%%%%%%%%%%%%%%%%%%%%%%%%%%%%
%%% Bibliography
%%%%%%%%%%%%%%%%%%%%%%%%%%%%%%%%%%%%%%%%%%%%%%%%%%%%%%%%%%%%%%%%%%%%%%%%%%%% 
\renewcommand*{\finalnamedelim}{\addcomma\space}
\defbibnote{prenote}{{\footnotesize (\enquote{de $X$} is listed under D,
    \enquote{van $X$} under V, and so on, regardless of national
    conventions.)\par}}
% \defbibnote{postnote}{\par\medskip\noindent{\footnotesize% Note:
%     \arxivp \mparcp \philscip \biorxivp}}

\printbibliography[prenote=prenote%,postnote=postnote
]

\end{document}

%%%%%%%%%%%%%%%%%%%%%%%%%%%%%%%%%%%%%%%%%%%%%%%%%%%%%%%%%%%%%%%%%%%%%%%%%%%%
%%% Cut text (won't be compiled)
%%%%%%%%%%%%%%%%%%%%%%%%%%%%%%%%%%%%%%%%%%%%%%%%%%%%%%%%%%%%%%%%%%%%%%%%%%%% 




Leaving for the moment
the definition of symbols to intuition, the theorem rewrites a joint
probability distribution as a law of total probability:
\begin{equation}
  \label{eq:definetti}
  \p( \X{1}\=\x{1},\; \X{2}\=\x{2},\; \dotsc \| \zI) =
  \int
  \prod_{i} f_{\x{i}}\;
  \pf(\fx \| \zI)\,\di\fx \;,
\end{equation}
where $\fx \defd (f_{x})$ is a distribution over the values that can be
assumed by each quantity $\X{i}$, and the integral is over the simplex of
such distributions.

The condition for the theorem to hold is that the joint distribution be
infinitely \emph{fully exchangeable}, that is, symmetric with respect to
permutations of the $\x{i}$, for any set of indices $\set{i}$. This formula
is the infinite limit of the sampling formula from an urn of unknown
content. From now on \enquote{exchangeable} shall be understood as
\enquote{infinitely exchangeable}.

A more general version of the theorem holds if the joint distribution is
\emph{partially exchangeable}, that is, if the set $\set{\X{i}}$ is divided
into two or more categories represented by subsets
$\set{\Y{j}}, \set{\Z{k}}, \dotsc$, and permutations are allowed within
each subset but not necessarily across subsets. The formula then becomes,
with a suitable re-indexing
$\set{1,2,\dotsc} \mapsto \set{1', 2', \dotsc, 1'', 2'', \dotsc}$,
\begin{multline}
  \label{eq:partialdefinetti}
  \p( \Y{1'}\=\y{1'},\; \Y{2'}\=\y{2'},\; \dotsc,\;
  \Z{1''}\=\z{1''},\; \Z{2''}\=\z{2''},\; \dotsc \| \zI) ={}\\
  \iint
  \prod_{j} g_{\y{j}}\;
  \prod_{k} h_{\z{k}}\;
  \pf(\bg,\bh \| \zI)\,\di\bg\,\di\bh \;,
\end{multline}
with distinct distributions $\bg$, $\bh$ for each category. If the density
$\pf(\bg,\bh \| \zI)\,\di\bg\,\di\bh$ is diagonal, that is, if it contains
a term $\delt(\bg-\bh)$, the fully exchangeable form~\eqref{eq:definetti}
is recovered.



With a little reflection we see that if we know that the quantities $X$
belong to category $Y$ in instances $1',2',\dotsc$, and to category $Z$ in
instances $1'',2'',\dotsc$, then
\begin{enumerate*}[label=(\alph*)]
\item there is some other quantity $C$ that allows us to distinguish the
  two categories, and \item the values $\C{i}\=\cc{i}$ of this quantity
  \emph{are known} for all instances.
\end{enumerate*}

Let us say, for example, that the quantities $\X{i}$ are the results of
animal treatments, with values \enquote{$\xs$}uccess and
\enquote{$\xf$}ailure. $Y$ refers to the results for treatments on
$\xY$aks, and $Z$ on $\xZ$ebras. If we write
$$\p(\Y{3}\=\xs,\;\Z{5}\=\xf \| \zI)=0.2\;,$$
then we must already know that animal number~$3$ is a yak, $\C{3}\=\xY$,
and animal number~$5$ is a zebra, $\C{5}\=\xZ$. This is clear from our very
notation, otherwise we would not have known whether to use the symbol $Y$
or $Z$ for those instances. This information is evidently implicit in our
background information $\zI$.

Let us make the information about the $\C{i}$ explicit. Their possible
values are \enquote{$\xY$} and \enquote{$\xZ$}. We rewrite the probability
in \eqn~\eqref{eq:partialdefinetti} as
\begin{equation}
  \label{eq:infoCexplicit}
  \begin{aligned}
  &\p( \Y{1'}\=\y{1'},\; %\Y{2'}\=\y{2'},\;
  \dotsc,\;
  \Z{1''}\=\z{1''},\; %\Z{2''}\=\z{2''},\;
  \dotsc \| \zI) \equiv{}\\
  &
\p( \X{1'}\=\x{1'},\; %\X{2'}\=\x{2'},\;
  \dotsc,\;
  \X{1''}\=\x{1''},\; %\X{2''}\=\x{2''},\;
  \dotsc \| %{}\\
  \C{1'}\=\xY,\; %\C{2'}\=\cc{2'},\;
  \dotsc,\;
  \C{1''}\=\xZ,\; %\C{2''}\=\cc{2''},\;
  \dotsc,\; I) \;.
  \end{aligned}
\end{equation}

Then it is clear that the partially exchangeable probability
distribution~\eqref{eq:partialdefinetti} or \eqref{eq:infoCexplicit} can
also be called \emph{conditionally exchangeable}.




The present work gives a representation formula for pairs of quantities
$(\X{i},\C{i})$ such that
\begin{enumerate}%[nosep]
\item the distribution for the $\set{\X{i}}$ is conditionally exchangeable,
  given $\set{\C{i}}$,
\item the distribution for the $\set{\C{i}}$ is fully exchangeable.
\end{enumerate}
The formula is:
%\begin{empheq}{equation}
\begin{subequations}
  \label{eq:result}
  \begin{gather}
  \p\bigl[ (\X{1}\eq\x{1}, \C{1}\eq\cc{1}),\;
  (\X{2}\eq\x{2}, \C{2}\eq\cc{2}),\; \dotsc \| I \bigr] =
 \int \prod_{i}f_{\x{i},\cc{i}}\;
  \pf(\fx\|I)\,\di\fx\\
  \label{eq:factorizable}
\text{with}\qquad \boxed{\;\vphantom{\int}\pf(\fx\|I)\,\di\fx =
  \pf[(f_{x \bcond c}) \|I]\,\di(f_{x \bcond c})
  \times
  \pf[(f_{,c}) \|I]\,\di(f_{,c})\;}
\end{gather}
\end{subequations}
%\end{empheq}%\\[-2\baselineskip]
where
\begin{itemize}%[label=\textendash]
\item $\fx \defd (f_{x, c})$ is a joint distribution over the set of values
  that can be assumed by each $(\X{i},\C{i})$ pair,
\item $\bigl(f_{,c} \defd \sum_{x}f_{x, c} \bigr)$ is the related
  \emph{marginal} frequency distribution for the $c$ values,
\item $\bigl(f_{x \bcond c} \defd f_{x, c}/f_{,c} \bigr)$ are the related
  \emph{conditional} frequency distributions of $x$ given $c$.
\end{itemize}



The formula~\eqref{eq:result} shows that the pairs $(\X{i},\C{i})$ have a
fully exchangeable distribution, with a representation analogous to
\eqn~\eqref{eq:definetti}. The noteworthy feature of this formula is that
\emph{the density $\pf(\fx\|I)\,\di\fx$ is factorizable into the product of
  a density for the conditional frequencies $(f_{x \bcond c})$ and a
  density for the marginal frequencies $(f_{,c})$}, according to
\eqn~\eqref{eq:factorizable}. This factorization expresses the conditional
exchangeability for $\set{\X{i}}$ given $\set{\C{i}}$.



The proof of formula is given in the next section. In the final section I
discuss possible generalizations and connections with Bayesian belief
networks.



%%% Local Variables: 
%%% mode: LaTeX
%%% TeX-PDF-mode: t
%%% TeX-master: t
%%% End: 
