\pdfoutput=1
%% Author: PGL  Porta Mana
%% Created: 2019-08-08T08:04:04+0200
%% Last-Updated: 2019-08-13T00:21:47+0200
%%%%%%%%%%%%%%%%%%%%%%%%%%%%%%%%%%%%%%%%%%%%%%%%%%%%%%%%%%%%%%%%%%%%%%%%%%%%
\newif\ifarxiv
\arxivfalse
\ifarxiv\pdfmapfile{+classico.map}\fi
\newif\ifafour
\afourfalse% true = A4, false = A5
\newif\iftypodisclaim % typographical disclaim on the side
\typodisclaimtrue
\newcommand*{\memfontfamily}{zplx}
\newcommand*{\memfontpack}{newpxtext}
\documentclass[\ifafour a4paper,12pt,\else a5paper,10pt,\fi%extrafontsizes,%
onecolumn,oneside,article,%french,italian,german,swedish,latin,
british%
]{memoir}
\newcommand*{\updated}{\today}
\newcommand*{\firstdraft}{8 August 2019}
\newcommand*{\firstpublished}{***}
\newcommand*{\propertitle}{A relation between log-likelihood\\ and cross-validation
  log-scores\\{\large (with some remarks on both)}%
}% title uses LARGE; set Large for smaller
\newcommand*{\pdftitle}{A relation between log-likelihood and cross-validation log-scores (with some remarks on both)}
\newcommand*{\headtitle}{log-likelihood and cross-validation}
\newcommand*{\pdfauthor}{P.G.L.  Porta Mana}
\newcommand*{\headauthor}{Porta Mana}
\newcommand*{\reporthead}{Open Science Framework \href{https://doi.org/10.31219/osf.io/***}{\textsc{doi}:10.31219/osf.io/***}}% Report number

%%%%%%%%%%%%%%%%%%%%%%%%%%%%%%%%%%%%%%%%%%%%%%%%%%%%%%%%%%%%%%%%%%%%%%%%%%%%
%%% Calls to packages (uncomment as needed)
%%%%%%%%%%%%%%%%%%%%%%%%%%%%%%%%%%%%%%%%%%%%%%%%%%%%%%%%%%%%%%%%%%%%%%%%%%%%

%\usepackage{pifont}

%\usepackage{fontawesome}

\usepackage[T1]{fontenc} 
\input{glyphtounicode} \pdfgentounicode=1

\usepackage[utf8]{inputenx}

%\usepackage{newunicodechar}
% \newunicodechar{Ĕ}{\u{E}}
% \newunicodechar{ĕ}{\u{e}}
% \newunicodechar{Ĭ}{\u{I}}
% \newunicodechar{ĭ}{\u{\i}}
% \newunicodechar{Ŏ}{\u{O}}
% \newunicodechar{ŏ}{\u{o}}
% \newunicodechar{Ŭ}{\u{U}}
% \newunicodechar{ŭ}{\u{u}}
% \newunicodechar{Ā}{\=A}
% \newunicodechar{ā}{\=a}
% \newunicodechar{Ē}{\=E}
% \newunicodechar{ē}{\=e}
% \newunicodechar{Ī}{\=I}
% \newunicodechar{ī}{\={\i}}
% \newunicodechar{Ō}{\=O}
% \newunicodechar{ō}{\=o}
% \newunicodechar{Ū}{\=U}
% \newunicodechar{ū}{\=u}
% \newunicodechar{Ȳ}{\=Y}
% \newunicodechar{ȳ}{\=y}

\newcommand*{\bmmax}{0} % reduce number of bold fonts, before font packages
\newcommand*{\hmmax}{0} % reduce number of heavy fonts, before font packages

\usepackage{textcomp}

%\usepackage[normalem]{ulem}% package for underlining
% \makeatletter
% \def\ssout{\bgroup \ULdepth=-.35ex%\UL@setULdepth
%  \markoverwith{\lower\ULdepth\hbox
%    {\kern-.03em\vbox{\hrule width.2em\kern1.2\p@\hrule}\kern-.03em}}%
%  \ULon}
% \makeatother

\usepackage{amsmath}

\usepackage{mathtools}
\addtolength{\jot}{\jot} % increase spacing in multiline formulae
\setlength{\multlinegap}{0pt}

%\usepackage{empheq}% automatically calls amsmath and mathtools
\newcommand*{\widefbox}[1]{\fbox{\hspace{1em}#1\hspace{1em}}}

%\usepackage{fancybox}

\usepackage{mdframed}

% \usepackage[misc]{ifsym} % for dice
% \newcommand*{\diceone}{{\scriptsize\Cube{1}}}

\usepackage{amssymb}

\usepackage{amsxtra}

\usepackage[main=british,french,italian,german,swedish,latin,esperanto]{babel}\selectlanguage{british}
\newcommand*{\langfrench}{\foreignlanguage{french}}
\newcommand*{\langgerman}{\foreignlanguage{german}}
\newcommand*{\langitalian}{\foreignlanguage{italian}}
\newcommand*{\langswedish}{\foreignlanguage{swedish}}
\newcommand*{\langlatin}{\foreignlanguage{latin}}
\newcommand*{\langnohyph}{\foreignlanguage{nohyphenation}}

\usepackage[autostyle=false,autopunct=false,english=british]{csquotes}
\setquotestyle{british}

\usepackage{amsthm}
\newcommand*{\QED}{\textsc{q.e.d.}}
\renewcommand*{\qedsymbol}{\QED}
\theoremstyle{remark}
\newtheorem{note}{Note}
\newtheorem*{remark}{Note}
\newtheoremstyle{innote}{\parsep}{\parsep}{\footnotesize}{}{}{}{0pt}{}
\theoremstyle{innote}
\newtheorem*{innote}{}

\usepackage[shortlabels,inline]{enumitem}
\SetEnumitemKey{para}{itemindent=\parindent,leftmargin=0pt,listparindent=\parindent,parsep=0pt,itemsep=\topsep}
% \begin{asparaenum} = \begin{enumerate}[para]
% \begin{inparaenum} = \begin{enumerate*}
\setlist[enumerate,2]{label=\alph*.}
\setlist[enumerate]{label=\arabic*.,leftmargin=1.5\parindent}
\setlist[itemize]{leftmargin=1.5\parindent}
\setlist[description]{leftmargin=1.5\parindent}
% old alternative:
% \setlist[enumerate,2]{label=\alph*.}
% \setlist[enumerate]{leftmargin=\parindent}
% \setlist[itemize]{leftmargin=\parindent}
% \setlist[description]{leftmargin=\parindent}

\usepackage[babel,theoremfont,largesc]{newpxtext}

\usepackage[bigdelims,nosymbolsc%,smallerops % probably arXiv doesn't have it
]{newpxmath}
\linespread{1.083}%\useosf
%% smaller operators for old version of newpxmath
\makeatletter
\def\re@DeclareMathSymbol#1#2#3#4{%
    \let#1=\undefined
    \DeclareMathSymbol{#1}{#2}{#3}{#4}}
%\re@DeclareMathSymbol{\bigsqcupop}{\mathop}{largesymbols}{"46}
%\re@DeclareMathSymbol{\bigodotop}{\mathop}{largesymbols}{"4A}
\re@DeclareMathSymbol{\bigoplusop}{\mathop}{largesymbols}{"4C}
\re@DeclareMathSymbol{\bigotimesop}{\mathop}{largesymbols}{"4E}
\re@DeclareMathSymbol{\sumop}{\mathop}{largesymbols}{"50}
\re@DeclareMathSymbol{\prodop}{\mathop}{largesymbols}{"51}
\re@DeclareMathSymbol{\bigcupop}{\mathop}{largesymbols}{"53}
\re@DeclareMathSymbol{\bigcapop}{\mathop}{largesymbols}{"54}
%\re@DeclareMathSymbol{\biguplusop}{\mathop}{largesymbols}{"55}
\re@DeclareMathSymbol{\bigwedgeop}{\mathop}{largesymbols}{"56}
\re@DeclareMathSymbol{\bigveeop}{\mathop}{largesymbols}{"57}
%\re@DeclareMathSymbol{\bigcupdotop}{\mathop}{largesymbols}{"DF}
%\re@DeclareMathSymbol{\bigcapplusop}{\mathop}{largesymbolsPXA}{"00}
%\re@DeclareMathSymbol{\bigsqcupplusop}{\mathop}{largesymbolsPXA}{"02}
%\re@DeclareMathSymbol{\bigsqcapplusop}{\mathop}{largesymbolsPXA}{"04}
%\re@DeclareMathSymbol{\bigsqcapop}{\mathop}{largesymbolsPXA}{"06}
\re@DeclareMathSymbol{\bigtimesop}{\mathop}{largesymbolsPXA}{"10}
%\re@DeclareMathSymbol{\coprodop}{\mathop}{largesymbols}{"60}
%\re@DeclareMathSymbol{\varprod}{\mathop}{largesymbolsPXA}{16}
\makeatother
%%
%% With euler font cursive for Greek letters - the [1] means 100% scaling
\DeclareFontFamily{U}{egreek}{\skewchar\font'177}%
\DeclareFontShape{U}{egreek}{m}{n}{<-6>s*[1]eurm5 <6-8>s*[1]eurm7 <8->s*[1]eurm10}{}%
\DeclareFontShape{U}{egreek}{m}{it}{<->s*[1]eurmo10}{}%
\DeclareFontShape{U}{egreek}{b}{n}{<-6>s*[1]eurb5 <6-8>s*[1]eurb7 <8->s*[1]eurb10}{}%
\DeclareFontShape{U}{egreek}{b}{it}{<->s*[1]eurbo10}{}%
\DeclareSymbolFont{egreeki}{U}{egreek}{m}{it}%
\SetSymbolFont{egreeki}{bold}{U}{egreek}{b}{it}% from the amsfonts package
\DeclareSymbolFont{egreekr}{U}{egreek}{m}{n}%
\SetSymbolFont{egreekr}{bold}{U}{egreek}{b}{n}% from the amsfonts package
% Take also \sum, \prod, \coprod symbols from Euler fonts
\DeclareFontFamily{U}{egreekx}{\skewchar\font'177}
\DeclareFontShape{U}{egreekx}{m}{n}{%
       <-7.5>s*[0.9]euex7%
    <7.5-8.5>s*[0.9]euex8%
    <8.5-9.5>s*[0.9]euex9%
    <9.5->s*[0.9]euex10%
}{}
\DeclareSymbolFont{egreekx}{U}{egreekx}{m}{n}
\DeclareMathSymbol{\sumop}{\mathop}{egreekx}{"50}
\DeclareMathSymbol{\prodop}{\mathop}{egreekx}{"51}
\DeclareMathSymbol{\coprodop}{\mathop}{egreekx}{"60}
\makeatletter
\def\sum{\DOTSI\sumop\slimits@}
\def\prod{\DOTSI\prodop\slimits@}
\def\coprod{\DOTSI\coprodop\slimits@}
\makeatother
\input{definegreek.tex}% Greek letters not usually given in LaTeX.

%\usepackage%[scaled=0.9]%
%{classico}%  Optima as sans-serif font
\renewcommand\sfdefault{uop}
\DeclareMathAlphabet{\mathsf}  {T1}{\sfdefault}{m}{sl}
\SetMathAlphabet{\mathsf}{bold}{T1}{\sfdefault}{b}{sl}
%\newcommand*{\mathte}[1]{\textbf{\textit{\textsf{#1}}}}
% Upright sans-serif math alphabet
% \DeclareMathAlphabet{\mathsu}  {T1}{\sfdefault}{m}{n}
% \SetMathAlphabet{\mathsu}{bold}{T1}{\sfdefault}{b}{n}

% DejaVu Mono as typewriter text
\usepackage[scaled=0.84]{DejaVuSansMono}

\usepackage{mathdots}

\usepackage[usenames]{xcolor}
% Tol (2012) colour-blind-, print-, screen-friendly colours, alternative scheme; Munsell terminology
\definecolor{mypurpleblue}{RGB}{68,119,170}
\definecolor{myblue}{RGB}{102,204,238}
\definecolor{mygreen}{RGB}{34,136,51}
\definecolor{myyellow}{RGB}{204,187,68}
\definecolor{myred}{RGB}{238,102,119}
\definecolor{myredpurple}{RGB}{170,51,119}
\definecolor{mygrey}{RGB}{187,187,187}
% Tol (2012) colour-blind-, print-, screen-friendly colours; Munsell terminology
% \definecolor{lbpurple}{RGB}{51,34,136}
% \definecolor{lblue}{RGB}{136,204,238}
% \definecolor{lbgreen}{RGB}{68,170,153}
% \definecolor{lgreen}{RGB}{17,119,51}
% \definecolor{lgyellow}{RGB}{153,153,51}
% \definecolor{lyellow}{RGB}{221,204,119}
% \definecolor{lred}{RGB}{204,102,119}
% \definecolor{lpred}{RGB}{136,34,85}
% \definecolor{lrpurple}{RGB}{170,68,153}
\definecolor{lgrey}{RGB}{221,221,221}
%\newcommand*\mycolourbox[1]{%
%\colorbox{mygrey}{\hspace{1em}#1\hspace{1em}}}
\colorlet{shadecolor}{lgrey}

\usepackage{bm}

\usepackage{microtype}

\usepackage[backend=biber,mcite,%subentry,
citestyle=authoryear-comp,bibstyle=pglpm-authoryear,autopunct=false,sorting=ny,sortcites=false,natbib=false,maxcitenames=1,maxbibnames=8,minbibnames=8,giveninits=true,uniquename=false,uniquelist=false,maxalphanames=1,block=space,hyperref=true,defernumbers=false,useprefix=true,sortupper=false,language=british,parentracker=false]{biblatex}
\DeclareSortingScheme{ny}{\sort{\field{sortname}\field{author}\field{editor}}\sort{\field{year}}}
\iffalse\makeatletter%%% replace parenthesis with brackets
\newrobustcmd*{\parentexttrack}[1]{%
  \begingroup
  \blx@blxinit
  \blx@setsfcodes
  \blx@bibopenparen#1\blx@bibcloseparen
  \endgroup}
\AtEveryCite{%
  \let\parentext=\parentexttrack%
  \let\bibopenparen=\bibopenbracket%
  \let\bibcloseparen=\bibclosebracket}
\makeatother\fi
\DefineBibliographyExtras{british}{\def\finalandcomma{\addcomma}}
\renewcommand*{\finalnamedelim}{\addcomma\space}
\setcounter{biburlnumpenalty}{1}
\setcounter{biburlucpenalty}{0}
\setcounter{biburllcpenalty}{1}
\DeclareDelimFormat{multicitedelim}{\addsemicolon\space}
\DeclareDelimFormat{compcitedelim}{\addsemicolon\space}
\DeclareDelimFormat{postnotedelim}{\space}
\ifarxiv\else\addbibresource{portamanabib.bib}\fi
\renewcommand{\bibfont}{\footnotesize}
%\appto{\citesetup}{\footnotesize}% smaller font for citations
\defbibheading{bibliography}[\bibname]{\section*{#1}\addcontentsline{toc}{section}{#1}%\markboth{#1}{#1}
}
\newcommand*{\citep}{\parencites}
\newcommand*{\citey}{\parencites*}
%\renewcommand*{\cite}{\parencite}
\renewcommand*{\cites}{\parencites}
\providecommand{\href}[2]{#2}
\providecommand{\eprint}[2]{\texttt{\href{#1}{#2}}}
\newcommand*{\amp}{\&}
% \newcommand*{\citein}[2][]{\textnormal{\textcite[#1]{#2}}%\addtocategory{extras}{#2}
% }
\newcommand*{\citein}[2][]{\textnormal{\textcite[#1]{#2}}%\addtocategory{extras}{#2}
}
\newcommand*{\citebi}[2][]{\textcite[#1]{#2}%\addtocategory{extras}{#2}
}
\newcommand*{\subtitleproc}[1]{}
\newcommand*{\chapb}{ch.}
%
% \def\arxivp{}
% \def\mparcp{}
% \def\philscip{}
% \def\biorxivp{}
% \newcommand*{\arxivsi}{\texttt{arXiv} eprints available at \url{http://arxiv.org/}.\\}
% \newcommand*{\mparcsi}{\texttt{mp\_arc} eprints available at \url{http://www.ma.utexas.edu/mp_arc/}.\\}
% \newcommand*{\philscisi}{\texttt{philsci} eprints available at \url{http://philsci-archive.pitt.edu/}.\\}
% \newcommand*{\biorxivsi}{\texttt{bioRxiv} eprints available at \url{http://biorxiv.org/}.\\}
\newcommand*{\arxiveprint}[1]{%\global\def\arxivp{\arxivsi}%\citeauthor{0arxivcite}\addtocategory{ifarchcit}{0arxivcite}%eprint
\texttt{\urlalt{https://arxiv.org/abs/#1}{arXiv:\hspace{0pt}#1}}%
%\texttt{\href{http://arxiv.org/abs/#1}{\protect\url{arXiv:#1}}}%
%\renewcommand{\arxivnote}{\texttt{arXiv} eprints available at \url{http://arxiv.org/}.}
}
\newcommand*{\haleprint}[1]{%\global\def\arxivp{\arxivsi}%\citeauthor{0arxivcite}\addtocategory{ifarchcit}{0arxivcite}%eprint
\texttt{\urlalt{https://hal.archives-ouvertes.fr/#1}{HAL:\hspace{0pt}#1}}%
%\texttt{\href{http://arxiv.org/abs/#1}{\protect\url{arXiv:#1}}}%
%\renewcommand{\arxivnote}{\texttt{arXiv} eprints available at \url{http://arxiv.org/}.}
}
\newcommand*{\mparceprint}[1]{%\global\def\mparcp{\mparcsi}%\citeauthor{0mparccite}\addtocategory{ifarchcit}{0mparccite}%eprint
\texttt{\urlalt{http://www.ma.utexas.edu/mp_arc-bin/mpa?yn=#1}{mp\_arc:\hspace{0pt}#1}}%
%\texttt{\href{http://www.ma.utexas.edu/mp_arc-bin/mpa?yn=#1}{\protect\url{mp_arc:#1}}}%
%\providecommand{\mparcnote}{\texttt{mp_arc} eprints available at \url{http://www.ma.utexas.edu/mp_arc/}.}
}
\newcommand*{\philscieprint}[1]{%\global\def\philscip{\philscisi}%\citeauthor{0philscicite}\addtocategory{ifarchcit}{0philscicite}%eprint
\texttt{\urlalt{http://philsci-archive.pitt.edu/archive/#1}{PhilSci:\hspace{0pt}#1}}%
%\texttt{\href{http://philsci-archive.pitt.edu/archive/#1}{\protect\url{PhilSci:#1}}}%
%\providecommand{\mparcnote}{\texttt{philsci} eprints available at \url{http://philsci-archive.pitt.edu/}.}
}
\newcommand*{\biorxiveprint}[1]{%\global\def\biorxivp{\biorxivsi}%\citeauthor{0arxivcite}\addtocategory{ifarchcit}{0arxivcite}%eprint
\texttt{\urlalt{https://doi.org/10.1101/#1}{bioRxiv doi:\hspace{0pt}10.1101/#1}}%
%\texttt{\href{http://arxiv.org/abs/#1}{\protect\url{arXiv:#1}}}%
%\renewcommand{\arxivnote}{\texttt{arXiv} eprints available at \url{http://arxiv.org/}.}
}
\newcommand*{\osfeprint}[1]{%
\texttt{\urlalt{https://doi.org/10.17605/osf.io/#1}{Open Science Framework doi:10.17605/osf.io/#1}}%
}

\usepackage{graphicx}

%\usepackage{wrapfig}

%\usepackage{tikz-cd}

\PassOptionsToPackage{hyphens}{url}\usepackage[hypertexnames=false]{hyperref}

\usepackage[depth=4]{bookmark}
\hypersetup{colorlinks=true,bookmarksnumbered,pdfborder={0 0 0.25},citebordercolor={0.2667 0.4667 0.6667},citecolor=mypurpleblue,linkbordercolor={0.6667 0.2 0.4667},linkcolor=myredpurple,urlbordercolor={0.1333 0.5333 0.2},urlcolor=mygreen,breaklinks=true,pdftitle={\pdftitle},pdfauthor={\pdfauthor}}
% \usepackage[vertfit=local]{breakurl}% only for arXiv
\providecommand*{\urlalt}{\href}

\usepackage[british]{datetime2}
\DTMnewdatestyle{mydate}%
{% definitions
\renewcommand*{\DTMdisplaydate}[4]{%
\number##3\ \DTMenglishmonthname{##2} ##1}%
\renewcommand*{\DTMDisplaydate}{\DTMdisplaydate}%
}
\DTMsetdatestyle{mydate}

%%%%%%%%%%%%%%%%%%%%%%%%%%%%%%%%%%%%%%%%%%%%%%%%%%%%%%%%%%%%%%%%%%%%%%%%%%%%
%%% Layout. I do not know on which kind of paper the reader will print the
%%% paper on (A4? letter? one-sided? double-sided?). So I choose A5, which
%%% provides a good layout for reading on screen and save paper if printed
%%% two pages per sheet. Average length line is 66 characters and page
%%% numbers are centred.
%%%%%%%%%%%%%%%%%%%%%%%%%%%%%%%%%%%%%%%%%%%%%%%%%%%%%%%%%%%%%%%%%%%%%%%%%%%%
\ifafour\setstocksize{297mm}{210mm}%{*}% A4
\else\setstocksize{210mm}{5.5in}%{*}% 210x139.7
\fi
\settrimmedsize{\stockheight}{\stockwidth}{*}
\setlxvchars[\normalfont] %313.3632pt for a 66-characters line
\setxlvchars[\normalfont]
\setlength{\trimtop}{0pt}
\setlength{\trimedge}{\stockwidth}
\addtolength{\trimedge}{-\paperwidth}
% The length of the normalsize alphabet is 133.05988pt - 10 pt = 26.1408pc
% The length of the normalsize alphabet is 159.6719pt - 12pt = 30.3586pc
% Bringhurst gives 32pc as boundary optimal with 69 ch per line
% The length of the normalsize alphabet is 191.60612pt - 14pt = 35.8634pc
\ifafour\settypeblocksize{*}{32pc}{1.618} % A4
%\setulmargins{*}{*}{1.667}%gives 5/3 margins % 2 or 1.667
\else\settypeblocksize{*}{26pc}{1.618}% nearer to a 66-line newpx and preserves GR
\fi
\setulmargins{*}{*}{1}%gives equal margins
\setlrmargins{*}{*}{*}
\setheadfoot{\onelineskip}{2.5\onelineskip}
\setheaderspaces{*}{2\onelineskip}{*}
\setmarginnotes{2ex}{10mm}{0pt}
\checkandfixthelayout[nearest]
\fixpdflayout
%%% End layout
%% this fixes missing white spaces
\pdfmapline{+dummy-space <dummy-space.pfb}\pdfinterwordspaceon%

%%% Sectioning
\newcommand*{\asudedication}[1]{%
{\par\centering\textit{#1}\par}}
\newenvironment{acknowledgements}{\section*{Thanks}\addcontentsline{toc}{section}{Thanks}}{\par}
\makeatletter\renewcommand{\appendix}{\par
  \bigskip{\centering
   \interlinepenalty \@M
   \normalfont
   \printchaptertitle{\sffamily\appendixpagename}\par}
  \setcounter{section}{0}%
  \gdef\@chapapp{\appendixname}%
  \gdef\thesection{\@Alph\c@section}%
  \anappendixtrue}\makeatother
\counterwithout{section}{chapter}
\setsecnumformat{\upshape\csname the#1\endcsname\quad}
\setsecheadstyle{\large\bfseries\sffamily%
\centering}
\setsubsecheadstyle{\bfseries\sffamily%
\raggedright}
%\setbeforesecskip{-1.5ex plus 1ex minus .2ex}% plus 1ex minus .2ex}
%\setaftersecskip{1.3ex plus .2ex }% plus 1ex minus .2ex}
%\setsubsubsecheadstyle{\bfseries\sffamily\slshape\raggedright}
%\setbeforesubsecskip{1.25ex plus 1ex minus .2ex }% plus 1ex minus .2ex}
%\setaftersubsecskip{-1em}%{-0.5ex plus .2ex}% plus 1ex minus .2ex}
\setsubsecindent{0pt}%0ex plus 1ex minus .2ex}
\setparaheadstyle{\bfseries\sffamily%
\raggedright}
\setcounter{secnumdepth}{2}
\setlength{\headwidth}{\textwidth}
\newcommand{\addchap}[1]{\chapter*[#1]{#1}\addcontentsline{toc}{chapter}{#1}}
\newcommand{\addsec}[1]{\section*{#1}\addcontentsline{toc}{section}{#1}}
\newcommand{\addsubsec}[1]{\subsection*{#1}\addcontentsline{toc}{subsection}{#1}}
\newcommand{\addpara}[1]{\paragraph*{#1.}\addcontentsline{toc}{subsubsection}{#1}}
\newcommand{\addparap}[1]{\paragraph*{#1}\addcontentsline{toc}{subsubsection}{#1}}

%%% Headers, footers, pagestyle
\copypagestyle{manaart}{plain}
\makeheadrule{manaart}{\headwidth}{0.5\normalrulethickness}
\makeoddhead{manaart}{%
{\footnotesize%\sffamily%
\scshape\headauthor}}{}{{\footnotesize\sffamily%
\headtitle}}
\makeoddfoot{manaart}{}{\thepage}{}
\newcommand*\autanet{\includegraphics[height=\heightof{M}]{autanet.pdf}}
\definecolor{mygray}{gray}{0.333}
\iftypodisclaim%
\ifafour\newcommand\addprintnote{\begin{picture}(0,0)%
\put(245,149){\makebox(0,0){\rotatebox{90}{\tiny\color{mygray}\textsf{This
            document is designed for screen reading and
            two-up printing on A4 or Letter paper}}}}%
\end{picture}}% A4
\else\newcommand\addprintnote{\begin{picture}(0,0)%
\put(176,112){\makebox(0,0){\rotatebox{90}{\tiny\color{mygray}\textsf{This
            document is designed for screen reading and
            two-up printing on A4 or Letter paper}}}}%
\end{picture}}\fi%afourtrue
\makeoddfoot{plain}{}{\makebox[0pt]{\thepage}\addprintnote}{}
\else
\makeoddfoot{plain}{}{\makebox[0pt]{\thepage}}{}
\fi%typodisclaimtrue
\makeoddhead{plain}{\scriptsize\reporthead}{}{}
% \copypagestyle{manainitial}{plain}
% \makeheadrule{manainitial}{\headwidth}{0.5\normalrulethickness}
% \makeoddhead{manainitial}{%
% \footnotesize\sffamily%
% \scshape\headauthor}{}{\footnotesize\sffamily%
% \headtitle}
% \makeoddfoot{manaart}{}{\thepage}{}

\pagestyle{manaart}

\setlength{\droptitle}{-3.9\onelineskip}
\pretitle{\begin{center}\LARGE\sffamily%
\bfseries}
\posttitle{\bigskip\end{center}}

\makeatletter\newcommand*{\atf}{\includegraphics[%trim=1pt 1pt 0pt 0pt,
totalheight=\heightof{@}]{atblack.png}}\makeatother
\providecommand{\affiliation}[1]{\textsl{\textsf{\footnotesize #1}}}
\providecommand{\epost}[1]{\texttt{\footnotesize\textless#1\textgreater}}
\providecommand{\email}[2]{\href{mailto:#1ZZ@#2 ((remove ZZ))}{#1\protect\atf#2}}

\preauthor{\vspace{-0.5\baselineskip}\begin{center}
\normalsize\sffamily%
\lineskip  0.5em}
\postauthor{\par\end{center}}
\predate{\DTMsetdatestyle{mydate}\begin{center}\footnotesize}
\postdate{\end{center}\vspace{-\medskipamount}}

\setfloatadjustment{figure}{\footnotesize}
\captiondelim{\quad}
\captionnamefont{\footnotesize\sffamily%
}
\captiontitlefont{\footnotesize}
\firmlists*
\midsloppy
% handling orphan/widow lines, memman.pdf
% \clubpenalty=10000
% \widowpenalty=10000
% \raggedbottom
% Downes, memman.pdf
\clubpenalty=9996
\widowpenalty=9999
\brokenpenalty=4991
\predisplaypenalty=10000
\postdisplaypenalty=1549
\displaywidowpenalty=1602
\raggedbottom
\selectlanguage{british}\frenchspacing

%%%%%%%%%%%%%%%%%%%%%%%%%%%%%%%%%%%%%%%%%%%%%%%%%%%%%%%%%%%%%%%%%%%%%%%%%%%%
%%% Paper's details
%%%%%%%%%%%%%%%%%%%%%%%%%%%%%%%%%%%%%%%%%%%%%%%%%%%%%%%%%%%%%%%%%%%%%%%%%%%%
\title{\propertitle}
\author{%
\hspace*{\stretch{1}}%
%% uncomment if additional authors present
% \parbox{0.5\linewidth}%\makebox[0pt][c]%
% {\protect\centering ***\\%
% \footnotesize\epost{\email{***}{***}}}%
% \hspace*{\stretch{1}}%
\parbox{0.75\linewidth}%\makebox[0pt][c]%
{\protect\centering P.G.L.  Porta Mana\\%
\footnotesize Kavli Institute, Trondheim, Norway \quad\epost{\email{piero.mana}{ntnu.no}}}%
\hspace*{\stretch{1}}%
%\quad\href{https://orcid.org/0000-0002-6070-0784}{\protect\includegraphics[scale=0.16]{orcid_32x32.png}\textsc{orcid}:0000-0002-6070-0784}%
}

\date{Draft of \today\ (first drafted \firstdraft)}
%\date{\firstpublished; updated \updated}

%%%%%%%%%%%%%%%%%%%%%%%%%%%%%%%%%%%%%%%%%%%%%%%%%%%%%%%%%%%%%%%%%%%%%%%%%%%%
%%% Macros @@@
%%%%%%%%%%%%%%%%%%%%%%%%%%%%%%%%%%%%%%%%%%%%%%%%%%%%%%%%%%%%%%%%%%%%%%%%%%%%

% Common ones - uncomment as needed
%\providecommand{\nequiv}{\not\equiv}
%\providecommand{\coloneqq}{\mathrel{\mathop:}=}
%\providecommand{\eqqcolon}{=\mathrel{\mathop:}}
%\providecommand{\varprod}{\prod}
\newcommand*{\de}{\partialup}%partial diff
\newcommand*{\pu}{\piup}%constant pi
\newcommand*{\delt}{\deltaup}%Kronecker, Dirac
%\newcommand*{\eps}{\varepsilonup}%Levi-Civita, Heaviside
%\newcommand*{\riem}{\zetaup}%Riemann zeta
%\providecommand{\degree}{\textdegree}% degree
%\newcommand*{\celsius}{\textcelsius}% degree Celsius
%\newcommand*{\micro}{\textmu}% degree Celsius
\newcommand*{\I}{\mathrm{i}}%imaginary unit
\newcommand*{\e}{\mathrm{e}}%Neper
\newcommand*{\di}{\mathrm{d}}%differential
%\newcommand*{\Di}{\mathrm{D}}%capital differential
%\newcommand*{\planckc}{\hslash}
%\newcommand*{\avogn}{N_{\textrm{A}}}
%\newcommand*{\NN}{\bm{\mathrm{N}}}
%\newcommand*{\ZZ}{\bm{\mathrm{Z}}}
%\newcommand*{\QQ}{\bm{\mathrm{Q}}}
\newcommand*{\RR}{\bm{\mathrm{R}}}
%\newcommand*{\CC}{\bm{\mathrm{C}}}
%\newcommand*{\nabl}{\bm{\nabla}}%nabla
%\DeclareMathOperator{\lb}{lb}%base 2 log
%\DeclareMathOperator{\tr}{tr}%trace
%\DeclareMathOperator{\card}{card}%cardinality
%\DeclareMathOperator{\im}{Im}%im part
%\DeclareMathOperator{\re}{Re}%re part
%\DeclareMathOperator{\sgn}{sgn}%signum
%\DeclareMathOperator{\ent}{ent}%integer less or equal to
%\DeclareMathOperator{\Ord}{O}%same order as
%\DeclareMathOperator{\ord}{o}%lower order than
%\newcommand*{\incr}{\triangle}%finite increment
\newcommand*{\defd}{\coloneqq}
\newcommand*{\defs}{\eqqcolon}
\newcommand*{\Land}{\bigwedge}
\newcommand*{\Lor}{\bigvee}
%\newcommand*{\lland}{\DOTSB\;\land\;}
%\newcommand*{\llor}{\DOTSB\;\lor\;}
\newcommand*{\limplies}{\mathbin{\Rightarrow}}%implies
%\newcommand*{\suchthat}{\mid}%{\mathpunct{|}}%such that (eg in sets)
%\newcommand*{\with}{\colon}%with (list of indices)
%\newcommand*{\mul}{\times}%multiplication
%\newcommand*{\inn}{\cdot}%inner product
%\newcommand*{\dotv}{\mathord{\,\cdot\,}}%variable place
%\newcommand*{\comp}{\circ}%composition of functions
%\newcommand*{\con}{\mathbin{:}}%scal prod of tensors
%\newcommand*{\equi}{\sim}%equivalent to 
\renewcommand*{\asymp}{\simeq}%equivalent to 
%\newcommand*{\corr}{\mathrel{\hat{=}}}%corresponds to
%\providecommand{\varparallel}{\ensuremath{\mathbin{/\mkern-7mu/}}}%parallel (tentative symbol)
\renewcommand*{\le}{\leqslant}%less or equal
\renewcommand*{\ge}{\geqslant}%greater or equal
%\DeclarePairedDelimiter\clcl{[}{]}
%\DeclarePairedDelimiter\clop{[}{[}
%\DeclarePairedDelimiter\opcl{]}{]}
%\DeclarePairedDelimiter\opop{]}{[}
\DeclarePairedDelimiter\abs{\lvert}{\rvert}
%\DeclarePairedDelimiter\norm{\lVert}{\rVert}
\DeclarePairedDelimiter\set{\{}{\}}
%\DeclareMathOperator{\pr}{P}%probability
\newcommand*{\pf}{\mathrm{p}}%probability
\newcommand*{\p}{\mathrm{P}}%probability
\newcommand*{\E}{\mathrm{E}}
%\renewcommand*{\|}{\nonscript\,\vert\nonscript\;\mathopen{}}
\renewcommand*{\|}[1][]{\nonscript\,#1\vert\nonscript\;\mathopen{}}
%\DeclarePairedDelimiterX{\cond}[2]{(}{)}{#1\nonscript\,\delimsize\vert\nonscript\;\mathopen{}#2}
\DeclarePairedDelimiterX{\condt}[2]{[}{]}{#1\nonscript\,\delimsize\vert\nonscript\;\mathopen{}#2}
%\DeclarePairedDelimiterX{\conds}[2]{\{}{\}}{#1\nonscript\,\delimsize\vert\nonscript\;\mathopen{}#2}
%\newcommand*{\+}{\lor}
%\renewcommand{\*}{\land}
\newcommand*{\sect}{\S}% Sect.~
\newcommand*{\sects}{\S\S}% Sect.~
\newcommand*{\chap}{ch.}%
\newcommand*{\chaps}{chs}%
\newcommand*{\bref}{ref.}%
\newcommand*{\brefs}{refs}%
%\newcommand*{\fn}{fn}%
\newcommand*{\eqn}{eq.}%
\newcommand*{\eqns}{eqs}%
\newcommand*{\fig}{fig.}%
\newcommand*{\figs}{figs}%
\newcommand*{\vs}{{vs}}
\newcommand*{\etc}{{etc.}}
%\newcommand*{\ie}{{i.e.}}
%\newcommand*{\ca}{{c.}}
\newcommand*{\eg}{{e.g.}}
\newcommand*{\foll}{{ff.}}
%\newcommand*{\viz}{{viz}}
\newcommand*{\cf}{{cf.}}
%\newcommand*{\Cf}{{Cf.}}
%\newcommand*{\vd}{{v.}}
\newcommand*{\etal}{{et al.}}
%\newcommand*{\etsim}{{et sim.}}
%\newcommand*{\ibid}{{ibid.}}
%\newcommand*{\sic}{{sic}}
%\newcommand*{\id}{\mathte{I}}%id matrix
%\newcommand*{\nbd}{\nobreakdash}%
%\newcommand*{\bd}{\hspace{0pt}}%
%\def\hy{-\penalty0\hskip0pt\relax}
%\newcommand*{\labelbis}[1]{\tag*{(\ref{#1})$_\text{r}$}}
%\newcommand*{\mathbox}[2][.8]{\parbox[t]{#1\columnwidth}{#2}}
%\newcommand*{\zerob}[1]{\makebox[0pt][l]{#1}}
\newcommand*{\tprod}{\mathop{\textstyle\prod}\nolimits}
\newcommand*{\tsum}{\mathop{\textstyle\sum}\nolimits}
%\newcommand*{\tint}{\begingroup\textstyle\int\endgroup\nolimits}
%\newcommand*{\tland}{\mathop{\textstyle\bigwedge}\nolimits}
\newcommand*{\tlor}{\mathop{\textstyle\bigvee}\nolimits}
%\newcommand*{\sprod}{\mathop{\textstyle\prod}}
%\newcommand*{\ssum}{\mathop{\textstyle\sum}}
%\newcommand*{\sint}{\begingroup\textstyle\int\endgroup}
%\newcommand*{\sland}{\mathop{\textstyle\bigwedge}}
%\newcommand*{\slor}{\mathop{\textstyle\bigvee}}
%\newcommand*{\T}{^\intercal}%transpose
%%\newcommand*{\QEM}%{\textnormal{$\Box$}}%{\ding{167}}
%\newcommand*{\qem}{\leavevmode\unskip\penalty9999 \hbox{}\nobreak\hfill
%\quad\hbox{\QEM}}

%%%%%%%%%%%%%%%%%%%%%%%%%%%%%%%%%%%%%%%%%%%%%%%%%%%%%%%%%%%%%%%%%%%%%%%%%%%%
%%% Custom macros for this file @@@
%%%%%%%%%%%%%%%%%%%%%%%%%%%%%%%%%%%%%%%%%%%%%%%%%%%%%%%%%%%%%%%%%%%%%%%%%%%%
 \definecolor{notecolour}{RGB}{68,170,153}
\newcommand*{\puzzle}{{\fontencoding{U}\fontfamily{fontawesometwo}\selectfont\symbol{225}}}
%\newcommand*{\puzzle}{\maltese}
\newcommand{\mynote}[1]{ {\color{notecolour}\puzzle\ #1}}
\newcommand*{\widebar}[1]{{\mkern1.5mu\skew{2}\overline{\mkern-1.5mu#1\mkern-1.5mu}\mkern 1.5mu}}

% \newcommand{\explanation}[4][t]{%\setlength{\tabcolsep}{-1ex}
% %\smash{
% \begin{tabular}[#1]{c}#2\\[0.5\jot]\rule{1pt}{#3}\\#4\end{tabular}}%}
% \newcommand*{\ptext}[1]{\text{\small #1}}
%\DeclareMathOperator*{\argsup}{arg\,sup}
\newcommand*{\dob}{degree of belief}
\newcommand*{\dobs}{degrees of belief}
\newcommand*{\yK}{I}
\newcommand*{\yO}{\mathrm{O}}
\newcommand*{\hH}{\Hat{H}}
%%% Custom macros end @@@

%%%%%%%%%%%%%%%%%%%%%%%%%%%%%%%%%%%%%%%%%%%%%%%%%%%%%%%%%%%%%%%%%%%%%%%%%%%%
%%% Beginning of document
%%%%%%%%%%%%%%%%%%%%%%%%%%%%%%%%%%%%%%%%%%%%%%%%%%%%%%%%%%%%%%%%%%%%%%%%%%%%
\firmlists
\begin{document}
\captiondelim{\quad}\captionnamefont{\footnotesize}\captiontitlefont{\footnotesize}
\selectlanguage{british}\frenchspacing
\maketitle

%%%%%%%%%%%%%%%%%%%%%%%%%%%%%%%%%%%%%%%%%%%%%%%%%%%%%%%%%%%%%%%%%%%%%%%%%%%%
%%% Abstract
%%%%%%%%%%%%%%%%%%%%%%%%%%%%%%%%%%%%%%%%%%%%%%%%%%%%%%%%%%%%%%%%%%%%%%%%%%%%
\abstractrunin
\abslabeldelim{}
\renewcommand*{\abstractname}{}
\setlength{\absleftindent}{0pt}
\setlength{\absrightindent}{0pt}
\setlength{\abstitleskip}{-\absparindent}
\begin{abstract}\labelsep 0pt%
  \noindent ***
\\\noindent\emph{\footnotesize Note: Dear Reader
    \amp\ Peer, this manuscript is being peer-reviewed by you. Thank you.}
% \par%\\[\jot]
% \noindent
% {\footnotesize PACS: ***}\qquad%
% {\footnotesize MSC: ***}%
%\qquad{\footnotesize Keywords: ***}
\end{abstract}
\selectlanguage{british}\frenchspacing

%%%%%%%%%%%%%%%%%%%%%%%%%%%%%%%%%%%%%%%%%%%%%%%%%%%%%%%%%%%%%%%%%%%%%%%%%%%%
%%% Epigraph
%%%%%%%%%%%%%%%%%%%%%%%%%%%%%%%%%%%%%%%%%%%%%%%%%%%%%%%%%%%%%%%%%%%%%%%%%%%%
% \asudedication{\small ***}
% \vspace{\bigskipamount}
% \setlength{\epigraphwidth}{.7\columnwidth}
% %\epigraphposition{flushright}
% \epigraphtextposition{flushright}
% %\epigraphsourceposition{flushright}
% \epigraphfontsize{\footnotesize}
% \setlength{\epigraphrule}{0pt}
% %\setlength{\beforeepigraphskip}{0pt}
% %\setlength{\afterepigraphskip}{0pt}
% \epigraph{\emph{text}}{source}



%%%%%%%%%%%%%%%%%%%%%%%%%%%%%%%%%%%%%%%%%%%%%%%%%%%%%%%%%%%%%%%%%%%%%%%%%%%%
%%% BEGINNING OF MAIN TEXT
%%%%%%%%%%%%%%%%%%%%%%%%%%%%%%%%%%%%%%%%%%%%%%%%%%%%%%%%%%%%%%%%%%%%%%%%%%%%

\section{***}
\label{sec:***}

The probability calculus unequivocally tells us how
$\p(H_{h} \| D \, \yK)$, our \dob\ in a hypothesis $H_{h}$ given data $D$
and background information or assumptions $\yK$, is related to
$\p(D \| H_{h} \, \yK)$, our \dob\ in observing those data when we
entertain that hypothesis as true:
\begin{subequations}
    \label{eq:posterior_hypothesis}
  \begin{align}
    \label{eq:posterior_hypothesis_universal}
    \p(H_{h} \| D \, \yK) &=
    \frac{\p(D \| H_{h} \, \yK)\;\p(H_{h} \| \yK)}{\p(D \| \yK)}\\
    \label{eq:posterior_hypothesis_sethypotheses}
    &=\frac{\p(D \| H_{h} \, \yK)\;\p(H_{h} \| \yK)}{\sum_{h'} \p(D \| H_{h'} \, \yK)\; \p(H_{h'} \| \yK)}.
  \end{align}
\end{subequations}
$D$, $H_{h}$, $\yK$ denote propositions, which usually are about numeric
quantities. I use the terms \enquote{\dob}, \enquote{belief}, and
\enquote{probability} as synonyms. By \enquote{hypothesis} I mean a
scientific (physical, biological, \etc) hypothesis, a state or development
of things capable of experimental verification, at least in a thought
experiment.
\begin{innote}
  Don't we too often abuse of the fact that the probability calculus, just
  like the truth calculus, proceeds purely syntactically rather than
  semantically? That is, if I tell you that $H$ and $H \limplies D$ are
  true, you can conclude that $D$ is true; similarly, if I tell you that
  $\p(H \| \yK) = p$ and $\p(H \limplies D \| \yK) = q$, you can conclude
  (try it as an exercise) that $\p(H\;D \| \yK) = p + q -1$. In either case
  you don't need to know what $H$ and $D$ are about -- they could be about
  Donald Duck or parallel universes. And so we often say \enquote{under
    model $\theta$ our belief about the value of quantity $x$ is expressed
    by such and such distribution $\pf(x \| \theta)=f(x)$}, without
  explaining what $\theta$ really is and why it leads to $f$. Aren't terms
  such as \enquote{model} and \enquote{hypothesis}, as often used in
  probability and statistics, convenient and respectable-looking carpets
  under which we can sweep the fact that we don't quite know what we're
  speaking about? The need to look under the carpet arises, though, the
  moment we have to specify our pre-data belief, the prior, about the
  mysterious $\theta$.
  
  But semantics can very well be a by-product of syntax, or the distinction
  between the two be a chimera
  \citep{wittgenstein1945_t1999,girard2001,girard2003}. Such important
  matters are unfortunately rarely discussed in probability and statistics.
\end{innote}


Expression~\eqref{eq:posterior_hypothesis_sethypotheses} assumes that
we have a set $\set{H_{h}}$ of mutually exclusive and exhaustive hypotheses
under consideration, which is implicit in our knowledge $\yK$ -- in fact,
the right side is only valid if
\begin{equation}
  \label{eq:implicit_knowledge}
  \p\bigl(\tlor_{h} H_{h} \| \yK\bigr) = 1,
  \qquad
  \p(H_{h} \land H_{h'} \| \yK) = 0 \quad \text{if $h \ne h'$}.
\end{equation}
Only in extremely rare cases does the set of hypotheses $\set{H_{h}}$
encompass and reflect the extremely complex and fuzzy hypotheses lying
in the backs of our minds. The background knowledge $\yK$ is therefore only
a simplified picture of our actual knowledge. That's why $\yK$ or the
hypotheses $\set{H_{h}}$ are often called \emph{models}.\; \enquote{A
  theory cannot duplicate nature, for if it did so in all respects, it
  would be isomorphic to nature itself and hence useless, a mere repetition
  of all the complexity which nature presents to us, that very complexity
  we frame theories to penetrate and set aside. If a theory were not
  simpler than the phenomena it was designed to model, it would serve no
  purpose. Like a portrait, it can represent only a part of the subject it
  pictures. This part it exaggerates, if only because it leaves out the
  rest. Its simplicity is its virtue, provided the aspect it portrays be
  that which we wish to study} \citep[Prologue p.~xvi]{truesdelletal1980}.

Expression~\eqref{eq:posterior_hypothesis_universal} is universally valid
instead, but it's rarely possible to quantify its denominator
$\p(D \| \yK)$ unless we simplify our inferential problem by introducing a
possibly unrealistic exhaustive set of hypotheses, thus falling back
to~\eqref{eq:posterior_hypothesis_sethypotheses}. We can bypass this
problem if we are content with comparing our beliefs about any two
hypotheses through their ratio, so that the term $\p(D \| \yK)$ cancels
out. See Jaynes's \citey[\sects~4.3--4.4]{jaynes1994_r2003} insightful
remarks about such binary comparisons, and also Good's
\citey[\sect~6.3--6.6]{good1950}.


\bigskip

If our problem is to finally choose a hypothesis, discarding its
competitors for future calculations, or more generally to make a decision
(for example, choice of medical treatment) based on the observed data, the
post-data belief~\eqref{eq:posterior_hypothesis} is necessary but not
sufficient. We also need to specify a utility or cost function to calculate
the expected gains of choosing one or another hypothesis or making one or
another decision
\citep{kadaneetal1980b,degroot1970_r2004}[\chap~2]{bernardoetal1994_r2000}.

If our problem has an exploratory nature instead -- for example, evaluating
which hypotheses to include in our simplified set, or examining whether a
hypothesis leads to peculiar beliefs for peculiar kinds of data -- then all
terms appearing in expression~\eqref{eq:posterior_hypothesis} are usually
freely examined. In particular the term $\p(D \| H_{h} \, \yK)$, called the
\emph{likelihood} of the hypothesis given the data \citep[\sect~6.1
p.~62]{good1950}, or its logarithm
\begin{equation}
  \label{eq:log-likelihood}
  \log\p(D \| H_{h} \, \yK).
\end{equation}


The likelihoods of several hypotheses are often compared, through their
ratios for example, called \emph{relative Bayes factor}, or its logarithm,
called \emph{relative weight of evidence}
\citep[\chap~6]{good1950}{good1975,good1981,good1985}[and many other works
in][]{good1983}[\sect~1.4]{osteyeeetal1974}{mackay1992,kassetal1995}[see
also][\chaps~V, VI, A]{jeffreys1939_r1983}. \enquote{It is historically
  interesting that the expression ``weight of evidence'', in its technical
  sense, anticipated the term ``likelihood'' by over forty years}
\citep[\sect~1.4.2 p.~12]{osteyeeetal1974}.
\begin{innote}
  Recent literature \citep[for example][]{kassetal1995} seems to
  exclusively deal with \emph{relative} Bayes factors, so I'd like to point
  out that the non-relative Bayes factor for a hypothesis $H_{h}$ provided
  by data $D$ is actually defined as \citep[\sect~2]{good1981}
  \begin{equation}
    \label{eq:proper_Bayes_factor}
    \frac{\p(D \| H_{h} \; \yK)}{\p(D \| \lnot H_{h} \; \yK)} \equiv
    \frac{\yO(H_{h} \| D \; \yK)}{\yO(H_{h} \| \yK)} =
    \frac{\p(D \| H_{h}\; \yK)\; [1- \p(H_{h} \| \yK)]}{
\sum_{h'}^{h' \ne h} \p(D \| H_{h'} \; \yK) \; \p(H_{h'} \| \yK)
    },
  \end{equation}
  where the \emph{odds} $\yO$ is defined as $\yO \defd \p/(1-\p)$. Looking
  at the expression on the right, which can be derived from the probability
  rules, it's clear that the Bayes factor for a hypothesis involves the
  likelihoods of \emph{all} other hypotheses as well as their pre-data
  probabilities. This quantity and its logarithm, the (non-relative) weight
  of evidence, have important properties which \emph{relative} Bayes factors
  don't enjoy. For example, the expected weight of evidence for a correct
  hypothesis is always positive, and for a wrong hypotheses always negative
  \citep[\sect~6.7]{good1950}. See Jaynes
  \citey[\sects~4.3--4.4]{jaynes1994_r2003} for further discussion and a
  numeric example.
\end{innote}

\bigskip

The literature in probability and statistics has also employed various
other ad-hoc measures to make exploratory analyses. Here I consider one in
particular: the \emph{leave-one-out cross-validation log-score}, which I'll
just call \enquote{log-score} for brevity:
\begin{equation}
  \label{eq:log-score}
  \frac{1}{d} \sum_{i=1}^{d} \log\p(D_{i} \| D_{-i} \, H_{h} \, \yK)
\end{equation}
where every $D_{i}$ is one datum in the data $D \equiv \Land_{i} D_{i}$,
and $D_{-i}$ denotes the data with datum $D_{i}$ excluded. The intuition
behind this score, cursorily speaking, is this: \enquote{let's see what my
  belief in one datum should be, on average, once I've observed the other
  data, if I consider $H_{h}$ as true}. \enquote{On average} means
considering such belief for every single datum in turn, and then taking the
geometric mean, which is the arithmetic mean on a log scale.

An ample literature discusses the properties and use of the log-score
\citep[for
example][]{geisseretal1979,vehtarietal2002,vehtarietal2012,krnjajicetal2011,krnjajicetal2014,gelmanetal2014,piironenetal2017,gronauetal2019,chandramoulietal2019}.
Gelman \etal\ \citey{gelmanetal2014} show among other things that it's
approximately equal to expected value of the post-data log-probability of a
new datum:
\begin{equation}
  \label{eq:logscore_approx_logposterior}
  \E\condt[\Big]{\log\p(D' \| D\; H_{h} \; \yK)}{ D\; H_{h} \; \yK},
\end{equation}
where $D'$ represents the new datum. Krnjaji\'c \etal\
\citey{krnjajicetal2011,krnjajicetal2014} numerically compare it with the
deviance information criterion. Regarding its uses I recommend reading the
recent debate among Gronau \etal\ \parentext{\cite*{gronauetal2019} \amp\
  \cite{chandramoulietal2019}}, where all authors give very insightful
remarks.
%  (There's an ambiguity in this
% definition, because we can ask: what's a \enquote{datum} in the case of
% multi-dimensional observations? a single numerical value? or a
% multidimensional point? Different interpretations lead to different
% log-scores. We'll come back to this point later.)

% This is a reasonable intuition, and the log-score~\eqref{eq:log-score} and
% post-data probability~\eqref{eq:posterior_hypothesis} often lead to
% qualitatively similar results in comparing two hypotheses. There are
% exceptions, though.

\bigskip

I now show an exact relation between the log-score~\eqref{eq:log-score} and
the log-likelihood~\eqref{eq:log-likelihood} which doesn't seem to appear
in the literature. I find this relation very intriguing: it says that
\emph{the log-likelihood is the sum of all averaged log-scores that can be formed
  from all data subsets}.

% My point of view, which hinges on the logical foundations of the
% probability calculus
% \citep{polya1939_t1941,polya1949,polya1954b_r1968,cox1946,hailperin1996,jaynes1994_r2003,paris1994_r2006,snow1998,tereninetal2015_r2017},
% is that every intuitively built quantitative assessment of belief is either
% \begin{enumerate*}[label=(\arabic*)]
% \item an approximation of a formula that can be derived from the
%   probability calculus, or
% \item wrong.
% \end{enumerate*}

% I shall now show that the log-score above can be viewed as an approximation
% of the log-likelihood $\p(D \| H_{h} \, \yK)$ of the post-data
% probability~\eqref{eq:posterior_hypothesis}; or, if you like, that the
% post-data probability can be seen as a refined version of the log-score.

\bigskip

We can obviously write the likelihood as the $d$th root of its $d$th power:
\begin{equation}
  \label{eq:root_product}
  \p(D \| H \, \yK) \equiv  \bigl[\,
  \underbracket{\p(D \| H \, \yK) \times \dotsm \times
  \p(D \| H \, \yK)}_{\text{$d$ times}}
  \,\bigr]^{1/d}
\end{equation}
where we have dropped the subscript ${}_{h}$ for simplicity. By the rules
of probability we have
\begin{equation}
  \label{eq:product_rule}
  \p(D \| H \, \yK) =
  \p(D_{i} \| D_{-i} \, H_{h} \, \yK) \times \p(D_{-i} \|  H_{h} \, \yK)
\end{equation}
no matter which specific $i \in \set{1, \dotsc, d}$ we choose (temporal
ordering and similar matters are completely irrelevant in the formula
above: it's a logical relation between propositions). So let's expand each
of the $d$ factors in the identity~\eqref{eq:root_product} using the
product rule~\eqref{eq:product_rule}, using a different $i$ for each of
them. The result can be thus displayed:
\begin{equation}
  \label{eq:product_2}
  \begin{aligned}
    \p(D \| H \, \yK) \equiv{}
    \bigl[\,&\p(D_{1} \| D_{-1} \, H \, \yK) \times
            \p(D_{-1} \|  H \, \yK) \times{}\\
          &\p(D_{2} \| D_{-2} \, H \, \yK) \times
            \p(D_{-2} \| H \, \yK)\times{}\\
          &\hphantom{\p(D_{2} \| D_{-2} \, H \, \yK)}
            \mathrlap{\dotso}\hphantom{{}\times  \p(D_{-2} \| H \, \yK)}\times{}\\
          &\p(D_{d} \|\underbracket[0pt][1ex]{ D_{-d} }_{\mathclap{%
              \substack{\Big\uparrow\\\text{this column contributes to the log-score}}%
}} H \, \yK) \times\p(D_{-d} \|  H \, \yK)
            \,\bigr]^{1/d}.
  \end{aligned}
\end{equation}
Upon taking the logarithm of this expression, the $d$ factors vertically
aligned on the left add up to the log-score~\eqref{eq:log-score}, as
indicated. But the mathematical reshaping we just did for
$\p(D \| H \, \yK)$ -- that is, the root-product
identity~\eqref{eq:root_product} and the expansion~\eqref{eq:product_2} --
can be done for each of the remaining factors $\p(D_{-i} \| H \, \yK)$
vertically aligned on the right in expression~\eqref{eq:product_2}; and so
on recursively. Here is an explicit example for $d=3$:
\begin{multline}
  \label{eq:example_further_expansion}
  \p(D \| H \, \yK) \equiv{}\\
  \begin{alignedat}[b]{2}
    \Bigl\{\,&\p(D_{1} \| D_{2} \,D_{3} \, H \, \yK) \times
    \bigl[ &&\p(D_{2} \| D_{3} \, H \, \yK) \times \p( D_{3} \| H \, \yK) \times{}\\[-0.5\jot]
    &&&\p(D_{3} \| D_{2} \, H \, \yK) \times \p( D_{2} \| H \, \yK)
    \bigr]^{1/2}\times{}\\[\jot]
 %
    &\p(D_{2} \| D_{1} \,D_{3} \, H \, \yK) \times
    \bigl[&&\p(D_{1} \| D_{3} \, H \, \yK) \times \p( D_{3} \| H \, \yK) \times{}\\[-0.5\jot]
    &&&\p(D_{3} \| D_{1} \, H \, \yK) \times \p( D_{1} \| H \, \yK)
    \bigr]^{1/2}\times{}\\[\jot]
 %
    &\p(D_{3} \| D_{1} \,D_{2} \, H \, \yK) \times
    \bigl[&&\p(D_{1} \| D_{2} \, H \, \yK) \times \p( D_{2} \| H \, \yK) \times{}\\[-0.5\jot]
    &&&\p(D_{2} \| D_{1} \, H \, \yK) \times \p( D_{1} \| H \, \yK)
    \bigr]^{1/2}\Bigr\}^{1/3}.
  \end{alignedat}
\end{multline}
In this example, the logarithm of the three vertically aligned factors in
the left column is, as already noted, the log-score~\eqref{eq:log-score}.
The logarithm of the six vertically aligned factors in the central column
is an average of the log-scores calculated for the three distinct subsets
of pairs of data $\set{D_{1}\,D_{2}}$, $\set{D_{1}\, D_{3}}$,
$\set{D_{2}\, D_{3}}$. Likewise, the logarithm of the six factors
vertically aligned on the right is the average of the log-scores for the
three subsets of data singletons $\set{D_{1}}$, $\set{D_{2}}$,
$\set{D_{3}}$.


In the general case with $d$ data there are $\binom{d}{k}$ subsets with $k$
data points. We therefore obtain
\begin{multline}
  \label{eq:general_equivalence}
  \log\p(D \| H \, \yK) \equiv
  \frac{1}{d} \sum_{i=1}^{d} \log\p(D_{i} \| D_{-i} \, H \, \yK)
+{}\\
\shoveright{\frac{1}{d} \sum_{i\in\set{1,\dotsc,d}} \frac{1}{d-1} \sum_{j\in\set{1,\dotsc,d}}^{j\ne i}  \!\!\! \log\p(D_{-i,j} \| D_{-i,-j} \, H \, \yK)
  +{}}\\[0.5\jot]
\shoveright{\binom{d}{d-2}^{-1} \!\!\! \sum_{i,j\in\set{1,\dotsc,d}}^{i<j} \frac{1}{d-2} \sum_{k\in\set{1,\dotsc,d}}^{k\ne i,j} \!\!\! \log\p(D_{-i,-j,k} \| D_{-i,-j,-k} \, H \, \yK)
      +{}}\\[\jot]
       {} \dotsb +{}\\
  % \binom{d}{k}^{-1}\frac{1}{k}\quad
  % \sum_{\mathclap{\text{$k$-tuples}}}   \log\p(D_{i_{1}} \| \underbracket{D_{i_{2}} \dotsm D_{i_{k}}}_{\mathclap{\text{order doesn't matter}}} \, H \, \yK)
  % +{}&\\[0.5\jot]
  % \dotsb +{}&\\[0.5\jot]
\shoveright{      \binom{d}{2}^{-1} \!\!\! \sum_{i,j\in\set{1,\dotsc,d}}^{i<j} \frac{1}{2}
%      \sum_{k\in\set{1,\dotsc,d}}^{k\ne i,j}  \!\!\! \log\p(D_{k} \| D_{i,j} \, H \, \yK)
      \bigl[ \log\p(D_{i} \| D_{j} \, H \, \yK) +
      \log\p(D_{j} \| D_{i} \, H \, \yK)\bigr]
  +{}}\\[0.5\jot]
  \frac{1}{d} \sum_{i=1}^{d} \log\p(D_{i} \| H \, \yK),
\end{multline}
which can be compactly written
\begin{mdframed}
  \begin{equation}
    \label{eq:general_equivalence_compact}
    \log\p(D \| H \, \yK) \equiv
    \sum_{k=1}^{d}
    \binom{d}{k}^{-1}
    \!\smashoperator{\sum_{\substack{\text{ordered}\\\text{$k$-tuples}}}} \quad
    \frac{1}{k}
    \quad\smashoperator{\sum_{\substack{\text{cyclic}\\\text{permutations}}}} 
    \log\p(D_{i_{1}} \| D_{i_{2}} \dotsm D_{i_{k}} \, H \, \yK).
  \end{equation}
\end{mdframed}
% The post-data log-probability for $H$ will be equal to this expression,
% plus the pre-data log-probability for $H$, plus a term which is the same for
% all hypotheses.
That is, \emph{the log-likelihood is the sum of all averaged log-scores
  that can be formed from all (non-empty) data subsets with $k$ elements},
the average for the $k$th-order log-scores being over the $\binom{d}{k}$
subsets having the same cardinality $k$.

There's also an equivalent form with a slightly different interpretation:
We take each datum $D_{i}$ in turn and calculate our log-belief in it
conditional on all possible subsets of remaining data, from the empty
subset with no data (term $k=0$), to the only subset $D_{-i}$ with all data
except $D_{i}$ (term $k=d-1$). These log-beliefs are averaged over the
$\binom{d-1}{k}$ subsets having the same cardinality $k$. The result can be
expressed as
\begin{mdframed}
  \begin{equation}
    \label{eq:general_equivalence_alternative}
    \log\p(D \| H \, \yK) \equiv
    \frac{1}{d}\sum_{i=1}^{d}
    \sum_{k=0}^{d-1}
    \binom{d-1}{k}^{-1}
    \smashoperator{\sum_{\substack{\text{ordered}\\\text{$k$-tuples,}\\\text{$i$ excluded}}}}
    \log\p(D_{i} \| D_{i_{1}} \dotsm D_{i_{k}} \, H \, \yK).
  \end{equation}
\end{mdframed}

It's remarkable that the individual log-scores in
expressions~\eqref{eq:general_equivalence_compact} and
\eqref{eq:general_equivalence_alternative} above are computationally
expensive, but their sum results in a less expensive quantity: the
log-likelihood.

I'd like to offer three ways of looking at the
relation~\eqref{eq:general_equivalence} between the log-likelihood and the
log-score.

First, we can see the log-likelihood as a refinement and improvement of the
log-score. The log-likelihood takes into account not only the log-score for
the whole data, but also the log-scores for all possible subsets of data.
Figuratively speaking it examines the relationship between hypothesis and
data locally, globally, and on all intermediate scales.

The second point of view only holds for hypotheses $\hH$ which make any
observed data irrelevant:
\begin{equation}
  \label{eq:irrelevance_data}
  \p(D \| D' \, \hH \, \yK) = \p(D \| \hH \, \yK)
  \qquad\text{if $D' \nRightarrow D$},
\end{equation}
or super-hypotheses $H$ about such hypotheses, leading to exchangeable
joint beliefs:
\begin{equation}
  \label{eq:superhypothesis_irrelevance_data}
  \p(D D' \| H \, \yK) = \sum_{h} \p(D \| \hH_{h} \, H \, \yK)\;
  \p(D' \| \hH_{h} \, H \, \yK)\;
  \p(\hH_{h} \| H \, \yK)
  \qquad\text{if $D' \nRightarrow D$}.
\end{equation}
In either case the log-score can be seen as an approximation of the log-likelihood;
more precisely of the log-likelihood per datum:
\begin{equation}
  \label{eq:logscore_approx_loglh}
  \frac{1}{d} \sum_{i=1}^{d} \log\p(D_{i} \| D_{-i} \, H \, \yK)
  \approx \frac{1}{d} \log\p(D \| H \, \yK).
\end{equation}
This is in fact an exact equality if property~\eqref{eq:irrelevance_data}
holds for $H$.

which lead to
exchangeable beliefs about the data

Second, we can see 
This approximation is only valid

This approximation is reasonable if the amount of data is large with
respect to the dimension of the space of a single datum, because *** (ref
to geisser, stone, gelfandetal)

*** remark that $D\,H$ is a \emph{new} probability model: which of the two
are we assessing? Connection with learning vs non-learning models which I
hope to take up in another work.


*** problems calculation with time-relevant hypotheses


\enquote{we cannot give a universal rule for them beyond the common-sense one, that if anybody does not know what his suggested value is, or whether there is one, he does not know what question he is asking and consequently does not know what his answer means} \citep[\sect~3.1 p.~124 ]{jeffreys1939_r1983}.


*** with similar procedure we can included all k-fold scores.


**



\textcolor{white}{If you find this you can claim a postcard from me.}






% \[
%   \begin{tikzcd}
%       M_{n,n}(\CC) \arrow{r}{R'_{a}(\Hat{U})} & M_{n,n}(\CC)
%     \\
%     L(\mathcal{H}) \arrow{r}{\Hat{U}} \arrow[swap]{d}{R_*}\arrow[swap]{u}{R'_*} & L(\mathcal{H}) \arrow{d}{R_*}\arrow{u}{R'_*} \\
%       M_{n,n}(\CC) \arrow{r}{R_{a}(\Hat{U})} & M_{n,n}(\CC)
%   \end{tikzcd}
% \]

% \[
%   \begin{tikzcd}
%       \CC^n \arrow{r}{R'_*(A)} & \CC^n
%     \\
%     \mathcal{H} \arrow{r}{A} \arrow[swap]{d}{R}\arrow[swap]{u}{R'} & \mathcal{H} \arrow{d}{R}\arrow{u}{R'} \\
%       \CC^n \arrow{r}{R_*(A)} & \CC^n
%   \end{tikzcd}
% \]


% \[
%   \begin{tikzcd}
%     \mathcal{H} \arrow{r}{A} \arrow[swap]{d}{R} & \mathcal{H} \arrow{d}{R} \\
%       \CC^n \arrow{r}{R_*(A)} & \CC^n
%   \end{tikzcd}
% \]

%%\setlength{\intextsep}{0.5ex}% with wrapfigure
%\begin{figure}[p!]%{r}{0.4\linewidth} % with wrapfigure
%  \centering\includegraphics[trim={12ex 0 18ex 0},clip,width=\linewidth]{maxent_saddle.png}\\
%\caption{caption}\label{fig:comparison_a5}
%\end{figure}% exp_family_maxent.nb


%%%%%%%%%%%%%%%%%%%%%%%%%%%%%%%%%%%%%%%%%%%%%%%%%%%%%%%%%%%%%%%%%%%%%%%%%%%%
%%% Acknowledgements
%%%%%%%%%%%%%%%%%%%%%%%%%%%%%%%%%%%%%%%%%%%%%%%%%%%%%%%%%%%%%%%%%%%%%%%%%%%% 
\iffalse
\begin{acknowledgements}
  \ldots to the Kavli Foundation and the Centre of Excellence scheme of the
  Research Council of Norway (Yasser Roudi group) for financial support. To
  Mari \amp\ Miri for continuous encouragement and affection, and to Buster
  Keaton and Saitama for filling life with awe and inspiration. To the
  developers and maintainers of Open Science Framework, \LaTeX, Emacs,
  AUC\TeX, Python, Inkscape, Sci-Hub for making a free and impartial
  scientific exchange possible.
%\rotatebox{15}{P}\rotatebox{5}{I}\rotatebox{-10}{P}\rotatebox{10}{\reflectbox{P}}\rotatebox{-5}{O}.
\sourceatright{\autanet}
\end{acknowledgements}
\fi

%%%%%%%%%%%%%%%%%%%%%%%%%%%%%%%%%%%%%%%%%%%%%%%%%%%%%%%%%%%%%%%%%%%%%%%%%%%%
%%% Appendices
%%%%%%%%%%%%%%%%%%%%%%%%%%%%%%%%%%%%%%%%%%%%%%%%%%%%%%%%%%%%%%%%%%%%%%%%%%%% 
\clearpage
% %\renewcommand*{\appendixpagename}{Appendix}
% %\renewcommand*{\appendixname}{Appendix}
% %\appendixpage
% \appendix

%%%%%%%%%%%%%%%%%%%%%%%%%%%%%%%%%%%%%%%%%%%%%%%%%%%%%%%%%%%%%%%%%%%%%%%%%%%%
%%% Bibliography
%%%%%%%%%%%%%%%%%%%%%%%%%%%%%%%%%%%%%%%%%%%%%%%%%%%%%%%%%%%%%%%%%%%%%%%%%%%% 
\defbibnote{prenote}{{\footnotesize (\enquote{de $X$} is listed under D,
    \enquote{van $X$} under V, and so on, regardless of national
    conventions.)\par}}
% \defbibnote{postnote}{\par\medskip\noindent{\footnotesize% Note:
%     \arxivp \mparcp \philscip \biorxivp}}

\printbibliography[prenote=prenote%,postnote=postnote
]

\end{document}

%%%%%%%%%%%%%%%%%%%%%%%%%%%%%%%%%%%%%%%%%%%%%%%%%%%%%%%%%%%%%%%%%%%%%%%%%%%%
%%% Cut text (won't be compiled)
%%%%%%%%%%%%%%%%%%%%%%%%%%%%%%%%%%%%%%%%%%%%%%%%%%%%%%%%%%%%%%%%%%%%%%%%%%%% 


%%% Local Variables: 
%%% mode: LaTeX
%%% TeX-PDF-mode: t
%%% TeX-master: t
%%% End: 
