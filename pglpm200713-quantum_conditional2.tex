\pdfoutput=1
%% Author: PGL  Porta Mana
%% Created: 2020-07-13T09:04:38+0200
%% Last-Updated: 2020-07-14T16:41:41+0200
%%%%%%%%%%%%%%%%%%%%%%%%%%%%%%%%%%%%%%%%%%%%%%%%%%%%%%%%%%%%%%%%%%%%%%%%%%%%
\newif\ifarxiv
\arxivfalse
\ifarxiv\pdfmapfile{+classico.map}\fi
\newif\ifafour
\afourfalse% true = A4, false = A5
\newif\iftypodisclaim % typographical disclaim on the side
\typodisclaimtrue
\newcommand*{\memfontfamily}{zplx}
\newcommand*{\memfontpack}{newpxtext}
\documentclass[\ifafour a4paper,12pt,\else a5paper,10pt,\fi%extrafontsizes,%
onecolumn,oneside,article,%french,italian,german,swedish,latin,
british%
]{memoir}
\newcommand*{\firstdraft}{13 July 2020}
\newcommand*{\firstpublished}{13 July 2020}
\newcommand*{\updated}{\ifarxiv***\else\today\fi}
\newcommand*{\propertitle}{The rule of conditional probability is valid\\in quantum theory%\\{\large ***}%
}% title uses LARGE; set Large for smaller
\newcommand*{\pdftitle}{The rule of conditional probability is valid in quantum theory}
\newcommand*{\headtitle}{Conditional probability}
\newcommand*{\pdfauthor}{P.G.L.  Porta Mana}
\newcommand*{\headauthor}{Porta Mana}
\newcommand*{\reporthead}{\ifarxiv\else Open Science Framework \href{https://doi.org/10.31219/osf.io/bsnh7}{\textsc{doi}:10.31219/osf.io/bsnh7}\fi}% Report number

%%%%%%%%%%%%%%%%%%%%%%%%%%%%%%%%%%%%%%%%%%%%%%%%%%%%%%%%%%%%%%%%%%%%%%%%%%%%
%%% Calls to packages (uncomment as needed)
%%%%%%%%%%%%%%%%%%%%%%%%%%%%%%%%%%%%%%%%%%%%%%%%%%%%%%%%%%%%%%%%%%%%%%%%%%%%

%\usepackage{pifont}

%\usepackage{fontawesome}

\usepackage[T1]{fontenc} 
\input{glyphtounicode} \pdfgentounicode=1

\usepackage[utf8]{inputenx}

%\usepackage{newunicodechar}
% \newunicodechar{Ĕ}{\u{E}}
% \newunicodechar{ĕ}{\u{e}}
% \newunicodechar{Ĭ}{\u{I}}
% \newunicodechar{ĭ}{\u{\i}}
% \newunicodechar{Ŏ}{\u{O}}
% \newunicodechar{ŏ}{\u{o}}
% \newunicodechar{Ŭ}{\u{U}}
% \newunicodechar{ŭ}{\u{u}}
% \newunicodechar{Ā}{\=A}
% \newunicodechar{ā}{\=a}
% \newunicodechar{Ē}{\=E}
% \newunicodechar{ē}{\=e}
% \newunicodechar{Ī}{\=I}
% \newunicodechar{ī}{\={\i}}
% \newunicodechar{Ō}{\=O}
% \newunicodechar{ō}{\=o}
% \newunicodechar{Ū}{\=U}
% \newunicodechar{ū}{\=u}
% \newunicodechar{Ȳ}{\=Y}
% \newunicodechar{ȳ}{\=y}

\newcommand*{\bmmax}{0} % reduce number of bold fonts, before font packages
\newcommand*{\hmmax}{0} % reduce number of heavy fonts, before font packages

\usepackage{textcomp}

%\usepackage[normalem]{ulem}% package for underlining
% \makeatletter
% \def\ssout{\bgroup \ULdepth=-.35ex%\UL@setULdepth
%  \markoverwith{\lower\ULdepth\hbox
%    {\kern-.03em\vbox{\hrule width.2em\kern1.2\p@\hrule}\kern-.03em}}%
%  \ULon}
% \makeatother

\usepackage{amsmath}

\usepackage{mathtools}
%\addtolength{\jot}{\jot} % increase spacing in multiline formulae
\setlength{\multlinegap}{0pt}

%\usepackage{empheq}% automatically calls amsmath and mathtools
%\newcommand*{\widefbox}[1]{\fbox{\hspace{1em}#1\hspace{1em}}}

%%%% empheq above seems more versatile than these:
%\usepackage{fancybox}
%\usepackage{framed}

% \usepackage[misc]{ifsym} % for dice
% \newcommand*{\diceone}{{\scriptsize\Cube{1}}}

\usepackage{amssymb}

\usepackage{amsxtra}

\usepackage[main=british,french,italian,german,swedish,latin,esperanto]{babel}\selectlanguage{british}
\newcommand*{\langfrench}{\foreignlanguage{french}}
\newcommand*{\langgerman}{\foreignlanguage{german}}
\newcommand*{\langitalian}{\foreignlanguage{italian}}
\newcommand*{\langswedish}{\foreignlanguage{swedish}}
\newcommand*{\langlatin}{\foreignlanguage{latin}}
\newcommand*{\langnohyph}{\foreignlanguage{nohyphenation}}

\usepackage[autostyle=false,autopunct=false,english=british]{csquotes}
\setquotestyle{american}
\newcommand*{\defquote}[1]{`\,#1\,'}

\usepackage{amsthm}
\newcommand*{\QED}{\textsc{q.e.d.}}
\renewcommand*{\qedsymbol}{\QED}
\theoremstyle{remark}
\newtheorem{note}{Note}
\newtheorem*{remark}{Note}
\newtheoremstyle{innote}{\parsep}{\parsep}{\footnotesize}{}{}{}{0pt}{}
\theoremstyle{innote}
\newtheorem*{innote}{}

\usepackage[shortlabels,inline]{enumitem}
\SetEnumitemKey{para}{itemindent=\parindent,leftmargin=0pt,listparindent=\parindent,parsep=0pt,itemsep=\topsep}
% \begin{asparaenum} = \begin{enumerate}[para]
% \begin{inparaenum} = \begin{enumerate*}
\setlist{itemsep=0pt,topsep=\parsep}
\setlist[enumerate,2]{label=\alph*.}
\setlist[enumerate]{label=\arabic*.,leftmargin=1.5\parindent}
\setlist[itemize]{leftmargin=1.5\parindent}
\setlist[description]{leftmargin=1.5\parindent}
% old alternative:
% \setlist[enumerate,2]{label=\alph*.}
% \setlist[enumerate]{leftmargin=\parindent}
% \setlist[itemize]{leftmargin=\parindent}
% \setlist[description]{leftmargin=\parindent}

\usepackage[babel,theoremfont,largesc]{newpxtext}

\usepackage[bigdelims,nosymbolsc%,smallerops % probably arXiv doesn't have it
]{newpxmath}
%\useosf
%\linespread{1.083}%
%\linespread{1.05}% widely used
\linespread{1.1}% best for text with maths
%% smaller operators for old version of newpxmath
\makeatletter
\def\re@DeclareMathSymbol#1#2#3#4{%
    \let#1=\undefined
    \DeclareMathSymbol{#1}{#2}{#3}{#4}}
%\re@DeclareMathSymbol{\bigsqcupop}{\mathop}{largesymbols}{"46}
%\re@DeclareMathSymbol{\bigodotop}{\mathop}{largesymbols}{"4A}
\re@DeclareMathSymbol{\bigoplusop}{\mathop}{largesymbols}{"4C}
\re@DeclareMathSymbol{\bigotimesop}{\mathop}{largesymbols}{"4E}
\re@DeclareMathSymbol{\sumop}{\mathop}{largesymbols}{"50}
\re@DeclareMathSymbol{\prodop}{\mathop}{largesymbols}{"51}
\re@DeclareMathSymbol{\bigcupop}{\mathop}{largesymbols}{"53}
\re@DeclareMathSymbol{\bigcapop}{\mathop}{largesymbols}{"54}
%\re@DeclareMathSymbol{\biguplusop}{\mathop}{largesymbols}{"55}
\re@DeclareMathSymbol{\bigwedgeop}{\mathop}{largesymbols}{"56}
\re@DeclareMathSymbol{\bigveeop}{\mathop}{largesymbols}{"57}
%\re@DeclareMathSymbol{\bigcupdotop}{\mathop}{largesymbols}{"DF}
%\re@DeclareMathSymbol{\bigcapplusop}{\mathop}{largesymbolsPXA}{"00}
%\re@DeclareMathSymbol{\bigsqcupplusop}{\mathop}{largesymbolsPXA}{"02}
%\re@DeclareMathSymbol{\bigsqcapplusop}{\mathop}{largesymbolsPXA}{"04}
%\re@DeclareMathSymbol{\bigsqcapop}{\mathop}{largesymbolsPXA}{"06}
\re@DeclareMathSymbol{\bigtimesop}{\mathop}{largesymbolsPXA}{"10}
%\re@DeclareMathSymbol{\coprodop}{\mathop}{largesymbols}{"60}
%\re@DeclareMathSymbol{\varprod}{\mathop}{largesymbolsPXA}{16}
\makeatother
%%
%% With euler font cursive for Greek letters - the [1] means 100% scaling
\DeclareFontFamily{U}{egreek}{\skewchar\font'177}%
\DeclareFontShape{U}{egreek}{m}{n}{<-6>s*[1]eurm5 <6-8>s*[1]eurm7 <8->s*[1]eurm10}{}%
\DeclareFontShape{U}{egreek}{m}{it}{<->s*[1]eurmo10}{}%
\DeclareFontShape{U}{egreek}{b}{n}{<-6>s*[1]eurb5 <6-8>s*[1]eurb7 <8->s*[1]eurb10}{}%
\DeclareFontShape{U}{egreek}{b}{it}{<->s*[1]eurbo10}{}%
\DeclareSymbolFont{egreeki}{U}{egreek}{m}{it}%
\SetSymbolFont{egreeki}{bold}{U}{egreek}{b}{it}% from the amsfonts package
\DeclareSymbolFont{egreekr}{U}{egreek}{m}{n}%
\SetSymbolFont{egreekr}{bold}{U}{egreek}{b}{n}% from the amsfonts package
% Take also \sum, \prod, \coprod symbols from Euler fonts
\DeclareFontFamily{U}{egreekx}{\skewchar\font'177}
\DeclareFontShape{U}{egreekx}{m}{n}{%
       <-7.5>s*[0.9]euex7%
    <7.5-8.5>s*[0.9]euex8%
    <8.5-9.5>s*[0.9]euex9%
    <9.5->s*[0.9]euex10%
}{}
\DeclareSymbolFont{egreekx}{U}{egreekx}{m}{n}
\DeclareMathSymbol{\sumop}{\mathop}{egreekx}{"50}
\DeclareMathSymbol{\prodop}{\mathop}{egreekx}{"51}
\DeclareMathSymbol{\coprodop}{\mathop}{egreekx}{"60}
\makeatletter
\def\sum{\DOTSI\sumop\slimits@}
\def\prod{\DOTSI\prodop\slimits@}
\def\coprod{\DOTSI\coprodop\slimits@}
\makeatother
\input{definegreek.tex}% Greek letters not usually given in LaTeX.

%\usepackage%[scaled=0.9]%
%{classico}%  Optima as sans-serif font
\renewcommand\sfdefault{uop}
\DeclareMathAlphabet{\mathsf}  {T1}{\sfdefault}{m}{sl}
\SetMathAlphabet{\mathsf}{bold}{T1}{\sfdefault}{b}{sl}
%\newcommand*{\mathte}[1]{\textbf{\textit{\textsf{#1}}}}
% Upright sans-serif math alphabet
% \DeclareMathAlphabet{\mathsu}  {T1}{\sfdefault}{m}{n}
% \SetMathAlphabet{\mathsu}{bold}{T1}{\sfdefault}{b}{n}

% DejaVu Mono as typewriter text
\usepackage[scaled=0.84]{DejaVuSansMono}

\usepackage{mathdots}

\usepackage[usenames]{xcolor}
% Tol (2012) colour-blind-, print-, screen-friendly colours, alternative scheme; Munsell terminology
\definecolor{mypurpleblue}{RGB}{68,119,170}
\definecolor{myblue}{RGB}{102,204,238}
\definecolor{mygreen}{RGB}{34,136,51}
\definecolor{myyellow}{RGB}{204,187,68}
\definecolor{myred}{RGB}{238,102,119}
\definecolor{myredpurple}{RGB}{170,51,119}
\definecolor{mygrey}{RGB}{187,187,187}
% Tol (2012) colour-blind-, print-, screen-friendly colours; Munsell terminology
% \definecolor{lbpurple}{RGB}{51,34,136}
% \definecolor{lblue}{RGB}{136,204,238}
% \definecolor{lbgreen}{RGB}{68,170,153}
% \definecolor{lgreen}{RGB}{17,119,51}
% \definecolor{lgyellow}{RGB}{153,153,51}
% \definecolor{lyellow}{RGB}{221,204,119}
% \definecolor{lred}{RGB}{204,102,119}
% \definecolor{lpred}{RGB}{136,34,85}
% \definecolor{lrpurple}{RGB}{170,68,153}
\definecolor{lgrey}{RGB}{221,221,221}
%\newcommand*\mycolourbox[1]{%
%\colorbox{mygrey}{\hspace{1em}#1\hspace{1em}}}
\colorlet{shadecolor}{lgrey}

\usepackage{bm}

\usepackage{microtype}

\usepackage[backend=biber,mcite,%subentry,
citestyle=authoryear-ibid,bibstyle=pglpm-authoryear,autopunct=false,sorting=ny,sortcites=false,natbib=false,maxcitenames=2,maxbibnames=8,minbibnames=8,giveninits=true,uniquename=false,uniquelist=false,maxalphanames=1,block=space,hyperref=true,defernumbers=false,useprefix=true,sortupper=false,language=british,parentracker=false]{biblatex}
\DeclareSortingScheme{ny}{\sort{\field{sortname}\field{author}\field{editor}}\sort{\field{year}}}
\iffalse\makeatletter%%% replace parenthesis with brackets
\newrobustcmd*{\parentexttrack}[1]{%
  \begingroup
  \blx@blxinit
  \blx@setsfcodes
  \blx@bibopenparen#1\blx@bibcloseparen
  \endgroup}
\AtEveryCite{%
  \let\parentext=\parentexttrack%
  \let\bibopenparen=\bibopenbracket%
  \let\bibcloseparen=\bibclosebracket}
\makeatother\fi
\DefineBibliographyExtras{british}{\def\finalandcomma{\addcomma}}
\renewcommand*{\finalnamedelim}{\addspace\amp\space}
%\renewcommand*{\finalnamedelim}{\addcomma\space}
\setcounter{biburlnumpenalty}{1}
\setcounter{biburlucpenalty}{0}
\setcounter{biburllcpenalty}{1}
\DeclareDelimFormat{multicitedelim}{\addsemicolon\addspace\space}
\DeclareDelimFormat{compcitedelim}{\addsemicolon\addspace\space}
\DeclareDelimFormat{postnotedelim}{\addspace}
\ifarxiv\else\addbibresource{portamanabib.bib}\fi
\renewcommand{\bibfont}{\footnotesize}
%\appto{\citesetup}{\footnotesize}% smaller font for citations
\defbibheading{bibliography}[\bibname]{\section*{#1}\addcontentsline{toc}{section}{#1}%\markboth{#1}{#1}
}
\newcommand*{\citep}{\footcites}
\newcommand*{\citey}{\parencites*}
\newcommand*{\ibid}{\unspace\addtocounter{footnote}{-1}\footnotemark{}}
%\renewcommand*{\cite}{\parencite}
%\renewcommand*{\cites}{\parencites}
\providecommand{\href}[2]{#2}
\providecommand{\eprint}[2]{\texttt{\href{#1}{#2}}}
\newcommand*{\amp}{\&}
% \newcommand*{\citein}[2][]{\textnormal{\textcite[#1]{#2}}%\addtocategory{extras}{#2}
% }
\newcommand*{\citein}[2][]{\textnormal{\textcite[#1]{#2}}%\addtocategory{extras}{#2}
}
\newcommand*{\citebi}[2][]{\textcite[#1]{#2}%\addtocategory{extras}{#2}
}
\newcommand*{\subtitleproc}[1]{}
\newcommand*{\chapb}{ch.}
%
% \def\arxivp{}
% \def\mparcp{}
% \def\philscip{}
% \def\biorxivp{}
% \newcommand*{\arxivsi}{\texttt{arXiv} eprints available at \url{http://arxiv.org/}.\\}
% \newcommand*{\mparcsi}{\texttt{mp\_arc} eprints available at \url{http://www.ma.utexas.edu/mp_arc/}.\\}
% \newcommand*{\philscisi}{\texttt{philsci} eprints available at \url{http://philsci-archive.pitt.edu/}.\\}
% \newcommand*{\biorxivsi}{\texttt{bioRxiv} eprints available at \url{http://biorxiv.org/}.\\}
\newcommand*{\arxiveprint}[1]{%\global\def\arxivp{\arxivsi}%\citeauthor{0arxivcite}\addtocategory{ifarchcit}{0arxivcite}%eprint
\texttt{arXiv:\urlalt{https://arxiv.org/abs/#1}{#1}}%
%\texttt{\href{http://arxiv.org/abs/#1}{\protect\url{arXiv:#1}}}%
%\renewcommand{\arxivnote}{\texttt{arXiv} eprints available at \url{http://arxiv.org/}.}
}
\newcommand*{\mparceprint}[1]{%\global\def\mparcp{\mparcsi}%\citeauthor{0mparccite}\addtocategory{ifarchcit}{0mparccite}%eprint
\texttt{mp\_arc:\urlalt{http://www.ma.utexas.edu/mp_arc-bin/mpa?yn=#1}{#1}}%
%\texttt{\href{http://www.ma.utexas.edu/mp_arc-bin/mpa?yn=#1}{\protect\url{mp_arc:#1}}}%
%\providecommand{\mparcnote}{\texttt{mp_arc} eprints available at \url{http://www.ma.utexas.edu/mp_arc/}.}
}
\newcommand*{\haleprint}[1]{%\global\def\arxivp{\arxivsi}%\citeauthor{0arxivcite}\addtocategory{ifarchcit}{0arxivcite}%eprint
\texttt{HAL:\urlalt{https://hal.archives-ouvertes.fr/#1}{#1}}%
%\texttt{\href{http://arxiv.org/abs/#1}{\protect\url{arXiv:#1}}}%
%\renewcommand{\arxivnote}{\texttt{arXiv} eprints available at \url{http://arxiv.org/}.}
}
\newcommand*{\philscieprint}[1]{%\global\def\philscip{\philscisi}%\citeauthor{0philscicite}\addtocategory{ifarchcit}{0philscicite}%eprint
\texttt{PhilSci:\urlalt{http://philsci-archive.pitt.edu/archive/#1}{#1}}%
%\texttt{\href{http://philsci-archive.pitt.edu/archive/#1}{\protect\url{PhilSci:#1}}}%
%\providecommand{\mparcnote}{\texttt{philsci} eprints available at \url{http://philsci-archive.pitt.edu/}.}
}
\newcommand*{\biorxiveprint}[1]{%\global\def\biorxivp{\biorxivsi}%\citeauthor{0arxivcite}\addtocategory{ifarchcit}{0arxivcite}%eprint
bioRxiv \texttt{doi:\urlalt{https://doi.org/10.1101/#1}{10.1101/#1}}%
%\texttt{\href{http://arxiv.org/abs/#1}{\protect\url{arXiv:#1}}}%
%\renewcommand{\arxivnote}{\texttt{arXiv} eprints available at \url{http://arxiv.org/}.}
}
\newcommand*{\osfeprint}[1]{%
Open Science Framework \texttt{doi:\urlalt{https://doi.org/10.17605/osf.io/#1}{10.17605/osf.io/#1}}%
}

\usepackage{graphicx}

%\usepackage{wrapfig}

%\usepackage{tikz-cd}

\PassOptionsToPackage{hyphens}{url}\usepackage[hypertexnames=false]{hyperref}

\usepackage[depth=4]{bookmark}
\hypersetup{colorlinks=true,bookmarksnumbered,pdfborder={0 0 0.25},citebordercolor={0.2667 0.4667 0.6667},citecolor=mypurpleblue,linkbordercolor={0.6667 0.2 0.4667},linkcolor=myredpurple,urlbordercolor={0.1333 0.5333 0.2},urlcolor=mygreen,breaklinks=true,pdftitle={\pdftitle},pdfauthor={\pdfauthor}}
% \usepackage[vertfit=local]{breakurl}% only for arXiv
\providecommand*{\urlalt}{\href}

\usepackage[british]{datetime2}
\DTMnewdatestyle{mydate}%
{% definitions
\renewcommand*{\DTMdisplaydate}[4]{%
\number##3\ \DTMenglishmonthname{##2} ##1}%
\renewcommand*{\DTMDisplaydate}{\DTMdisplaydate}%
}
\DTMsetdatestyle{mydate}

%%%%%%%%%%%%%%%%%%%%%%%%%%%%%%%%%%%%%%%%%%%%%%%%%%%%%%%%%%%%%%%%%%%%%%%%%%%%
%%% Layout. I do not know on which kind of paper the reader will print the
%%% paper on (A4? letter? one-sided? double-sided?). So I choose A5, which
%%% provides a good layout for reading on screen and save paper if printed
%%% two pages per sheet. Average length line is 66 characters and page
%%% numbers are centred.
%%%%%%%%%%%%%%%%%%%%%%%%%%%%%%%%%%%%%%%%%%%%%%%%%%%%%%%%%%%%%%%%%%%%%%%%%%%%
\ifafour\setstocksize{297mm}{210mm}%{*}% A4
\else\setstocksize{210mm}{5.5in}%{*}% 210x139.7
\fi
\settrimmedsize{\stockheight}{\stockwidth}{*}
\setlxvchars[\normalfont] %313.3632pt for a 66-characters line
\setxlvchars[\normalfont]
\setlength{\trimtop}{0pt}
\setlength{\trimedge}{\stockwidth}
\addtolength{\trimedge}{-\paperwidth}
% The length of the normalsize alphabet is 133.05988pt - 10 pt = 26.1408pc
% The length of the normalsize alphabet is 159.6719pt - 12pt = 30.3586pc
% Bringhurst gives 32pc as boundary optimal with 69 ch per line
% The length of the normalsize alphabet is 191.60612pt - 14pt = 35.8634pc
\ifafour\settypeblocksize{*}{32pc}{1.618} % A4
%\setulmargins{*}{*}{1.667}%gives 5/3 margins % 2 or 1.667
\else\settypeblocksize{*}{26pc}{1.618}% nearer to a 66-line newpx and preserves GR
\fi
\setulmargins{*}{*}{1}%gives equal margins
\setlrmargins{*}{*}{*}
\setheadfoot{\onelineskip}{2.5\onelineskip}
\setheaderspaces{*}{2\onelineskip}{*}
\setmarginnotes{2ex}{10mm}{0pt}
\checkandfixthelayout[nearest]
\fixpdflayout
%%% End layout
%% this fixes missing white spaces
\pdfmapline{+dummy-space <dummy-space.pfb}\pdfinterwordspaceon%

%%% Sectioning
\newcommand*{\asudedication}[1]{%
{\par\centering\textit{#1}\par}}
\newenvironment{acknowledgements}{\section*{Thanks}\addcontentsline{toc}{section}{Thanks}}{\par}
\makeatletter\renewcommand{\appendix}{\par
  \bigskip{\centering
   \interlinepenalty \@M
   \normalfont
   \printchaptertitle{\sffamily\appendixpagename}\par}
  \setcounter{section}{0}%
  \gdef\@chapapp{\appendixname}%
  \gdef\thesection{\@Alph\c@section}%
  \anappendixtrue}\makeatother
\counterwithout{section}{chapter}
\setsecnumformat{\upshape\csname the#1\endcsname\quad}
\setsecheadstyle{\large\bfseries\sffamily%
\centering}
\setsubsecheadstyle{\bfseries\sffamily%
\raggedright}
%\setbeforesecskip{-1.5ex plus 1ex minus .2ex}% plus 1ex minus .2ex}
%\setaftersecskip{1.3ex plus .2ex }% plus 1ex minus .2ex}
%\setsubsubsecheadstyle{\bfseries\sffamily\slshape\raggedright}
%\setbeforesubsecskip{1.25ex plus 1ex minus .2ex }% plus 1ex minus .2ex}
%\setaftersubsecskip{-1em}%{-0.5ex plus .2ex}% plus 1ex minus .2ex}
\setsubsecindent{0pt}%0ex plus 1ex minus .2ex}
\setparaheadstyle{\bfseries\sffamily%
\raggedright}
\setcounter{secnumdepth}{2}
\setlength{\headwidth}{\textwidth}
\newcommand{\addchap}[1]{\chapter*[#1]{#1}\addcontentsline{toc}{chapter}{#1}}
\newcommand{\addsec}[1]{\section*{#1}\addcontentsline{toc}{section}{#1}}
\newcommand{\addsubsec}[1]{\subsection*{#1}\addcontentsline{toc}{subsection}{#1}}
\newcommand{\addpara}[1]{\paragraph*{#1.}\addcontentsline{toc}{subsubsection}{#1}}
\newcommand{\addparap}[1]{\paragraph*{#1}\addcontentsline{toc}{subsubsection}{#1}}

%%% Headers, footers, pagestyle
\copypagestyle{manaart}{plain}
\makeheadrule{manaart}{\headwidth}{0.5\normalrulethickness}
\makeoddhead{manaart}{%
{\footnotesize%\sffamily%
\scshape\headauthor}}{}{{\footnotesize\sffamily%
\headtitle}}
\makeoddfoot{manaart}{}{\thepage}{}
\newcommand*\autanet{\includegraphics[height=\heightof{M}]{autanet.pdf}}
\definecolor{mygray}{gray}{0.333}
\iftypodisclaim%
\ifafour\newcommand\addprintnote{\begin{picture}(0,0)%
\put(245,149){\makebox(0,0){\rotatebox{90}{\tiny\color{mygray}\textsf{This
            document is designed for screen reading and
            two-up printing on A4 or Letter paper}}}}%
\end{picture}}% A4
\else\newcommand\addprintnote{\begin{picture}(0,0)%
\put(176,112){\makebox(0,0){\rotatebox{90}{\tiny\color{mygray}\textsf{This
            document is designed for screen reading and
            two-up printing on A4 or Letter paper}}}}%
\end{picture}}\fi%afourtrue
\makeoddfoot{plain}{}{\makebox[0pt]{\thepage}\addprintnote}{}
\else
\makeoddfoot{plain}{}{\makebox[0pt]{\thepage}}{}
\fi%typodisclaimtrue
\makeoddhead{plain}{\scriptsize\reporthead}{}{}
% \copypagestyle{manainitial}{plain}
% \makeheadrule{manainitial}{\headwidth}{0.5\normalrulethickness}
% \makeoddhead{manainitial}{%
% \footnotesize\sffamily%
% \scshape\headauthor}{}{\footnotesize\sffamily%
% \headtitle}
% \makeoddfoot{manaart}{}{\thepage}{}

\pagestyle{manaart}

\setlength{\droptitle}{-3.9\onelineskip}
\pretitle{\begin{center}\LARGE\sffamily%
\bfseries}
\posttitle{\bigskip\end{center}}

\makeatletter\newcommand*{\atf}{\includegraphics[%trim=1pt 1pt 0pt 0pt,
totalheight=\heightof{@}]{atblack.png}}\makeatother
\providecommand{\affiliation}[1]{\textsl{\textsf{\footnotesize #1}}}
\providecommand{\epost}[1]{\texttt{\footnotesize\textless#1\textgreater}}
\providecommand{\email}[2]{\href{mailto:#1ZZ@#2 ((remove ZZ))}{#1\protect\atf#2}}

\preauthor{\vspace{-0.5\baselineskip}\begin{center}
\normalsize\sffamily%
\lineskip  0.5em}
\postauthor{\par\end{center}}
\predate{\DTMsetdatestyle{mydate}\begin{center}\footnotesize}
\postdate{\end{center}\vspace{-\medskipamount}}

\setfloatadjustment{figure}{\footnotesize}
\captiondelim{\quad}
\captionnamefont{\footnotesize\sffamily%
}
\captiontitlefont{\footnotesize}
%\firmlists*
\midsloppy
% handling orphan/widow lines, memman.pdf
% \clubpenalty=10000
% \widowpenalty=10000
% \raggedbottom
% Downes, memman.pdf
\clubpenalty=9996
\widowpenalty=9999
\brokenpenalty=4991
\predisplaypenalty=10000
\postdisplaypenalty=1549
\displaywidowpenalty=1602
\raggedbottom

\paragraphfootnotes
\setlength{\footmarkwidth}{2ex}
% \threecolumnfootnotes
%\setlength{\footmarksep}{0em}
\footmarkstyle{\textsuperscript{%\color{myred}
\scriptsize\bfseries#1}~}
%\footmarkstyle{\textsuperscript{\color{myred}\scriptsize\bfseries#1}~}
%\footmarkstyle{\textsuperscript{[#1]}~}

\selectlanguage{british}\frenchspacing

%%%%%%%%%%%%%%%%%%%%%%%%%%%%%%%%%%%%%%%%%%%%%%%%%%%%%%%%%%%%%%%%%%%%%%%%%%%%
%%% Paper's details
%%%%%%%%%%%%%%%%%%%%%%%%%%%%%%%%%%%%%%%%%%%%%%%%%%%%%%%%%%%%%%%%%%%%%%%%%%%%
\title{\propertitle}
\author{%
\hspace*{\stretch{1}}%
%% uncomment if additional authors present
% \parbox{0.5\linewidth}%\makebox[0pt][c]%
% {\protect\centering ***\\%
% \footnotesize\epost{\email{***}{***}}}%
% \hspace*{\stretch{1}}%
\parbox{0.75\linewidth}%\makebox[0pt][c]%
{\protect\centering P.G.L.  Porta Mana  \href{https://orcid.org/0000-0002-6070-0784}{\protect\includegraphics[scale=0.16]{orcid_32x32.png}}\\%
\footnotesize Kavli Institute, Trondheim\quad\epost{\email{pgl}{portamana.org}}}%
%% uncomment if additional authors present
% \hspace*{\stretch{1}}%
% \parbox{0.5\linewidth}%\makebox[0pt][c]%
% {\protect\centering ***\\%
% \footnotesize\epost{\email{***}{***}}}%
\hspace*{\stretch{1}}%
}

%\date{Draft of \today\ (first drafted \firstdraft)}
\date{\firstpublished; updated \updated}

%%%%%%%%%%%%%%%%%%%%%%%%%%%%%%%%%%%%%%%%%%%%%%%%%%%%%%%%%%%%%%%%%%%%%%%%%%%%
%%% Macros @@@
%%%%%%%%%%%%%%%%%%%%%%%%%%%%%%%%%%%%%%%%%%%%%%%%%%%%%%%%%%%%%%%%%%%%%%%%%%%%

% Common ones - uncomment as needed
%\providecommand{\nequiv}{\not\equiv}
%\providecommand{\coloneqq}{\mathrel{\mathop:}=}
%\providecommand{\eqqcolon}{=\mathrel{\mathop:}}
%\providecommand{\varprod}{\prod}
\newcommand*{\de}{\partialup}%partial diff
\newcommand*{\pu}{\piup}%constant pi
\newcommand*{\delt}{\deltaup}%Kronecker, Dirac
%\newcommand*{\eps}{\varepsilonup}%Levi-Civita, Heaviside
%\newcommand*{\riem}{\zetaup}%Riemann zeta
%\providecommand{\degree}{\textdegree}% degree
%\newcommand*{\celsius}{\textcelsius}% degree Celsius
%\newcommand*{\micro}{\textmu}% degree Celsius
\newcommand*{\I}{\mathrm{i}}%imaginary unit
\newcommand*{\e}{\mathrm{e}}%Neper
\newcommand*{\di}{\mathrm{d}}%differential
%\newcommand*{\Di}{\mathrm{D}}%capital differential
%\newcommand*{\planckc}{\hslash}
%\newcommand*{\avogn}{N_{\textrm{A}}}
%\newcommand*{\NN}{\bm{\mathrm{N}}}
%\newcommand*{\ZZ}{\bm{\mathrm{Z}}}
%\newcommand*{\QQ}{\bm{\mathrm{Q}}}
\newcommand*{\RR}{\bm{\mathrm{R}}}
%\newcommand*{\CC}{\bm{\mathrm{C}}}
%\newcommand*{\nabl}{\bm{\nabla}}%nabla
%\DeclareMathOperator{\lb}{lb}%base 2 log
%\DeclareMathOperator{\tr}{tr}%trace
%\DeclareMathOperator{\card}{card}%cardinality
%\DeclareMathOperator{\im}{Im}%im part
%\DeclareMathOperator{\re}{Re}%re part
%\DeclareMathOperator{\sgn}{sgn}%signum
%\DeclareMathOperator{\ent}{ent}%integer less or equal to
%\DeclareMathOperator{\Ord}{O}%same order as
%\DeclareMathOperator{\ord}{o}%lower order than
%\newcommand*{\incr}{\triangle}%finite increment
\newcommand*{\defd}{\coloneqq}
\newcommand*{\defs}{\eqqcolon}
%\newcommand*{\Land}{\bigwedge}
%\newcommand*{\Lor}{\bigvee}
%\newcommand*{\lland}{\DOTSB\;\land\;}
%\newcommand*{\llor}{\DOTSB\;\lor\;}
%\newcommand*{\limplies}{\mathbin{\Rightarrow}}%implies
%\newcommand*{\suchthat}{\mid}%{\mathpunct{|}}%such that (eg in sets)
%\newcommand*{\with}{\colon}%with (list of indices)
%\newcommand*{\mul}{\times}%multiplication
%\newcommand*{\inn}{\cdot}%inner product
%\newcommand*{\dotv}{\mathord{\,\cdot\,}}%variable place
%\newcommand*{\comp}{\circ}%composition of functions
%\newcommand*{\con}{\mathbin{:}}%scal prod of tensors
%\newcommand*{\equi}{\sim}%equivalent to 
\renewcommand*{\asymp}{\simeq}%equivalent to 
%\newcommand*{\corr}{\mathrel{\hat{=}}}%corresponds to
%\providecommand{\varparallel}{\ensuremath{\mathbin{/\mkern-7mu/}}}%parallel (tentative symbol)
\renewcommand*{\le}{\leqslant}%less or equal
\renewcommand*{\ge}{\geqslant}%greater or equal
%\DeclarePairedDelimiter\clcl{[}{]}
%\DeclarePairedDelimiter\clop{[}{[}
%\DeclarePairedDelimiter\opcl{]}{]}
%\DeclarePairedDelimiter\opop{]}{[}
\DeclarePairedDelimiter\abs{\lvert}{\rvert}
%\DeclarePairedDelimiter\norm{\lVert}{\rVert}
\DeclarePairedDelimiter\set{\{}{\}}
%\DeclareMathOperator{\pr}{P}%probability
\newcommand*{\pf}{\mathrm{p}}%probability
\newcommand*{\p}{\mathrm{P}}%probability
%\newcommand*{\E}{\mathrm{E}}
%\renewcommand*{\|}{\nonscript\,\vert\nonscript\;\mathopen{}}
\renewcommand*{\|}[1][]{\nonscript\,#1\vert\nonscript\;\mathopen{}}
%\DeclarePairedDelimiterX{\cond}[2]{(}{)}{#1\nonscript\,\delimsize\vert\nonscript\;\mathopen{}#2}
%\DeclarePairedDelimiterX{\condt}[2]{[}{]}{#1\nonscript\,\delimsize\vert\nonscript\;\mathopen{}#2}
%\DeclarePairedDelimiterX{\conds}[2]{\{}{\}}{#1\nonscript\,\delimsize\vert\nonscript\;\mathopen{}#2}
%\newcommand*{\+}{\lor}
%\renewcommand{\*}{\land}
%% symbol = for equality statements within probabilities
%% from https://tex.stackexchange.com/a/484142/97039
%\newcommand*{\eq}{\mathrel{\!=\!}}
%\let\texteq\=
%\renewcommand*{\=}{\TextOrMath\texteq\eq}
%%
\newcommand*{\sect}{\S}% Sect.~
\newcommand*{\sects}{\S\S}% Sect.~
\newcommand*{\chap}{ch.}%
\newcommand*{\chaps}{chs}%
\newcommand*{\bref}{ref.}%
\newcommand*{\brefs}{refs}%
%\newcommand*{\fn}{fn}%
\newcommand*{\eqn}{eq.}%
\newcommand*{\eqns}{eqs}%
\newcommand*{\fig}{fig.}%
\newcommand*{\figs}{figs}%
\newcommand*{\vs}{{vs}}
\newcommand*{\eg}{{e.g.}}
\newcommand*{\etc}{{etc.}}
\newcommand*{\ie}{{i.e.}}
%\newcommand*{\ca}{{c.}}
\newcommand*{\foll}{{ff.}}
%\newcommand*{\viz}{{viz}}
\newcommand*{\cf}{{cf.}}
%\newcommand*{\Cf}{{Cf.}}
%\newcommand*{\vd}{{v.}}
\newcommand*{\etal}{{et al.}}
%\newcommand*{\etsim}{{et sim.}}
%\newcommand*{\ibid}{{ibid.}}
%\newcommand*{\sic}{{sic}}
%\newcommand*{\id}{\mathte{I}}%id matrix
%\newcommand*{\nbd}{\nobreakdash}%
%\newcommand*{\bd}{\hspace{0pt}}%
%\def\hy{-\penalty0\hskip0pt\relax}
%\newcommand*{\labelbis}[1]{\tag*{(\ref{#1})$_\text{r}$}}
%\newcommand*{\mathbox}[2][.8]{\parbox[t]{#1\columnwidth}{#2}}
%\newcommand*{\zerob}[1]{\makebox[0pt][l]{#1}}
\newcommand*{\tprod}{\mathop{\textstyle\prod}\nolimits}
\newcommand*{\tsum}{\mathop{\textstyle\sum}\nolimits}
%\newcommand*{\tint}{\begingroup\textstyle\int\endgroup\nolimits}
%\newcommand*{\tland}{\mathop{\textstyle\bigwedge}\nolimits}
%\newcommand*{\tlor}{\mathop{\textstyle\bigvee}\nolimits}
%\newcommand*{\sprod}{\mathop{\textstyle\prod}}
%\newcommand*{\ssum}{\mathop{\textstyle\sum}}
%\newcommand*{\sint}{\begingroup\textstyle\int\endgroup}
%\newcommand*{\sland}{\mathop{\textstyle\bigwedge}}
%\newcommand*{\slor}{\mathop{\textstyle\bigvee}}
%\newcommand*{\T}{^\intercal}%transpose
%%\newcommand*{\QEM}%{\textnormal{$\Box$}}%{\ding{167}}
%\newcommand*{\qem}{\leavevmode\unskip\penalty9999 \hbox{}\nobreak\hfill
%\quad\hbox{\QEM}}

%%%%%%%%%%%%%%%%%%%%%%%%%%%%%%%%%%%%%%%%%%%%%%%%%%%%%%%%%%%%%%%%%%%%%%%%%%%%
%%% Custom macros for this file @@@
%%%%%%%%%%%%%%%%%%%%%%%%%%%%%%%%%%%%%%%%%%%%%%%%%%%%%%%%%%%%%%%%%%%%%%%%%%%%
\definecolor{notecolour}{RGB}{68,170,153}
\newcommand*{\puzzle}{{\fontencoding{U}\fontfamily{fontawesometwo}\selectfont\symbol{225}}}
%\newcommand*{\puzzle}{\maltese}
\newcommand{\mynote}[1]{ {\color{notecolour}\puzzle\ #1}}
\newcommand*{\widebar}[1]{{\mkern1.5mu\skew{2}\overline{\mkern-1.5mu#1\mkern-1.5mu}\mkern 1.5mu}}

% \newcommand{\explanation}[4][t]{%\setlength{\tabcolsep}{-1ex}
% %\smash{
% \begin{tabular}[#1]{c}#2\\[0.5\jot]\rule{1pt}{#3}\\#4\end{tabular}}%}
% \newcommand*{\ptext}[1]{\text{\small #1}}
%\DeclareMathOperator*{\argsup}{arg\,sup}
\newcommand*{\dob}{degree of belief}
\newcommand*{\dobs}{degrees of belief}
\newcommand*{\yDa}{D_{\textrm{a}}}
\newcommand*{\yDb}{D_{\textrm{b}}}
\newcommand*{\yDc}{D_{\textrm{c}}}
\newcommand*{\yDd}{D_{\textrm{d}}}
%%% Custom macros end @@@

%%%%%%%%%%%%%%%%%%%%%%%%%%%%%%%%%%%%%%%%%%%%%%%%%%%%%%%%%%%%%%%%%%%%%%%%%%%%
%%% Beginning of document
%%%%%%%%%%%%%%%%%%%%%%%%%%%%%%%%%%%%%%%%%%%%%%%%%%%%%%%%%%%%%%%%%%%%%%%%%%%%
%\firmlists
\begin{document}
\captiondelim{\quad}\captionnamefont{\footnotesize}\captiontitlefont{\footnotesize}
\selectlanguage{british}\frenchspacing
\maketitle

%%%%%%%%%%%%%%%%%%%%%%%%%%%%%%%%%%%%%%%%%%%%%%%%%%%%%%%%%%%%%%%%%%%%%%%%%%%%
%%% Abstract
%%%%%%%%%%%%%%%%%%%%%%%%%%%%%%%%%%%%%%%%%%%%%%%%%%%%%%%%%%%%%%%%%%%%%%%%%%%%
\iffalse\abstractrunin
\abslabeldelim{}
\renewcommand*{\abstractname}{}
\setlength{\absleftindent}{0pt}
\setlength{\absrightindent}{0pt}
\setlength{\abstitleskip}{-\absparindent}
\begin{abstract}\labelsep 0pt%
  \noindent ***
% \\\noindent\emph{\footnotesize Note: Dear Reader
%     \amp\ Peer, this manuscript is being peer-reviewed by you. Thank you.}
% \par%\\[\jot]
% \noindent
% {\footnotesize PACS: ***}\qquad%
% {\footnotesize MSC: ***}%
%\qquad{\footnotesize Keywords: ***}
\end{abstract}\fi
\selectlanguage{british}\frenchspacing

%%%%%%%%%%%%%%%%%%%%%%%%%%%%%%%%%%%%%%%%%%%%%%%%%%%%%%%%%%%%%%%%%%%%%%%%%%%%
%%% Epigraph
%%%%%%%%%%%%%%%%%%%%%%%%%%%%%%%%%%%%%%%%%%%%%%%%%%%%%%%%%%%%%%%%%%%%%%%%%%%%
% \asudedication{\small ***}
% \vspace{\bigskipamount}
% \setlength{\epigraphwidth}{.7\columnwidth}
% %\epigraphposition{flushright}
% \epigraphtextposition{flushright}
% %\epigraphsourceposition{flushright}
% \epigraphfontsize{\footnotesize}
% \setlength{\epigraphrule}{0pt}
% %\setlength{\beforeepigraphskip}{0pt}
% %\setlength{\afterepigraphskip}{0pt}
% \epigraph{\emph{text}}{source}



%%%%%%%%%%%%%%%%%%%%%%%%%%%%%%%%%%%%%%%%%%%%%%%%%%%%%%%%%%%%%%%%%%%%%%%%%%%%
%%% BEGINNING OF MAIN TEXT
%%%%%%%%%%%%%%%%%%%%%%%%%%%%%%%%%%%%%%%%%%%%%%%%%%%%%%%%%%%%%%%%%%%%%%%%%%%%

In a recent manuscript, Gelman \amp\ Yao \citey{gelmanetal2020} claim that
\enquote{the usual rules of conditional probability fail in the quantum
  realm} and purport to support such statement with an example. Such a
statement is false. I would like to recall some literature that shows why
it is false, and to sum up the fallacy underlying their example.

Let me point out at the outset that the rule of conditional probability and
the other two rules (sum and negation) are in fact routinely used in
quantum theory, especially in problems of state \enquote{retrodiction} and
measurement reconstruction
\parencites{jones1991b,slater1995b}[\chaps~7,8]{demuynck2002b}{barnettetal2003,zimanetal2004_r2006,darianoetal2004}[see][\sect~1
for many further references]{maanssonetal2006}, for example to infer the
state of a quantum laser given its output through different optical
apparatus \parencite{leonhardt1997}.

Similar incorrect claims with similar examples have appeared before in the
quantum literature. Bernard O. Koopman \parencite[of the Pitman-Koopman
theorem for sufficient statistics,][]{koopman1936} discussed the
falsity of such claims already in \cite*{koopman1957}. The Introduction in
his work is very clear:
\begin{quotation}\footnotesize
  Ever since the advent of modern quantum mechanics in the late
  1920's, the idea has been prevalent that the classical laws of
  probability cease, in some sense, to be valid in the new theory. More or
  less explicit statements to this effect have been made in large number
  and by many of the most eminent workers in the new physics \textelp{}.
  Some authors have even gone farther and stated that the formal structure
  of logic must be altered to conform to the terms of reference of quantum
  physics \textelp{}.

  Such a thesis is surprising, to say the least, to anyone holding more or
  less conventional views regarding the positions of logic, probability,
  and experimental science: many of us have been apt -- perhaps too naively
  -- to assume that experiments can lead to conclusions only when worked up
  by means of logic and probability, whose laws seem to be on a different
  level from those of physical science.

  The primary object of this presentation is to show that the thesis in
  question is entirely without validity and is the product of a confused
  view of the laws of probability.
\end{quotation}

A claim similar to Gelman \amp\ Yao's, with a similar supporting example,
was made in a work by Brukner \amp\ Zeilinger \citey{brukneretal2001},
somewhat famous in the quantum community; although their focus was on an
alleged inconsistency of some properties of the Shannon entropy in quantum
theory.

The fallacy in the reasoning of Brukner \amp\ Zeilinger and Gelman \amp\
Yao rests in the neglect of the experimental setup, leading to an incorrect
calculation of conditional probabilities. Such fallacy was exposed and
discussed at length by Porta Mana \citey{portamana2003_r2004} through a
step-by-step analysis and calculation. This work also showed, through
simple examples \parencite[\sect~IV]{portamana2003_r2004}, that the same
incorrect statements can be obtained \emph{with completely classical
  systems}, such as drawing from an urn, if the setup is neglected. Here is
a simple example.

Consider an urn with one $B$lue and one $R$ed ball. There are two possible
drawing setups:
\begin{enumerate}[label=$D_{\textrm{\alph*}}$]
\item\label{item:repB} With replacement for blue, without replacement for red. That is, if
  blue is drawn, it is put back before the next draw (and the urn is
  shaken); if red is drawn, it is thrown away before the next draw.
\item \label{item:repR} With replacement for red, without replacement for blue.
\end{enumerate}
The two setups are obviously mutually exclusive.

We can easily calculate what is the unconditional probability for blue at
the second draw in the setup $\yDa$:
\begin{equation}
  \label{eq:A_red2}
  \p(B_{2} \| \yDa) = \tfrac{3}{4} \;,
\end{equation}
and the conditional probabilities for blue at the second draw, conditional
on the first draw, in the setup $\yDb$:
\begin{equation}
  \label{eq:A_red2}
  \p(B_{2} \| B_{1} \land \yDb) = 0 \qquad
  \p(B_{2} \| R_{1} \land \yDb) = \tfrac{1}{2} \;.
\end{equation}
We  find that
\begin{equation}
  \label{eq:cross_combine}
  \p(B_{2} \| \yDa) \mathrel{\bm{\ne}}
  \p(B_{2} \| B_{1} \land \yDb) \; \p( B_{1} \| \yDb) +
  \p(B_{2} \| R_{1} \land \yDb)  \; \p( R_{1} \| \yDb) \;.
\end{equation}
This inequality is no surprising. And it does not contradict the rule of
conditional probability, because that rule is supposed to be used within
the same probability space. You can call the inequality above
\enquote{interference} if you like; for further examples see Kirkpatrick
\citey{kirkpatrick2001_r2003,kirkpatrick2002_r2003} and Porta Mana
\citey[\sect~IV]{portamana2003_r2004}.

Note that, had we considered a drawing setup $\yDc$ with replacement for both
colours, and a setup $\yDd$ without replacement for either colour, we would
have found
\begin{equation}
  \label{eq:cross_combine_same}
  \p(B_{2} \| \yDc) \mathrel{\bm{=}}
  \p(B_{2} \| B_{1} \land \yDd) \; \p( B_{1} \| \yDd) +
  \p(B_{2} \| R_{1} \land \yDd)  \; \p( R_{1} \| \yDd) \;.
\end{equation}
The equality above, however, is \emph{not} and expression of the
conditional-probability rule, because the probability spaces are different.
It is simply a peculiar equality contingent on the two specific setups. The
probability calculus handles correctly situations such
as~\eqref{eq:cross_combine} or \eqref{eq:cross_combine_same}.

In fact, strictly speaking it is wrong to use the expression
\defquote{$B_{2}$} for all these setups, because \defquote{$B_{2}$} in the
one setup denotes a different statement (or random variable) than in
another. Just like \enquote{it rains (on 2020-07-14T09:00+0200 in
  Trondheim)} is different from \enquote{it rains (on 2019-01-20T18:00+0200
  in Rome)}. I should have used different symbols. The explicit presence of
\defquote{$D_{\dotso}$}, which represents given information, luckily
avoided any ambiguities. But if in our formulae we omit the notation of the
setup \emph{and} we use the same notation for actually different statements
or random variables, then we're in for trouble and for incorrect
applications of the probability rules.

\medskip

The example above stresses the importance of the probability space -- even
in \enquote{non-quantum} situations. But it is not meant as a parallel of
Gelman \amp\ Yao's \citey[\sect~2]{gelmanetal2020} example. In fact their
example has important differences and their analysis of it is incorrect
from the point of view of quantum theory. Here are the main points:

\begin{enumerate}[label=(\roman*)]
\item\label{item:q_details} It \emph{does} matter whether many photons are
  sent at once, or one at a time, as well as their wavelength, temporal
  spread, and so on. These details lead to different probabilities
  distributions of detection at the screen
  \citep[\eg][]{mandeletal1965,morganetal1966,paul1982,jacobsonetal1995}[and
  textbooks such
  as][]{loudon1973_r2000,mandeletal1995_r2008,scullyetal1997_r2001,bachoretal1998_r2004,wallsetal1994}.
  More precisely, the spatio-temporal dependence of the optic-field
  operator \parencite[which defines the quantum system:][]{vanenk2003b}
  must be specified.

  All these details are not \enquote{latent variables}: they are the
  initial and boundary conditions that define the physical system. They
  correspond to the different drawing setups in the urn example above -- we
  would not call the specification \enquote{drawing with replacement} a
  latent variable. The rules of probability apply seamlessly in each case.

  It is also possible to consider situations where part of the setup, such
  as slit width or presence or absence of detectors, is unknown. This is
  similar, in the urn example above, to not knowing whether $\yDa$ or
  $\yDb$ applies. In this case one can make inferences by giving the
  probability for each setup, \eg\ $\p(\yDa)$, and applying the
  conditional-probability rule. The same rule is indeed applied in the
  analogous quantum situation \parencite[\eg][]{zimanetal2004_r2006}.
  Again, no violations of the probability rules in the quantum realm.

\item\label{item:q_prob_freq} Owing to the point above, it is important not
  to conflate the \emph{probability} for single-photon detections and the
  \emph{frequency} distribution of a long-run of such detections, as Gelman
  \amp\ Yao \citey[p.~1]{gelmanetal2020} instead do. For example, in some
  setups we can have a detection probability density $\pf(y_{1})$ for the
  first photon, and a \emph{different} density for the second photon
  $\pf(y_{2} \| y_{1})$, conditional on the detection of the first -- both
  being different from the long-run joint density of detections $f(y)$
  \parentext{see \eg\ the phenomena of higher-order coherence and bunching
    in the references above\ibid}. The rules of the probability calculus
  also apply in such situations. We can infer, for example, the position of
  the first photon detection given the second from
  $\pf(y_{1} \| y_{2}) \propto \pf(y_{2} \| y_{1})\;\pf(y_{1})$.
  
\item\label{item:q_slitclosed_vs_detect} The long-run frequency
  distribution for screen detection in the setup with one slit open
  (denoted \defquote{$p_1(y)$} by Gelman \amp\ Yao), and the one
  conditional on slit detection in the setup with detectors at the slits
  (\defquote{$p_{4}(y \| x)$}) are a priori unrelated, since they belong to
  different setups. In some situations (but not always, see points
  \ref{item:q_details}, \ref{item:q_prob_freq} above) they experimentally
  turn out to be equal. But this was not required by the probability
  calculus. Such equality is similar to \eqn~\eqref{eq:cross_combine_same}
  between the with-replacement and without-replacement urn setups, which is
  not an a priori rule of the probability calculus, and in fact does not
  hold for the setups in \eqn~\eqref{eq:cross_combine}.

\item In the setup with one or both slits open, the quantity
  \enquote{detection at slit $x$} \emph{does not exist}, because no such
  measurement is included in the setup. So in these cases the idea of a
  marginal probability does not even apply.

  Of course one is free to introduce such a quantity within some
  hidden-variable theory (which goes beyond the quantum realm), such as
  Bohmian mechanics \parencite[\!; this theory includes as hidden variables
  the positions of the particles and the spatial configuration of a
  so-called pilot wave]{berndletal1995,duerretal2004,valentinietal2005}.
  But then note, first, that such theory gives experimental predictions in
  complete agreement with standard quantum theory; and second, that the
  rules of probability theory are again used without violations in this
  theory (\eg\ the relation between joint and conditional distributions of
  the particles' positions, and so on).
%
% \item There is a continuum of possible experimental setups, which includes
%   as special cases the setups without slit detectors, with only one slit
%   open, and with slit detectors.\ibid The corresponding long-run frequency
%   distributions also vary continuously.
\end{enumerate}

I may add that the idea and parlance of \enquote{photons passing through
  slits} are used today only out of tradition; maybe a little poetically.
The technical parlance, as routinely used in quantum-optics labs for
example \parencite{leonhardt1997,bachoretal1998_r2004}, has a different
underlying picture. The basic \enquote{system} in a quantum-optics
experiment are not photons, but the modes of the field-configuration
operator (note that this is not yet Quantum ElectroDynamics).
\enquote{Photon numbers} denote the discrete outcomes of a specific
energy-measurement operator; \enquote{photon states} denote specific states
of the field operators. As another example, \enquote{entanglement} is
strictly speaking not among photons, but among modes of the field operator
\parencite{vanenk2003b}. Several quantum physicists indeed oppose the idea
and parlance of \enquote{photons}, owing to the confusion it leads to. Lamb
\parencite[of the Lamb shift,][]{lambetal1947} wrote in \cite*{lamb1995}:
\begin{quote}
  \footnotesize the author does not like the use of the word ``photon'',
  which dates from 1926. In his view, there is no such thing as a photon.
  Only a comedy of errors and historical accidents led to its popularity
  among physicists and optical scientists.
\end{quote}
Wald \citey{wald1994} warns:
\begin{quote}
  \footnotesize standard treatments of quantum field theory in flat
  spacetime rely heavily on Poincar\'e symmetry (usually entering the
  analysis implicitly via plane-wave expansions) and interpret the theory
  primarily in terms of a notion of ``particles''. Neither Poincar\'e (or
  other) symmetry nor a useful notion of ``particles'' exists in a general,
  curved spacetime, so a number of the familiar tools and concepts of field
  theory must be ``unlearned'' in order to have a clear grasp of quantum
  field theory in curved spacetime. [p.~ix] \textelp{} the notion of
  ``particles'' plays no fundamental role either in the formulation or
  interpretation of the theory. [p.~2]
\end{quote}



First, in the experiments with only one slit open or both slits open, the outcome
space is $\set{\textrm{\defquote{no event}}} \cup \RR$, because either an
emulsion is produced at some point on the screen, or none is produced.





After all, we do not expect that marginal probabilities from, say, a
drawing-without-replacement urn setup should be obtainable from the joint
distribution of a drawing-with-replacement setup, or of P\'olya drawings.
These setups are mutually exclusive. A random variable of one of them is
not the same random variable of another. If you thought that you could
consistently combine probabilities from such different urn-drawing setups
and find that you actually cannot, well, too bad for you. The probability
calculus, in fact, makes clear at the outset that the probabilities of these
setups cannot generally be combined. It is thus somewhat funny that one
ends up blaming the probability calculus, which makes a clear distinction,
for one's neglect of that distinction.


Likewise, measurement setups
in quantum theory -- and in many classical-physics situations -- are
generally mutually exclusive.

This kind of incorrect claims about

I refer to the work just cited for the full analysis and counterexamples.
Here I summarize the basic fallacy with a simpler counterexample.

\medskip

The basic fallacy is the confusion of the probability conditional with a
temporal ordering.

Take the conditional probability $\p(A \| B)$, where the statement or event
$A$ refers to a time $t_{2}$, and $B$ to a time $t_{1}$ that precedes
$t_{2}$. This probability is related to the reverse conditional
$\p(B \| A)$ by
\begin{equation}
  \label{eq:conditional_prob}
  \p(A \| B)\; \p(B) = \p(B \| A) \; \p(A) = \p(A \land B) \;.
\end{equation}
It goes without saying that \emph{the statements or events $A$ and $B$ must
  be the same on both sides of each equation}. In particular, in the
conditional $\p(B \| A)$ the statement $B$ still refers to the time
$t_{1}$, and $A$ to the time $t_{2}$, with $t_{1}$ preceding $t_{2}$. We
can represent these times explicitly and
rewrite~\eqref{eq:conditional_prob} as
\begin{equation}
  \label{eq:conditional_prob_time}
  \p(A_{2} \| B_{1})\; \p(B_{1}) = \p(B_{1} \| A_{2}) \; \p(A_{2})
  = \p(A_{2} \land B_{1}) \;.
\end{equation}

The fallacy is to think that, in calculating $\p(B \| A)$, we should now
ensure that $B$ refers to time $t_{2}$, and $A$ to time $t_{1}$, \emph{swapping
the times}. But these would be \emph{different} events, not the original
events. In symbols, we would be calculating $\p(B_{2} \| A_{1})$, which is
different from $\p(B_{1} \| A_{2})$, to which the conditional-probability
rule~\eqref{eq:conditional_prob} refers.

Probability theory has nothing to say, a priori, about the relation between
$\p(B_{1} \| A_{2})$ and $\p(B_{2} \| A_{1})$. They are two logically
different situations. A simple example can illustrate this point.

Consider an urn with $R$ed and $B$lue balls. The drawing scheme is with
replacement for blue, and without replacement for red. That is, if a blue
ball is drawn, it is put back in the urn (and the urn shaken) before the
next draw. If a red ball is drawn, it is not put back before the next draw.

The urn initially has one blue and one red ball. The probabilities for the
first draw are straightforward:
\begin{equation}
  \label{eq:1st_draw}
  \p(B_{1}) = \tfrac{1}{2} \qquad \p(R_{1}) = \tfrac{1}{2} \;,
\end{equation}
as are the conditional probabilities for the second draw, conditional on
the first:
\begin{subequations}
  \label{eq:2nd_draw}
  \begin{align}
    \p(B_{2} \|B_{1}) &= \tfrac{1}{2}
    & \p(R_{2} \| R_{1}) &= 0
      \label{eq:2nd_draw_equal}
    \\[\jot]
    \p(R_{2} \|B_{1}) &= \tfrac{1}{2}
    & \p(B_{2} \| R_{1}) &= 1 \;.
      \label{eq:2nd_draw_mixed}
  \end{align}
\end{subequations}

These probabilities can be visualized with the \enquote{equiprobable
  worlds} diagram of \fig~\ref{fig:worlds}. In half of the worlds the first
draw yields red; in the other half, blue, according to~\eqref{eq:1st_draw}.
In all worlds with red first, the second must yield blue. In half of the
worlds with blue first, the second yields blue; and in the other half, red;
as for \eqns~\eqref{eq:2nd_draw}.
\begin{figure}[t]
  \begin{framed}
    \centering
    \caption{\enquote{Equiprobable worlds} diagram}\label{fig:worlds}
    \begin{tabular}{p{5em} p{1em} p{1em} p{1em} p{1em} }
      % &\multicolumn{4}{c}{\!\!\!\!\enquote{equiprobable worlds}}
      % \\[1em]
      {{\footnotesize 1st draw}}& R& R & B & B
      \\[1ex]
        & $\Big\downarrow$& $\Big\downarrow$
                                   & $\Big\downarrow$& $\Big\downarrow$
      \\[2ex]
      {{\footnotesize 2nd draw}}& B& B & B & R
    \end{tabular}
  \end{framed}
\end{figure}


Let us now calculate the conditional probabilities for the first draw
conditional on the second (imagine the first draw was hidden from you, and
upon seeing the second you're asked to guess what the first was). Using
the conditional-probability rule~\eqref{eq:conditional_prob_time} in the form of
Bayes's theorem,
\begin{equation}
  \label{eq:bayes}
  \p(X_{1} \| Y_{2}) = \frac{\p(Y_{2} \| X_{1})\, \p(X_{1})}{
   \sum_{X} \p(Y_{2} \| X_{1})\, \p(X_{1})} \;,
\end{equation}
we obtain
\begin{subequations}
  \label{eq:2nd_draw_reverse}
  \begin{align}
    \p(B_{1} \|B_{2}) &= \tfrac{1}{3}
    & \p(R_{1} \| R_{2}) &= 0
    \\[\jot]
    \p(R_{1} \|B_{2}) &= \tfrac{2}{3}
    & \p(B_{1} \| R_{2}) &= 1 \;.
      \label{eq:2nd_draw__reverse_mixed}
  \end{align}
\end{subequations}
These conditional probabilities are intuitively correct, as can be checked
by simple enumeration of the possible cases in \fig~\ref{fig:worlds}. For
example, among all three worlds that have blue at the second draw, two of
them have red at the first; hence $\p(R_{1} \|B_{2}) = 2/3$. Also, if we
have red at the second draw, then logically red cannot have been drawn at
the first, otherwise there would not have been any red left; hence blue
must have been drawn at the first: $\p(B_{1} \| R_{2}) = 1$.

\medskip

The following two relations between the conditional
probabilities~\eqref{eq:2nd_draw_mixed} and
\eqref{eq:2nd_draw__reverse_mixed} are relevant to our discussion:
\begin{equation}
  \label{eq:compare_conditionals}
  \p(R_{2} \|B_{1}) \mathrel{\bm{\ne}} \p(R_{1} \|B_{2}) \qquad
    \p(B_{2} \| R_{1}) \mathrel{\bm{=}} \p(B_{1} \| R_{2}) \;.
\end{equation}
As we see, the probability-calculus does not prescribe the equality nor the
inequality between probabilities for events of similar kind but at
different times (which therefore are \emph{not} the same event). Any
relations of this kind will depend on the specific situation. You can
check, for example, that in a scheme of drawing with replacement for both
blue and red, we would obtain two equalities in place
of~\eqref{eq:compare_conditionals}. In a scheme of P\'olya draws for blue
(blue is returned plus an additional blue) and replacement draws for red,
instead, we would obtain two inequalities in place
of~\eqref{eq:compare_conditionals}.

Something analogous \parencite[see][for an exact parallel
see]{portamana2003_r2004} happens in inferences about quantum measurements.
The conditional probability for the result of measurement $R$ made at time
$t_{2}$, given the result of measurement $B$ made at time $t_{1}$, is not
necessarily the same as that for measurement $R$ made at time $t_{1}$,
given measurement $B$ at time $t_{2}$. The time ordering of measurements is
extremely important in quantum theory (as it is in the urn example above).
In particular, if our goal is to actually \emph{retrodict} the result of
a previous measurement or state from the information gained in a subsequent
measurement, we must be very careful not to confuse the conditional
probabilities. The times of the two measurements cannot be swapped.

It should be noted that the conditional-probability
rule~\eqref{eq:conditional_prob} is in fact routinely used in quantum
theory in problems of state retrodiction and measurement reconstruction
\parencites{jones1991b,slater1995b}[\chaps~7,8]{demuynck2002b}{zimanetal2004_r2006,darianoetal2004}[see][\sect~1
for many further references]{maanssonetal2006},
for example to infer the state of a quantum laser given its output through
different optical apparatus \parencite{leonhardt1997}.

\textcolor{white}{If you find this you can claim a postcard from me.}



%%%% examples use empheq
%   \begin{empheq}[left={\mathllap{\begin{aligned}    \de\yF_{\yc}/\de\yp&=0\text{:} \\
%         \de\yF_{\yc}/\de\ym&=0\text{:}\\ \de\yF_{\yc}/\de\yl&=0\text{:}\end{aligned}}\qquad}\empheqlbrace]{align}
%     \label{eq:con_p}
% %    \de\yF_{\yc}/\de\yp &\equiv
%     -\ln\yp + \ln\yq + \yl\yM + \ym\yu &=0,\\
%     \label{eq:con_u}
% %    \de\yF_{\yc}/\de\ym &\equiv
%     \yu\yp-1 &=0,\\
%     \label{eq:con_l}
%     %\de\yF_{\yc}/\de\yl &\equiv
%     \yM\yp-\yc &=0.
%   \end{empheq}
%%%%
% \begin{empheq}[box=\widefbox]{equation}
%   \label{eq:maxent_question}
%   \p\bigl[\yE{N+1}{k} \bigcond \tsum\yo\yf{N}\in\yA, \yM\bigr] = \mathord{?}
% \end{empheq}



% \[
%   \begin{tikzcd}
%       M_{n,n}(\CC) \arrow{r}{R'_{a}(\Hat{U})} & M_{n,n}(\CC)
%     \\
%     L(\mathcal{H}) \arrow{r}{\Hat{U}} \arrow[swap]{d}{R_*}\arrow[swap]{u}{R'_*} & L(\mathcal{H}) \arrow{d}{R_*}\arrow{u}{R'_*} \\
%       M_{n,n}(\CC) \arrow{r}{R_{a}(\Hat{U})} & M_{n,n}(\CC)
%   \end{tikzcd}
% \]

% \[
%   \begin{tikzcd}
%       \CC^n \arrow{r}{R'_*(A)} & \CC^n
%     \\
%     \mathcal{H} \arrow{r}{A} \arrow[swap]{d}{R}\arrow[swap]{u}{R'} & \mathcal{H} \arrow{d}{R}\arrow{u}{R'} \\
%       \CC^n \arrow{r}{R_*(A)} & \CC^n
%   \end{tikzcd}
% \]


% \[
%   \begin{tikzcd}
%     \mathcal{H} \arrow{r}{A} \arrow[swap]{d}{R} & \mathcal{H} \arrow{d}{R} \\
%       \CC^n \arrow{r}{R_*(A)} & \CC^n
%   \end{tikzcd}
% \]

%%\setlength{\intextsep}{0ex}% with wrapfigure
%%\setlength{\columnsep}{0ex}% with wrapfigure
%\begin{figure}[p!]% with figure
%\begin{wrapfigure}{r}{0.4\linewidth} % with wrapfigure
%  \centering\includegraphics[trim={12ex 0 18ex 0},clip,width=\linewidth]{maxent_saddle.png}\\
%\caption{caption}\label{fig:comparison_a5}
%\end{figure}% exp_family_maxent.nb


%%%%%%%%%%%%%%%%%%%%%%%%%%%%%%%%%%%%%%%%%%%%%%%%%%%%%%%%%%%%%%%%%%%%%%%%%%%%
%%% Acknowledgements
%%%%%%%%%%%%%%%%%%%%%%%%%%%%%%%%%%%%%%%%%%%%%%%%%%%%%%%%%%%%%%%%%%%%%%%%%%%% 
\iffalse
\begin{acknowledgements}
  \ldots to Mari \amp\ Miri for continuous encouragement and affection, and
  to Buster Keaton and Saitama for filling life with awe and inspiration.
  To the developers and maintainers of \LaTeX, Emacs, AUC\TeX, Open Science
  Framework, R, Python, Inkscape, Sci-Hub for making a free and impartial
  scientific exchange possible.
%\rotatebox{15}{P}\rotatebox{5}{I}\rotatebox{-10}{P}\rotatebox{10}{\reflectbox{P}}\rotatebox{-5}{O}.
%\sourceatright{\autanet}
\mbox{}\hfill\autanet
\end{acknowledgements}
\fi

%%%%%%%%%%%%%%%%%%%%%%%%%%%%%%%%%%%%%%%%%%%%%%%%%%%%%%%%%%%%%%%%%%%%%%%%%%%%
%%% Appendices
%%%%%%%%%%%%%%%%%%%%%%%%%%%%%%%%%%%%%%%%%%%%%%%%%%%%%%%%%%%%%%%%%%%%%%%%%%%% 
%\clearpage
\bigskip
% %\renewcommand*{\appendixpagename}{Appendix}
% %\renewcommand*{\appendixname}{Appendix}
% %\appendixpage
% \appendix

%%%%%%%%%%%%%%%%%%%%%%%%%%%%%%%%%%%%%%%%%%%%%%%%%%%%%%%%%%%%%%%%%%%%%%%%%%%%
%%% Bibliography
%%%%%%%%%%%%%%%%%%%%%%%%%%%%%%%%%%%%%%%%%%%%%%%%%%%%%%%%%%%%%%%%%%%%%%%%%%%% 
\renewcommand*{\finalnamedelim}{\addcomma\space}
\defbibnote{prenote}{{\footnotesize (\enquote{de $X$} is listed under D,
    \enquote{van $X$} under V, and so on, regardless of national
    conventions.)\par}}
% \defbibnote{postnote}{\par\medskip\noindent{\footnotesize% Note:
%     \arxivp \mparcp \philscip \biorxivp}}

\printbibliography[prenote=prenote%,postnote=postnote
]

\end{document}

%%%%%%%%%%%%%%%%%%%%%%%%%%%%%%%%%%%%%%%%%%%%%%%%%%%%%%%%%%%%%%%%%%%%%%%%%%%%
%%% Cut text (won't be compiled)
%%%%%%%%%%%%%%%%%%%%%%%%%%%%%%%%%%%%%%%%%%%%%%%%%%%%%%%%%%%%%%%%%%%%%%%%%%%% 


%%% Local Variables: 
%%% mode: LaTeX
%%% TeX-PDF-mode: t
%%% TeX-master: t
%%% End: 
