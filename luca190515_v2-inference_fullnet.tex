\pdfoutput=1
%% Author: PGL  Porta Mana
%% Created: 2015-05-01T20:53:34+0200
%% Last-Updated: 2020-02-29T18:11:58+0100
%%%%%%%%%%%%%%%%%%%%%%%%%%%%%%%%%%%%%%%%%%%%%%%%%%%%%%%%%%%%%%%%%%%%%%%%%%%%
\newif\ifarxiv
\arxivfalse
\ifarxiv\pdfmapfile{+classico.map}\fi
\newif\ifafour
\afourfalse% true = A4, false = A5
\newif\iftypodisclaim % typographical disclaim on the side
\typodisclaimtrue
\newcommand*{\memfontfamily}{zplx}
\newcommand*{\memfontpack}{newpxtext}
\documentclass[\ifafour a4paper,12pt,\else a5paper,10pt,\fi%extrafontsizes,%
onecolumn,oneside,article,%french,italian,german,swedish,latin,
british%
]{memoir}
\newcommand*{\firstdraft}{15 May 2019}
\newcommand*{\firstpublished}{15 May 2019}
\newcommand*{\updated}{\ifarxiv***\else\today\fi}
\newcommand*{\propertitle}{Inferring the total activity of a large neuronal population\\ from a small sample\\ II: full Bayesian approach [draft]%\\{\large ***}%
}% title uses LARGE; set Large for smaller
\newcommand*{\pdftitle}{\propertitle}
\newcommand*{\headtitle}{Bayesian inferences about large neuronal populations}
\newcommand*{\pdfauthor}{P.G.L.  Porta Mana}
\newcommand*{\headauthor}{Porta Mana}
\newcommand*{\reporthead}{\ifarxiv\else Open Science Framework \href{https://doi.org/10.31219/osf.io/***}{\textsc{doi}:10.31219/osf.io/***}\fi}% Report number

%%%%%%%%%%%%%%%%%%%%%%%%%%%%%%%%%%%%%%%%%%%%%%%%%%%%%%%%%%%%%%%%%%%%%%%%%%%%
%%% Calls to packages (uncomment as needed)
%%%%%%%%%%%%%%%%%%%%%%%%%%%%%%%%%%%%%%%%%%%%%%%%%%%%%%%%%%%%%%%%%%%%%%%%%%%%

%\usepackage{pifont}

%\usepackage{fontawesome}

\usepackage[T1]{fontenc} 
\input{glyphtounicode} \pdfgentounicode=1

\usepackage[utf8]{inputenx}

%\usepackage{newunicodechar}
% \newunicodechar{Ĕ}{\u{E}}
% \newunicodechar{ĕ}{\u{e}}
% \newunicodechar{Ĭ}{\u{I}}
% \newunicodechar{ĭ}{\u{\i}}
% \newunicodechar{Ŏ}{\u{O}}
% \newunicodechar{ŏ}{\u{o}}
% \newunicodechar{Ŭ}{\u{U}}
% \newunicodechar{ŭ}{\u{u}}
% \newunicodechar{Ā}{\=A}
% \newunicodechar{ā}{\=a}
% \newunicodechar{Ē}{\=E}
% \newunicodechar{ē}{\=e}
% \newunicodechar{Ī}{\=I}
% \newunicodechar{ī}{\={\i}}
% \newunicodechar{Ō}{\=O}
% \newunicodechar{ō}{\=o}
% \newunicodechar{Ū}{\=U}
% \newunicodechar{ū}{\=u}
% \newunicodechar{Ȳ}{\=Y}
% \newunicodechar{ȳ}{\=y}

\newcommand*{\bmmax}{0} % reduce number of bold fonts, before font packages
\newcommand*{\hmmax}{0} % reduce number of heavy fonts, before font packages

\usepackage{textcomp}

%\usepackage[normalem]{ulem}% package for underlining
% \makeatletter
% \def\ssout{\bgroup \ULdepth=-.35ex%\UL@setULdepth
%  \markoverwith{\lower\ULdepth\hbox
%    {\kern-.03em\vbox{\hrule width.2em\kern1.2\p@\hrule}\kern-.03em}}%
%  \ULon}
% \makeatother

\usepackage{amsmath}

\usepackage{mathtools}
\addtolength{\jot}{\jot} % increase spacing in multiline formulae
\setlength{\multlinegap}{0pt}

\usepackage{empheq}% automatically calls amsmath and mathtools
\newcommand*{\widefbox}[1]{\fbox{\hspace{1em}#1\hspace{1em}}}

%%%% empheq above seems more versatile than these:
%\usepackage{fancybox}
%\usepackage{framed}

% \usepackage[misc]{ifsym} % for dice
% \newcommand*{\diceone}{{\scriptsize\Cube{1}}}

\usepackage{amssymb}

\usepackage{amsxtra}

\usepackage[main=british,french,italian,german,swedish,latin,esperanto]{babel}\selectlanguage{british}
\newcommand*{\langfrench}{\foreignlanguage{french}}
\newcommand*{\langgerman}{\foreignlanguage{german}}
\newcommand*{\langitalian}{\foreignlanguage{italian}}
\newcommand*{\langswedish}{\foreignlanguage{swedish}}
\newcommand*{\langlatin}{\foreignlanguage{latin}}
\newcommand*{\langnohyph}{\foreignlanguage{nohyphenation}}

\usepackage[autostyle=false,autopunct=false,english=british]{csquotes}
\setquotestyle{british}

\usepackage{amsthm}
\newcommand*{\QED}{\textsc{q.e.d.}}
\renewcommand*{\qedsymbol}{\QED}
\theoremstyle{remark}
\newtheorem{note}{Note}
\newtheorem*{remark}{Note}
\newtheoremstyle{innote}{\parsep}{\parsep}{\footnotesize}{}{}{}{0pt}{}
\theoremstyle{innote}
\newtheorem*{innote}{}

\usepackage[shortlabels,inline]{enumitem}
\SetEnumitemKey{para}{itemindent=\parindent,leftmargin=0pt,listparindent=\parindent,parsep=0pt,itemsep=\topsep}
% \begin{asparaenum} = \begin{enumerate}[para]
% \begin{inparaenum} = \begin{enumerate*}
\setlist{itemsep=0pt,topsep=\parsep}
\setlist[enumerate,2]{label=\alph*.}
\setlist[enumerate]{label=\arabic*.,leftmargin=1.5\parindent}
\setlist[itemize]{leftmargin=1.5\parindent}
\setlist[description]{leftmargin=1.5\parindent}
% old alternative:
% \setlist[enumerate,2]{label=\alph*.}
% \setlist[enumerate]{leftmargin=\parindent}
% \setlist[itemize]{leftmargin=\parindent}
% \setlist[description]{leftmargin=\parindent}

\usepackage[babel,theoremfont,largesc]{newpxtext}

\usepackage[bigdelims,nosymbolsc%,smallerops % probably arXiv doesn't have it
]{newpxmath}
\linespread{1.083}%\useosf
%% smaller operators for old version of newpxmath
\makeatletter
\def\re@DeclareMathSymbol#1#2#3#4{%
    \let#1=\undefined
    \DeclareMathSymbol{#1}{#2}{#3}{#4}}
%\re@DeclareMathSymbol{\bigsqcupop}{\mathop}{largesymbols}{"46}
%\re@DeclareMathSymbol{\bigodotop}{\mathop}{largesymbols}{"4A}
\re@DeclareMathSymbol{\bigoplusop}{\mathop}{largesymbols}{"4C}
\re@DeclareMathSymbol{\bigotimesop}{\mathop}{largesymbols}{"4E}
\re@DeclareMathSymbol{\sumop}{\mathop}{largesymbols}{"50}
\re@DeclareMathSymbol{\prodop}{\mathop}{largesymbols}{"51}
\re@DeclareMathSymbol{\bigcupop}{\mathop}{largesymbols}{"53}
\re@DeclareMathSymbol{\bigcapop}{\mathop}{largesymbols}{"54}
%\re@DeclareMathSymbol{\biguplusop}{\mathop}{largesymbols}{"55}
\re@DeclareMathSymbol{\bigwedgeop}{\mathop}{largesymbols}{"56}
\re@DeclareMathSymbol{\bigveeop}{\mathop}{largesymbols}{"57}
%\re@DeclareMathSymbol{\bigcupdotop}{\mathop}{largesymbols}{"DF}
%\re@DeclareMathSymbol{\bigcapplusop}{\mathop}{largesymbolsPXA}{"00}
%\re@DeclareMathSymbol{\bigsqcupplusop}{\mathop}{largesymbolsPXA}{"02}
%\re@DeclareMathSymbol{\bigsqcapplusop}{\mathop}{largesymbolsPXA}{"04}
%\re@DeclareMathSymbol{\bigsqcapop}{\mathop}{largesymbolsPXA}{"06}
\re@DeclareMathSymbol{\bigtimesop}{\mathop}{largesymbolsPXA}{"10}
%\re@DeclareMathSymbol{\coprodop}{\mathop}{largesymbols}{"60}
%\re@DeclareMathSymbol{\varprod}{\mathop}{largesymbolsPXA}{16}
\makeatother
%%
%% With euler font cursive for Greek letters - the [1] means 100% scaling
\DeclareFontFamily{U}{egreek}{\skewchar\font'177}%
\DeclareFontShape{U}{egreek}{m}{n}{<-6>s*[1]eurm5 <6-8>s*[1]eurm7 <8->s*[1]eurm10}{}%
\DeclareFontShape{U}{egreek}{m}{it}{<->s*[1]eurmo10}{}%
\DeclareFontShape{U}{egreek}{b}{n}{<-6>s*[1]eurb5 <6-8>s*[1]eurb7 <8->s*[1]eurb10}{}%
\DeclareFontShape{U}{egreek}{b}{it}{<->s*[1]eurbo10}{}%
\DeclareSymbolFont{egreeki}{U}{egreek}{m}{it}%
\SetSymbolFont{egreeki}{bold}{U}{egreek}{b}{it}% from the amsfonts package
\DeclareSymbolFont{egreekr}{U}{egreek}{m}{n}%
\SetSymbolFont{egreekr}{bold}{U}{egreek}{b}{n}% from the amsfonts package
% Take also \sum, \prod, \coprod symbols from Euler fonts
\DeclareFontFamily{U}{egreekx}{\skewchar\font'177}
\DeclareFontShape{U}{egreekx}{m}{n}{%
       <-7.5>s*[0.9]euex7%
    <7.5-8.5>s*[0.9]euex8%
    <8.5-9.5>s*[0.9]euex9%
    <9.5->s*[0.9]euex10%
}{}
\DeclareSymbolFont{egreekx}{U}{egreekx}{m}{n}
\DeclareMathSymbol{\sumop}{\mathop}{egreekx}{"50}
\DeclareMathSymbol{\prodop}{\mathop}{egreekx}{"51}
\DeclareMathSymbol{\coprodop}{\mathop}{egreekx}{"60}
\makeatletter
\def\sum{\DOTSI\sumop\slimits@}
\def\prod{\DOTSI\prodop\slimits@}
\def\coprod{\DOTSI\coprodop\slimits@}
\makeatother
\input{definegreek.tex}% Greek letters not usually given in LaTeX.

%\usepackage%[scaled=0.9]%
%{classico}%  Optima as sans-serif font
\renewcommand\sfdefault{uop}
\DeclareMathAlphabet{\mathsf}  {T1}{\sfdefault}{m}{sl}
\SetMathAlphabet{\mathsf}{bold}{T1}{\sfdefault}{b}{sl}
\newcommand*{\mathte}[1]{\textbf{\textit{\textsf{#1}}}}
% Upright sans-serif math alphabet
% \DeclareMathAlphabet{\mathsu}  {T1}{\sfdefault}{m}{n}
% \SetMathAlphabet{\mathsu}{bold}{T1}{\sfdefault}{b}{n}

% DejaVu Mono as typewriter text
\usepackage[scaled=0.84]{DejaVuSansMono}

\usepackage{mathdots}

\usepackage[usenames]{xcolor}
% Tol (2012) colour-blind-, print-, screen-friendly colours, alternative scheme; Munsell terminology
\definecolor{mypurpleblue}{RGB}{68,119,170}
\definecolor{myblue}{RGB}{102,204,238}
\definecolor{mygreen}{RGB}{34,136,51}
\definecolor{myyellow}{RGB}{204,187,68}
\definecolor{myred}{RGB}{238,102,119}
\definecolor{myredpurple}{RGB}{170,51,119}
\definecolor{mygrey}{RGB}{187,187,187}
% Tol (2012) colour-blind-, print-, screen-friendly colours; Munsell terminology
% \definecolor{lbpurple}{RGB}{51,34,136}
% \definecolor{lblue}{RGB}{136,204,238}
% \definecolor{lbgreen}{RGB}{68,170,153}
% \definecolor{lgreen}{RGB}{17,119,51}
% \definecolor{lgyellow}{RGB}{153,153,51}
% \definecolor{lyellow}{RGB}{221,204,119}
% \definecolor{lred}{RGB}{204,102,119}
% \definecolor{lpred}{RGB}{136,34,85}
% \definecolor{lrpurple}{RGB}{170,68,153}
\definecolor{lgrey}{RGB}{221,221,221}
%\newcommand*\mycolourbox[1]{%
%\colorbox{mygrey}{\hspace{1em}#1\hspace{1em}}}
\colorlet{shadecolor}{lgrey}

\usepackage{bm}

\usepackage{microtype}

\usepackage[backend=biber,mcite,%subentry,
citestyle=authoryear-comp,bibstyle=pglpm-authoryear,autopunct=false,sorting=ny,sortcites=false,natbib=false,maxcitenames=2,maxbibnames=8,minbibnames=8,giveninits=true,uniquename=false,uniquelist=false,maxalphanames=1,block=space,hyperref=true,defernumbers=false,useprefix=true,sortupper=false,language=british,parentracker=false]{biblatex}
\DeclareSortingScheme{ny}{\sort{\field{sortname}\field{author}\field{editor}}\sort{\field{year}}}
\iffalse\makeatletter%%% replace parenthesis with brackets
\newrobustcmd*{\parentexttrack}[1]{%
  \begingroup
  \blx@blxinit
  \blx@setsfcodes
  \blx@bibopenparen#1\blx@bibcloseparen
  \endgroup}
\AtEveryCite{%
  \let\parentext=\parentexttrack%
  \let\bibopenparen=\bibopenbracket%
  \let\bibcloseparen=\bibclosebracket}
\makeatother\fi
\DefineBibliographyExtras{british}{\def\finalandcomma{\addcomma}}
\renewcommand*{\finalnamedelim}{\addspace\amp\space}
\setcounter{biburlnumpenalty}{1}
\setcounter{biburlucpenalty}{0}
\setcounter{biburllcpenalty}{1}
\DeclareDelimFormat{multicitedelim}{\addsemicolon\addspace\space}
\DeclareDelimFormat{compcitedelim}{\addsemicolon\addspace\space}
\DeclareDelimFormat{postnotedelim}{\addspace}
\ifarxiv\else\addbibresource{portamanabib.bib}\fi
\renewcommand{\bibfont}{\footnotesize}
%\appto{\citesetup}{\footnotesize}% smaller font for citations
\defbibheading{bibliography}[\bibname]{\section*{#1}\addcontentsline{toc}{section}{#1}%\markboth{#1}{#1}
}
\newcommand*{\citep}{\footcites}
\newcommand*{\citey}{\footcites}%{\parencites*}
\newcommand*{\ibid}{\unspace\addtocounter{footnote}{-1}\footnotemark{}}
%\renewcommand*{\cite}{\parencite}
%\renewcommand*{\cites}{\parencites}
\providecommand{\href}[2]{#2}
\providecommand{\eprint}[2]{\texttt{\href{#1}{#2}}}
\newcommand*{\amp}{\&}
% \newcommand*{\citein}[2][]{\textnormal{\textcite[#1]{#2}}%\addtocategory{extras}{#2}
% }
\newcommand*{\citein}[2][]{\textnormal{\textcite[#1]{#2}}%\addtocategory{extras}{#2}
}
\newcommand*{\citebi}[2][]{\textcite[#1]{#2}%\addtocategory{extras}{#2}
}
\newcommand*{\subtitleproc}[1]{}
\newcommand*{\chapb}{ch.}
%
% \def\arxivp{}
% \def\mparcp{}
% \def\philscip{}
% \def\biorxivp{}
% \newcommand*{\arxivsi}{\texttt{arXiv} eprints available at \url{http://arxiv.org/}.\\}
% \newcommand*{\mparcsi}{\texttt{mp\_arc} eprints available at \url{http://www.ma.utexas.edu/mp_arc/}.\\}
% \newcommand*{\philscisi}{\texttt{philsci} eprints available at \url{http://philsci-archive.pitt.edu/}.\\}
% \newcommand*{\biorxivsi}{\texttt{bioRxiv} eprints available at \url{http://biorxiv.org/}.\\}
\newcommand*{\arxiveprint}[1]{%\global\def\arxivp{\arxivsi}%\citeauthor{0arxivcite}\addtocategory{ifarchcit}{0arxivcite}%eprint
\texttt{arXiv:\urlalt{https://arxiv.org/abs/#1}{#1}}%
%\texttt{\href{http://arxiv.org/abs/#1}{\protect\url{arXiv:#1}}}%
%\renewcommand{\arxivnote}{\texttt{arXiv} eprints available at \url{http://arxiv.org/}.}
}
\newcommand*{\mparceprint}[1]{%\global\def\mparcp{\mparcsi}%\citeauthor{0mparccite}\addtocategory{ifarchcit}{0mparccite}%eprint
\texttt{mp\_arc:\urlalt{http://www.ma.utexas.edu/mp_arc-bin/mpa?yn=#1}{#1}}%
%\texttt{\href{http://www.ma.utexas.edu/mp_arc-bin/mpa?yn=#1}{\protect\url{mp_arc:#1}}}%
%\providecommand{\mparcnote}{\texttt{mp_arc} eprints available at \url{http://www.ma.utexas.edu/mp_arc/}.}
}
\newcommand*{\haleprint}[1]{%\global\def\arxivp{\arxivsi}%\citeauthor{0arxivcite}\addtocategory{ifarchcit}{0arxivcite}%eprint
\texttt{HAL:\urlalt{https://hal.archives-ouvertes.fr/#1}{#1}}%
%\texttt{\href{http://arxiv.org/abs/#1}{\protect\url{arXiv:#1}}}%
%\renewcommand{\arxivnote}{\texttt{arXiv} eprints available at \url{http://arxiv.org/}.}
}
\newcommand*{\philscieprint}[1]{%\global\def\philscip{\philscisi}%\citeauthor{0philscicite}\addtocategory{ifarchcit}{0philscicite}%eprint
\texttt{PhilSci:\urlalt{http://philsci-archive.pitt.edu/archive/#1}{#1}}%
%\texttt{\href{http://philsci-archive.pitt.edu/archive/#1}{\protect\url{PhilSci:#1}}}%
%\providecommand{\mparcnote}{\texttt{philsci} eprints available at \url{http://philsci-archive.pitt.edu/}.}
}
\newcommand*{\biorxiveprint}[1]{%\global\def\biorxivp{\biorxivsi}%\citeauthor{0arxivcite}\addtocategory{ifarchcit}{0arxivcite}%eprint
\texttt{bioRxiv doi:\urlalt{https://doi.org/10.1101/#1}{10.1101/#1}}%
%\texttt{\href{http://arxiv.org/abs/#1}{\protect\url{arXiv:#1}}}%
%\renewcommand{\arxivnote}{\texttt{arXiv} eprints available at \url{http://arxiv.org/}.}
}
\newcommand*{\osfeprint}[1]{%
\texttt{Open Science Framework doi:\urlalt{https://doi.org/10.17605/osf.io/#1}{10.17605/osf.io/#1}}%
}

\usepackage{graphicx}

%\usepackage{wrapfig}

%\usepackage{tikz-cd}

\PassOptionsToPackage{hyphens}{url}\usepackage[hypertexnames=false]{hyperref}

\usepackage[depth=4]{bookmark}
\hypersetup{colorlinks=true,bookmarksnumbered,pdfborder={0 0 0.25},citebordercolor={0.2667 0.4667 0.6667},citecolor=mypurpleblue,linkbordercolor={0.6667 0.2 0.4667},linkcolor=myredpurple,urlbordercolor={0.1333 0.5333 0.2},urlcolor=mygreen,breaklinks=true,pdftitle={\pdftitle},pdfauthor={\pdfauthor}}
% \usepackage[vertfit=local]{breakurl}% only for arXiv
\providecommand*{\urlalt}{\href}

\usepackage[british]{datetime2}
\DTMnewdatestyle{mydate}%
{% definitions
\renewcommand*{\DTMdisplaydate}[4]{%
\number##3\ \DTMenglishmonthname{##2} ##1}%
\renewcommand*{\DTMDisplaydate}{\DTMdisplaydate}%
}
\DTMsetdatestyle{mydate}

%%%%%%%%%%%%%%%%%%%%%%%%%%%%%%%%%%%%%%%%%%%%%%%%%%%%%%%%%%%%%%%%%%%%%%%%%%%%
%%% Layout. I do not know on which kind of paper the reader will print the
%%% paper on (A4? letter? one-sided? double-sided?). So I choose A5, which
%%% provides a good layout for reading on screen and save paper if printed
%%% two pages per sheet. Average length line is 66 characters and page
%%% numbers are centred.
%%%%%%%%%%%%%%%%%%%%%%%%%%%%%%%%%%%%%%%%%%%%%%%%%%%%%%%%%%%%%%%%%%%%%%%%%%%%
\ifafour\setstocksize{297mm}{210mm}%{*}% A4
\else\setstocksize{210mm}{5.5in}%{*}% 210x139.7
\fi
\settrimmedsize{\stockheight}{\stockwidth}{*}
\setlxvchars[\normalfont] %313.3632pt for a 66-characters line
\setxlvchars[\normalfont]
\setlength{\trimtop}{0pt}
\setlength{\trimedge}{\stockwidth}
\addtolength{\trimedge}{-\paperwidth}
% The length of the normalsize alphabet is 133.05988pt - 10 pt = 26.1408pc
% The length of the normalsize alphabet is 159.6719pt - 12pt = 30.3586pc
% Bringhurst gives 32pc as boundary optimal with 69 ch per line
% The length of the normalsize alphabet is 191.60612pt - 14pt = 35.8634pc
\ifafour\settypeblocksize{*}{32pc}{1.618} % A4
%\setulmargins{*}{*}{1.667}%gives 5/3 margins % 2 or 1.667
\else\settypeblocksize{*}{26pc}{1.618}% nearer to a 66-line newpx and preserves GR
\fi
\setulmargins{*}{*}{1}%gives equal margins
\setlrmargins{*}{*}{*}
\setheadfoot{\onelineskip}{2.5\onelineskip}
\setheaderspaces{*}{2\onelineskip}{*}
\setmarginnotes{2ex}{10mm}{0pt}
\checkandfixthelayout[nearest]
\fixpdflayout
%%% End layout
%% this fixes missing white spaces
\pdfmapline{+dummy-space <dummy-space.pfb}\pdfinterwordspaceon%

%%% Sectioning
\newcommand*{\asudedication}[1]{%
{\par\centering\textit{#1}\par}}
\newenvironment{acknowledgements}{\section*{Thanks}\addcontentsline{toc}{section}{Thanks}}{\par}
\makeatletter\renewcommand{\appendix}{\par
  \bigskip{\centering
   \interlinepenalty \@M
   \normalfont
   \printchaptertitle{\sffamily\appendixpagename}\par}
  \setcounter{section}{0}%
  \gdef\@chapapp{\appendixname}%
  \gdef\thesection{\@Alph\c@section}%
  \anappendixtrue}\makeatother
\counterwithout{section}{chapter}
\setsecnumformat{\upshape\csname the#1\endcsname\quad}
\setsecheadstyle{\large\bfseries\sffamily%
\centering}
\setsubsecheadstyle{\bfseries\sffamily%
\raggedright}
%\setbeforesecskip{-1.5ex plus 1ex minus .2ex}% plus 1ex minus .2ex}
%\setaftersecskip{1.3ex plus .2ex }% plus 1ex minus .2ex}
%\setsubsubsecheadstyle{\bfseries\sffamily\slshape\raggedright}
%\setbeforesubsecskip{1.25ex plus 1ex minus .2ex }% plus 1ex minus .2ex}
%\setaftersubsecskip{-1em}%{-0.5ex plus .2ex}% plus 1ex minus .2ex}
\setsubsecindent{0pt}%0ex plus 1ex minus .2ex}
\setparaheadstyle{\bfseries\sffamily%
\raggedright}
\setcounter{secnumdepth}{2}
\setlength{\headwidth}{\textwidth}
\newcommand{\addchap}[1]{\chapter*[#1]{#1}\addcontentsline{toc}{chapter}{#1}}
\newcommand{\addsec}[1]{\section*{#1}\addcontentsline{toc}{section}{#1}}
\newcommand{\addsubsec}[1]{\subsection*{#1}\addcontentsline{toc}{subsection}{#1}}
\newcommand{\addpara}[1]{\paragraph*{#1.}\addcontentsline{toc}{subsubsection}{#1}}
\newcommand{\addparap}[1]{\paragraph*{#1}\addcontentsline{toc}{subsubsection}{#1}}

%%% Headers, footers, pagestyle
\copypagestyle{manaart}{plain}
\makeheadrule{manaart}{\headwidth}{0.5\normalrulethickness}
\makeoddhead{manaart}{%
{\footnotesize%\sffamily%
\scshape\headauthor}}{}{{\footnotesize\sffamily%
\headtitle}}
\makeoddfoot{manaart}{}{\thepage}{}
\newcommand*\autanet{\includegraphics[height=\heightof{M}]{autanet.pdf}}
\definecolor{mygray}{gray}{0.333}
\iftypodisclaim%
\ifafour\newcommand\addprintnote{\begin{picture}(0,0)%
\put(245,149){\makebox(0,0){\rotatebox{90}{\tiny\color{mygray}\textsf{This
            document is designed for screen reading and
            two-up printing on A4 or Letter paper}}}}%
\end{picture}}% A4
\else\newcommand\addprintnote{\begin{picture}(0,0)%
\put(176,112){\makebox(0,0){\rotatebox{90}{\tiny\color{mygray}\textsf{This
            document is designed for screen reading and
            two-up printing on A4 or Letter paper}}}}%
\end{picture}}\fi%afourtrue
\makeoddfoot{plain}{}{\makebox[0pt]{\thepage}\addprintnote}{}
\else
\makeoddfoot{plain}{}{\makebox[0pt]{\thepage}}{}
\fi%typodisclaimtrue
\makeoddhead{plain}{\scriptsize\reporthead}{}{}
% \copypagestyle{manainitial}{plain}
% \makeheadrule{manainitial}{\headwidth}{0.5\normalrulethickness}
% \makeoddhead{manainitial}{%
% \footnotesize\sffamily%
% \scshape\headauthor}{}{\footnotesize\sffamily%
% \headtitle}
% \makeoddfoot{manaart}{}{\thepage}{}

\pagestyle{manaart}

\setlength{\droptitle}{-3.9\onelineskip}
\pretitle{\begin{center}\Large\sffamily%
\bfseries}
\posttitle{\bigskip\end{center}}

\makeatletter\newcommand*{\atf}{\includegraphics[%trim=1pt 1pt 0pt 0pt,
totalheight=\heightof{@}]{atblack.png}}\makeatother
\providecommand{\affiliation}[1]{\textsl{\textsf{\footnotesize #1}}}
\providecommand{\epost}[1]{\texttt{\footnotesize\textless#1\textgreater}}
\providecommand{\email}[2]{\href{mailto:#1ZZ@#2 ((remove ZZ))}{#1\protect\atf#2}}

\preauthor{\vspace{-0.5\baselineskip}\begin{center}
\normalsize\sffamily%
\lineskip  0.5em}
\postauthor{\par\end{center}}
\predate{\DTMsetdatestyle{mydate}\begin{center}\footnotesize}
\postdate{\end{center}\vspace{-\medskipamount}}

\setfloatadjustment{figure}{\footnotesize}
\captiondelim{\quad}
\captionnamefont{\footnotesize\sffamily%
}
\captiontitlefont{\footnotesize}
%\firmlists*
\midsloppy
% handling orphan/widow lines, memman.pdf
% \clubpenalty=10000
% \widowpenalty=10000
% \raggedbottom
% Downes, memman.pdf
\clubpenalty=9996
\widowpenalty=9999
\brokenpenalty=4991
\predisplaypenalty=10000
\postdisplaypenalty=1549
\displaywidowpenalty=1602
\raggedbottom

\paragraphfootnotes
%\setlength{\footmarkwidth}{0em}
% \threecolumnfootnotes
%\setlength{\footmarksep}{0em}
\footmarkstyle{\textsuperscript{%\color{myred}
\scriptsize\bfseries#1}~}
%\footmarkstyle{\textsuperscript{\color{myred}\scriptsize\bfseries#1}~}
%\footmarkstyle{\textsuperscript{[#1]}~}

\selectlanguage{british}\frenchspacing

%%%%%%%%%%%%%%%%%%%%%%%%%%%%%%%%%%%%%%%%%%%%%%%%%%%%%%%%%%%%%%%%%%%%%%%%%%%%
%%% Paper's details
%%%%%%%%%%%%%%%%%%%%%%%%%%%%%%%%%%%%%%%%%%%%%%%%%%%%%%%%%%%%%%%%%%%%%%%%%%%%
\title{\propertitle}
\author{%
\hspace*{\stretch{1}}%
%% uncomment if additional authors present
% \parbox{0.5\linewidth}%\makebox[0pt][c]%
% {\protect\centering ***\\%
% \footnotesize\epost{\email{***}{***}}}%
% \hspace*{\stretch{1}}%
\parbox{0.75\linewidth}%\makebox[0pt][c]%
{\protect\centering P.G.L.  Porta Mana  \href{https://orcid.org/0000-0002-6070-0784}{\protect\includegraphics[scale=0.16]{orcid_32x32.png}}\\%
\footnotesize Kavli Institute, Trondheim\quad\epost{\email{piero.mana}{ntnu.no}}}%
%% uncomment if additional authors present
% \hspace*{\stretch{1}}%
% \parbox{0.5\linewidth}%\makebox[0pt][c]%
% {\protect\centering ***\\%
% \footnotesize\epost{\email{***}{***}}}%
\hspace*{\stretch{1}}%
}

%\date{Draft of \today\ (first drafted \firstdraft)}
\date{\firstpublished; updated \updated}

%%%%%%%%%%%%%%%%%%%%%%%%%%%%%%%%%%%%%%%%%%%%%%%%%%%%%%%%%%%%%%%%%%%%%%%%%%%%
%%% Macros @@@
%%%%%%%%%%%%%%%%%%%%%%%%%%%%%%%%%%%%%%%%%%%%%%%%%%%%%%%%%%%%%%%%%%%%%%%%%%%%

% Common ones - uncomment as needed
%\providecommand{\nequiv}{\not\equiv}
%\providecommand{\coloneqq}{\mathrel{\mathop:}=}
%\providecommand{\eqqcolon}{=\mathrel{\mathop:}}
%\providecommand{\varprod}{\prod}
\newcommand*{\de}{\partialup}%partial diff
\newcommand*{\pu}{\piup}%constant pi
\newcommand*{\delt}{\deltaup}%Kronecker, Dirac
%\newcommand*{\eps}{\varepsilonup}%Levi-Civita, Heaviside
%\newcommand*{\riem}{\zetaup}%Riemann zeta
%\providecommand{\degree}{\textdegree}% degree
%\newcommand*{\celsius}{\textcelsius}% degree Celsius
%\newcommand*{\micro}{\textmu}% degree Celsius
\newcommand*{\I}{\mathrm{i}}%imaginary unit
\newcommand*{\e}{\mathrm{e}}%Neper
\newcommand*{\di}{\mathrm{d}}%differential
%\newcommand*{\Di}{\mathrm{D}}%capital differential
%\newcommand*{\planckc}{\hslash}
%\newcommand*{\avogn}{N_{\textrm{A}}}
%\newcommand*{\NN}{\bm{\mathrm{N}}}
%\newcommand*{\ZZ}{\bm{\mathrm{Z}}}
%\newcommand*{\QQ}{\bm{\mathrm{Q}}}
\newcommand*{\RR}{\bm{\mathrm{R}}}
%\newcommand*{\CC}{\bm{\mathrm{C}}}
%\newcommand*{\nabl}{\bm{\nabla}}%nabla
%\DeclareMathOperator{\lb}{lb}%base 2 log
%\DeclareMathOperator{\tr}{tr}%trace
%\DeclareMathOperator{\card}{card}%cardinality
%\DeclareMathOperator{\im}{Im}%im part
%\DeclareMathOperator{\re}{Re}%re part
%\DeclareMathOperator{\sgn}{sgn}%signum
%\DeclareMathOperator{\ent}{ent}%integer less or equal to
%\DeclareMathOperator{\Ord}{O}%same order as
%\DeclareMathOperator{\ord}{o}%lower order than
%\newcommand*{\incr}{\triangle}%finite increment
\newcommand*{\defd}{\coloneqq}
\newcommand*{\defs}{\eqqcolon}
%\newcommand*{\Land}{\bigwedge}
%\newcommand*{\Lor}{\bigvee}
%\newcommand*{\lland}{\DOTSB\;\land\;}
%\newcommand*{\llor}{\DOTSB\;\lor\;}
%\newcommand*{\limplies}{\mathbin{\Rightarrow}}%implies
%\newcommand*{\suchthat}{\mid}%{\mathpunct{|}}%such that (eg in sets)
%\newcommand*{\with}{\colon}%with (list of indices)
%\newcommand*{\mul}{\times}%multiplication
%\newcommand*{\inn}{\cdot}%inner product
%\newcommand*{\dotv}{\mathord{\,\cdot\,}}%variable place
%\newcommand*{\comp}{\circ}%composition of functions
%\newcommand*{\con}{\mathbin{:}}%scal prod of tensors
%\newcommand*{\equi}{\sim}%equivalent to 
\renewcommand*{\asymp}{\simeq}%equivalent to 
%\newcommand*{\corr}{\mathrel{\hat{=}}}%corresponds to
%\providecommand{\varparallel}{\ensuremath{\mathbin{/\mkern-7mu/}}}%parallel (tentative symbol)
\renewcommand*{\le}{\leqslant}%less or equal
\renewcommand*{\ge}{\geqslant}%greater or equal
\DeclarePairedDelimiter\clcl{[}{]}
%\DeclarePairedDelimiter\clop{[}{[}
%\DeclarePairedDelimiter\opcl{]}{]}
%\DeclarePairedDelimiter\opop{]}{[}
\DeclarePairedDelimiter\abs{\lvert}{\rvert}
%\DeclarePairedDelimiter\norm{\lVert}{\rVert}
\DeclarePairedDelimiter\set{\{}{\}}
%\DeclareMathOperator{\pr}{P}%probability
\newcommand*{\pf}{\mathrm{p}}%probability
\newcommand*{\p}{\mathrm{P}}%probability
%\newcommand*{\E}{\mathrm{E}}
%\renewcommand*{\|}{\nonscript\,\vert\nonscript\;\mathopen{}}
\renewcommand*{\|}[1][]{\nonscript\,#1\vert\nonscript\;\mathopen{}}
%\DeclarePairedDelimiterX{\cond}[2]{(}{)}{#1\nonscript\,\delimsize\vert\nonscript\;\mathopen{}#2}
%\DeclarePairedDelimiterX{\condt}[2]{[}{]}{#1\nonscript\,\delimsize\vert\nonscript\;\mathopen{}#2}
%\DeclarePairedDelimiterX{\conds}[2]{\{}{\}}{#1\nonscript\,\delimsize\vert\nonscript\;\mathopen{}#2}
%\newcommand*{\+}{\lor}
%\renewcommand{\*}{\land}
\newcommand*{\sect}{\S}% Sect.~
\newcommand*{\sects}{\S\S}% Sect.~
\newcommand*{\chap}{ch.}%
\newcommand*{\chaps}{chs}%
\newcommand*{\bref}{ref.}%
\newcommand*{\brefs}{refs}%
%\newcommand*{\fn}{fn}%
\newcommand*{\eqn}{eq.}%
\newcommand*{\eqns}{eqs}%
\newcommand*{\fig}{fig.}%
\newcommand*{\figs}{figs}%
\newcommand*{\vs}{{vs}}
%\newcommand*{\etc}{{etc.}}
%\newcommand*{\ie}{{i.e.}}
%\newcommand*{\ca}{{c.}}
%\newcommand*{\eg}{{e.g.}}
\newcommand*{\foll}{{ff.}}
%\newcommand*{\viz}{{viz}}
\newcommand*{\cf}{{cf.}}
%\newcommand*{\Cf}{{Cf.}}
%\newcommand*{\vd}{{v.}}
\newcommand*{\etal}{{et al.}}
%\newcommand*{\etsim}{{et sim.}}
%\newcommand*{\ibid}{{ibid.}}
%\newcommand*{\sic}{{sic}}
%\newcommand*{\id}{\mathte{I}}%id matrix
%\newcommand*{\nbd}{\nobreakdash}%
%\newcommand*{\bd}{\hspace{0pt}}%
%\def\hy{-\penalty0\hskip0pt\relax}
%\newcommand*{\labelbis}[1]{\tag*{(\ref{#1})$_\text{r}$}}
%\newcommand*{\mathbox}[2][.8]{\parbox[t]{#1\columnwidth}{#2}}
%\newcommand*{\zerob}[1]{\makebox[0pt][l]{#1}}
\newcommand*{\tprod}{\mathop{\textstyle\prod}\nolimits}
\newcommand*{\tsum}{\mathop{\textstyle\sum}\nolimits}
%\newcommand*{\tint}{\begingroup\textstyle\int\endgroup\nolimits}
%\newcommand*{\tland}{\mathop{\textstyle\bigwedge}\nolimits}
%\newcommand*{\tlor}{\mathop{\textstyle\bigvee}\nolimits}
%\newcommand*{\sprod}{\mathop{\textstyle\prod}}
%\newcommand*{\ssum}{\mathop{\textstyle\sum}}
%\newcommand*{\sint}{\begingroup\textstyle\int\endgroup}
%\newcommand*{\sland}{\mathop{\textstyle\bigwedge}}
%\newcommand*{\slor}{\mathop{\textstyle\bigvee}}
%\newcommand*{\T}{^\intercal}%transpose
%%\newcommand*{\QEM}%{\textnormal{$\Box$}}%{\ding{167}}
%\newcommand*{\qem}{\leavevmode\unskip\penalty9999 \hbox{}\nobreak\hfill
%\quad\hbox{\QEM}}

%%%%%%%%%%%%%%%%%%%%%%%%%%%%%%%%%%%%%%%%%%%%%%%%%%%%%%%%%%%%%%%%%%%%%%%%%%%%
%%% Custom macros for this file @@@
%%%%%%%%%%%%%%%%%%%%%%%%%%%%%%%%%%%%%%%%%%%%%%%%%%%%%%%%%%%%%%%%%%%%%%%%%%%%
 \definecolor{notecolour}{RGB}{68,170,153}
\newcommand*{\puzzle}{{\fontencoding{U}\fontfamily{fontawesometwo}\selectfont\symbol{225}}}
%\newcommand*{\puzzle}{\maltese}
\newcommand{\mynote}[1]{ {\color{notecolour}\puzzle\ #1}}
\newcommand*{\widebar}[1]{{\mkern1.5mu\skew{2}\overline{\mkern-1.5mu#1\mkern-1.5mu}\mkern 1.5mu}}

% \newcommand{\explanation}[4][t]{%\setlength{\tabcolsep}{-1ex}
% %\smash{
% \begin{tabular}[#1]{c}#2\\[0.5\jot]\rule{1pt}{#3}\\#4\end{tabular}}%}
% \newcommand*{\ptext}[1]{\text{\small #1}}
%\DeclareMathOperator*{\argsup}{arg\,sup}
\newcommand*{\dob}{degree of belief}
\newcommand*{\dobs}{degrees of belief}

\newcommand*{\yFF}{F}
\newcommand*{\yff}{f}
\newcommand*{\yF}{\bm{\yFF}}
\newcommand*{\yf}{\bm{\yff}}
\newcommand*{\yH}{\varIota}
\newcommand*{\yjj}{\phi}
\newcommand*{\yj}{\bm{\yjj}}
\newcommand*{\yAv}{A}
\newcommand*{\yA}{\yAv}%pop av value
\newcommand*{\yAs}{\widebar{A}}%pop av value
\newcommand*{\yav}{a}
\newcommand*{\ya}{\yav}%subpop av value
\newcommand*{\yas}{\widebar{a}}%pop av value
\newcommand*{\ypp}{G}
\newcommand*{\ynuu}{\nu}
\newcommand*{\ynu}{\bm{\ynuu}}
\newcommand*{\ysh}{H}
\newcommand*{\yRR}{R}
\newcommand*{\yR}{\bm{\yRR}}
\newcommand{\appropto}{\mathrel{\vcenter{
  \offinterlineskip\halign{\hfil$##$\cr
    \propto\cr\noalign{\kern2pt}\sim\cr\noalign{\kern-2pt}}}}}
% https://tex.stackexchange.com/a/33547/97039
\newcommand*{\yeq}{\mathrel{\!=\!}}
\newcommand*{\yJJ}{J}
\newcommand*{\yJ}{\bm{\yJJ}}
\newcommand*{\yI}{\varIota}
\newcommand*{\yG}{\bm{G}}
\newcommand*{\yr}{\bm{r}}



%%% Custom macros end @@@

%%%%%%%%%%%%%%%%%%%%%%%%%%%%%%%%%%%%%%%%%%%%%%%%%%%%%%%%%%%%%%%%%%%%%%%%%%%%
%%% Beginning of document
%%%%%%%%%%%%%%%%%%%%%%%%%%%%%%%%%%%%%%%%%%%%%%%%%%%%%%%%%%%%%%%%%%%%%%%%%%%%
%\firmlists
\begin{document}
\captiondelim{\quad}\captionnamefont{\footnotesize}\captiontitlefont{\footnotesize}
\selectlanguage{british}\frenchspacing
\maketitle

%%%%%%%%%%%%%%%%%%%%%%%%%%%%%%%%%%%%%%%%%%%%%%%%%%%%%%%%%%%%%%%%%%%%%%%%%%%%
%%% Abstract
%%%%%%%%%%%%%%%%%%%%%%%%%%%%%%%%%%%%%%%%%%%%%%%%%%%%%%%%%%%%%%%%%%%%%%%%%%%%
\abstractrunin
\abslabeldelim{}
\renewcommand*{\abstractname}{}
\setlength{\absleftindent}{0pt}
\setlength{\absrightindent}{0pt}
\setlength{\abstitleskip}{-\absparindent}
\begin{abstract}\labelsep 0pt%
  \noindent ***
% \\\noindent\emph{\footnotesize Note: Dear Reader
%     \amp\ Peer, this manuscript is being peer-reviewed by you. Thank you.}
% \par%\\[\jot]
% \noindent
% {\footnotesize PACS: ***}\qquad%
% {\footnotesize MSC: ***}%
%\qquad{\footnotesize Keywords: ***}
\end{abstract}
\selectlanguage{british}\frenchspacing

%%%%%%%%%%%%%%%%%%%%%%%%%%%%%%%%%%%%%%%%%%%%%%%%%%%%%%%%%%%%%%%%%%%%%%%%%%%%
%%% Epigraph
%%%%%%%%%%%%%%%%%%%%%%%%%%%%%%%%%%%%%%%%%%%%%%%%%%%%%%%%%%%%%%%%%%%%%%%%%%%%
% \asudedication{\small ***}
% \vspace{\bigskipamount}
% \setlength{\epigraphwidth}{.7\columnwidth}
% %\epigraphposition{flushright}
% \epigraphtextposition{flushright}
% %\epigraphsourceposition{flushright}
% \epigraphfontsize{\footnotesize}
% \setlength{\epigraphrule}{0pt}
% %\setlength{\beforeepigraphskip}{0pt}
% %\setlength{\afterepigraphskip}{0pt}
% \epigraph{\emph{text}}{source}



%%%%%%%%%%%%%%%%%%%%%%%%%%%%%%%%%%%%%%%%%%%%%%%%%%%%%%%%%%%%%%%%%%%%%%%%%%%%
%%% BEGINNING OF MAIN TEXT
%%%%%%%%%%%%%%%%%%%%%%%%%%%%%%%%%%%%%%%%%%%%%%%%%%%%%%%%%%%%%%%%%%%%%%%%%%%%

\section{Introduction}
\label{sec:introduction}

The goal of many experimental recordings of neuronal activity can be
likened to \emph{survey sampling}:


\textcolor{white}{Did you find this? Congrats, you won a postcard! Feel
  free to send me your address.}


Here are the distributions for our \dobs\ about three different quantities,
assuming an entropic pre-data distribution for the long-run frequencies of
the full network:

\textbf{The frequency distribution $\yF$ of the full-network activity
  \emph{during the recording}}

\begin{equation}
  \label{eq:fullfreq_from_samplefreq}
  \begin{split}
  \p(\yF \|\yf,\yH) &\propto
  \!\begin{multlined}[t][0.8\linewidth]
  \int\!\di\ynu\;  \tsum_{\yj}
  \delt(\tsum_{\yA}\yjj_{\ya\yA} = \yff_{\ya})\;
  \delt(\tsum_{\ya}\yjj_{\ya\yA} = \yFF_{\yA})\times{}\\
  \binom{T}{T\yj}\;
  \biggl[\prod_{\ya\yA}
(\ypp_{\ya\yA}\ynuu_{\yA})^{T\yjj_{\ya\yA}}
\biggr]\;
\exp[-L\;\ysh(\ynu;\yR)]
\end{multlined}
\\
&\appropto \delt(\tsum_{\yA}\ypp_{\ya\yA}\yFF_{\yA} = \yff_{\ya})\;
\exp[-L\;\ysh(\yF;\yR)]
\end{split}
\end{equation}

\bigskip

\textbf{The long-run frequency distribution $\ynu$ of the full-network
  activity}

\begin{equation}
  \label{eq:longrun_from_samplefreq}
  \begin{split}
  \p(\ynu \|\yf,\yH) &\propto
    \binom{T}{T\yf}\;
  \biggl[\prod_{\ya}
(\tsum_{\yA}\ypp_{\ya\yA}\ynuu_{\yA})^{T\yff_{\ya}}
\biggr]\;
\exp[-L\;\ysh(\ynu;\yR)]
\\
&\appropto 
\exp[-T\;\ysh(\yf; \mathte{\ypp}\ynu)
-L\;\ysh(\ynu;\yR)]
\end{split}
\end{equation}
Note that if some $\yff_{\ya}$ are zero, then the first exponential may
be badly approximated by a delta, because the constraints lie on a facet of the
simplex of frequencies $\set{\ynu}$.

\bigskip

\textbf{The activity $\yA'$ of the full-network
  in a \emph{new} time bin}

\begin{equation}
  \label{eq:newbin_from_samplefreq}
  \begin{split}
  \p(\yA' \|\yf,\yH) &\propto
  \int\!\di\ynu\;\ynuu_{\yA'}\;
  \binom{T}{T\yf}\;
  \biggl[\prod_{\ya}
(\tsum_{\yA}\ypp_{\ya\yA}\ynuu_{\yA})^{T\yff_{\ya}}
\biggr]\;
\exp[-L\;\ysh(\ynu;\yR)]
\\
&\appropto 
\int\!\di\ynu\;\ynuu_{\yA'}\;
\exp[-T\;\ysh(\yf; \mathte{\ypp}\ynu)
-L\;\ysh(\ynu;\yR)]
\end{split}
\end{equation}
Note that if some $\yff_{\ya}$ are zero, then the first exponential may
be badly approximated by a delta, because the constraints lie on a facet of the
simplex of frequencies $\set{\ynu}$.


\section{Derivation of the probability distribution\\ and of its approximations}
\label{sec:derivation}

\subsection{Notation}
\label{sec:notation}

\begin{itemize}[wide,label={}]
\item $N=1000$: size of the full network.

\item $n=65$: size of the sample.

\item $T=417\,641$: number of bins in the recording.

Bins are indexed by $t\in\set{1,\dotsc,T}$.

\item $A\in\set{0,1,\dotsc,N}$: levels of total activity of the full network.

\item $A_{t}\in\set{0,1,\dotsc,N}$: total activity of the full network at bin $t$.

\item $\yAs \defd (A_{1}, A_{2}, \dotsc, A_{T})$: sequence of activities of full
network during the recording.

\item $F_{A} \in \set{0,1/T,\dotsc,1}$: relative frequency of activity level $A$
of full network during the recording. That is, activity level $A$ appears
in $T F_{A}$ out of $T$ bins: $T F_{A} = \sum_{t}\delt(A_{t}\yeq A)$.

\item $\yF \defd (F_{0},F_{1}, \dotsc,F_{N})$: frequency distribution.

  Note that $\yAs$ completely determines $\yF$. We write the $\yF$
  determined by $\yAs$ as $\yF(\yAs)$.

\item $\nu_{A} \in \clcl{0,1}$: long-run relative frequency of activity level $A$
of full network during all hypothetical repetitions of the same recording
in the same conditions.

\item $\ynu \defd (\nu_{0}, \nu_{1}, \dotsc, \nu_{N})$: frequency distribution.

\item $a\in\set{0,1,\dotsc,n}$: levels of total activity of the sample.

\item $a_{t}\in\set{0,1,\dotsc,n}$: total activity of the sample at bin $t$.

\item $\yas \defd (a_{1}, a_{2}, \dotsc, a_{T})$: sequence of activities of
sample during the recording.

\item $f_{a} \in \set{0,1/T,\dotsc,1}$: relative frequency of activity level $a$
of sample during the recording.

\item $\yf \defd (f_{0},f_{1}, \dotsc,f_{n})$: frequency distribution.

  Note that $\yas$ completely determines $\yf$.  We write the $\yf$
  determined by $\yas$ as $\yf(\yas)$.

  
\item $J_{a\,A} \in \set{0,1/T,\dotsc,1}$: joint relative frequency of
  activity levels $a$ and $A$ for sample and full network during the
  recording. That is, the pair $(a,A)$ appears in $T J_{a\,A}$ out of $T$
  bins:
  $T J_{a\,A} = \sum_{t}\delt(a_{t} \yeq  a)\, \delt(A_{t}
  \yeq  A)$.

  Note that the full network can't have fewer active neurons or fewer
  silent neurons than its sample (for example, if $A_{t}=0$ then
  $a_{t}=0$, and if $A_{t}=N$ then $a_{t}=n$), so $J_{a\,A}\equiv 0$ for
  $a>A$ or $n-a>N-A$.

  Obviously $F_{A}\equiv\sum_{a}J_{a\,A}$ and $f_{a}\equiv\sum_{A}J_{a\,A}$.

\item $\yJ \defd (J_{0\,0},J_{0\,1}, \dotsc,J_{n\,N})$: joint frequency
  distribution.

  Note that $(\yas,\yAs)$ together completely determine $\yJ$.  We write the $\yJ$
  determined by $(\yas,\yAs)$ as $\yJ(\yas,\yAs)$.

\item
  $G_{a\,A} \defd 
  \binom{n}{a}\binom{N-n}{A-a}\binom{N}{A}^{-1} \equiv
  \binom{A}{a}\binom{N-A}{n-a}\binom{N}{n}^{-1}$: hypergeometric
  distribution.
  
\item $\yI$: background information and assumptions; includes knowledge of
  $N$ and $n$.
\end{itemize}

\subsection{Derivation}
\label{sec:derivation_steps}

We want to know which distribution of frequencies of full-network
activities occurred during the recording, given our observations of the
sample. Namely, $\pf(\yF \| \yas,\yI)$. This can be obtained as the
marginal of the probability distribution for the joint frequencies,
$\pf(\yJ \| \yas, \yI)$. If this probability distribution is represented by
a set of Monte Carlo samples $\set{\yJ^{(i)}}$, then
$\set{\yF^{(i)}\equiv (\sum_{a}J^{(i)}_{a\,A})}$ are automatically samples
of $\pf(\yF \| \yas,\yI)$.

By the theorem of total probability,
\begin{equation}
  \label{eq:total_prob_As}
  \pf(\yJ \| \yas, \yI) =
  \sum_{\yAs}  \pf(\yJ \| \yas, \yAs, \yI)\; \pf(\yAs \| \yas, \yI),
\end{equation}
the sum being over all possible sequences $\set{\yAs}$ of total activities.

The first factor is a singular probability distribution, because $\yas$,
$\yAs$ jointly determine $\yJ$:
\begin{equation}
  \label{eq:singular_J}
  \begin{split}
  \pf(\yJ \| \yas, \yAs, \yI) 
&= \delt\bigl[\yJ \yeq \yJ(\yas,\yAs)\bigr]\\
&\equiv \prod_{a\,A}  \delt\bigl[T J_{a\,A} \yeq  \tsum_{t}
  \delt(a_{t} \yeq  a)\, \delt(A_{t} \yeq  A)\bigr].
\end{split}
\end{equation}

The second factor is derived from Bayes's theorem:
\begin{equation}
  \label{eq:bayes_sequences}
  \pf(\yAs \| \yas, \yI) =
  \frac{\pf(\yas \| \yAs, \yI)\;\pf(\yAs \| \yI)}
  {\sum_{\yAs}\pf(\yas \| \yAs, \yI)\;\pf(\yAs \| \yI)}.
\end{equation}

\bigskip

In the last formula, let's first derive $\pf(\yas \| \yAs, \yI)$. We make
the following
\begin{equation}
  \label{eq:assumption_irrelevance_t}
  \parbox{0.9\linewidth}{\textbf{assumption}: if $A_{t}$ is known, then
    knowledge of $a_{t'}$ or $A_{t'}$ with $t'\ne t$ is irrelevant for our
    inferences about $a_{t}$.}  
\end{equation}
It isn't a realistic assumption, but it's the one behind the
maximum-entropy method in this kind of applications. From this assumption
we have, using the product rule,
\begin{equation}
  \label{eq:factor_at_At}
  \pf(\yas \| \yAs, \yI) = \prod_{t} \pf(a_{t} \| A_{t}, \yI),
\end{equation}
and from simple sampling theory
$\pf(a_{t} \| A_{t}, \yI) = G_{a_{t}\,A_{t}}$ for each $t$, so that
\begin{equation}
  \label{eq:factor_at_At}
  \pf(\yas \| \yAs, \yI) = \prod_{t} G_{a_{t}\,A_{t}}.
\end{equation}
The latter product can be rewritten by grouping together all $t$ in which
the same $(a,A)$ pair appears: there are $T J_{a\,A}(\yas,\yAs)$ such $t$.
Then we consider all such pairs:
\begin{equation}
  \label{eq:factor_at_At_grouped}
  \pf(\yas \| \yAs, \yI) = \prod_{a,A} {G_{a\,A}}^{T J_{a\,A}(\yas,\yAs)}
\end{equation}
where $\prod_{a,A} \defd \prod_{a}\prod_{A}$.

\medskip

Now let's decide upon $\pf(\yAs \| \yI)$ in
formula~\eqref{eq:bayes_sequences}. We assume that this probability has the
same value for all sequences $\yAs$ having the same frequency distribution
$\yF(\yAs)$; that is, it functionally depends on $\yAs$ only through
$\yF(\yAs)$. More specifically we assume that it can be hierarchically
written as
\begin{equation}
  \label{eq:p_as_exchangeable}
  \pf(\yAs \| \yI) = \pi(\yF) =
  \int\!\!\di\ynu\;q(\ynu)\;\prod_{A}{\nu_{A}}^{T F_{A}}
\end{equation}
for some density function $q(\ynu)\,\di\ynu$. The integral expression is
simply a mathematical way to write $\pi(\yF)$, and doesn't need to be
further interpreted (the integral disappears if we can solve it
analytically). But it's also the formula for infinite exchangeability, and
from this point of view $\ynu$ can be interpreted as the long-run frequency
distribution of the activities in all experiments performed in the same
conditions -- imagining to join together their recording times. This is the
point of view underlying the maximum-entropy method.

\medskip

Replacing \eqref{eq:factor_at_At_grouped} and \eqref{eq:p_as_exchangeable}
into~\eqref{eq:bayes_sequences} and rearranging we find
\begin{equation}
  \label{eq:bayes_sequences_developed}
  \begin{split}
  \pf(\yAs \| \yas, \yI) &= \frac{1}{Z}
  \int\!\!\di\ynu\;q(\ynu)\; %\delt\bigl[\yJ \yeq \yJ(\yas,\yAs)\bigr]\;
  \prod_{a,A} \Bigl[{G_{a\,A}}^{T J_{a\,A}(\yas,\yAs)}\Bigr]
  \;\prod_{A}{\nu_{A}}^{T F_{A}}
  \\
  &\equiv
  \frac{1}{Z}
  \int\!\!\di\ynu\;q(\ynu)\; %\delt\bigl[\yJ \yeq \yJ(\yas,\yAs)\bigr]\;
  \prod_{a,A} \bigl(G_{a\,A}\,\nu_{A}\bigr)^{T J_{a\,A}(\yas,\yAs)},
\end{split}
\end{equation}
where $Z\defd \sum_{\yAs}\pf(\yas \| \yAs, \yI)\;\pf(\yAs \| \yI)$ is the
normalization factor and the second equality comes from rewriting
\begin{equation}
  \label{eq:rewrite_nu_power}
  {\nu_{A}}^{T F_{A}} \equiv {\nu_{A}}^{T \sum_{a}J_{a\,A}}
  \equiv \prod_{a}{\nu_{A}}^{T J_{a\,A}}.
\end{equation}
Note that the normalization factor is independent of $\yAs$ and only
depends on $\yas$, which is given and fixed in our inference.

\bigskip

We can now return to our initial probability
distribution~\eqref{eq:total_prob_As}. Replacing~\eqref{eq:singular_J} and
\eqref{eq:bayes_sequences_developed} into it and rearranging we have
\begin{equation}
  \label{eq:next-final_expression}
    \pf(\yJ \| \yas, \yI) =
  \frac{1}{Z}
  \int\!\!\di\ynu\;q(\ynu)\;
  \sum_{\yAs}\delt\bigl[\yJ \yeq \yJ(\yas,\yAs)\bigr]\;
  \prod_{a,A} \bigl(G_{a\,A}\,\nu_{A}\bigr)^{T J_{a\,A}(\yas,\yAs)}.
\end{equation}

This expression can be considered as the marginal distribution of the joint
density function
\begin{equation}
  \label{eq:joint_q_J_almost-final}
      \pf(\yJ, \ynu\| \yas, \yI) =
  \frac{1}{Z}\;
  q(\ynu)\;
  \sum_{\yAs}\delt\bigl[\yJ \yeq \yJ(\yas,\yAs)\bigr]\;
  \prod_{a,A} \bigl(G_{a\,A}\,\nu_{A}\bigr)^{T J_{a\,A}(\yas,\yAs)}.
\end{equation}
If we represent this density by a set of Monte Carlo samples for
$(\yJ,\ynu)$, we automatically also have samples for the distribution for
$\yJ$, that for $\yF$, and that for $\ynu$.




\medskip

Let's consider the sum over all possible sequences, $\sum_{\yAs}$, in the
joint density~\eqref{eq:joint_q_J_almost-final}. Owing to the delta in the
summand, the only $\yAs$s that contribute to the sum are those for which
$\yJ(\yas, \yAs)=\yJ$. This also means that all terms in the sum have the
same numerical value, because their values depend on $\yAs$ only through
$\yJ(\yas, \yAs)$, which is fixed. We must therefore only find how many
terms there are in the sum, and multiply the value of one term for the
number of terms.

With $\yas$ and $\yJ$ fixed, we are asking how many $\yAs$s satisfy
$\yJ(\yas, \yAs)=\yJ$. Consider the problem from the following point of
view. We have a grid of boxes with $(n+1)$ rows and $(N+1)$ columns,
indexed by $(a,A)$. A sequence of $T$ balls go into the boxes: if the $t$th
ball goes into the $(a,A)$ box, it means that at the $t$th time bin the
activities of sample and full network were $a_{t}=a$ and $A_{t}=A$. In the
typical combinatorial problem we would ask in how many ways we can fill the
boxes, with $T J_{a\,A}$ balls in the box $(a,A)$, by throwing the $T$
balls. The number would be given by all $T!$ possible permutations of the
balls, but considering permutations of two or more balls within the same
box as equivalent, thus finding the multinomial coefficient
\begin{equation}
  \label{eq:multinomial_coefficient_generic}
  \binom{T}{T\yJ} \equiv
  \binom{T}{T J_{0\,0}, \dotsc, T J_{n\,N}} 
  \defd
  \frac{T!}{\prod_{a,A}(T J_{a\,A})!}.
\end{equation}
In our case, however, we have one constraint: $\yas$ is fixed, which means
that the \emph{row} in which each ball must fall is determined and fixed.
The $t$th ball must perforce fall into row $a_{t}$. In considering all
possible permutations we must therefore exclude those that change the rows
of the balls. This means that only within-row permutations are allowed.
Within each row the counting proceeds as usual, so that for row $a$, which
has a total of $T\sum_{A}J_{a\,A}\equiv T f_{a}$ balls, we have
\begin{equation}
  \label{eq:multinomial_coefficient_row}
  \binom{T f_{a}}{T J_{a\,0}, \dotsc, T J_{a\,N}} =
  \frac{(T f_{a})!}{\prod_{A}(T J_{a\,A})!}.
\end{equation}
We therefore find that the number of terms in the sum
of~\eqref{eq:joint_q_J_almost-final} is
\begin{equation}
  \label{eq:multinomial_coefficient_withinrow}
  \prod_{a}\binom{T f_{a}}{T J_{a\,0}, \dotsc, T J_{a\,N}}.
\end{equation}

If $T$ is large the logarithm of the multinomial coefficient can be
approximated using the Shannon entropy $\ysh$, which for a generic
distribution $(x_{i})$ satisfies the bounds \citep[Lemma 2.2 pp.
429--430 in][]{csiszaretal2004b}
\begin{equation}
  \label{eq:approx_multinomial}
  T \ysh(x_{0},\dotsc,x_{N}) - \ln\binom{T+N}{T} \le
  \ln\binom{T}{Tx_{0},\dotsc,Tx_{N}} \le T \ysh(x_{0},\dotsc,x_{N}).
\end{equation}
We therefore approximate the
multiplicity~\eqref{eq:multinomial_coefficient_withinrow} as
\begin{multline}
  \label{eq:multinomial_coefficient_withinrow_approx}
  \exp\Biggl[T\sum_{a} f_{a}\;\ysh\Biggl(
\frac{J_{a\,0}}{f_{a}}, \dotsc,
\frac{J_{a\,N}}{f_{a}}\Biggr)
    \Biggr] \equiv
  \exp\Biggl[-T\sum_{a} f_{a} \sum_{A}
    \frac{J_{a\,A}}{f_{a}}\ln\frac{J_{a\,A}}{f_{a}}
    \Biggr]
    \equiv{}\\
    \exp\Biggl[-T \sum_{a,A} J_{a\,A}\ln J_{a\,A} + T  \sum_{a} f_{a}\ln f_{a}\Biggr]
    \equiv
    \exp[T\,\ysh(\yJ) - T\,\ysh(\yf)],
\end{multline}
where the second line is obtained by combining the sums, simplifying, and
using the definition of Shannon entropy. \mynote{Is it possible that this
  approximation affects the result? -- Update: the discrepancy is less than
0.1\%.}

\bigskip

The sum over $\yAs$ in formula~\eqref{eq:joint_q_J_almost-final} can
therefore be replaced by any single summand multiplied by the
multiplicity~\eqref{eq:multinomial_coefficient_withinrow_approx} above. The
marginal frequency distribution $\yf$ is still fixed, determined by $\yas$,
so we still have the constraint $\sum_{A}J_{a\,A}=f_{a}$ for each $a$,
which we compactly write $\delt(\tsum\yJ\yeq \yf)$. We find
\begin{multline}
  \label{eq:final_expression_multiplicity}
      \pf(\yJ, \ynu \| \yas, \yI) \\
  \begin{aligned}
  &=\frac{1}{Z}\;\delt(\tsum\yJ \yeq \yf)
  % \int\!\!\di\ynu
  \;q(\ynu)\;
  %\sum_{\yAs}\delt\bigl[\yJ \yeq \yJ(\yas,\yAs)\bigr]\;
  \exp[T\,\ysh(\yJ) - T\,\ysh(\yf)]\;
  \prod_{a,A} \bigl(G_{a\,A}\,\nu_{A}\bigr)^{T J_{a\,A}(\yas,\yAs)}
  \\
     &=
%     \begin{multlined}[t][0.7\linewidth]
  \frac{1}{Z}\;\delt(\tsum\yJ \yeq \yf)
  % \int\!\!\di\ynu
  \;q(\ynu)\;%\times{}\\
  \exp\bigl\{T\bigl[
  \ysh(\yJ) - \ysh(\yf)
  + \tsum_{a,A} J_{a\,A}\ln(G_{a\,A}\,\nu_{A})
  \bigr]\bigr\},
%\end{multlined}
\end{aligned}
\end{multline}
where the second equality comes from re-expressing the product over $(a,A)$
in exponential-logarithm form.

Now we note that $G_{a\,A}\,\nu_{A}$ is a normalized distribution in
$(a,A)$ owing to the properties of the hypergeometric distribution and of
$\ynu$. We denote it $\yG\cdot\ynu$. We can also use the definition of
relative entropy
\begin{equation}
  \label{eq:rel_entropy}
  \ysh[(x_{i}); (y_{i})] \defd \tsum_{i} x_{i} \ln\frac{x_{i}}{y_{i}}
\equiv -\ysh[(x_{i})] -\tsum_{i} x_{i} \ln y_{i},
\end{equation}
to combine the first and last terms within the exponential. The logarithm
of our joint density~\eqref{eq:joint_q_J_almost-final} can then finally be
written as
\begin{empheq}[box=\widefbox]{equation}
  \label{eq:log-density_final}
  \begin{multlined}[][0.9\linewidth]
    \ln \pf(\yJ, \ynu \| \yas, \yI) =
\ln\delt(\tsum\yJ \yeq \yf) + 
    \ln q(\ynu)
  -T\,\ysh(\yJ; \yG\cdot\ynu)
\\[\jot]{}  - T\,\ysh(\yf) - \ln Z(\yas)
%  \text{with $\tsum_{a}\yJ=\yf(\yas)$}
\end{multlined}
\end{empheq}

The last two terms are constants: they only depend on $\yas$ and
$\yf(\yas)$, which are given and fixed in our problem. They can therefore
be discarded in Monte Carlo sampling. In sampling, the delta term is taken
care of by restricting the sampling to the set of allowed $\yJ$.

\medskip

The density $q(\ynu)$ remains to be specified. We use an entropic density
\begin{equation}
  \label{eq:density_nu_entropic}
  q(\ynu) \propto \exp[-L\;\ysh(\ynu;\yr)]
\end{equation}
where $\yr$ is a reference distribution -- we take the uniform one for
simplicity -- and $L$ is of order unity, say less than $10$. This density
is motivated by the consideration of multiplicities: a frequency
distribution, like $\ynu$, can be realized in a number of ways equal to a
multinomial coefficient. This coefficient, by
formula~\eqref{eq:approx_multinomial}, approximately equal to the
exponential of the Shannon entropy of the distribution. So the
density~\eqref{eq:density_nu_entropic} keeps track of this multiplicity,
but regulates its importance via the parameter $L$. We can also choose
$L=0$, which leads to $q(\ynu)$ being uniform in $\ynu$.





%%%% examples use empheq
%   \begin{empheq}[left={\mathllap{\begin{aligned}    \de\yF_{\yc}/\de\yp&=0\text{:} \\
%         \de\yF_{\yc}/\de\ym&=0\text{:}\\ \de\yF_{\yc}/\de\yl&=0\text{:}\end{aligned}}\qquad}\empheqlbrace]{align}
%     \label{eq:con_p}
% %    \de\yF_{\yc}/\de\yp &\equiv
%     -\ln\yp + \ln\yq + \yl\yM + \ym\yu &=0,\\
%     \label{eq:con_u}
% %    \de\yF_{\yc}/\de\ym &\equiv
%     \yu\yp-1 &=0,\\
%     \label{eq:con_l}
%     %\de\yF_{\yc}/\de\yl &\equiv
%     \yM\yp-\yc &=0.
%   \end{empheq}
%%%%
% \begin{empheq}[box=\widefbox]{equation}
%   \label{eq:maxent_question}
%   \p\bigl[\yE{N+1}{k} \bigcond \tsum\yo\yf{N}\in\yA, \yM\bigr] = \mathord{?}
% \end{empheq}



% \[
%   \begin{tikzcd}
%       M_{n,n}(\CC) \arrow{r}{R'_{a}(\Hat{U})} & M_{n,n}(\CC)
%     \\
%     L(\mathcal{H}) \arrow{r}{\Hat{U}} \arrow[swap]{d}{R_*}\arrow[swap]{u}{R'_*} & L(\mathcal{H}) \arrow{d}{R_*}\arrow{u}{R'_*} \\
%       M_{n,n}(\CC) \arrow{r}{R_{a}(\Hat{U})} & M_{n,n}(\CC)
%   \end{tikzcd}
% \]

% \[
%   \begin{tikzcd}
%       \CC^n \arrow{r}{R'_*(A)} & \CC^n
%     \\
%     \mathcal{H} \arrow{r}{A} \arrow[swap]{d}{R}\arrow[swap]{u}{R'} & \mathcal{H} \arrow{d}{R}\arrow{u}{R'} \\
%       \CC^n \arrow{r}{R_*(A)} & \CC^n
%   \end{tikzcd}
% \]


% \[
%   \begin{tikzcd}
%     \mathcal{H} \arrow{r}{A} \arrow[swap]{d}{R} & \mathcal{H} \arrow{d}{R} \\
%       \CC^n \arrow{r}{R_*(A)} & \CC^n
%   \end{tikzcd}
% \]

%%\setlength{\intextsep}{0.5ex}% with wrapfigure
%\begin{figure}[p!]%{r}{0.4\linewidth} % with wrapfigure
%  \centering\includegraphics[trim={12ex 0 18ex 0},clip,width=\linewidth]{maxent_saddle.png}\\
%\caption{caption}\label{fig:comparison_a5}
%\end{figure}% exp_family_maxent.nb


%%%%%%%%%%%%%%%%%%%%%%%%%%%%%%%%%%%%%%%%%%%%%%%%%%%%%%%%%%%%%%%%%%%%%%%%%%%%
%%% Acknowledgements
%%%%%%%%%%%%%%%%%%%%%%%%%%%%%%%%%%%%%%%%%%%%%%%%%%%%%%%%%%%%%%%%%%%%%%%%%%%% 
\begin{acknowledgements}
  \ldots to Mari \amp\ Miri for continuous encouragement and affection, and
  to Buster Keaton and Saitama for filling life with awe and inspiration.
  To the developers and maintainers of \LaTeX, Emacs, AUC\TeX, Open Science
  Framework, R, Python, Inkscape, Sci-Hub for making a free and impartial
  scientific exchange possible.
%\rotatebox{15}{P}\rotatebox{5}{I}\rotatebox{-10}{P}\rotatebox{10}{\reflectbox{P}}\rotatebox{-5}{O}.
%\sourceatright{\autanet}
\mbox{}\hfill\autanet

This work is financially supported by the Kavli Foundation and the Centre
  of Excellence scheme of the  Research Council of Norway (Yasser Roudi group).
\end{acknowledgements}

%%%%%%%%%%%%%%%%%%%%%%%%%%%%%%%%%%%%%%%%%%%%%%%%%%%%%%%%%%%%%%%%%%%%%%%%%%%%
%%% Appendices
%%%%%%%%%%%%%%%%%%%%%%%%%%%%%%%%%%%%%%%%%%%%%%%%%%%%%%%%%%%%%%%%%%%%%%%%%%%% 
\bigskip
% %\renewcommand*{\appendixpagename}{Appendix}
% %\renewcommand*{\appendixname}{Appendix}
% %\appendixpage
% \appendix

%%%%%%%%%%%%%%%%%%%%%%%%%%%%%%%%%%%%%%%%%%%%%%%%%%%%%%%%%%%%%%%%%%%%%%%%%%%%
%%% Bibliography
%%%%%%%%%%%%%%%%%%%%%%%%%%%%%%%%%%%%%%%%%%%%%%%%%%%%%%%%%%%%%%%%%%%%%%%%%%%% 
\renewcommand*{\finalnamedelim}{\addcomma\space}
\defbibnote{prenote}{{\footnotesize (\enquote{de $X$} is listed under D,
    \enquote{van $X$} under V, and so on, regardless of national
    conventions.)\par}}
% \defbibnote{postnote}{\par\medskip\noindent{\footnotesize% Note:
%     \arxivp \mparcp \philscip \biorxivp}}

\printbibliography[prenote=prenote%,postnote=postnote
]

\end{document}

%%%%%%%%%%%%%%%%%%%%%%%%%%%%%%%%%%%%%%%%%%%%%%%%%%%%%%%%%%%%%%%%%%%%%%%%%%%%
%%% Cut text (won't be compiled)
%%%%%%%%%%%%%%%%%%%%%%%%%%%%%%%%%%%%%%%%%%%%%%%%%%%%%%%%%%%%%%%%%%%%%%%%%%%% 


%%% Local Variables: 
%%% mode: LaTeX
%%% TeX-PDF-mode: t
%%% TeX-master: t
%%% End: 
