\pdfoutput=1
%% Author: PGL  Porta Mana
%% Created: 2019-01-11T10:39:40+0100
%% Last-Updated: 2019-07-31T23:10:40+0200
%%%%%%%%%%%%%%%%%%%%%%%%%%%%%%%%%%%%%%%%%%%%%%%%%%%%%%%%%%%%%%%%%%%%%%%%%%%%
\newif\ifarxiv
\arxivfalse
\ifarxiv\pdfmapfile{+classico.map}\fi
\newif\ifafour
\afourfalse% true = A4, false = A5
\newif\iftypodisclaim % typographical disclaim on the side
\typodisclaimtrue
\newcommand*{\memfontfamily}{zplx}
\newcommand*{\memfontpack}{newpxtext}
\documentclass[\ifafour a4paper,12pt,\else a5paper,10pt,\fi%extrafontsizes,%
onecolumn,oneside,article,%french,italian,german,swedish,latin,
british%
]{memoir}
\newcommand*{\updated}{\today}
\newcommand*{\firstdraft}{11 January 2019}
\newcommand*{\firstpublished}{\firstdraft}
\newcommand*{\propertitle}{The simplex and probability%\\{\large ***}%
}% title uses LARGE; set Large for smaller
\newcommand*{\pdftitle}{\propertitle}
\newcommand*{\headtitle}{Simplex and probability}
\newcommand*{\pdfauthor}{P.G.L.  Porta Mana}
\newcommand*{\headauthor}{Porta Mana}
\newcommand*{\reporthead}{}% Report number

%%%%%%%%%%%%%%%%%%%%%%%%%%%%%%%%%%%%%%%%%%%%%%%%%%%%%%%%%%%%%%%%%%%%%%%%%%%%
%%% Calls to packages (uncomment as needed)
%%%%%%%%%%%%%%%%%%%%%%%%%%%%%%%%%%%%%%%%%%%%%%%%%%%%%%%%%%%%%%%%%%%%%%%%%%%%

%\usepackage{pifont}

%\usepackage{fontawesome}

\usepackage[T1]{fontenc} 
\input{glyphtounicode} \pdfgentounicode=1

\usepackage[utf8]{inputenx}

%\usepackage{newunicodechar}
% \newunicodechar{Ĕ}{\u{E}}
% \newunicodechar{ĕ}{\u{e}}
% \newunicodechar{Ĭ}{\u{I}}
% \newunicodechar{ĭ}{\u{\i}}
% \newunicodechar{Ŏ}{\u{O}}
% \newunicodechar{ŏ}{\u{o}}
% \newunicodechar{Ŭ}{\u{U}}
% \newunicodechar{ŭ}{\u{u}}
% \newunicodechar{Ā}{\=A}
% \newunicodechar{ā}{\=a}
% \newunicodechar{Ē}{\=E}
% \newunicodechar{ē}{\=e}
% \newunicodechar{Ī}{\=I}
% \newunicodechar{ī}{\={\i}}
% \newunicodechar{Ō}{\=O}
% \newunicodechar{ō}{\=o}
% \newunicodechar{Ū}{\=U}
% \newunicodechar{ū}{\=u}
% \newunicodechar{Ȳ}{\=Y}
% \newunicodechar{ȳ}{\=y}

\newcommand*{\bmmax}{0} % reduce number of bold fonts, before font packages
\newcommand*{\hmmax}{0} % reduce number of heavy fonts, before font packages

\usepackage{textcomp}

%\usepackage[normalem]{ulem}% package for underlining
% \makeatletter
% \def\ssout{\bgroup \ULdepth=-.35ex%\UL@setULdepth
%  \markoverwith{\lower\ULdepth\hbox
%    {\kern-.03em\vbox{\hrule width.2em\kern1.2\p@\hrule}\kern-.03em}}%
%  \ULon}
% \makeatother

\usepackage{amsmath}

\usepackage{mathtools}
\addtolength{\jot}{\jot} % increase spacing in multiline formulae
\setlength{\multlinegap}{0pt}

%\usepackage{empheq}% automatically calls amsmath and mathtools
\newcommand*{\widefbox}[1]{\fbox{\hspace{1em}#1\hspace{1em}}}

%\usepackage{fancybox}

%\usepackage{framed}

% \usepackage[misc]{ifsym} % for dice
% \newcommand*{\diceone}{{\scriptsize\Cube{1}}}

\usepackage{amssymb}

\usepackage{amsxtra}

\usepackage[main=british,french,italian,german,swedish,latin,esperanto]{babel}\selectlanguage{british}
\newcommand*{\langfrench}{\foreignlanguage{french}}
\newcommand*{\langgerman}{\foreignlanguage{german}}
\newcommand*{\langitalian}{\foreignlanguage{italian}}
\newcommand*{\langswedish}{\foreignlanguage{swedish}}
\newcommand*{\langlatin}{\foreignlanguage{latin}}
\newcommand*{\langnohyph}{\foreignlanguage{nohyphenation}}

\usepackage[autostyle=false,autopunct=false,english=british]{csquotes}
\setquotestyle{british}

\usepackage{amsthm}
\newcommand*{\QED}{\textsc{q.e.d.}}
\renewcommand*{\qedsymbol}{\QED}
\theoremstyle{remark}
\newtheorem{note}{Note}
\newtheorem*{remark}{Note}
\newtheoremstyle{innote}{\parsep}{\parsep}{\footnotesize}{}{}{}{0pt}{}
\theoremstyle{innote}
\newtheorem*{innote}{}

\usepackage[shortlabels,inline]{enumitem}
\SetEnumitemKey{para}{itemindent=\parindent,leftmargin=0pt,listparindent=\parindent,parsep=0pt,itemsep=\topsep}
% \begin{asparaenum} = \begin{enumerate}[para]
% \begin{inparaenum} = \begin{enumerate*}
\setlist[enumerate,2]{label=\alph*.}
\setlist[enumerate]{label=\arabic*.,leftmargin=1.5\parindent,topsep=0.5\baselineskip}
\setlist[itemize]{leftmargin=1.5\parindent,topsep=0.5\baselineskip}
\setlist[description]{leftmargin=1.5\parindent,topsep=0.5\baselineskip}
% old alternative:
% \setlist[enumerate,2]{label=\alph*.}
% \setlist[enumerate]{leftmargin=\parindent}
% \setlist[itemize]{leftmargin=\parindent}
% \setlist[description]{leftmargin=\parindent}

\usepackage[babel,theoremfont,largesc]{newpxtext}

\usepackage[bigdelims,nosymbolsc%,smallerops % probably arXiv doesn't have it
]{newpxmath}
\linespread{1.083}%\useosf
%% smaller operators for old version of newpxmath
\makeatletter
\def\re@DeclareMathSymbol#1#2#3#4{%
    \let#1=\undefined
    \DeclareMathSymbol{#1}{#2}{#3}{#4}}
%\re@DeclareMathSymbol{\bigsqcupop}{\mathop}{largesymbols}{"46}
%\re@DeclareMathSymbol{\bigodotop}{\mathop}{largesymbols}{"4A}
\re@DeclareMathSymbol{\bigoplusop}{\mathop}{largesymbols}{"4C}
\re@DeclareMathSymbol{\bigotimesop}{\mathop}{largesymbols}{"4E}
\re@DeclareMathSymbol{\sumop}{\mathop}{largesymbols}{"50}
\re@DeclareMathSymbol{\prodop}{\mathop}{largesymbols}{"51}
\re@DeclareMathSymbol{\bigcupop}{\mathop}{largesymbols}{"53}
\re@DeclareMathSymbol{\bigcapop}{\mathop}{largesymbols}{"54}
%\re@DeclareMathSymbol{\biguplusop}{\mathop}{largesymbols}{"55}
\re@DeclareMathSymbol{\bigwedgeop}{\mathop}{largesymbols}{"56}
\re@DeclareMathSymbol{\bigveeop}{\mathop}{largesymbols}{"57}
%\re@DeclareMathSymbol{\bigcupdotop}{\mathop}{largesymbols}{"DF}
%\re@DeclareMathSymbol{\bigcapplusop}{\mathop}{largesymbolsPXA}{"00}
%\re@DeclareMathSymbol{\bigsqcupplusop}{\mathop}{largesymbolsPXA}{"02}
%\re@DeclareMathSymbol{\bigsqcapplusop}{\mathop}{largesymbolsPXA}{"04}
%\re@DeclareMathSymbol{\bigsqcapop}{\mathop}{largesymbolsPXA}{"06}
\re@DeclareMathSymbol{\bigtimesop}{\mathop}{largesymbolsPXA}{"10}
%\re@DeclareMathSymbol{\coprodop}{\mathop}{largesymbols}{"60}
%\re@DeclareMathSymbol{\varprod}{\mathop}{largesymbolsPXA}{16}
\makeatother
%%
%% With euler font cursive for Greek letters - the [1] means 100% scaling
\DeclareFontFamily{U}{egreek}{\skewchar\font'177}%
\DeclareFontShape{U}{egreek}{m}{n}{<-6>s*[1]eurm5 <6-8>s*[1]eurm7 <8->s*[1]eurm10}{}%
\DeclareFontShape{U}{egreek}{m}{it}{<->s*[1]eurmo10}{}%
\DeclareFontShape{U}{egreek}{b}{n}{<-6>s*[1]eurb5 <6-8>s*[1]eurb7 <8->s*[1]eurb10}{}%
\DeclareFontShape{U}{egreek}{b}{it}{<->s*[1]eurbo10}{}%
\DeclareSymbolFont{egreeki}{U}{egreek}{m}{it}%
\SetSymbolFont{egreeki}{bold}{U}{egreek}{b}{it}% from the amsfonts package
\DeclareSymbolFont{egreekr}{U}{egreek}{m}{n}%
\SetSymbolFont{egreekr}{bold}{U}{egreek}{b}{n}% from the amsfonts package
% Take also \sum, \prod, \coprod symbols from Euler fonts
\DeclareFontFamily{U}{egreekx}{\skewchar\font'177}
\DeclareFontShape{U}{egreekx}{m}{n}{%
       <-7.5>s*[0.9]euex7%
    <7.5-8.5>s*[0.9]euex8%
    <8.5-9.5>s*[0.9]euex9%
    <9.5->s*[0.9]euex10%
}{}
\DeclareSymbolFont{egreekx}{U}{egreekx}{m}{n}
\DeclareMathSymbol{\sumop}{\mathop}{egreekx}{"50}
\DeclareMathSymbol{\prodop}{\mathop}{egreekx}{"51}
\DeclareMathSymbol{\coprodop}{\mathop}{egreekx}{"60}
\makeatletter
\def\sum{\DOTSI\sumop\slimits@}
\def\prod{\DOTSI\prodop\slimits@}
\def\coprod{\DOTSI\coprodop\slimits@}
\makeatother
\input{definegreek.tex}% Greek letters not usually given in LaTeX.

%\usepackage%[scaled=0.9]%
%{classico}%  Optima as sans-serif font
\renewcommand\sfdefault{uop}
\DeclareMathAlphabet{\mathsf}  {T1}{\sfdefault}{m}{sl}
\SetMathAlphabet{\mathsf}{bold}{T1}{\sfdefault}{b}{sl}
\newcommand*{\mathte}[1]{\textbf{\textit{\textsf{#1}}}}
% Upright sans-serif math alphabet
% \DeclareMathAlphabet{\mathsu}  {T1}{\sfdefault}{m}{n}
% \SetMathAlphabet{\mathsu}{bold}{T1}{\sfdefault}{b}{n}

% DejaVu Mono as typewriter text
\usepackage[scaled=0.84]{DejaVuSansMono}

\usepackage{mathdots}

\usepackage[usenames]{xcolor}
% Tol (2012) colour-blind-, print-, screen-friendly colours, alternative scheme; Munsell terminology
\definecolor{mypurpleblue}{RGB}{68,119,170}
\definecolor{myblue}{RGB}{102,204,238}
\definecolor{mygreen}{RGB}{34,136,51}
\definecolor{myyellow}{RGB}{204,187,68}
\definecolor{myred}{RGB}{238,102,119}
\definecolor{myredpurple}{RGB}{170,51,119}
\definecolor{mygrey}{RGB}{187,187,187}
% Tol (2012) colour-blind-, print-, screen-friendly colours; Munsell terminology
% \definecolor{lbpurple}{RGB}{51,34,136}
% \definecolor{lblue}{RGB}{136,204,238}
% \definecolor{lbgreen}{RGB}{68,170,153}
% \definecolor{lgreen}{RGB}{17,119,51}
% \definecolor{lgyellow}{RGB}{153,153,51}
% \definecolor{lyellow}{RGB}{221,204,119}
% \definecolor{lred}{RGB}{204,102,119}
% \definecolor{lpred}{RGB}{136,34,85}
% \definecolor{lrpurple}{RGB}{170,68,153}
\definecolor{lgrey}{RGB}{221,221,221}
%\newcommand*\mycolourbox[1]{%
%\colorbox{mygrey}{\hspace{1em}#1\hspace{1em}}}
\colorlet{shadecolor}{lgrey}

\usepackage{bm}

\usepackage{microtype}

\usepackage[backend=biber,mcite,%subentry,
citestyle=authoryear-comp,bibstyle=pglpm-authoryear,autopunct=false,sorting=ny,sortcites=false,natbib=false,maxcitenames=1,maxbibnames=8,minbibnames=8,giveninits=true,uniquename=false,uniquelist=false,maxalphanames=1,block=space,hyperref=true,defernumbers=false,useprefix=true,sortupper=false,language=british,parentracker=false]{biblatex}
\DeclareSortingScheme{ny}{\sort{\field{sortname}\field{author}\field{editor}}\sort{\field{year}}}
\iffalse\makeatletter%%% replace parenthesis with brackets
\newrobustcmd*{\parentexttrack}[1]{%
  \begingroup
  \blx@blxinit
  \blx@setsfcodes
  \blx@bibopenparen#1\blx@bibcloseparen
  \endgroup}
\AtEveryCite{%
  \let\parentext=\parentexttrack%
  \let\bibopenparen=\bibopenbracket%
  \let\bibcloseparen=\bibclosebracket}
\makeatother\fi
\DefineBibliographyExtras{british}{\def\finalandcomma{\addcomma}}
\renewcommand*{\finalnamedelim}{\addcomma\space}
\setcounter{biburlnumpenalty}{1}
\setcounter{biburlucpenalty}{0}
\setcounter{biburllcpenalty}{1}
\DeclareDelimFormat{multicitedelim}{\addsemicolon\space}
\DeclareDelimFormat{compcitedelim}{\addsemicolon\space}
\DeclareDelimFormat{postnotedelim}{\space}
\ifarxiv\else\addbibresource{portamanabib.bib}\fi
\renewcommand{\bibfont}{\footnotesize}
%\appto{\citesetup}{\footnotesize}% smaller font for citations
\defbibheading{bibliography}[\bibname]{\section*{#1}\addcontentsline{toc}{section}{#1}%\markboth{#1}{#1}
}
\newcommand*{\citep}{\parencites}
\newcommand*{\citey}{\parencites*}
%\renewcommand*{\cite}{\parencite}
\renewcommand*{\cites}{\parencites}
\providecommand{\href}[2]{#2}
\providecommand{\eprint}[2]{\texttt{\href{#1}{#2}}}
\newcommand*{\amp}{\&}
% \newcommand*{\citein}[2][]{\textnormal{\textcite[#1]{#2}}%\addtocategory{extras}{#2}
% }
\newcommand*{\citein}[2][]{\textnormal{\textcite[#1]{#2}}%\addtocategory{extras}{#2}
}
\newcommand*{\citebi}[2][]{\textcite[#1]{#2}%\addtocategory{extras}{#2}
}
\newcommand*{\subtitleproc}[1]{}
\newcommand*{\chapb}{ch.}
%
% \def\arxivp{}
% \def\mparcp{}
% \def\philscip{}
% \def\biorxivp{}
% \newcommand*{\arxivsi}{\texttt{arXiv} eprints available at \url{http://arxiv.org/}.\\}
% \newcommand*{\mparcsi}{\texttt{mp\_arc} eprints available at \url{http://www.ma.utexas.edu/mp_arc/}.\\}
% \newcommand*{\philscisi}{\texttt{philsci} eprints available at \url{http://philsci-archive.pitt.edu/}.\\}
% \newcommand*{\biorxivsi}{\texttt{bioRxiv} eprints available at \url{http://biorxiv.org/}.\\}
\newcommand*{\arxiveprint}[1]{%\global\def\arxivp{\arxivsi}%\citeauthor{0arxivcite}\addtocategory{ifarchcit}{0arxivcite}%eprint
\texttt{\urlalt{https://arxiv.org/abs/#1}{arXiv:\hspace{0pt}#1}}%
%\texttt{\href{http://arxiv.org/abs/#1}{\protect\url{arXiv:#1}}}%
%\renewcommand{\arxivnote}{\texttt{arXiv} eprints available at \url{http://arxiv.org/}.}
}
\newcommand*{\haleprint}[1]{%\global\def\arxivp{\arxivsi}%\citeauthor{0arxivcite}\addtocategory{ifarchcit}{0arxivcite}%eprint
\texttt{\urlalt{https://hal.archives-ouvertes.fr/#1}{HAL:\hspace{0pt}#1}}%
%\texttt{\href{http://arxiv.org/abs/#1}{\protect\url{arXiv:#1}}}%
%\renewcommand{\arxivnote}{\texttt{arXiv} eprints available at \url{http://arxiv.org/}.}
}
\newcommand*{\mparceprint}[1]{%\global\def\mparcp{\mparcsi}%\citeauthor{0mparccite}\addtocategory{ifarchcit}{0mparccite}%eprint
\texttt{\urlalt{http://www.ma.utexas.edu/mp_arc-bin/mpa?yn=#1}{mp\_arc:\hspace{0pt}#1}}%
%\texttt{\href{http://www.ma.utexas.edu/mp_arc-bin/mpa?yn=#1}{\protect\url{mp_arc:#1}}}%
%\providecommand{\mparcnote}{\texttt{mp_arc} eprints available at \url{http://www.ma.utexas.edu/mp_arc/}.}
}
\newcommand*{\philscieprint}[1]{%\global\def\philscip{\philscisi}%\citeauthor{0philscicite}\addtocategory{ifarchcit}{0philscicite}%eprint
\texttt{\urlalt{http://philsci-archive.pitt.edu/archive/#1}{PhilSci:\hspace{0pt}#1}}%
%\texttt{\href{http://philsci-archive.pitt.edu/archive/#1}{\protect\url{PhilSci:#1}}}%
%\providecommand{\mparcnote}{\texttt{philsci} eprints available at \url{http://philsci-archive.pitt.edu/}.}
}
\newcommand*{\biorxiveprint}[1]{%\global\def\biorxivp{\biorxivsi}%\citeauthor{0arxivcite}\addtocategory{ifarchcit}{0arxivcite}%eprint
\texttt{\urlalt{https://doi.org/10.1101/#1}{bioRxiv doi:\hspace{0pt}10.1101/#1}}%
%\texttt{\href{http://arxiv.org/abs/#1}{\protect\url{arXiv:#1}}}%
%\renewcommand{\arxivnote}{\texttt{arXiv} eprints available at \url{http://arxiv.org/}.}
}
\newcommand*{\osfeprint}[1]{%
\texttt{\urlalt{https://doi.org/10.17605/osf.io/#1}{Open Science Framework doi:10.17605/osf.io/#1}}%
}

\usepackage{graphicx}

%\usepackage{wrapfig}

%\usepackage{tikz-cd}

\PassOptionsToPackage{hyphens}{url}\usepackage[hypertexnames=false]{hyperref}

\usepackage[depth=4]{bookmark}
\hypersetup{colorlinks=true,bookmarksnumbered,pdfborder={0 0 0.25},citebordercolor={0.2667 0.4667 0.6667},citecolor=mypurpleblue,linkbordercolor={0.6667 0.2 0.4667},linkcolor=myredpurple,urlbordercolor={0.1333 0.5333 0.2},urlcolor=mygreen,breaklinks=true,pdftitle={\pdftitle},pdfauthor={\pdfauthor}}
% \usepackage[vertfit=local]{breakurl}% only for arXiv
\providecommand*{\urlalt}{\href}

\usepackage[british]{datetime2}
\DTMnewdatestyle{mydate}%
{% definitions
\renewcommand*{\DTMdisplaydate}[4]{%
\number##3\ \DTMenglishmonthname{##2} ##1}%
\renewcommand*{\DTMDisplaydate}{\DTMdisplaydate}%
}
\DTMsetdatestyle{mydate}

%%%%%%%%%%%%%%%%%%%%%%%%%%%%%%%%%%%%%%%%%%%%%%%%%%%%%%%%%%%%%%%%%%%%%%%%%%%%
%%% Layout. I do not know on which kind of paper the reader will print the
%%% paper on (A4? letter? one-sided? double-sided?). So I choose A5, which
%%% provides a good layout for reading on screen and save paper if printed
%%% two pages per sheet. Average length line is 66 characters and page
%%% numbers are centred.
%%%%%%%%%%%%%%%%%%%%%%%%%%%%%%%%%%%%%%%%%%%%%%%%%%%%%%%%%%%%%%%%%%%%%%%%%%%%
\ifafour\setstocksize{297mm}{210mm}%{*}% A4
\else\setstocksize{210mm}{5.5in}%{*}% 210x139.7
\fi
\settrimmedsize{\stockheight}{\stockwidth}{*}
\setlxvchars[\normalfont] %313.3632pt for a 66-characters line
\setxlvchars[\normalfont]
\setlength{\trimtop}{0pt}
\setlength{\trimedge}{\stockwidth}
\addtolength{\trimedge}{-\paperwidth}
% The length of the normalsize alphabet is 133.05988pt - 10 pt = 26.1408pc
% The length of the normalsize alphabet is 159.6719pt - 12pt = 30.3586pc
% Bringhurst gives 32pc as boundary optimal with 69 ch per line
% The length of the normalsize alphabet is 191.60612pt - 14pt = 35.8634pc
\ifafour\settypeblocksize{*}{32pc}{1.618} % A4
%\setulmargins{*}{*}{1.667}%gives 5/3 margins % 2 or 1.667
\else\settypeblocksize{*}{26pc}{1.618}% nearer to a 66-line newpx and preserves GR
\fi
\setulmargins{*}{*}{1}%gives equal margins
\setlrmargins{*}{*}{*}
\setheadfoot{\onelineskip}{2.5\onelineskip}
\setheaderspaces{*}{2\onelineskip}{*}
\setmarginnotes{2ex}{10mm}{0pt}
\checkandfixthelayout[nearest]
\fixpdflayout
%%% End layout
%% this fixes missing white spaces
\pdfmapline{+dummy-space <dummy-space.pfb}\pdfinterwordspaceon%

%%% Sectioning
\newcommand*{\asudedication}[1]{%
{\par\centering\textit{#1}\par}}
\newenvironment{acknowledgements}{\section*{Thanks}\addcontentsline{toc}{section}{Thanks}}{\par}
\makeatletter\renewcommand{\appendix}{\par
  \bigskip{\centering
   \interlinepenalty \@M
   \normalfont
   \printchaptertitle{\sffamily\appendixpagename}\par}
  \setcounter{section}{0}%
  \gdef\@chapapp{\appendixname}%
  \gdef\thesection{\@Alph\c@section}%
  \anappendixtrue}\makeatother
\counterwithout{section}{chapter}
\setsecnumformat{\upshape\csname the#1\endcsname\quad}
\setsecheadstyle{\large\bfseries\sffamily%
\centering}
\setsubsecheadstyle{\bfseries\sffamily%
\raggedright}
%\setbeforesecskip{-1.5ex plus 1ex minus .2ex}% plus 1ex minus .2ex}
%\setaftersecskip{1.3ex plus .2ex }% plus 1ex minus .2ex}
%\setsubsubsecheadstyle{\bfseries\sffamily\slshape\raggedright}
%\setbeforesubsecskip{1.25ex plus 1ex minus .2ex }% plus 1ex minus .2ex}
%\setaftersubsecskip{-1em}%{-0.5ex plus .2ex}% plus 1ex minus .2ex}
\setsubsecindent{0pt}%0ex plus 1ex minus .2ex}
\setparaheadstyle{\bfseries\sffamily%
\raggedright}
\setcounter{secnumdepth}{2}
\setlength{\headwidth}{\textwidth}
\newcommand{\addchap}[1]{\chapter*[#1]{#1}\addcontentsline{toc}{chapter}{#1}}
\newcommand{\addsec}[1]{\section*{#1}\addcontentsline{toc}{section}{#1}}
\newcommand{\addsubsec}[1]{\subsection*{#1}\addcontentsline{toc}{subsection}{#1}}
\newcommand{\addpara}[1]{\paragraph*{#1.}\addcontentsline{toc}{subsubsection}{#1}}
\newcommand{\addparap}[1]{\paragraph*{#1}\addcontentsline{toc}{subsubsection}{#1}}

%%% Headers, footers, pagestyle
\copypagestyle{manaart}{plain}
\makeheadrule{manaart}{\headwidth}{0.5\normalrulethickness}
\makeoddhead{manaart}{%
{\footnotesize%\sffamily%
\scshape\headauthor}}{}{{\footnotesize\sffamily%
\headtitle}}
\makeoddfoot{manaart}{}{\thepage}{}
\newcommand*\autanet{\includegraphics[height=\heightof{M}]{autanet.pdf}}
\definecolor{mygray}{gray}{0.333}
\iftypodisclaim%
\ifafour\newcommand\addprintnote{\begin{picture}(0,0)%
\put(245,149){\makebox(0,0){\rotatebox{90}{\tiny\color{mygray}\textsf{This
            document is designed for screen reading and
            two-up printing on A4 or Letter paper}}}}%
\end{picture}}% A4
\else\newcommand\addprintnote{\begin{picture}(0,0)%
\put(176,112){\makebox(0,0){\rotatebox{90}{\tiny\color{mygray}\textsf{This
            document is designed for screen reading and
            two-up printing on A4 or Letter paper}}}}%
\end{picture}}\fi%afourtrue
\makeoddfoot{plain}{}{\makebox[0pt]{\thepage}\addprintnote}{}
\else
\makeoddfoot{plain}{}{\makebox[0pt]{\thepage}}{}
\fi%typodisclaimtrue
\makeoddhead{plain}{}{}{\footnotesize\reporthead}
% \copypagestyle{manainitial}{plain}
% \makeheadrule{manainitial}{\headwidth}{0.5\normalrulethickness}
% \makeoddhead{manainitial}{%
% \footnotesize\sffamily%
% \scshape\headauthor}{}{\footnotesize\sffamily%
% \headtitle}
% \makeoddfoot{manaart}{}{\thepage}{}

\pagestyle{manaart}

\setlength{\droptitle}{-3.9\onelineskip}
\pretitle{\begin{center}\LARGE\sffamily%
\bfseries}
\posttitle{\bigskip\end{center}}

\makeatletter\newcommand*{\atf}{\includegraphics[%trim=1pt 1pt 0pt 0pt,
totalheight=\heightof{@}]{atblack.png}}\makeatother
\providecommand{\affiliation}[1]{\textsl{\textsf{\footnotesize #1}}}
\providecommand{\epost}[1]{\texttt{\footnotesize\textless#1\textgreater}}
\providecommand{\email}[2]{\href{mailto:#1ZZ@#2 ((remove ZZ))}{#1\protect\atf#2}}

\preauthor{\vspace{-0.5\baselineskip}\begin{center}
\normalsize\sffamily%
\lineskip  0.5em}
\postauthor{\par\end{center}}
\predate{\DTMsetdatestyle{mydate}\begin{center}\footnotesize}
\postdate{\end{center}\vspace{-\medskipamount}}

\setfloatadjustment{figure}{\footnotesize}
\captiondelim{\quad}
\captionnamefont{\footnotesize\sffamily%
}
\captiontitlefont{\footnotesize}
\firmlists*
\midsloppy
% handling orphan/widow lines, memman.pdf
% \clubpenalty=10000
% \widowpenalty=10000
% \raggedbottom
% Downes, memman.pdf
\clubpenalty=9996
\widowpenalty=9999
\brokenpenalty=4991
\predisplaypenalty=10000
\postdisplaypenalty=1549
\displaywidowpenalty=1602
\raggedbottom
\selectlanguage{british}\frenchspacing

%%%%%%%%%%%%%%%%%%%%%%%%%%%%%%%%%%%%%%%%%%%%%%%%%%%%%%%%%%%%%%%%%%%%%%%%%%%%
%%% Paper's details
%%%%%%%%%%%%%%%%%%%%%%%%%%%%%%%%%%%%%%%%%%%%%%%%%%%%%%%%%%%%%%%%%%%%%%%%%%%%
\title{\propertitle}
\author{%
\hspace*{\stretch{1}}%
%% uncomment if additional authors present
% \parbox{0.5\linewidth}%\makebox[0pt][c]%
% {\protect\centering ***\\%
% \footnotesize\epost{\email{***}{***}}}%
% \hspace*{\stretch{1}}%
\parbox{0.75\linewidth}%\makebox[0pt][c]%
{\protect\centering P.G.L.  Porta Mana\\%
\footnotesize Kavli Institute, Trondheim \quad \epost{\email{pgl}{portamana.org}}}%
\hspace*{\stretch{1}}%
%\quad\href{https://orcid.org/0000-0002-6070-0784}{\protect\includegraphics[scale=0.16]{orcid_32x32.png}\textsc{orcid}:0000-0002-6070-0784}%
}

\date{\firstpublished; updated \updated}

%%%%%%%%%%%%%%%%%%%%%%%%%%%%%%%%%%%%%%%%%%%%%%%%%%%%%%%%%%%%%%%%%%%%%%%%%%%%
%%% Macros @@@
%%%%%%%%%%%%%%%%%%%%%%%%%%%%%%%%%%%%%%%%%%%%%%%%%%%%%%%%%%%%%%%%%%%%%%%%%%%%

% Common ones - uncomment as needed
%\providecommand{\nequiv}{\not\equiv}
%\providecommand{\coloneqq}{\mathrel{\mathop:}=}
%\providecommand{\eqqcolon}{=\mathrel{\mathop:}}
%\providecommand{\varprod}{\prod}
\newcommand*{\de}{\partialup}%partial diff
\newcommand*{\pu}{\piup}%constant pi
\newcommand*{\delt}{\deltaup}%Kronecker, Dirac
%\newcommand*{\eps}{\varepsilonup}%Levi-Civita, Heaviside
%\newcommand*{\riem}{\zetaup}%Riemann zeta
%\providecommand{\degree}{\textdegree}% degree
%\newcommand*{\celsius}{\textcelsius}% degree Celsius
%\newcommand*{\micro}{\textmu}% degree Celsius
\newcommand*{\I}{\mathrm{i}}%imaginary unit
\newcommand*{\e}{\mathrm{e}}%Neper
\newcommand*{\di}{\mathrm{d}}%differential
%\newcommand*{\Di}{\mathrm{D}}%capital differential
%\newcommand*{\planckc}{\hslash}
%\newcommand*{\avogn}{N_{\textrm{A}}}
%\newcommand*{\NN}{\bm{\mathrm{N}}}
%\newcommand*{\ZZ}{\bm{\mathrm{Z}}}
%\newcommand*{\QQ}{\bm{\mathrm{Q}}}
\newcommand*{\RR}{\bm{\mathrm{R}}}
%\newcommand*{\CC}{\bm{\mathrm{C}}}
%\newcommand*{\nabl}{\bm{\nabla}}%nabla
%\DeclareMathOperator{\lb}{lb}%base 2 log
%\DeclareMathOperator{\tr}{tr}%trace
%\DeclareMathOperator{\card}{card}%cardinality
%\DeclareMathOperator{\im}{Im}%im part
%\DeclareMathOperator{\re}{Re}%re part
%\DeclareMathOperator{\sgn}{sgn}%signum
%\DeclareMathOperator{\ent}{ent}%integer less or equal to
%\DeclareMathOperator{\Ord}{O}%same order as
%\DeclareMathOperator{\ord}{o}%lower order than
%\newcommand*{\incr}{\triangle}%finite increment
\newcommand*{\defd}{\coloneqq}
\newcommand*{\defs}{\eqqcolon}
%\newcommand*{\Land}{\bigwedge}
%\newcommand*{\Lor}{\bigvee}
%\newcommand*{\lland}{\DOTSB\;\land\;}
%\newcommand*{\llor}{\DOTSB\;\lor\;}
%\newcommand*{\limplies}{\mathbin{\Rightarrow}}%implies
\newcommand*{\suchthat}{\mid}%{\mathpunct{|}}%such that (eg in sets)
%\newcommand*{\with}{\colon}%with (list of indices)
%\newcommand*{\mul}{\times}%multiplication
%\newcommand*{\inn}{\cdot}%inner product
%\newcommand*{\dotv}{\mathord{\,\cdot\,}}%variable place
%\newcommand*{\comp}{\circ}%composition of functions
%\newcommand*{\con}{\mathbin{:}}%scal prod of tensors
%\newcommand*{\equi}{\sim}%equivalent to 
\renewcommand*{\asymp}{\simeq}%equivalent to 
%\newcommand*{\corr}{\mathrel{\hat{=}}}%corresponds to
%\providecommand{\varparallel}{\ensuremath{\mathbin{/\mkern-7mu/}}}%parallel (tentative symbol)
\renewcommand*{\le}{\leqslant}%less or equal
\renewcommand*{\ge}{\geqslant}%greater or equal
%\DeclarePairedDelimiter\clcl{[}{]}
\DeclarePairedDelimiter\clop{[}{[}
%\DeclarePairedDelimiter\opcl{]}{]}
%\DeclarePairedDelimiter\opop{]}{[}
\DeclarePairedDelimiter\abs{\lvert}{\rvert}
%\DeclarePairedDelimiter\norm{\lVert}{\rVert}
\DeclarePairedDelimiter\set{\{}{\}}
%\DeclareMathOperator{\pr}{P}%probability
\newcommand*{\pf}{\mathrm{p}}%probability
\newcommand*{\p}{\mathrm{P}}%probability
\newcommand*{\E}{\mathrm{E}}
%\renewcommand*{\|}{\nonscript\,\vert\nonscript\;\mathopen{}}
\renewcommand*{\|}[1][]{\nonscript\,#1\vert\nonscript\;\mathopen{}}
%\DeclarePairedDelimiterX{\cond}[2]{(}{)}{#1\nonscript\,\delimsize\vert\nonscript\;\mathopen{}#2}
%\DeclarePairedDelimiterX{\condt}[2]{[}{]}{#1\nonscript\,\delimsize\vert\nonscript\;\mathopen{}#2}
%\DeclarePairedDelimiterX{\conds}[2]{\{}{\}}{#1\nonscript\,\delimsize\vert\nonscript\;\mathopen{}#2}
%\newcommand*{\+}{\lor}
%\renewcommand{\*}{\land}
\newcommand*{\sect}{\S}% Sect.~
\newcommand*{\sects}{\S\S}% Sect.~
\newcommand*{\chap}{ch.}%
\newcommand*{\chaps}{chs}%
\newcommand*{\bref}{ref.}%
\newcommand*{\brefs}{refs}%
%\newcommand*{\fn}{fn}%
\newcommand*{\eqn}{eq.}%
\newcommand*{\eqns}{eqs}%
\newcommand*{\fig}{fig.}%
\newcommand*{\figs}{figs}%
\newcommand*{\vs}{{vs}}
%\newcommand*{\etc}{{etc.}}
%\newcommand*{\ie}{{i.e.}}
%\newcommand*{\ca}{{c.}}
%\newcommand*{\eg}{{e.g.}}
\newcommand*{\foll}{{ff.}}
%\newcommand*{\viz}{{viz}}
\newcommand*{\cf}{{cf.}}
%\newcommand*{\Cf}{{Cf.}}
%\newcommand*{\vd}{{v.}}
\newcommand*{\etal}{{et al.}}
%\newcommand*{\etsim}{{et sim.}}
%\newcommand*{\ibid}{{ibid.}}
%\newcommand*{\sic}{{sic}}
%\newcommand*{\id}{\mathte{I}}%id matrix
%\newcommand*{\nbd}{\nobreakdash}%
%\newcommand*{\bd}{\hspace{0pt}}%
%\def\hy{-\penalty0\hskip0pt\relax}
%\newcommand*{\labelbis}[1]{\tag*{(\ref{#1})$_\text{r}$}}
%\newcommand*{\mathbox}[2][.8]{\parbox[t]{#1\columnwidth}{#2}}
%\newcommand*{\zerob}[1]{\makebox[0pt][l]{#1}}
\newcommand*{\tprod}{\mathop{\textstyle\prod}\nolimits}
\newcommand*{\tsum}{\mathop{\textstyle\sum}\nolimits}
%\newcommand*{\tint}{\begingroup\textstyle\int\endgroup\nolimits}
%\newcommand*{\tland}{\mathop{\textstyle\bigwedge}\nolimits}
%\newcommand*{\tlor}{\mathop{\textstyle\bigvee}\nolimits}
%\newcommand*{\sprod}{\mathop{\textstyle\prod}}
%\newcommand*{\ssum}{\mathop{\textstyle\sum}}
%\newcommand*{\sint}{\begingroup\textstyle\int\endgroup}
%\newcommand*{\sland}{\mathop{\textstyle\bigwedge}}
%\newcommand*{\slor}{\mathop{\textstyle\bigvee}}
\newcommand*{\T}{^{\mathord{\!\intercal}}}%transpose
%%\newcommand*{\QEM}%{\textnormal{$\Box$}}%{\ding{167}}
%\newcommand*{\qem}{\leavevmode\unskip\penalty9999 \hbox{}\nobreak\hfill
%\quad\hbox{\QEM}}

%%%%%%%%%%%%%%%%%%%%%%%%%%%%%%%%%%%%%%%%%%%%%%%%%%%%%%%%%%%%%%%%%%%%%%%%%%%%
%%% Custom macros for this file @@@
%%%%%%%%%%%%%%%%%%%%%%%%%%%%%%%%%%%%%%%%%%%%%%%%%%%%%%%%%%%%%%%%%%%%%%%%%%%%
 \definecolor{notecolour}{RGB}{68,170,153}
\newcommand*{\puzzle}{{\fontencoding{U}\fontfamily{fontawesometwo}\selectfont\symbol{225}}}
%\newcommand*{\puzzle}{\maltese}
\newcommand{\mynote}[1]{ {\color{notecolour}\puzzle\ #1}}
\newcommand*{\widebar}[1]{{\mkern1.5mu\skew{2}\overline{\mkern-1.5mu#1\mkern-1.5mu}\mkern 1.5mu}}

% \newcommand{\explanation}[4][t]{%\setlength{\tabcolsep}{-1ex}
% %\smash{
% \begin{tabular}[#1]{c}#2\\[0.5\jot]\rule{1pt}{#3}\\#4\end{tabular}}%}
% \newcommand*{\ptext}[1]{\text{\small #1}}
%\DeclareMathOperator*{\argsup}{arg\,sup}
\newcommand*{\dob}{degree of belief}
\newcommand*{\dobs}{degrees of belief}
\newcommand*{\dd}{\textsc{d}}
\newcommand*{\yq}{\bm{q}}
\newcommand*{\ysim}{\varDelta}
\newcommand*{\ysum}{\bm{\tsum}}
\newcommand*{\sh}{\Eta}
\newcommand*{\yL}{L}
\newcommand*{\yu}{\bm{u}}
\newcommand*{\yuu}{\mathte{U}} % this is the "u-u" one
\newcommand*{\yui}{\mathte{U}^{\mathord{\parallel}}}
\newcommand*{\yid}{\mathte{I}}
\newcommand*{\yx}{\bm{x}}
\newcommand*{\ymm}{\mathte{M}}
\newcommand*{\yqr}{\yq^{*}}
\newcommand*{\ySigma}{\bm{\varSigma}}
\newcommand*{\ySigmab}{\Hat{\ySigma}\vphantom{\ySigma}}
\newcommand*{\yDelta}{\bm{\varDelta}}
\newcommand*{\yya}{\bm{h}}
\newcommand*{\yyb}{k}
%%% Custom macros end @@@

%%%%%%%%%%%%%%%%%%%%%%%%%%%%%%%%%%%%%%%%%%%%%%%%%%%%%%%%%%%%%%%%%%%%%%%%%%%%
%%% Beginning of document
%%%%%%%%%%%%%%%%%%%%%%%%%%%%%%%%%%%%%%%%%%%%%%%%%%%%%%%%%%%%%%%%%%%%%%%%%%%%
\firmlists
\begin{document}
\captiondelim{\quad}\captionnamefont{\footnotesize}\captiontitlefont{\footnotesize}
\selectlanguage{british}\frenchspacing
\maketitle

%%%%%%%%%%%%%%%%%%%%%%%%%%%%%%%%%%%%%%%%%%%%%%%%%%%%%%%%%%%%%%%%%%%%%%%%%%%%
%%% Abstract
%%%%%%%%%%%%%%%%%%%%%%%%%%%%%%%%%%%%%%%%%%%%%%%%%%%%%%%%%%%%%%%%%%%%%%%%%%%%
\abstractrunin
\abslabeldelim{}
\renewcommand*{\abstractname}{}
\setlength{\absleftindent}{0pt}
\setlength{\absrightindent}{0pt}
\setlength{\abstitleskip}{-\absparindent}
\begin{abstract}\labelsep 0pt%
  \noindent A collection of geometric and mathematical facts about
  simplices of any dimension and about probability distributions on them.
  This collection is regularly updated \\\noindent\emph{\footnotesize Note:
    Dear Reader \amp\ Peer, this manuscript is being peer-reviewed by you.
    Thank you.}
% \par%\\[\jot]
% \noindent
% {\footnotesize PACS: ***}\qquad%
% {\footnotesize MSC: ***}%
%\qquad{\footnotesize Keywords: ***}
\end{abstract}
\selectlanguage{british}\frenchspacing

%%%%%%%%%%%%%%%%%%%%%%%%%%%%%%%%%%%%%%%%%%%%%%%%%%%%%%%%%%%%%%%%%%%%%%%%%%%%
%%% Epigraph
%%%%%%%%%%%%%%%%%%%%%%%%%%%%%%%%%%%%%%%%%%%%%%%%%%%%%%%%%%%%%%%%%%%%%%%%%%%%
% \asudedication{\small ***}
% \vspace{\bigskipamount}
% \setlength{\epigraphwidth}{.7\columnwidth}
% %\epigraphposition{flushright}
% \epigraphtextposition{flushright}
% %\epigraphsourceposition{flushright}
% \epigraphfontsize{\footnotesize}
% \setlength{\epigraphrule}{0pt}
% %\setlength{\beforeepigraphskip}{0pt}
% %\setlength{\afterepigraphskip}{0pt}
% \epigraph{\emph{text}}{source}



%%%%%%%%%%%%%%%%%%%%%%%%%%%%%%%%%%%%%%%%%%%%%%%%%%%%%%%%%%%%%%%%%%%%%%%%%%%%
%%% BEGINNING OF MAIN TEXT
%%%%%%%%%%%%%%%%%%%%%%%%%%%%%%%%%%%%%%%%%%%%%%%%%%%%%%%%%%%%%%%%%%%%%%%%%%%%

\bigskip

I collect here some geometric and mathematical facts about simplices of any
dimensions, and about distributions of probability on them. I often use
them in my research and thought they can be useful to other people. I'll
update this collection regularly. Almost no explicit proofs are given,
although sometimes I give the idea of the proof. Please let me know if you
find any mistakes or if you think some other fact should be included.



\section{Simplex}
\label{sec:base}

Consider $N+1$ mutually exclusive and exhaustive propositions. The
distributions of their relative frequencies form an $N$-dimensional
simplex. Label the propositions $\set{0,\dotsc,N}$, denote a
relative-frequency distribution by $(q_0, \dotsc, q_N) \defs \yq$. In the
rest of this memo it's always implicitly assumed that $q_i\ge 0$, also in
the integration domains. Denote by
\begin{equation}
  \label{eq:def_orthant}
  \ysim_N \defd \set{(x_{0},\dotsc,x_{N}) \suchthat x_i\ge 0, \tsum_i x_i = 1}
\end{equation}
the $N$-dimensional simplex, or $N$\dd-simplex, which can be thought of as
embedded in the non-negative orthant of $\RR^{N+1}$. It is a convex space,
and therefore part of an affine space \citep{portamana2011_r2019}.


\section{Geometric properties}
\label{sec:geometry}

Let's examine some geometric properties of the $N$\dd-simplex, embedded as a
symmetric $(N+1)$-hyperhedron (triangle, tetrahedron, and so on) in
$N$-dimensional Euclidean space -- that is, a space with a flat metric.
Suppose its edges have length $\yL$; note that when it's naturally embedded
in the nonnegative orthant, then $\yL=\sqrt{2}$.

\begin{itemize}[wide]
\item Its hypervolume is
  \begin{equation}
    \label{eq:volume}
    \Biggl(\frac{\yL}{\sqrt{2}}\Biggr)^{N}\frac{\sqrt{N+1}}{N!} \;.
  \end{equation}
\item The hyperarea of a facet (equal to the hypervolume of one less
  dimension) is
  \begin{equation}
    \label{eq:area_facet}
    \Biggl(\frac{\yL}{\sqrt{2}}\Biggr)^{N-1}\frac{\sqrt{N}}{(N-1)!} \;.
  \end{equation}
\item The distance between the centre and a vertex is
  \begin{equation}
    \label{eq:centre_vertex}
    \frac{\yL}{\sqrt{2}} \sqrt{\frac{N}{N+1}} \;,
    \end{equation}
    which is also the radius of its circum-hypersphere.
\item The distance between the centre and the centre of a facet is
  \begin{equation}
    \label{eq:centre_facet}
    \frac{\yL}{\sqrt{2}}\frac{1}{\sqrt{N\,(N+1)}} \;,
  \end{equation}
  which is also the radius of its in-hypersphere.
\item The distance between a vertex and its opposite facet is therefore
  \begin{equation}
    \label{eq:vertex_facet}
    \frac{\yL}{\sqrt{2}}\sqrt{\frac{N+1}{N}} \;.
  \end{equation}
\item The distance between the centre and the origin in the embedding
  orthant is
  \begin{equation}
    \label{eq:centre_origin}
    \frac{\yL}{\sqrt{2}} \frac{1}{\sqrt{N+1}}\;.
  \end{equation}
  \mbox{}
\end{itemize}

Note the following curious facts as $N$ grows:
\begin{itemize}[wide]
\item the hypervolume is smaller and smaller compared to that of the
  embedding hypercube;
\item the ratio of the radii of circum- and in-hypersphere is $N$,
  therefore the ratio between their hypervolumes is $N^{N}$: most of the
  hypervolume of the $N$\dd-simplex is \enquote{at the corners};
\item the centre becomes closer and closer to the origin of the embedding
  orthant, with respect to the size of the edges;
\item the centre becomes closer and closer to the facets with respect to
  the distance to the edges;
\item almost all of the segment from a vertex to its opposite facet is
  between the vertex and the centre.
\end{itemize}







\section{Densities}
\label{sec:densities}

As basic
density (that is, volume element) we can take either
\begin{gather}
  \label{eq:vol_element_1N}
  \di q_1 \dotsm \di q_N\;,
  \qquad (q_1,\dotsc,q_N) \in \ysim_N\;,
\\\shortintertext{or}
  \label{eq:vol_element_0N}
  \di q_0 \dotsm \di q_N\;\delt(1-\ysum\yq)\;,
  \qquad (q_0,\dotsc,q_N) \in \clop{0,+\infty}^{N+1}\;,
\end{gather}
which is borrowed from a Euclidean volume element of the embedding space.
The latter leads to more symmetric formulae. The two densities are
equivalent, and their integration gives $1/N!$, as can be proven
inductively ($\ysim_k$ is the base of $\ysim_{k+1}$: multiply its
$k$-volume by a unit height and divide by $k+1$) or as shown in
\textcite[\sect~18.10]{jaynes1994_r2003}. Let's denote either density by
$\di\yq$. When~\eqref{eq:vol_element_1N} is intended, any $q_0$ that
appears in the integral must be understood as $q_0 \equiv 1-\sum_{i=1}^N q_i$.

\section{Flat prior}
\label{sec:flat}


The $N$-simplex has a natural convex structure. Thus the ratio of two
$N$-volumes is well-defined. There's only one normalized density that assigns
the same \dob\ to any two $N$-volumes having unit ratio:
\begin{equation}
  \label{eq:canonical_unif_density}
  % \frac{\sqrt{2}^N\,N!}{\sqrt{N+1}}\;
  N! \;\di\yq\;,
\end{equation}
called the flat prior.

\section{Jeffreys prior}
\label{sec:jeffreys}

It is also possible to embed the $N$-simplex into a hyperspherical surface
in $\clop{0,+\infty}^{N+1}$ via
\begin{equation}
  \label{eq:sphere_embedding}
  (q_1, \dotsc, q_N) \mapsto (x_0, x_1,\dotsc, x_N) =
  \bigl( \sqrt{1-q_1 -\dotsb -q_N}, \sqrt{q_1}, \dotsc, \sqrt{q_N} \bigr)\;.
\end{equation}
The normalized density induced by the Euclidean one is in this case
\begin{equation}
  \label{eq:jeffreys_prior}
  \Gamma\biggl( \frac{N+1}{2} \biggr)\;
  \frac{\di\yq}{\prod_{i=0}^{N}\sqrt{\pu q_i}}
\end{equation}

\section{Entropic prior}
\label{sec:entropic}

The relative-frequency distribution $\yq$ in $m$ observations can be
obtained in $\binom{m}{m\yq}$ ways, where
\begin{equation}
  \label{eq:multin_coeff}
  \binom{m}{m\yq} \defd
  \frac{m!}{(m q_0)! \dotsm (m q_N)!}
  \approx \exp[m \sh(\yq)]
\end{equation}
is the multinomial coefficient. If we have equal beliefs in the occurrence
of these ways and $m$ is very large, our belief about the relative
frequency $\yq$ can be approximated by the density
\begin{equation}
  \label{eq:entropy_prior}
  \frac{\exp[m \sh(\yq)]}{\int\!\di\yq\; \exp[m \sh(\yq)]}\;\di\yq\;,
\end{equation}
called the entropic prior.

% This density can be found as follows. $\ysim_N$ has $(N+1)$-volume
% $1/(N+1)!$, as can be easily proven inductively. This volume can be obtained
% as $1/(N+1)$ times the product of the $N$-volume, $\sqrt{2}^N\,V$ of the
% $N$-simplex with edge-length $\sqrt{2}$ symmetrically embedded as
% $\set{\yx \suchthat \yx\ge 0, \ysum\yx =1}$ and the height, the distance to
% the origin, which is the $(N+1)$-vector
% $\bigl(1/(N+1), \dotsc, 1/(N+1)\bigr)$ of length $1/\sqrt{N+1}$:
% \begin{equation}
%   \label{eq:volume_Np1_from_N}
%   \frac{1}{(N+1)!} = \frac{1}{N+1}\;\sqrt{2}^N\,V\;\frac{1}{\sqrt{N+1}}\;,
% \end{equation}
% from which we obtain $V = \frac{\sqrt{N+1}}{\sqrt{2}^N\,N!}$. Thus $1/V$ is
% the normalization of the density uniform in $(q_1, \dotsc, q_N)$.


\textcolor{white}{If you find this you can claim a postcard from me.}

\section{Metrics}
\label{sec:metrics}

A density on the simplex doesn't induce any canonical density on a
lower-dimensional subset. One way to induce a density on every
lower-dimensional subset is to equip the simplex with a metric
\citep[\chap~V]{choquetbruhatetal1977_r1996}[\chap~4]{bossavit1991}. We can
define a metric either in terms of intrinsic properties of the simplex or by embedding the simplex in a metric space.

\subsection{Flat metric}
\label{sec:flat_metric}

The flat metric is the one that respects the convex structure of the
simplex, in the sense that every two parallel $d$-dimensional subsets whose
$d$-volumes are in a ratio of one-to-one -- this can be calculated using
only the convex structure \citep{portamana2011_r2019} -- are given equal
$d$-volumes by the metric. The metric also allows the comparison of
non-parallel $d$-volumes, something that can't be done with the convex
structure alone. This metric can also be obtained by embedding the
$N$-simplex into the Euclidean space $\clop{0,+\infty}^{N+1}$ via
\begin{equation}
  \label{eq:flat_embedding}
  (q_1, \dotsc, q_N) \mapsto (x_0, x_1,\dotsc, x_N) =
  \bigl( 1-q_1 -\dotsb -q_N, q_1, \dotsc, q_N \bigr)
\end{equation}
and pulling-back the metric.

The embedding above has tangent map
\begin{gather}
  \label{eq:tangent_map_canonical}
  \begin{pmatrix}
-\yu \\ \yid_N    
\end{pmatrix}
\\\shortintertext{with}
\label{eq:def_u}
\yu \defd
\underbrace{\begin{pmatrix}
  1 &\dotso &1
\end{pmatrix}}_{N}\;,
\qquad
\yid_{N} \defd \text{identity $N$-matrix}\;.
\end{gather}

In coordinates $\yx$ the metric is represented by the identity matrix
$\yid_{N+1}$. The representation of its pull-back is therefore
\begin{equation}
  \label{eq:metric_pullback}
  \begin{pmatrix}
-\yu\T & \yid_N    
\end{pmatrix}
\;
\yid_{N+1}
\;
  \begin{pmatrix}
-\yu\T \\ \yid_N    
\end{pmatrix}
= \yid_N + \yu\T\yu\;.
\end{equation}
This is a matrix with all unit elements outside the diagonal and $2$ on the
diagonal. Its determinant can be found with Sylvester's theorem
\citep{sylvester1851,akritasetal1996}:
\begin{equation}
  \label{eq:sylvester}
  \det(\yid_N + \yu\T\yu) = \det(\yid_1 + \yu\yu\T) = 1 + N\;.
\end{equation}
This peculiar expression for the metric comes from the fact that $q_0$ is
a function of the other $q_i$. If we use $(q_0, q_1, \dotsc, q_N)$
as fictitious coordinates, multiplying by a delta for normalization
(\sect~\ref{sec:base}), then the metric is simply expressed by the identity
matrix $\yid_{N+1}$.

\section{Projections}
\label{sec:projections}

***

\section{Truncated normal distributions}
\label{sec:normal_distr}

A truncated normal distribution on the simplex is -- we would think --
proportional to
\begin{equation}
  \label{eq:normal}
\exp\Bigl[-\tfrac{1}{2}
(\yq-\yqr)\T \, \ySigma^{-1}\, (\yq-\yqr)
  \Bigr] \;\di\yq\;.
\end{equation}
It's truncated because it's defined on the simplex, a bounded space.

Let's first recall some facts about truncated normal distributions on any
bounded space, not just on the simplex:
\begin{itemize}
\item The normalization constant is not the one of the standard
  normal on $\RR^{N}$, for obvious reasons of integration domain.
\item The centre $\yqr$ is \emph{not} the mean of $\yq$:
  \begin{equation}
    \label{eq:normal_not_mean}
    \E(\yq) \ne \yqr \;!
  \end{equation}
  That's why it's better to call $\yqr$ \enquote{centre} (owing to the
  symmetries of the distribution function) or \enquote{location parameter}.
\item The scale matrix $\ySigma$ is \emph{not} the covariance of $\yq$:
  \begin{equation}
    \label{eq:normal_not_cov}
    \E(\yq\yq\T) \ne \ySigma \;!
  \end{equation}
  That's why it's better to call $\ySigma$ \enquote{scale form} than
  \enquote{covariance matrix}.
\item The scale form $\ySigma$ can also be negative-definite or
  indefinite. In the directions where its singular values are negative, the
  resulting distribution function in $\yq$ coordinates is $\cup$-shaped and
  has maxima at the boundaries. (Positive-definiteness is only required on
  unbounded spaces such as $\RR^{N}$: negative values would try to push the
  probability mass to infinity.)
\item If most mass is enough away from the boundaries (the distribution
  function is very peaked), so that there's little difference for the
  integrals if we extend them outside the simplex, then the covariance
  matrix will be approximately equal to the scale form $\ySigma$.
\end{itemize}

There are additional subtleties with the mathematical
expression~\eqref{eq:normal}, though, owing to the fact that it's defined
on a convex space, that is, a subset of an affine space.

First of all, remember from \sect~\ref{sec:projections} that
$\yu\T\yq = 1$ for every point $\yq$ and $\yq''$ of the simplex.
This means that -- as long as our integrals are strictly over the simplex
-- we can equivalently rewrite the normal~\eqref{eq:normal} as
\begin{equation}
  \label{eq:normal_equivalent}
  \begin{gathered}
\exp\Bigl[-\tfrac{1}{2}
(\yq-\yqr)\T \, \ySigmab^{-1}\, (\yq-\yqr)
\Bigr] \;\di\yq
\\
\text{with}\quad
\ySigmab^{-1} =
\ySigma^{-1} + \yya\yu\T + \yu\yya\T % + \yyb\yui
\end{gathered}
%\quad\text{\footnotesize equivalent to \eqref{eq:normal}}
\end{equation}
with \emph{arbitrary} $(N+1)$-vector $\yya$. Note that
$\ySigma^{-1}$ and $\ySigmab^{-1}$ have generally different (pseudo)inverses.

Second, from expression~\eqref{eq:normal_equivalent} we see that any
truncated normal can also be equivalently written in the form
\begin{equation}
  \label{eq:normal_simple}
  \exp\Bigl(-\tfrac{1}{2}
\yq\T \yDelta^{-1} \yq
  \Bigr) \;\di\yq
\end{equation}
with
\begin{equation}
  \label{eq:delta_simple_normal}
  \yDelta^{-1} \defd \ySigma^{-1} - \yu\yqr{}\T\ySigma^{-1} - \ySigma^{-1}\yqr\yu\T
  + \yu\yqr{}\T\ySigma^{-1}\yqr\yu\T \;.
\end{equation}
Also in this case $\yDelta^{-1}$ is determined but by an additional
$\yyb\yuu$ term which leads to a constant term that can be absorbed by the
normalization constant.

These two subtleties raise two questions. If we see expressions such
as~\eqref{eq:normal} or~\eqref{eq:normal_equivalent}, how do we know that
its $\ySigma^{-1}$ or $\yDelta^{-1}$ is \emph{really} an approximation of
the inverse covariance matrix? what if a $k\yuu$ term was actually added to
it? And how do we know what the centre $\yqr$ of the distribution is in the
expression~\eqref{eq:normal_equivalent}?

The possibility of expression~\eqref{eq:normal_equivalent} can be
understood intuitively as follows. A truncated normal on an $N$-dimensional
space has $N$ location parameters and $N\,(N+1)/2$ scale parameters. But
$\yDelta^{-1}$ in expression~\eqref{eq:normal_equivalent} comprises
$(N+1)\,(N+2)/2 \equiv N\,(N+1)/2 + N + 1$ parameters: of the additional
$N+1$, one is superfluous -- leading to the freedom of a constant additive
term -- and the rest encode the location parameter.



\section{Diverse}
\label{sec:diverse}

Suppose we have a covariance matrix -- equivalent to the inverse of a
quadratic form -- on a simplex. In the canonical embedding orthant, this
covariance matrix has a zero axis, corresponding to the plane of the
embedded simplex. Taking the \emph{pseudo}inverse leaves this axis to zero
while inverting the others to give the quadratic form. This doesn't affect
the results for vectors tangent to the simplex, since they have no
components along that axis.

Consider the projection from a higher-dimensional simplex to the previous
one via a hypergeometric distribution (\enquote{drawing without
  replacement}). This is done via a projection matrix $\ymm$, with the
hypergeometric distribution as its entries.

We can pull back the quadratic form from the lower-D to the higher-D
simplex, obtaining a degenerate quadratic form. This is done by multiplying
the quadratic form on the left and right by $\ymm\T$ and $\ymm$. This form
has a covariance matrix with several infinite axes, corresponding to the
kernel of the projection, and the zero axis corresponding to the direction
outside of the simplex.

\begin{itemize}
\item The projection matrix has a right inverse (pseudoinverse, and
  different from the transpose).
\item The projection matrix maps the uniform distribution to the uniform
  distribution; the inverse projection also maps the uniform distribution
  to the uniform distribution.
\item The previous fact implies that the projection and its pseudoinverse
  map the outside axes of covariance matrices in the two simplices to each other.
\item We can \enquote{add thickness} to the axis outside of the simplex by
  adding a constant term to \emph{all} entries of the covariance matrix.
\item The projection and its pull-back map the outside axis to itself.
\item Therefore we can add a constant term to the low-D covariance matrix
  or to its pull-back.
\item The pseudoinverse of the pulled-back quadratic form -- that is, its
  covariance matrix -- can be obtained by multiplying original covariance
  matrix by the pseudoinverses of the projectors. This, however, can only
  be done as long as we leave the axis outside the simplex equal to zero
  (so that the pseudoinverse doesn't touch it).
\item The pseudoinverse maps a spher
\end{itemize}





%%\setlength{\intextsep}{0.5ex}% with wrapfigure
%\begin{figure}[p!]%{r}{0.4\linewidth} % with wrapfigure
%  \centering\includegraphics[trim={12ex 0 18ex 0},clip,width=\linewidth]{maxent_saddle.png}\\
%\caption{caption}\label{fig:comparison_a5}
%\end{figure}% exp_family_maxent.nb


%%%%%%%%%%%%%%%%%%%%%%%%%%%%%%%%%%%%%%%%%%%%%%%%%%%%%%%%%%%%%%%%%%%%%%%%%%%%
%%% Acknowledgements
%%%%%%%%%%%%%%%%%%%%%%%%%%%%%%%%%%%%%%%%%%%%%%%%%%%%%%%%%%%%%%%%%%%%%%%%%%%% 
\iffalse
\begin{acknowledgements}
  \ldots to Mari \amp\ Miri for continuous encouragement and affection, and
  to Buster Keaton and Saitama for filling life with awe and inspiration.
  To the developers and maintainers of \LaTeX, Emacs, AUC\TeX, Open Science
  Framework, Python, Inkscape, Sci-Hub for making a free and unfiltered
  scientific exchange possible.
%\rotatebox{15}{P}\rotatebox{5}{I}\rotatebox{-10}{P}\rotatebox{10}{\reflectbox{P}}\rotatebox{-5}{O}.
\sourceatright{\autanet}
\end{acknowledgements}
\fi

%%%%%%%%%%%%%%%%%%%%%%%%%%%%%%%%%%%%%%%%%%%%%%%%%%%%%%%%%%%%%%%%%%%%%%%%%%%%
%%% Appendices
%%%%%%%%%%%%%%%%%%%%%%%%%%%%%%%%%%%%%%%%%%%%%%%%%%%%%%%%%%%%%%%%%%%%%%%%%%%% 
\clearpage
% %\renewcommand*{\appendixpagename}{Appendix}
% %\renewcommand*{\appendixname}{Appendix}
% %\appendixpage
% \appendix

%%%%%%%%%%%%%%%%%%%%%%%%%%%%%%%%%%%%%%%%%%%%%%%%%%%%%%%%%%%%%%%%%%%%%%%%%%%%
%%% Bibliography
%%%%%%%%%%%%%%%%%%%%%%%%%%%%%%%%%%%%%%%%%%%%%%%%%%%%%%%%%%%%%%%%%%%%%%%%%%%% 
\defbibnote{prenote}{{\footnotesize (\enquote{de $X$} is listed under D,
    \enquote{van $X$} under V, and so on, regardless of national
    conventions.)\par}}
% \defbibnote{postnote}{\par\medskip\noindent{\footnotesize% Note:
%     \arxivp \mparcp \philscip \biorxivp}}

\printbibliography[prenote=prenote%,postnote=postnote
]

\end{document}

%%%%%%%%%%%%%%%%%%%%%%%%%%%%%%%%%%%%%%%%%%%%%%%%%%%%%%%%%%%%%%%%%%%%%%%%%%%%
%%% Cut text (won't be compiled)
%%%%%%%%%%%%%%%%%%%%%%%%%%%%%%%%%%%%%%%%%%%%%%%%%%%%%%%%%%%%%%%%%%%%%%%%%%%% 


%%% Local Variables: 
%%% mode: LaTeX
%%% TeX-PDF-mode: t
%%% TeX-master: t
%%% End: 
